\documentclass{article}
\usepackage{doublespace}
\usepackage{fullpage}
\usepackage{epsfig}

\title{HDTV Encoder/Decoder writeup/description}
\author{Andrew Lamb}
\date{\today}


\begin{document}
\maketitle
\newpage

\section{Introduction/Overview}
This article presents a partial implementation in StreamIT of the HDTV standard 
for transmitting digital television signals as described in 
the ASTC Standard\cite{atsc:a53b}.




\section{Background on Encoding/Decoding}

\subsection{Data Randomizer}

\subsection{Reed-Solomon Codes}

\subsection{Data Interleaving (Convolutional Interleaving)}

\subsection{Trellis Coder}
\begin{figure}
\center
\epsfxsize=5.5in
\epsfbox{images/trellis-encoder.eps}
\caption{8VSB trellis encoder, precoder, and symbol mapper.}
\label{fig:trellis-system}
\end{figure}

Trellis coding (also known as convolution coding) 
is a method of encoding where the value of a single input 
bit affects the value of several output bits. The encoders
keep state and use that state to generate the output of the 
encoder. 

A particular trellis code is 
specified by at least three paramters. The first, the rate, is typically expressed as
a fraction n/k. n data bits are provided to the convolution encoder and k bits are produced,
where n>= k. The second, the reach K, is the number of output bits that the
each input bit affects (eg the amount of memory that is in the system). 
The third, is a set of generator polynomials which specify the system connections.
For more information see \cite{fleming:tutorial}.

Figure~\ref{fig:trellis-system} shows the trellis encoder used for the HDTV system. 
The total trellis encoder is actually a rate 2/3 (2 input bits, 3 output bits) 
encoder which combines a precoder (1/1) and an ``Optimal 4 bit 
Ungerboeck code,'' (1/2).

Each three bit 'symbol' produced by the trellis coder is then mapped to a value
that is transmitted over the transmission medium (cable or satellite) to 
a HDTV receiver. 

The combination of the precoder and Ungerboeck coder followed by the symbol mapper
have several desirable properties for error correction. I do not have sufficient
theoritcal background to explain exactly what the properties are, or why they are good,
so you will have to take it on faith as have I.


\subsubsection{Precoding}
\begin{figure}
\center
\epsfxsize=3.5in
\epsfbox{images/precoder.eps}
\caption{8VSB precoder}
\label{fig:precoder}
\end{figure}

\begin{table}
\center
\begin{tabular}{c|c|c}
input(x2) & output(y2) & state  \\
\hline
0 & 0 & 0 \\
1 & 1 & 0 \\
1 & 0 & 1 \\
0 & 0 & 0 \\
1 & 1 & 0 \\
0 & 1 & 1 \\
1 & 0 & 1 \\
1 & 1 & 0 \\
1 & 0 & 1 \\
0 & 0 & 0 \\
\end{tabular}
\caption{Example of precoding 01101011110:}
\label{tbl:precoding_example}
\end{table}

A precoder is a simple circuit which XORs the current input with the previous
output to get the current output. Figure~\ref{fig:precoder} is a diagram of the
precoder used in the HDTV encoding pipeline. Table~\ref{tbl:precoding_example} 
contains a worked out example of the input, output, and state of the precoder
for coding the data \texttt{01101011110}.


\begin{figure}
\center
\epsfxsize=3.5in
\epsfbox{images/de-precoder.eps}
\caption{8VSB de-precoder}
\label{fig:deprecoder}
\end{figure}

\begin{table}
\center
\begin{tabular}{c|c|c}
input(y2) & output(x2) & state  \\
\hline
0 & 0 & 0 \\
1 & 1 & 0 \\
0 & 1 & 1 \\
0 & 0 & 0 \\
1 & 1 & 0 \\
1 & 0 & 1 \\
0 & 1 & 1 \\
1 & 1 & 0 \\
0 & 1 & 1 \\
0 & 0 & 0 \\
\end{tabular}
\caption{Example of de-precoding 0100110100:}
\label{tbl:de_precoding_example}
\end{table}

To decode the precoded data, we simply reverse the process. It might be the case that 
the Viterbi algorithm should be applied to recover the input sequence, and we will
look into it. Figure~\ref{fig:deprecoder} is a schematic of the de precoder that was
implemented, and Table~\ref{tbl:de_precoding_example} contains a worked out example 
for the data \texttt{0100110100}.


\subsubsection{"Optimal 4 state Ungerboeck code"}
The internal Ungerboeck trellis encoder used in the HDTV encoder is
a 1/2 trellis code with a reach of 3. 

\begin{table}
\center
\begin{tabular}{c|c|c}
input & state   & output  \\
      & (s0,s1) & (z1,z0) \\
\hline
0 & 00 & 00 \\
1 & 01 & 10 \\
1 & 11 & 11 \\
0 & 11 & 01 \\
1 & 10 & 11 \\
0 & 01 & 00 \\
1 & 11 & 11 \\
1 & 10 & 11 \\
1 & 00 & 10 \\
0 & 00 & 00 \\
\end{tabular}
\caption{Example of encoding 0110101110 with an Ungerboeck code.}
\label{tbl:ungerboeck_example}
\end{table}

Decoding a bit stream that has been encoded with a trellis encoder is usually
done using the Virterbi decoding algorithm. The Virterbi decoding algorithm
is a well known algorithm in communication coding which uses a dynamic programming
approach to determine the sent signal. Chip Fleming has a great 
tutorial\cite{fleming:tutorial} which contains a walk through of the implementation
of a trellis encoder and a Viterbi decoder for a slightly different 1/2 trellis code. 



\begin{figure}
\center
\epsfxsize=5.5in
\epsfbox{images/trellis-state-transition.eps}
\caption{Trellis Encoder State transition diagram.}
\label{fig:trellis-state-diagram}
\end{figure}



\section{StreamIT Implementation}



\begin{small}
\begin{singlespace}
\bibliographystyle{abbrv}
\bibliography{references}
\end{singlespace}
\end{small}




\end{document}


