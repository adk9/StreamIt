\documentclass{article}
%\usepackage{doublespace}
%\usepackage{fullpage}
\usepackage{lscape}
\usepackage{epsfig}

\title{HDTV Encoder/Decoder writeup/description}
\author{Andrew Lamb}
\date{\today}


\begin{document}
\maketitle
\newpage

\section{Introduction/Overview}
This article presents a partial implementation in StreamIT of the HDTV standard 
for transmitting digital television signals as described in 
the ASTC Standard\cite{atsc:a53b}.

\section{Background on Encoding/Decoding}

\subsection{Data Randomizer}

\subsection{Reed-Solomon Codes}

\subsection{Data Interleaving (Convolutional Interleaving)}
\begin{figure}
\center
\epsfxsize=5.5in
\epsfbox{images/convolution-interleaver.eps}
\caption{8VSB convolutional interleaver and associated de-interleaver.}
\label{fig:convolution-interleaver}
\end{figure}

Convolutional Interleaving is a way of interleaving data that apparently makes
it rather immune to noise. All of the delay blocks in Figure~\ref{fig:convolution-interleaver}
are initialized to 0. All references to convolutional interleaving that I could find
on the internet were either too vague (eg descriptions of what particular ASIC designs did
or only vaguely described the encoder. I could not find a reference to what a decoder looked 
like. I sat down and sketeched out the diagram with the appropriate delay blocks, and I soon 
saw that it is obvious once you see what the interleaver is doing what the deinterleaver should
be doing. I decided that since I didn't see it immediately when first presented with a
diagram of a convolution encoder other readers might not either.

Therefore, Figure~\ref{fig:convolution-interleaver} shows an example of a convolutional
interleaver with 5 paths. For the encoder, at each each time step the switch goes to the
next position and gets the next byte of data. In exactly the same manner, the output
switch advances one tick each time step and after grabbing data from that delay block.
The Ds are delay blocks, and not registers. In particular, they actually delay the signal
rather than gating their input at each clock cycle.

The deinterleaver part of Figure~\ref{fig:convolution-interleaver} shows how the 
original signal is recovered from the interleaved version. 





\subsection{Trellis Coder}
\begin{figure}
\center
\epsfxsize=5.5in
\epsfbox{images/trellis-encoder.eps}
\caption{8VSB trellis encoder, precoder, and symbol mapper.}
\label{fig:trellis-system}
\end{figure}

Trellis coding (also known as convolution coding) 
is a method of encoding where the value of a single input 
bit affects the value of several output bits. The encoders
keep state and use that state to generate the output of the 
encoder. 

A particular trellis code is 
specified by at least three paramters. The first, the rate, is typically expressed as
a fraction n/k. n data bits are provided to the convolution encoder and k bits are produced,
where n>= k. The second, the reach K, is the number of output bits that the
each input bit affects (eg the amount of memory that is in the system). 
The third, is a set of generator polynomials which specify the system connections.
For more information see \cite{fleming:tutorial}.

Figure~\ref{fig:trellis-system} shows the trellis encoder used for the HDTV system. 
The total trellis encoder is actually a rate 2/3 (2 input bits, 3 output bits) 
encoder which combines a precoder (1/1) and an ``Optimal 4 bit 
Ungerboeck code,'' (1/2).

Each three bit 'symbol' produced by the trellis coder is then mapped to a value
that is transmitted over the transmission medium (cable or satellite) to 
a HDTV receiver. 

The combination of the precoder and Ungerboeck coder followed by the symbol mapper
have several desirable properties for error correction. I do not have sufficient
theoritcal background to explain exactly what the properties are, or why they are good,
so you will have to take it on faith as have I.


\subsection{Precoding}
\begin{figure}
\center
\epsfxsize=3.5in
\epsfbox{images/precoder.eps}
\caption{8VSB precoder}
\label{fig:precoder}
\end{figure}

\begin{table}
\center
	\begin{tabular}{c|c|c}
input(x2) & output(y2) & state  \\
\hline
0 & 0 & 0 \\
1 & 1 & 0 \\
1 & 0 & 1 \\
0 & 0 & 0 \\
1 & 1 & 0 \\
0 & 1 & 1 \\
1 & 0 & 1 \\
1 & 1 & 0 \\
1 & 0 & 1 \\
0 & 0 & 0 \\
\end{tabular}
\caption{Example of precoding 01101011110:}
\label{tbl:precoding_example}
\end{table}

A precoder is a simple circuit which XORs the current input with the previous
output to get the current output. Figure~\ref{fig:precoder} is a diagram of the
precoder used in the HDTV encoding pipeline. Table~\ref{tbl:precoding_example} 
contains a worked out example of the input, output, and state of the precoder
for coding the data \texttt{01101011110}.


\begin{figure}
\center
\epsfxsize=3.5in
\epsfbox{images/de-precoder.eps}
\caption{8VSB de-precoder}
\label{fig:deprecoder}
\end{figure}

\begin{table}
\center
\begin{tabular}{c|c|c}
input(y2) & output(x2) & state  \\
\hline
0 & 0 & 0 \\
1 & 1 & 0 \\
0 & 1 & 1 \\
0 & 0 & 0 \\
1 & 1 & 0 \\
1 & 0 & 1 \\
0 & 1 & 1 \\
1 & 1 & 0 \\
0 & 1 & 1 \\
0 & 0 & 0 \\
\end{tabular}
\caption{Example of de-precoding 0100110100:}
\label{tbl:de_precoding_example}
\end{table}

To decode the precoded data, we simply reverse the process. It might be the case that 
the Viterbi algorithm should be applied to recover the input sequence, and we will
look into it. Figure~\ref{fig:deprecoder} is a schematic of the de precoder that was
implemented, and Table~\ref{tbl:de_precoding_example} contains a worked out example 
for the data \texttt{0100110100}.


\subsection{``Optimal 4 state Ungerboeck code''}

\subsubsection{Encoder}

\begin{figure}
\center
\epsfxsize=3.5in
\epsfbox{images/ungerboeck.eps}
\caption{Optimal 4 state Ungerboeck Encoder}
\label{fig:ungerboeck}
\end{figure}

\begin{figure}
\center
\epsfxsize=3.5in
\epsfbox{images/trellis-state-transition.eps}
\caption{Trellis Encoder State transition diagram.}
\label{fig:trellis-state-diagram}
\end{figure}

The internal Ungerboeck encoder used in the HDTV encoder is
a 1/2 convolution code with a reach of 3. Figure~\ref{fig:ungerboeck}
shows a schematic diagram of an Ungerboeck encoder. For each input bit
that appears at the input Y1, two output bits are produces at Z1 and Z0.
The internal D (delay) blocks are registers which each hold state, marked
S0 and S1. We can also think about the encoder as a finite state machine.
Figure~\ref{fig:trellis-state-diagram} is a fsm representation of the ungerboeck
encoder used in HDTV. Note that the initial state is 00, corresponding to 
both registers in Figure~\ref{fig:ungerboeck} being set to zero.

\begin{figure}
\center
\epsfxsize=5.5in
\epsfbox{images/trellis-blank.eps}
\caption{Trellis diagram for the 4 state Ungerboeck Code}
\label{fig:trellis-blank}
\end{figure}

The reason that convolution codes are often called trellis codes is shown by 
Figure~\ref{fig:trellis-example}. The x axis represents time (discrete points)
and the y axis represents the possible states of the encoder. The solid lines
represent state transitions for an input of 0. The dashed lines represent
state transitions for an input of 1. As the input
data is encoded, a path is traced through the trellis. Each input sequence maps 
to exactly one path and each path maps exactly to one input sequence. It
is also important to note that each state only has 

\begin{figure}
\center
\epsfxsize=5.5in
\epsfbox{images/trellis-example.eps}
\caption{Path traced through the trellis for input sequence 01101011110}
\label{fig:trellis-example}
\end{figure}

\begin{table}
\center
\begin{tabular}{c|c|c}
input & state   & output  \\
      & (s0,s1) & (z1,z0) \\
\hline
0 & 00 & 00 \\
1 & 01 & 10 \\
1 & 11 & 11 \\
0 & 11 & 01 \\
1 & 10 & 11 \\
0 & 01 & 00 \\
1 & 11 & 11 \\
1 & 10 & 11 \\
1 & 00 & 10 \\
0 & 00 & 00 \\
\end{tabular}
\caption{Example of encoding 0110101110 with an Ungerboeck code.}
\label{tbl:ungerboeck_example}
\end{table}

Table~\ref{tbl:ungerboeck_example} works out an example of encoding 
the data \texttt{0110101110} with our encoder,
and Figure~\ref{fig:trellis-example} shows the corresponding path through
the trellis. The grey states correspond to the state of the encoder 
at that time. 

\subsubsection{Decoding}

Decoding a bit stream that has been encoded with a convolution code is usually
done using the Virterbi decoding algorithm. The Virterbi decoding algorithm
is a well known algorithm in communication coding which is used in CDMA cellphones
and satellite receivers among other things. The Viterbi algorithm uses a dynamic programming
approach to determine the most likely signal sent given a recieved (possibly corrupted) signal. 
Chip Fleming has a great tutorial\cite{fleming:tutorial} which contains a walk 
through of the implementation of a trellis encoder and a Viterbi decoder 
for a slightly different 1/2 trellis code. His tutorial was the inspiration for this document.

Since each input bit in a convolution code affects several output bits
the Viterbi decoding algorithm reads in several input symbols in order to determine
several output symbols (eg it processes several symbols at a timie). In the
signal processing literature (reference?) it appears that one can get most
of the error correcting advantages of the convolution code by processing 7K input symbols
(where K is the reach of the code as described above). 5K symbols also yields
acceptable results.

Since each input sequence maps to a particular path through the trellis,
The main idea of the Vertibi decoding algorithm is to determine the most
likely path throught the trellis given the received symbols. The straightforward
approach would be to examine \textbf{all} possible paths through the trellis, and
choose the path's whose output is closest to the received symbols. The
straightforward approach is rediculously computationally intensive, 
and totally uncecessary. The Viterbi algorithm (named after Andrew Viterbi, a founder of Qualcomm)
uses a dynamic programming aproach to trace out the paths of the encoder's state. 

We will perform 7K steps in each iteration of the algorithm. 
At each step, we read 7K 2-bit symbol
from the input and we output 7K 1-bit symbols at the output. 
We are tasked with filling out a table: each element in the table, $c[s,t]$, 
represents the accumulated error if the encoder were in state s at time t. 
Accumulated error is definied as hamming distance
(or euclidian distance) of received symbol to the expected symbol.

Our problem: we are sitting at the decoder, and we receive symbol $R_t$ at time $t$. We want to fill in 
$c[s,t*]$ for all possible states $\mathcal{S}$ (the Ungerboeck encoder we have 4 states).

Algorithm:
\begin{enumerate}
\item For each received Symbol $R_t$ at time $t$
	\begin{enumerate}
	\item Compute $c[s,t] \forall s \in \mathcal{S}$ (eg compute the entries
 	       of c[s,t] for each of the possible 4 states.)
	\item Remember which predecessor state was used to compute each $c[s,t]$ in
	       a predecessor table.
	\end{enumerate}
\item Reconstruct the path through the trellis with c and the predecessor states.
	\begin{enumerate}
	\item find $s^*$ such that $c[s^*,7K]$ is minimum. (eg at the end of the trellis, 
	       after we have looked at all input symbols, choose the state we would 
	       be in at the last time. 
	\item We then use the predecessor table to walk back through the trellis
	       along the path which yields the least possible error (which we have 
	       been calculating along the way.
	\item Once we have recovered our most likely path, we can recreate the
	       input sequence that was most likely sent.
	\end{enumerate} 
\end{enumerate}

Determining $c[s,t]$ for all possible states (in 1(a) above) is not obvious.
The algorithm to determine it is explained below.  

\begin{enumerate}
\item For each received symbol $R_t$ received at time $t$ we compute 
       $c[s,t] \forall s \in \mathcal{S}$ by using the entries of $c$ for 
       the previous time (eg $c[s,t-1]$).
	\begin{enumerate}
	\item For each $s \in \mathcal{S}$
		\begin{enumerate}
		\item Let $q^0$ and $q^1$ be the two possible previous states
		       that lead to $s$. $q^0\rightarrow s$ when an input of 0
		       was received, and $q^1\rightarrow s$ when an input of 1 
		       was received. These transitions produce symbols $R^0$ and $R^1$
		       respectively.
		\item Let $e^0$ be the hamming distance between $R_t$ and $R^0$ (eg the hamming
		       distance between the received symbol and the expected symbol if the input was 0). 
		       Similarly, let $e^1$ be the hamming distance between $R_t$ and $R^1$.
		\item Now, calculate $c[s,t] = min(c[q^0,t-1] + e^0, c[q^1,t-1] + e^1)$
		\item And remember which state ($q^0$ or $q^1$) was a predecessor state to s in the 
		       predecessor table.
		\end{enumerate}
	\end{enumerate}
\end{enumerate}

Since that discussion was filled with lots of math and symbols, 
lets make the algorithm concrete with a worked through example.

We will first decode the input sequence of symbols
\texttt{00 10 11 01 11 00 11 11 10 00} which we know was
produced by encoding the data \texttt{0110101110} from before.

We will fill two tables as we walk ourselves through the trellis:
the accumulated error table, $c[state,time]$ and the predecessor state
table, $psh[state,time]$. Since we know that the decoder is in state 00 
initially (eg time=0) we let $c[s,0]=\infty, for s\in\{01,10,11\}$ (eg accumulated errors 
for initial states other than 00 are set to $\infty$

\paragraph{Time 1, received Symbol = 00}
We are sitting at the decoder, and at time 1 the symbol 00 comes in. According to
the Viterbi algorithm, we look at all possible transitions that could have occured 
at time 1, calculate the accumulated error metric for each possible transition and fill
out our chart for $c[s,1]$ accordingly.

\begin{table}
\center
\textbf{Possible Transitions}
\begin{tabular}{l|l|l|l|l}
(input)$s0 \rightarrow s1$ & hamming distance between   & accumulated error  & Choice & Action \\
              (output)     & output and received symbol & if this transition &        &        \\
\hline
$(0) 00\rightarrow00 (00)$ & $ham(00, 00) = 0$ & $c[00][0] + 0 = 0 + 0 = 0$           & $*$ & $c[00][1] \leftarrow 0$ \\
$(1) 10\rightarrow00 (10)$ & $ham(00, 10) = 1$ & $c[10][0] + 0 = \infty + 0 = \infty$ &     & $psh[00][1] \leftarrow 00$ \\
\hline
$(0) 10\rightarrow01 (00)$ & $ham(00, 00) = 0$ & $c[10][0] + 0 = \infty + 0 = \infty$ &     & $c[01][1] \leftarrow 1$ \\
$(1) 00\rightarrow01 (10)$ & $ham(00, 10) = 1$ & $c[00][0] + 0 = 0 + 1 = 1$           & $*$ & $psh[01][1] \leftarrow 00$ \\
\hline
$(0) 01\rightarrow10 (01)$ & $ham(00, 01) = 1$ & $c[01][0] + 0 = \infty + 1 = \infty + 1$ & $*$ & $c[10][1] \leftarrow \infty + 1$ \\
$(1) 11\rightarrow10 (11)$ & $ham(00, 11) = 2$ & $c[11][0] + 0 = \infty + 2 = \infty + 2$ &     & $psh[10][1] \leftarrow 01$ \\
\hline
$(0) 11\rightarrow11 (01)$ & $ham(00, 01) = 1$ & $c[11][0] + 0 = \infty + 1 = \infty + 1$ & $*$ & $c[11][1] \leftarrow \infty + 1$ \\
$(1) 01\rightarrow11 (11)$ & $ham(00, 11) = 2$ & $c[01][0] + 0 = \infty + 2 = \infty + 2$ &     & $psh[11][1] \leftarrow 11$ \\
\end{tabular}
\caption{Decoding \texttt{00101101110011111000} at time = 1.}
\label{tbl:decode_time1}
\end{table}

\begin{figure}
\center
\epsfxsize=5.5in
\epsfbox{images/trellis-decode-1.eps}
\caption{Path traced through the trellis for input sequence 00101101110011111000, time = 1}
\label{fig:decode_time1}
\end{figure}




Table~\ref{tbl:decode_time1} demonstrates the calculations that we go through to 
determine the next row of the c table. The ``Choice'' column represents the choice of
$min(c[q^0,t-1] + e^0, c[q^1,t-1] + e^1)$ from the algorithm, and the right most column
represents the updates that we make to both tables.

Figure~\ref{fig:decode_time1} represents the path through the trellis	 
that we have started to explore. Paths with accumulated errors of $\infty$ are ignored.



 

\paragraph{Time 2, received Symbol = 10}

\begin{table}
\center
\textbf{Possible Transitions}
\begin{tabular}{l|l|l|l|l}
(input)$s0 \rightarrow s1$ & hamming distance between   & accumulated error  & Choice & Action \\
              (output)     & output and received symbol & if this transition &        &        \\
\hline
$(0) 00\rightarrow00 (00)$ & $ham(10, 00) = 1$ & $c[00][0] + 1 = 0 + 1 = 1$                     & $*$ & $c[00][2] \leftarrow 1$ \\
$(1) 10\rightarrow00 (10)$ & $ham(10, 10) = 0$ & $c[10][0] + 0 = (\infty + 1) + 0 = \infty + 1$ &     & $psh[00][2] \leftarrow 00$ \\
\hline
$(0) 10\rightarrow01 (00)$ & $ham(10, 00) = 1$ & $c[10][0] + 1 = (\infty + 1) + 1 = \infty + 1$ &     & $c[01][2] \leftarrow 0$ \\
$(1) 00\rightarrow01 (10)$ & $ham(10, 10) = 0$ & $c[00][0] + 0 = 0 + 0 = 0$                     & $*$ & $psh[01][2] \leftarrow 00$ \\
\hline
$(0) 01\rightarrow10 (01)$ & $ham(10, 01) = 2$ & $c[01][0] + 2 = 1 + 2 = 3$                     & $*$ & $c[10][2] \leftarrow 3$ \\
$(1) 11\rightarrow10 (11)$ & $ham(10, 11) = 1$ & $c[11][0] + 1 = (\infty + 1) + 1 = \infty + 2$ &     & $psh[10][2] \leftarrow 01$ \\
\hline
$(0) 11\rightarrow11 (01)$ & $ham(10, 01) = 2$ & $c[11][0] + 2 = (\infty + 1) + 2 = \infty + 3$ &     & $c[11][2] \leftarrow 2$ \\
$(1) 01\rightarrow11 (11)$ & $ham(10, 11) = 1$ & $c[01][0] + 1 = 1 + 1 = 2                    $ & $*$ & $psh[11][2] \leftarrow 01$ \\
\end{tabular}
\caption{Decoding \texttt{00101101110011111000} at time = 2.}
\label{tbl:decode_time2}
\end{table}

\begin{figure}
\center
\epsfxsize=5.5in
\epsfbox{images/trellis-decode-2.eps}
\caption{Decode paths traced through the trellis for input sequence 00101101110011111000, time = 2}
\label{fig:decode_time2}
\end{figure}





\paragraph{Time 3, received Symbol = 11}
\begin{table}
\center
\textbf{Possible Transitions}
\begin{tabular}{l|l|l|l|l}
(input)$s0 \rightarrow s1$ & hamming distance between   & accumulated error  & Choice & Action \\
              (output)     & output and received symbol & if this transition &        &        \\
\hline
$(0) 00\rightarrow00 (00)$ & $ham(11, 00) = 2$ & $c[00][0] + 2 = 1 + 2 = 3$ & $*$ & $c[00][2] \leftarrow 3$ \\
$(1) 10\rightarrow00 (10)$ & $ham(11, 10) = 1$ & $c[10][0] + 1 = 3 + 1 = 4$ &     & $psh[00][2] \leftarrow 00$ \\
\hline
$(0) 10\rightarrow01 (00)$ & $ham(11, 00) = 2$ & $c[10][0] + 2 = 3 + 2 = 5$ &     & $c[01][2] \leftarrow 2$ \\
$(1) 00\rightarrow01 (10)$ & $ham(11, 10) = 1$ & $c[00][0] + 1 = 1 + 1 = 2$ & $*$ & $psh[01][2] \leftarrow 00$ \\
\hline
$(0) 01\rightarrow10 (01)$ & $ham(11, 01) = 1$ & $c[01][0] + 1 = 0 + 1 = 1$ & $*$ & $c[10][2] \leftarrow 1$ \\
$(1) 11\rightarrow10 (11)$ & $ham(11, 11) = 0$ & $c[11][0] + 0 = 2 + 0 = 2$ &     & $psh[10][2] \leftarrow 01$ \\
\hline
$(0) 11\rightarrow11 (01)$ & $ham(11, 01) = 1$ & $c[11][0] + 1 = 2 + 1 = 3$ &     & $c[11][2] \leftarrow 0$ \\
$(1) 01\rightarrow11 (11)$ & $ham(11, 11) = 0$ & $c[01][0] + 0 = 0 + 0 = 2$ & $*$ & $psh[11][2] \leftarrow 01$ \\
\end{tabular}
\caption{Decoding \texttt{00101101110011111000} at time = 3.}
\label{tbl:decode_time3}
\end{table}

\begin{figure}
\center
\epsfxsize=5.5in
\epsfbox{images/trellis-decode-3.eps}
\caption{Decode paths traced through the trellis for input sequence 00101101110011111000, time = 3}
\label{fig:decode_time3}
\end{figure}



\paragraph{Time 4, received Symbol = 01}
\begin{table}
\center
\textbf{Possible Transitions}
\begin{tabular}{l|l|l|l|l}
(input)$s0 \rightarrow s1$ & hamming distance between   & accumulated error  & Choice & Action \\
              (output)     & output and received symbol & if this transition &        &        \\
\hline
$(0) 00\rightarrow00 (00)$ & $ham(01, 00) = 1$ & $c[00][0] + 1 = 3 + 1 = 4$ &     & $c[00][2] \leftarrow 3$ \\
$(1) 10\rightarrow00 (10)$ & $ham(01, 10) = 2$ & $c[10][0] + 2 = 1 + 2 = 3$ & $*$ & $psh[00][2] \leftarrow 10$ \\
\hline
$(0) 10\rightarrow01 (00)$ & $ham(01, 00) = 1$ & $c[10][0] + 1 = 1 + 1 = 2$ & $*$ & $c[01][2] \leftarrow 2$ \\
$(1) 00\rightarrow01 (10)$ & $ham(01, 10) = 2$ & $c[00][0] + 2 = 3 + 2 = 5$ &     & $psh[01][2] \leftarrow 10$ \\
\hline
$(0) 01\rightarrow10 (01)$ & $ham(01, 01) = 0$ & $c[01][0] + 0 = 2 + 0 = 2$ &     & $c[10][2] \leftarrow 1$ \\
$(1) 11\rightarrow10 (11)$ & $ham(01, 11) = 1$ & $c[11][0] + 1 = 0 + 1 = 1$ & $*$ & $psh[10][2] \leftarrow 11$ \\
\hline
$(0) 11\rightarrow11 (01)$ & $ham(01, 01) = 0$ & $c[11][0] + 0 = 0 + 0 = 0$ & $*$ & $c[11][2] \leftarrow 0$ \\
$(1) 01\rightarrow11 (11)$ & $ham(01, 11) = 1$ & $c[01][0] + 1 = 1 + 1 = 2$ &     & $psh[11][2] \leftarrow 11$ \\
\end{tabular}
\caption{Decoding \texttt{00101101110011111000} at time = 4.}
\label{tbl:decode_time4}
\end{table}

\begin{figure}
\center
\epsfxsize=5.5in
\epsfbox{images/trellis-decode-4.eps}
\caption{Decode paths traced through the trellis for input sequence 00101101110011111000, time = 4}
\label{fig:decode_time4}
\end{figure}


\paragraph{Time 5, received Symbol = 11}

\begin{table}
\center
\textbf{Possible Transitions}
\begin{tabular}{l|l|l|l|l}
(input)$s0 \rightarrow s1$ & hamming distance between   & accumulated error  & Choice & Action \\
              (output)     & output and received symbol & if this transition &        &        \\
\hline
$(0) 00\rightarrow00 (00)$ & $ham(11, 00) = 2$ & $c[00][0] + 2 = 3 + 2 = 5$ &     & $c[00][2] \leftarrow 2$ \\
$(1) 10\rightarrow00 (10)$ & $ham(11, 10) = 1$ & $c[10][0] + 1 = 1 + 1 = 2$ & $*$ & $psh[00][2] \leftarrow 10$ \\
\hline
$(0) 10\rightarrow01 (00)$ & $ham(11, 00) = 2$ & $c[10][0] + 2 = 1 + 2 = 3$ & $*$ & $c[01][2] \leftarrow 3$ \\
$(1) 00\rightarrow01 (10)$ & $ham(11, 10) = 1$ & $c[00][0] + 1 = 3 + 1 = 4$ &     & $psh[01][2] \leftarrow 10$ \\
\hline
$(0) 01\rightarrow10 (01)$ & $ham(11, 01) = 1$ & $c[01][0] + 1 = 2 + 1 = 3$ &     & $c[10][2] \leftarrow 0$ \\
$(1) 11\rightarrow10 (11)$ & $ham(11, 11) = 0$ & $c[11][0] + 0 = 0 + 0 = 0$ & $*$ & $psh[10][2] \leftarrow 11$ \\
\hline
$(0) 11\rightarrow11 (01)$ & $ham(11, 01) = 1$ & $c[11][0] + 1 = 0 + 1 = 1$ & $*$ & $c[11][2] \leftarrow 1$ \\
$(1) 01\rightarrow11 (11)$ & $ham(11, 11) = 0$ & $c[01][0] + 0 = 2 + 0 = 2$ &     & $psh[11][2] \leftarrow 11$ \\
\end{tabular}
\caption{Decoding \texttt{00101101110011111000} at time = 5.}
\label{tbl:decode_time5}
\end{table}

\begin{figure}
\center
\epsfxsize=5.5in
\epsfbox{images/trellis-decode-5.eps}
\caption{Decode paths traced through the trellis for input sequence 00101101110011111000, time = 5}
\label{fig:decode_time5}
\end{figure}


\paragraph{Time 6, received Symbol = 00}
(...)
\paragraph{Time 7, received Symbol = 11}
(...)
\paragraph{Time 8, received Symbol = 11}
(...)
\paragraph{Time 9, received Symbol = 10}
(...)
\paragraph{Time 10, received Symbol = 00}
\begin{figure}
\center
\epsfxsize=5.5in
\epsfbox{images/trellis-decode-10.eps}
\caption{Decode paths traced through the trellis for input sequence 00101101110011111000, time = 10}
\label{fig:decode_time10}
\end{figure}





\begin{figure}
\center
\epsfxsize=5.5in
\epsfbox{images/trellis-decode-traceback.eps}
\caption{Traceback of decode path. We choose to start at state 00 at time=10 \
         because it has the least accumulated error. Following the arrows backwards through \
	 the predecessor state table yields the same path that the encoder went through.}
\label{fig:decode_traceback}
\end{figure}



\begin{table}
\center
\textbf{Accumulated Errors ($c[state,time]$)}

\begin{tabular}{c|c|c|c|c|c|c|c|c|c|c|c}
state/time & 0 & 1 & 2 & 3 & 4 & 5 & 6 & 7 & 8 & 9 & 10 \\
\hline
\textbf{11} & $\infty$ & $\infty+1$ & 2 & 0 & 0 & 1 & 2 & 0 & 1 & 0 & 0 \\
\hline
\textbf{10} & $\infty$ & $\infty+1$ & 3 & 1 & 1 & 0 & 3 & 1 & 0 & 1 & 1 \\
\hline
\textbf{01} & $\infty$ &          1 & 0 & 2 & 2 & 3 & 0 & 2 & 3 & 2 & 2 \\
\hline
\textbf{00} &        0 &          0 & 1 & 3 & 3 & 2 & 1 & 3 & 2 & 3 & 3 \\
\hline
\end{tabular}

\textbf{Previous State History}

\begin{tabular}{c|c|c|c|c|c|c|c|c|c|c|c}
state/time & 0 & 1 & 2 & 3 & 4 & 5 & 6 & 7 & 8 & 9 & 10 \\
\hline
\textbf{11} & & 11 & 01 & 01 & 11 & 11 & 11 & 01 & 11 & 11 & 01 \\
\hline
\textbf{10} & & 01 & 01 & 01 & 11 & 11 & 11 & 01 & 11 & 11 & 01 \\
\hline
\textbf{01} & & 00 & 00 & 00 & 10 & 10 & 10 & 00 & 10 & 10 & 00 \\
\hline
\textbf{00} & & 00 & 00 & 00 & 10 & 10 & 10 & 00 & 10 & 10 & 00 \\
\hline
\end{tabular}

\caption{Completed Error and State History tables for example.}
\label{tbl:decode-state-tables}
\end{table}



\subsection{Multiple Trellis Encoders and Interleaving}

\begin{figure}
\center
\epsfxsize=5.5in
\epsfbox{images/multiple-trellis.eps}
\caption{}
\label{fig:multiple-trellis}
\end{figure}

\begin{figure}
\center
\epsfxsize=5.5in
\epsfbox{images/trellis-interleaving.eps}
\caption{Data reordering from the trellis encoders}
\label{fig:trellis-interleaving}
\end{figure}


The trellis encoding is done by 12 encoders in parallel shown in
Figure. The
order in which 






\section{StreamIT Implementation}


\subsection{Trellis Encoder/Decoder Implementation}

\subsubsection{Ungerboeck Decoder Implementation}
Each table is two dimensional: states and time. Each unit of time corresponds to
receiving 1 symbol (so in this case receiving 2 bits). Each cell represents something
for a particular state at a particular time. The algorithm fills up both tables as
time progresses. When the tables are full, it traces back through the table and finds the
path that results in the fewest errors occuring.

The error metric that is used can either be hamming distance or euclidian distance (basically
hamming distance squared).

The first table keeps track of accumulated errors. accumulated\_error[s,t] is the hamming distance
between the output that the current min path to state s at time t would generate and the
symbols that have been received. 

The second table keeps track of the previous states, so that the min path
can be reconstructed. predecessor\_states[s,t] represents the state s' of the encoder
in the minimum path that leads to s at time (t-1) 




\begin{small}
%\begin{singlespace}
\bibliographystyle{abbrv}
\bibliography{references}
%\end{singlespace}
\end{small}


\section{Appendix: Stream Graph}

\begin{landscape}
\begin{figure}
\center
\epsfxsize=8in
\epsfbox{images/hdtv-stream-graph.eps}
\caption{Stream Graph of the HDTV application.}
\label{fig:hdtv-stream-graph}
\end{figure}
\end{landscape}


\end{document}


