\begin{figure}
\center
\epsfxsize=4.5in
\epsfbox{images/convolution-interleaver.eps}
\caption{8VSB convolutional interleaver and associated de-interleaver.}
\label{fig:convolution-interleaver}
\end{figure}

Convolutional Interleaving is a way of interleaving data that apparently makes
it rather immune to noise. All of the delay blocks in Figure~\ref{fig:convolution-interleaver}
are initialized to 0. All references to convolutional interleaving that I could find
on the internet were either too vague (eg descriptions of what particular ASIC designs did
or only vaguely described the encoder. I could not find a reference to what a decoder looked 
like. I sat down and sketeched out the diagram with the appropriate delay blocks, and I soon 
saw that it is obvious once you see what the interleaver is doing what the deinterleaver should
be doing. I decided that since I didn't see it immediately when first presented with a
diagram of a convolution encoder other readers might not either.

Therefore, Figure~\ref{fig:convolution-interleaver} shows an example of a convolutional
interleaver with 5 paths. For the encoder, at each each time step the switch goes to the
next position and gets the next byte of data. In exactly the same manner, the output
switch advances one tick each time step and after grabbing data from that delay block.
The Ds are delay blocks, and not registers. In particular, they actually delay the signal
rather than gating their input at each clock cycle.

The deinterleaver part of Figure~\ref{fig:convolution-interleaver} shows how the 
original signal is recovered from the interleaved version. 

\begin{figure}
\center
\epsfxsize=3.0in
\epsfbox{streamgraphs/SGConvolutionalInterleaver.eps}
\epsfxsize=3.0in
\epsfbox{streamgraphs/SGConvolutionalDeinterleaver.eps}
\caption{Stream graph of the Convoltional Interleaver(left) and Deinterleaver(right).}
\label{fig:sg-convolution-interleave}
\end{figure}


