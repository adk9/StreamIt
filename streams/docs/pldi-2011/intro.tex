\section{Introduction}

The domain of stream programs is important because it stands at the
intersection of trends in applications and architectures.  Stream
programming naturally represents applications such as audio, video,
digital signal processing, and data analysis; applications that are
increasing prevalent as computing moves towards data-centric
applications and to the mobile and embedded space.  Also, by virtue of
their structure -- a graph of independent computational nodes (termed
{\it filters} with explicit and regular communication -- stream
programs are a natural fit for exploiting coarse-grained parallelism
suitable for multicore architectures.  The interest in streaming
applications has spawned a number of streaming languages that target
the streaming domain, including StreamIt~\cite{streamitcc},
Brook~\cite{brook04}, Cg~\cite{cg03}, Baker~\cite{Baker},
SPUR~\cite{spur05samos}, Spidle~\cite{spidle03}, Lime~\cite{lime10},
and SPL~\cite{spl09}.

In a stream program, filters define an atomic execution step that
repeats for many iterations (because data is continuously feed into
the application); each execution step discards a number of data items
from filter's input edge.  Often, a filter does not discard all the
data items that it read for the current execution step, requiring
these inspected (but not discarded) items for a future iteration (or
iterations) of the filter.  This type of filter is described as
performing a sliding window computation on its input. Sliding window
computations are prevalent in stream programs.  Examples of sliding
window computations include FIR filters; moving averages and
differences; error correcting codes; motion estimation; and network
packet inspection.  Some programming languages (e.g., Brook, Lime,
StreamIt, and IBM SPL) go so far as to include idioms that directly
represent sliding window computation, allowing the programmer to
specify, for each filter, the size of the window and the number of
items discarded after an execution of the filter.  Furthermore, a
recent study of a large streaming benchmark suite written in the
StreamIt programming language finds that 17 of the 30 real-world
benchmarks include at least one filter that performs a sliding window
computation~\cite{streamit-suite}.

A goal of stream programming is to directly expose to the software
layer the necessary information to enable automatic management of
coarse-grained parallelism.  Stream programs expose multiple forms of
parallelism: pipeline parallelism that exists between producers and
consumers; task parallelism that exists between pairs of filters on
parallel branches of the stream graph; and data parallelism that
exists when a filter is stateless and can thus be replicated in a SPMD
fashion.  Data parallelism is the most attractive, as it provides
load-balanced and limitless parallelism (as long as input data is
available).  A filter that is stateful, and cannot be
data-parallelized, becomes a limit to parallelization scalability, as
the work of that filter cannot be divided; the most load-intensive
stateful filter becomes the bottleneck for steady-state execution.

This paper presents a compiler framework for data-parallelizing
filters that perform sliding window computations when the properties
of the sliding window can be calculated statically.   Without explicitly
representing sliding windows in the language, a sliding window would
have to be expressed as state retained between iterations of a
filter.  If sliding window filters required state, this state would
represent a new parallelization bottleneck in 11 of the
17 real-world benchmarks in the StreamIt Benchmark Suite 
that contain sliding windows~\cite{streamit-suite}.  Focusing
on the Channelvocoder benchmark, this state would limit scalability to
18 cores, whereas our techniques scale to at least 64 cores.

Data-parallelizing a filter is performed via a transformation termed
{\it fission} (verb form {\it fiss})~\cite{streamit-asplos}.  Fission
is the process of data-parallelizing a stateless filter by duplicating
the filter a certain number of ways, assigning duplicates to distinct
cores, and correctly distributed input data to and collecting the
output data from the duplicates.  The duplicated filters are referred
to as {\it products}.  Fission requires that the sliding window
properties and the output rates of a filter are statically
determinable and constant.  When a sliding window is present, fission
is accomplished by duplicating certain input items since they are
required by multiple products.  This duplication translates into
inter-core communication, possibly a limiting factor for scalability when
targeting multicore architectures.

Previous approaches duplicate each input data item to all products,
with products ignoring (decimating) items that are not
needed~\cite{streamit-asplos}.  We will show that this strategy limits
scalability for certain multicores by requiring too much inter-core
communication.  In contrast, our strategy precisely routes each input
item to the minimal set of product filters that requires the item.
Unlike previous work, our techniques are defined on
multiple input and multiple output filters, removing the need
introduce synchronization filters that serialize input data before and
output data after the product filters.  

Our techniques operate on {\it static-rate} stream graphs, meaning
that the number of items produced, the number of items consumed, and
the number of items inspected by each filter can be determined
statically.  Because of this property, a steady-state schedule of
filter firings can be calculated that does not grow buffers and can be
executed indefinitely~\cite{lee87}.  Our techniques are conscious of
the spatial locality between producers and consumers.  Our framework
includes techniques that can determine when spatial locality can be
increased by altering the steady-state schedule.  When applicable, our
techniques can reduce the overall sharing (and thus inter-core
communication) requirement to below a threshold percent of the total
input communication for each sliding window filter that is
data-parallelized.  This technique trades latency for reductions in
inter-core communication.

The framework presented is defined on a model of computation that is
agnostic of source language.  To evaluate our techniques we have
implemented them in the context of the StreamIt compiler
infrastructure~\cite{gordon-asplos06}.  Our transformations are guided
by the parallelization management techniques presented
in~\cite{gordon-asplos06}.  We employ 3 real-world benchmarks from the
StreamIt Benchmark Suite~\cite{streamit-suite} that include of sliding
window computation.  We demonstrate the effectiveness of our
techniques by comparing them to previously published techniques on 2
multicore architectures: a 16-core SMP shared-memory multicore and a
64-core distributed-memory multicore.  We show that our techniques are
required to achieve scalable parallelization on both architectures,
achieving a 5.5x mean speedup on the 16-core SMP and a 1.8x mean
speedup on the 64-core distributed memory multicore over a previously
published technique.

\subsection{Contributions}
This paper makes the following contributions:
\begin{itemize}
  \myitem{Motivation for Exposing Sliding Windows in Stream
    Languages}: Without exposing sliding windows in the language, it
  requires heroic effort by the compiler to analyze the input patterns
  of such a filter. Without success, the compiler will not be able to
  data-parallelize these filters.  This will severely limit
  parallelization scalability for streaming applications with sliding
  window computations.

  \myitem{Generalized Fission of Sliding Window Filters}: We present a
  transformation that fisses sliding window filters with multiple
  input and multiple outputs.  The technique also supports filters
  that with multiple schedules of execution.  General fission defines
  a precise pattern of communication of input data to the products
  that can be reasoned upon by our other techniques.

  \myitem{Sharing Reduction}: We present a technique that decides when
  it is possible to decrease the amount of sharing between fission
  products by altering the steady-state of a stream graph, thus
  decreasing inter-core communication.  The technique reasons about
  all the sliding window filters of the stream graph, and when
  possible, reduces the sharing requirement to below a given threshold
  percent of the total input of the filters.

  \myitem{Data Parallelization of Stream Graph}: We present a
  framework for data-parallelizing all of the filters of a stream
  graph employing the fission transformation on individual filters and
  applying sharing reduction when possible.  This framework optimizes
  for spatial locality and enables the compiler to automatically and
  effectively manage parallelization across varying multicore
  architectures.
\end{itemize}
