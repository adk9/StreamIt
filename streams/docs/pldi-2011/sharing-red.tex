\section{Sharing Reduction}

Since streaming applications typically execute for
many iterations of the steady-state, it is legal to increase the
steady state by multiplying each filter's steady-state multiplicity by
the same constant.  As long as this constant $c$ is less than the number
of steady-state iterations $I$ of the application, and $c$ is a factor of
$I$, then the transformation is legal.  As mentioned above, we call
this {\it increasing} the steady-state by $c$.  This transformation
increases the latency of the steady-state.

By recognizing this property, we can directly reduce the amount of
inter-core communication that occurs between the fission products of a
fissed peeking filter for certain cases of general fission, including
the common case.  As we have seen above, the duplicated shared items
translate directly into inter-core communication for the common case
of fissing a producer and consumer by the same width.  For producer
$f$ and consumer $g$, fissed by $P$, where each $f_i$ is mapped the to
same core as $g_i$, Equation~\ref{eq:min-dup} gives the formula for
calculating the percentage of the total items that are duplicated for
a fission application of $g$.  In the equation we can directly control
the steady-state multiplicity of $g$, $M(S,g)$.  We can calculate a
constant $c_g$ such that the percentage is less than a threshold
$T_{\mt{sharing}}$:

\begin{equation}
\label{eq:ic-thresh}
c_g = \frac{1}{T_{\mt{sharing}}} \cdot \frac{C(g) \cdot P}{M(S,g) \cdot o(W,g)}
\end{equation}

\noindent where $0.0 < T_{\mt{sharing}} \le \mt{InterCore}$. Increasing the
steady-state of the graph by $c_g$ before general fission is applied
will assure that the percentage of items duplicated will be equal to
$T_{\mt{sharing}}$.  For each producer $f$ and consumer $g$ that is
duplicated by the same amount, and mapped to cores such that $f_i$ is
mapped to same core as $g_i$, this will ensure that at most
$T_{\mt{sharing}}$ fraction of total items are communicated inter-core.

The analysis of the inter-core communication requirements of the
fission of producer $f$ and consumer $g$ has so far assumed that $C(g)
= \mt{dup}_g $.  This is not always the case however.  If $C(g) >
\mt{dup}_g$, the communication between the fission products of $f$ and
$g$ will be unaligned.  Figure~\ref{fig:remaining-dup} illustrates
this case.  In the graph before fission is applied, $C(g)$ items will
remain in the input buffer between steady-state iterations.  After
fission by $P$, for each steady-state, $g_1$ will first consume the
$C(f)$ items that $f_p$ sent it from the previous steady-state
iteration.  $g_1$ will then require:

\[ M(S, g)/P \cdot o(W, g) + \mt{dup} - C(g)\]

\noindent additional items from $f_1$.\footnote{It is guaranteed that
  $M(S, g)/P \cdot o(W, g) > C(g)$ by the precondition of general
  fission.} This requirement is less than the number of items that
$f_1$ produces $M(S, g)/P \cdot o(W, g)$ since $C(g) > \mt{dup}$. So,
$C(g) - \mt{dup}$ items must be transferred from $f_1$ to $g_2$ in
addition to the shared and duplicated $\mt{dup}$. In all, $C(g)$ items
must be transferred from $f_i$ to $g_{i+1}$. Thus, even in the general
case where $C(g) > \mt{dup}_g$, Equation~\ref{eq:min-dup} calculates
the percentage of total communication that is inter-core for the
fission of $g$ by $P$ given the original steady-state. Furthermore,
Equation~\ref{eq:ic-thresh} calculates the constant $c_g$ by which to
increase the steady-state so that inter-core communication resulting
from the fission of $g$ is equal to $T_{\mt{sharing}}$ fraction of
total communication.

\subsection{Sharing Reduction for Other Fission Cases}

The sharing reduction optimization does not apply as simply to all
cases of fission of a peeking filter.  Consider the case where we have
single output producer $f$ and single input consumer $g$ with $(f,g)
\in E$, but $f$ and $g$ are fissed by differing widths, i.e., $P_f \ne
P_g$.  This case engenders inter-core communication not only because
of sharing but because of the communication required by the differing
fission widths.  For simplicity of analysis, let us assume that $C(g)
= 0$, and $P_f$ is a multiple of $P_g$. Figure~\ref{fig:diff-widths}
presents an example with these assumptions.  In the example, $f$ and
$g$ are fissed by 8 and 4 respectively.  Since we want to maintain
data parallelism, each $f_i$, $1 \le i \le P_f$, is mapped to a
distinct core, as is each $g_j$, $1 \le j \le P_g$.  Since each $f_i$
produces half the number of items required by each $g_j$, half the
communication in this case is inter-core, caused by the misalignment
of the rates of the producer and consumer fission products.

Going forward, let us assume that $P_f > P_g$.  If $P_f$ is a multiple
of $P_g$, the analysis is straightforward: the percentage of inter-core
communication to total communication is $(1 - \frac{P_g}{P_f}$). If $P_f$ is
not a multiple of $P_g$ misalignment also occurs because for one or
more $g_j$, it's input does not begin in alignment with any $f_i$'s.
The analysis of the percent of inter-core communication for this latter
case is complex, but it suffices for our purposes to bound it by
$(1 - \frac{P_g}{P_f})$.

Additionally, if $C(g) > 0$, it is required to share $C(g)$ items
between two cores.  There are $P_g$ sections of $C(g)$ items, and this
data has to be communicated inter-core to one other core.  Given this
analysis, we can quantify an {\it approximation} of the percentage of
inter-core communication for the fission application of $g$ when $P_f
> P_g$:

\begin{align}
\mt{InterCore}(g) & = & \frac{(1 - \frac{P_g}{P_f}) \cdot M(S, g) \cdot o(W, g) +
P_g \cdot C(g)}{M(S, g) \cdot o(W, g)} \\
\label{eq:diff-fiss}
\mt{InterCore}(g) & = & (1 - \frac{P_g}{P_f}) +
\frac{P_g \cdot C(g)}{M(S, g) \cdot o(W, g)}
\end{align}

The first term on the RHS of Equation~\ref{eq:diff-fiss} gives the
percentage of inter-core communication produced by the fission width
inequality.  The second term quantifies the percentage of inter-core
communication caused by the sharing due to peeking.  

Consider the case where $g$ has multiple inputs. In this case $g$
receives $\mt{RI}(f, g, S) \cdot M(S, g) \cdot o(W,g)$ items from $f$
during the steady-state, and $(1 -\mt{RI}(f, g, S) \cdot M(S, g) \cdot
o(W,g))$ from its other producers. There is a choice: for which
producer of $g$ to optimize the communication of $g$? If $f$ is
chosen, then the assignment of filters to cores should minimize the
inter-core communication by co-locating the fission products of $f$
and $g$ just as the assignment would when $g$ is single input.
However, the calculation must now account for the inter-core
communication of the other producers sending to the fission products
of $g$. We define $\mt{InterCore}(g, f)$ to approximate the percentage
of inter-core communication to total communication for the input of
the fission products of $g$ optimizing for the placement of the
fission products of $f$:

\begin{equation}
\label{eq:tcomm-fopt1}
 \mt{InterCore}(g,f)   = (1 - \mt{RI}(f,
   g, S))  + (1 - \frac{P_g}{P_f}) \cdot \mt{RI}(f,
   g, S)   + \frac{P_g \cdot C(g)}{M(S, g) \cdot o(W, g)} 
\end{equation}

\noindent The first term accounts for the percentage of items that $g$'s fission
products receive from producers that are not products of $f$.  The
second term, the percentage items that are communicated inter-core
because of the fission width mis-alignment, must now account for the
fact that not all items are from $f$.  Equation~\ref{eq:tcomm-fopt1}
can be simplified to:

\begin{equation}
\label{eq:tcomm-fopt}
 \mt{InterCore}(g, f)   =  (1 -\mt{RI}(f,g, S) \cdot \frac{P_g}{P_f})
 + \frac{P_g \cdot C(g)}{M(S, g) \cdot o(W, g)}
\end{equation}

How do we decide whether it is worthwhile to apply the sharing
reduction optimization to $g$ given Equation~\ref{eq:tcomm-fopt}? When
the sharing reduction optimization is applied, the steady-state
multiplicity of $g$ is increased by a constant $c$.  If this is
applied to Equation~\ref{eq:tcomm-fopt}, the first term will be
unaffected, however the second term can be reduced because a smaller
percentage of steady-state items needs to be duplicated.  By Amdahl's
law, sharing reduction can only decrease $\mt{InterCore}(g,
f)$ to under $T_{\mt{sharing}}$ if the second term of Equation~\ref{eq:tcomm-fopt}
accounts for more than $(1-T_{\mt{sharing}} )$ of the duplication.  We
define a constant $T_{\mt{apply}}$ such that sharing
reduction should only be applied to $g$ optimizing for $f$ if:

\begin{equation}
T_{\mt{sharing}}  >  T_{\mt{apply}} \ge (1 -\mt{RI}(f,g, S) \cdot
\frac{P_g}{P_f})
\end{equation}

Finally, since it was assumed that $P_f > P_g$, in order to generalize
the above equation to account for the opposite case, we must choose
the appropriate fraction of local communication by taking the minimum
of the two fission width ratios:

\begin{equation}
\label{eq:apply-sharing}
T_{\mt{sharing}}  >  T_{\mt{apply}} \ge (1 -\mt{RI}(f,g, S) \cdot
\min(\frac{P_g}{P_f}, \frac{P_f}{P_g}))
\end{equation}

\noindent As $T_{\mt{apply}} \rightarrow 0.0$ the application of
sharing reduction to $g$ will approach $T_{\mt{sharing}}$. Notice in
the common case where $f$ and $g$ are fissed by the same width, and
$f$ has single output and $g$ has single input, the RHS of
Equation~\ref{eq:apply-sharing} evaluates to 0, indicating to always
apply sharing reduction for this case.

\subsection{Sharing Reduction Applied to the Stream Graph}
So far we have considered the application of the sharing reduction
optimization to a single filter $g$ in the stream graph.  Now we will
cover how to apply sharing reduction across all the filters of the
stream graph for which it is appropriate.  The goal is reduce the
percentage of inter-core communication due to the sharing between
fission products of a peeking filter to {\it approximately}
$T_{\mt{sharing}}$.  Sharing reduction may not achieve
$T_{\mt{sharing}}$ because there may exist peeking filters that are to
be fissed for which Equation~\ref{eq:apply-sharing} cannot be
satisfied.  For these filters, we do not apply the sharing reduction
optimization.

The process of applying sharing reduction to the entire stream graph
consists of determining which filters are appropriate and determining
a steady-state multiplier by which to increase the graph.  To
determine the steady-state graph multiplier, the compiler calculates
the percentage of sharing over all of the filters which adhere to
Equation~\ref{eq:apply-sharing}.  Let $\Phi$ denote the set of filters
for which Equation~\ref{eq:apply-sharing} holds and that we seek to
fiss:

\begin{equation}
\label{eq:sr-mult}
c = \frac{1}{T_{\mt{sharing}}} \cdot \frac{\sum_{g \in \Phi} P_g \cdot C(g) }{\sum _{g \in
    \Phi} M(S, g) \cdot o(W, g)}
\end{equation}

\noindent Increasing the steady-state by $c$ for all filters of the
graph will reduce the total sharing for the filters of $\Phi$ to
$T_{\mt{sharing}}$.  