\section{Running the StreamIt Compiler and StreamIt Programs}

\subsection{Introduction}
The actual compiler takes StreamIt Java programs and makes either 
uniprocessor C code, or files ready for the RAW simulator. Most
of the old applications are written in the old Java syntax. 
New programs should be written in the new StreamIt syntax.
We have a nice new syntax spec (see {\tt streams/docs/syntax}).


\subsection{{\tt knit}}
The simplest way to run the StreamIt compiler is to run the 
{\tt knit} program on an {\tt .str} file. 

To try it out, go to the {\tt streams/apps/examples/hello}
directory and type {\tt knit HelloWorld6.str}. The HelloWorld6 example
will be our example program throughout the rest of this section. 
It is the simplest program that you can write in StreamIt. It doesn't
actually print HelloWorld, it merely prints out numbers in
increasing order, starting from 0.

The knit script does the following three things for you:
\begin{enumerate}
  \item Convert {\tt HelloWorld6.str} to {\tt HelloWorld6.java} with the frontend.
  \item Compile {\tt HelloWorld6.java} to {\tt HelloWorld6.c} with the compiler.
  \item Compile {\tt HelloWorld6.c} to {\tt a.out} using gcc and linking against the C library.
\end{enumerate}

\subsection{Viewing the Stream Graph}
Following the execution of the compiler, you can see graphs of various
views of the program in the {\tt .dot} files in the directory.  For
example, the original stream graph of the application is in {\tt
before.dot}.  These dot files can be viewed following the directions
in Section~\ref{sec:dot}.

\subsection{Running Compiled StreamIt Programs}
StreamIt programs are infinite loops -- if you simply execute the {\tt a.out} 
program generated from HelloWorld6, you will see a counter increasing 
rapidly and it won't stop until you hit CTL-C.

To force a StreamIt program's main loop to execute a finite number
of times, you can include the {\tt -i NUM} command line argument.
This will cause the StreamIt program to execute {\tt NUM} steady state
{\it (i)terations} and quit. For example, running {\tt a.out -i 10} will 
cause the following output.

\begin{verbatim}
cagfarm-46 % ./a.out -i 10
0
1
2
3
4
5
6
7
8
9
\end{verbatim}



\subsection{Frontend {\tt .str} $\rightarrow$ {\tt .java}}
To manually convert a {\tt .str} file into a StreamIt {\tt .java} 
file, you can use the following command:

\begin{verbatim}
java streamit.frontend.ToJava input.str > output.java
\end{verbatim}

The frontend will convert all of the files on the command line
into a single StreamIt {\tt .java} file which will appear on
standard out (STDOUT). You can either redirect STDOUT to a file, 
or you can use the command line option {\tt --output filename}.
To convert the HelloWorld6 program from {\tt .str} to {\tt .java}
you would use the following command:

\begin{verbatim}
java streamit.frontend.ToJava HelloWorld6.str > HelloWorld6.java
\end{verbatim}

\subsection{Java Library}
To execute a StreamIt {\tt .java} program using the Java library
you should simply be able to compile your {\tt .java} file using
{\tt jikes}. In the HelloWorld6 example from above,
you could do the following steps to compile and run the HelloWorld
application for 10 iterations.

\begin{verbatim}
java streamit.frontend.ToJava HelloWorld6.str > HelloWorld6.java
jikes HelloWorld6.java
java HelloWorld6 -i 10
\end{verbatim}

\subsection{StreamIt Compiler}
You can run the StreamIt compiler by issuing the following command
\begin{verbatim}
java -Xmx1500M  at.dms.kjc.Main -s filename.java
\end{verbatim}

Currently, the compiler is rather space inefficient, so to get large 
programs to compile you sometimes need to use the {\tt -Xmx1500M} option,
which sets the JVM's max heap size to 1.5GB.
Most people have the above command aliased to something like {\tt s}, which
can be accomplished by putting the following line in your {\tt .cshrc} file.

\begin{verbatim}
alias s "java -Xmx1500M  at.dms.kjc.Main -s "
\end{verbatim}


\subsubsection{{\tt .java} $\rightarrow$ {\tt .c}}
With no arguments other than file names, the compiler will produce
a C program on STDOUT. This C program can be compiled 
and linked against the C library to generate executables for a 
uniprocessor. To actually compile the C program with gcc 
you need to execute the following command to include the appropriate
headers and link against the appropriate libraries.

\begin{verbatim}
gcc -O2 -I/u/username/streams/library/c -L/u/username/streams/library/c \
    HelloWorld6.c -lstreamit -lsrfftw -lsfftw -lm
\end{verbatim}

{\tt gcc} will generate {\tt a.out} as an executable. If you want a different
name try using the {\tt -o filename.exe} command link argument.
Again, most people have a script that
automatically includes all of the required arguments.
Carrying on the same HelloWorld example from above, you could generate
and run the program for 10 executions using the C library with the following
commands:

\begin{verbatim}
java streamit.frontend.ToJava HelloWorld6.str > HelloWorld6.java
java -Xmx1500M  at.dms.kjc.Main -s HelloWorld6.java > HelloWorld6.c
gcc -O2 -I/u/username/streams/library/c -L/u/username/streams/library/c \
    HelloWorld6.c -lstreamit -lsrfftw -lsfftw -lm
./a.out -i 10
\end{verbatim}


\subsubsection{{\tt .java} $\rightarrow$ {\tt raw}}
To generate programs that will execute on the RAW simulator
you can pass the {\tt --raw X} command line argument to the compiler,
where {\tt X} specifies that you want to use an {\tt X}x{\tt X}
RAW machine. Unless you
have good reason, you should use {\tt --raw 4} which
corresponds to the size of the actual chip. After compiling
for RAW, there will be several new files in your directory.
To execute the simulator (assuming you have BTL set up correctly)
you can use one of the following commands:

\begin{verbatim}
make -f Makefile.streamit
\end{verbatim}

\begin{verbatim}
make -f Makefile.streamit debug
\end{verbatim}

\begin{verbatim}
make -f Makefile.streamit run
\end{verbatim}


So, one final time, using the HelloWorld6 application,
to compile and run on the RAW simulator {\tt btl}, 
you would use the following sequence of commands.

\begin{verbatim}
java streamit.frontend.ToJava HelloWorld6.str > HelloWorld6.java
java -Xmx1500M  at.dms.kjc.Main -s --raw 4 HelloWorld6.java
make -f Makefile.streamit run
\end{verbatim}


\subsubsection{Command Line Arguments}
The command line arguments for the StreamIt compiler are
very ephemeral -- they change from day to day. The best way 
to figure out which ones are available is to look at the
source file where they are defined. This file is
{\tt /streams/compiler/at/dms/kjc/KjcOptions.opt}.