\section{Background and Related Work}
\label{sec:related}

\subsection{Synchronous Dataflow}

Strictly speaking, synchronous dataflow (SDF) refers to a class of
dataflow graphs where each node can consume and produce multiple items
per firing, and the number of items produced and consumed is constant
from one invocation to the next~\cite{LM87-i,LM87-ii}.  For SDF
graphs, it is straightforward to determine a legal steady state
schedule using balance equations~\cite{leesdf} (see
Section~\ref{sec:balance}).  The SDF scheduling community focuses on
``single-appearance schedules'' (SAS) in which each filter appears at
only one position in the loop nest denoting the schedule.  The
interest in SAS is because the runtime overhead of a method call is
prohibitive, so the code for each block is inlined.  However, code
size is so important that they can't afford to duplicate the inlining,
thus leading to the restriction that each filter appear only once.
They have demonstrated that minimizing buffer size in general graphs
is NP-complete, but they explore two heuristics to do it in acyclic
graphs~\cite{Bhatta97}.

Cyclo-static dataflow (CSDF) is the generalization of SDF to the case
where each node has a set of phases with different I/O rates that are
cycled through repeatedly~\cite{BELP96,Parks95}.  Compared to SDF,
CSDF is more natural for the programmer for writing phased filters,
since periodic tasks can be separated.  For instance, consider a
channel decoder that inputs a large matrix of coefficients at regular
intervals, but decodes one element at a time for $N$ invocations in
between each set of coefficients.  By having two separate phases, the
programmer doesn't have to interleave this functionality in a single,
convoluted work function.  Also, CSDF can have lower latency and more
fine-grained parallelism than SDF.  However, current scheduling
techniques for CSDF involve converting each phase to an SDF actor,
which can cause deadlock even for legal graphs~\cite{BELP96} and does
not take advantage of the fine-grained dataflow exposed by CSDF.

%% The computation graph (CG) was described by Karp and
%% Miller~\cite{Karp67} as a model of computation that, unlike the two
%% above, allows the nodes to {\it peek}.  By ``peek'' we mean that a
%% node can read and use a value from a channel without consuming
%% (popping) it from the channel; peeking is good for the programmer
%% because it automates the buffer management for input items that are
%% used on multiple firings, as in an FIR filter.  However, peeking
%% complicates the schedule because an initialization schedule is needed
%% to ensure that enough items are on the channel in order to
%% fire~\cite{Gordo02}.  In~\cite{Karp67}, conditions are given for when
%% a computation graph terminates and when its data queues are bounded.

A dependence graph (DG) is often presented as an alternate
representation for a System of Uniform Recurrence
Equations~\cite{Karp67,DRV00}.  Sometimes a DG is collapsed into a
reduced dependence graph (RDG) in which there is only one node per
variable~\cite{DRV00}.  Of course, these graphs have (by definition) a
one-to-one correspondence with a set of recurrence equations, but they
are more trivial than the graphs described above: they only produce
and consume one item per firing and do not have a concept of
cyclic phases.

Govindarajan et al. develop a linear programming framework for
determining the ``rate-optimal schedule'' with the minimal memory
requirement~\cite{GGD94}.  A rate-optimal schedule is one that takes
advantage of parallel resources to execute the graph with the maximal
throughput.  Though their formulation is attractive and bears some
likeness to ours, it is specific for rate-optimal schedules and does
not directly relate to the polyhedral model.  It also can result in a
code size blow up, as the same node could be executed in many
different contexts.

\subsection{Systems of Affine Recurrence Equations}

A System of Affine Recurrence Equations (SARE) is a set of equations
${\cal E}_1 \dots {\cal E}_{m_{\cal E}}$ of the following
form~\cite{MQRS90,Feautrier92i}:
\begin{equation}
\forall {\vec i} \in {\cal D_{\cal E}}:~~X({\vec i}) = f_{\cal E}(\dots, Y({\vec h}_{{\cal
E}, Y}({\vec i})), \dots)
\label{eq:sare}
\end{equation}
In this equation, our notation is as follows:
\begin{itemize}

\item $\{X, Y, \dots\}$ is the set of {\it variables} defined by the
SARE.  Variables can be considered as multi-dimensional arrays mapping
a vector of indices ${\vec i}$ to a value in some space ${\cal V}$.

\item ${\cal D_{\cal E}}$ is a polyhedron representing the {\it
domain} of the equation ${\cal E}$.  Each equation for a variable $X$
has a disjoint domain; the domain $D_X$ for variable $X$ is taken as
the convex union of all domains over which $X$ is defined.

\item ${\vec h}_{{\cal E}, Y}$ is a vector-valued affine function,
giving the index of variable $Y$ that $X({\vec i})$ depends on.  A
vector-valued function ${\vec h}$ is affine if it can be written as
${\vec h}({\vec i}) = C{\vec i} + {\vec d}$, where C is a constant
array and ${\vec d}$ is a constant vector that do not vary with ${\vec
i}$.

\item $f_{\cal E}$ is a strict function used to compute the elements
of $X$.

\end{itemize}

Intuitively, the SARE can be considered as follows.  Given an array
element at a known index $\vec i$, the SARE gives the set of arrays
and indices that this element depends on.  Furthermore, each index
expression on the right hand side is an affine expression of $\vec i$.
Note that in order to map directly from an array element to the
elements that it depends on, the index expression on the left hand
side must exactly match the domain quantifier $\vec i$.

A SARE has no notion of internal state per se; however, state can be
modeled by introducing a self-loop in the graph of variables.  If the
dependence function ${\vec h}$ happens to be a translation (${\vec
h}({\vec i}) = {\vec i} - {\vec k}$ for constant ${\vec k}$), then the
SARE is {\it uniform}, and referred to as a SURE or
SRE~\cite{Karp67}. SURE's can be used for systolic array
synthesis~\cite{Quinton84}, and there are techniques to uniformize
SARE's into SURE's~\cite{Manjun00}.  There has been a large body of
research on scheduling and placement for recurrence equations;
Rajopadhye gives an excellent overview~\cite{Raj96}.

SARE's became interesting from the standpoint of program analysis upon
Feautrier's discovery~\cite{Feautrier92i,Feautrier92ii} that a SARE is
mathematically equivalent to a ``static control flow program''.  A
program has static control flow if the entire path of control can be
determined at compile time (see~\cite{Feautrier92i,DRV00} for
details).  Feautrier showed that using a SARE, one can use linear
programming to produce a {\it parameterized} schedule for a program.
That is, if a loop bound is unknown at compile time, then the symbolic
bound enters the schedule and it does not affect the complexity of the
technique.  Feautrier's framework is very flexible in that it allows
one to select a desirable affine schedule by any linear heuristic.

Lim and Lam~\cite{Lim01} build on Feautrier's technique with an affine
partitioning algorithm that maximizes the degree of parallelism while
minimizing the degree of synchronization.  Affine partitioning--which
subsumes loop fusion, reversal, permutation, and reindexing--operates
on a form that is equivalent to a SARE.  Affine representations of
scientific programs have also been utilized for powerful storage
optimizations~\cite{Lim01,Quillere,Thies01,Lefebvre98}.
