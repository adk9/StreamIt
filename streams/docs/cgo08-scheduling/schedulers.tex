\section{Scheduling Paradigms}

%% A schedule $\phi$ implies buffering requirements between actors, or in
%% other words, it imposes a minimum bound on the size of each FIFO
%% between pairs of actors. An actor cannot fire until its input channel
%% carries enough data for the phase to complete. An actor that consumes
%% a variable amount of data can be modelled as an actor that always
%% consumes a single item from its input, buffers the item internally and
%% updates its internal state. When it has buffered a sufficient quantity
%% of items of its phase to fire, it will then execute.

It is possible to devise a schedule statically (e.g., compile time or
at graph creation time) or dynamically (e.g., runtime). The goal of a
scheduler is simply to maximize the throughput of the streaming
application. In the age of multicore architectures, a scheduler will
need to utilize the cores to increase concurrency and hence improve
the throughput of a given streaming application.  We assume a
decoupled multicore architecture where each processing core has its
own local storage, and can only access data in its local memory. The
architecture provides some mechanism for moving data between cores or
from main memory to the core. Examples of decoupled multicores
architectures are Cell~\cite{cell} and Tile64~\cite{tilera}.

A dynamic scheduler is conceptually easy to understand. The scheduler
maintains an internal representation of a given stream graph (e.g.,
list or priority queue). In a multicore architecture, when there is a
core available, the scheduler scans its internal representation of the
stream graph, and determines which actor is ready to fire; recall that
an actor is ready to execute when there is sufficient buffering on its
input channel (i.e., data dependence is satisfied).  To simplify
matters, we only consider actors where the I/O rates are known and do
not vary from one instance of a phase to the next instance of the same
phase. 
%%; we will relax this constraint in later sections of the paper.

The scheduler assigns the actor to the available core. The core is
also informed where the input buffer for the actor resides in memory,
and where in memory to commit (buffer) the output of the actor for its
successors. The scheduler then updates its internal representation to
indicate the actor firings and implement a fairness policy to assure
overall progress (e.g., a round-robin scheduler).

The dynamic scheduler essentially maximizes utilization by pipelining
the execution of the stream graph and overlapping actor firings. For
example, consider a stream graph $A\rightarrow B$ where $\rightarrow$
denotes a FIFO connection between two actors, here $A$ and $B$. A
scheduler for a two processor multicore ($P_1$ and $P_2$) will assign
$A$ to processor $P_1$ and when $A$ completes its firing, the
scheduler assigns $B$ to processor $P_2$. While $B$ is running
however, the scheduler can determine that another firing of $A$ is
possible and hence it instructs $P_1$ to execute the actor again, and
the output is buffered such that the corresponding (i.e., second)
firing of $B$ consumes the correct data from the second instance of
$A$.

%Let the expected (mean) work units required by actor $X$ to complete
%running a phase be $W_X$. In the example above, when $W_A > W_B$
%the processors will proceed at different rates and $P_1$ becomes a
%bottleneck. 

The are numerous heuristics and optimizations that a dynamic scheduler
can employ to orchestrate the execution of a stream graph on a
multicore. Ultimately the goals of the scheduler are to maximize
utilization and throughput, and it strives to do so by balancing the
load on the processors, maintaining adequate buffering between actors,
and trying to do so with little runtime overhead.

There are several concerns that arise when constructing a
schedule. Since a dynamic scheduler is running along with the
application, it must be efficient in its orchestration of the
schedule. Large stream graphs and massive multicores present dual
obstacles that limit the scalability of a centralized dynamic
scheduler. It is possible to devise distributed or hierarchical
dynamic schedulers but that comes with additional implementation
complexity. Other issues include limited scheduling foresight that
increase buffer lifetimes and the total buffer footprint in
memory. This can arise when an actor fires too soon in the schedule
and its output must be preserved for a long period of time before the
consumer begins to process the data. This is akin to agressive
instruction-level speculation that increases register lifetimes and
hence register pressure.

In the case of stream graph scheduling, one can argue that the
virtualization of memory easily ameloriates the issue of long buffer
lifetimes. However, when the scheduler operates in a context with
finite memory, the scheduling decisions can become more complex,
leading to greater runtime overhead.

We implemented an general scheduling framework for stream graphs that
admits statically orchestrated schedules as well as dynamic scheduling
heuristics. We use the scheduling framework to evaluate static and
dynamic scheduling paradigms. In general we found that our static
scheduling methodology is comparable to our dynamic scheduler, both
achieving high performance on our set of streaming benchmarks.
