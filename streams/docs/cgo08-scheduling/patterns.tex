\section{Fundamental Scheduling Patterns}

%% In this paper, we require that the push, pop, and peek rates of each
%% filter are known at compile time.  This enables the compiler to
%% calculate a steady-state for the stream graph: a repetition of each
%% filter that does not change the number of items buffered on any data
%% channel~\cite{lee87,karczmarek:lctes:2003}. In combination with a
%% simple program analysis that estimates the number of operations
%% performed on each invocation of a given work function, the
%% steady-state repetitions offer an estimate of the work performed by a
%% given filter as a fraction of the overall program execution.  This
%% estimate is important for our software pipelining technique.

It may appear that there is a complex space of tradeoffs to explore
when devising a schedule. What we show in this section is that the
scheduling decisions can be broken down to to reasoning about simple
stream graph patterns. The simple decomposition of stream graphs into
these patterns faciliates more scheduling freedom, especially for a
static scheduler. We will show later that a static scheduler that
makes use of these patterns yields results that are comparable to
dynamic scheduler.

We present eight primitive patterns. Each of the patterns represents a
small stream graph consisting of one or more actors. The patterns
distinguish between statless and stateful actors. A stateless actor
does not maintain any internal state that is modified from one firing
to the next. A stateless actor is essentially a data parallel actor,
whereas a stateful actor requires a serialization of its firings so
that its internal state is always consistent.

The insight enabled by these patterns is that it is sufficient to
construct a static schedule for each pattern such that the schedule
achieves maximum utilization on just {\it two} processors. We call
such a schedule a {\it maximum efficiency static schedule} (MESS). It
is then possible to compose these schedules recursively to schedule
larger stream graphs on a larger number of processors.
%% XXX even number?

A MESS is a static schedule that simply makes decisions on the
assignment of actors to processors, and the ordering of actor firings
within a processor. The actor dynamic firings are guarded by the
dataflow dependencies, and such dependences are the only
synchronization mechanisms required for proper execution. 
%%Without loss
%%of generality, we assume that all actors consume and produce an equal
%%number of data items per firing...
This is different from a full dynamic scheduler which also must decide
on the assignment and ordering of actors, or full static schedulers
that require synchronization barriers. The combination of static
assignment and ordering and dynamic issue offers a scalable solution
for multicore architectures.

The patterns themselves were initially inspired by observations
derived from studying the characteristics of streaming application. We
believe that in their current formulation, they present a foundation
to further explore the complex space of scheduling tradeoffs for
static and dynamic paradigms alike.

\begin{table*}[t]
\center
{\small
\begin{tabular}{|c|c|c|c|} \hline
                 & {\bf bottleneck} & ${\bf P_1}$  & ${\bf P_2}$ \\ \hline
{\bf Pattern 1}  & ---              & $A$     & $A$ \\ \hline
{\bf Pattern 2}  & ---              & $A$     & --- \\ \hline
{\bf Pattern 3}  & ---              & $A, B$  & $A, B$ \\ \hline
{\bf Pattern 4}  & ---              & $A$     & $B$ \\ \hline
{\bf Pattern 5}  & $W_A \le W_B$    & $A$     & $B$ \\ \cline{2-4}
                 & $W_A > W_B$      & $(W_A - W_B) A, (2W_A) B$ & $(W_A + W_B) B$ \\ \hline
{\bf Pattern 6}  & $W_A \ge W_B$    & $A$     & $B$ \\ \cline{2-4}
                 & $W_A < W_B$      & $(2W_B) A, (W_B - W_A) B$ & $(W_A + W_B) B$ \\ \hline
\end{tabular}}
\caption{\small Maximum efficiency static schedules for the stream graph patterns.}
\label{tab:pattern-sched}
\end{table*}


\subsection{Pattern 1}

The simplest pattern is shown in Figure~\ref{fig:pattern1}. It
represents a singular {\it stateless} actor $A$ that consumes no
input, and produces no output.
%%and requires $W_A$ units of work to execute
This patterns is trivial to schedule statically for two
processors. Since the actor is stateless, and has no dependences, each
of the two processors executes an instance of $A$ as shown Table~\ref{tab:pattern-sched}.

\subsection{Pattern 2}

A related pattern is shown in Figure~\ref{fig:pattern2}. It also
represents a singular {\it stateful} actor $A$ that consumes no input
and produces no output.
%%, and requires $W_A$ units of work to execute.
This patterns is also trivial to schedule statically. Since there is a
dependence between each of the actor firings, the only feasible
schedule is to sequentially execute the actor on a single
processor. The binding of the actor to a specific and unique
processor, the schedule avoids inter-processor synchronization, and
potentially costly communication to mutate the state of the actor
appropriately between processors.

\subsection{Pattern 3}

The third pattern is shown if Figure~\ref{fig:pattern3}. It consists
of two actors $A\rightarrow B$ connected through a FIFO to form a two
stage {\it pipeline}. Both of the actors are stateless, and hence data
parallel. In this pattern, the maximum efficiency schedule runs one
instance of the pipeline on processor $P_1$ and another independent
instance of the pipeline on the second processor $P_2$.  This schedule
requires no synchronization between processors.

\subsection{Pattern 4}

The fourth pattern is shown if Figure~\ref{fig:pattern4}. It is
similar to the previous pattern with the disctintion that the actors
in the pipeline $A\rightarrow B$ are stateful. The maximum efficiency
schedule for this pattern is to place one actor on one processor and
the other on the second processor. This requires synchronization
between the two processors in terms of data exchange along the FIFO
channel. In other words, an instance $i$ of actor $B$ can not execute
until instance $i$ of actor $A$ has completed its execution.  If the
expected (mean) work units required by actors $A$ and $B$ are $W_A$
and $W_B$ respectively, and $W_A > W_B$, the static schedule can
run a prologue stage that buffers the output of $A$ ahead of time to
create slack between the two processors.

\subsection{Pattern 5}

The fifth pattern is shown if Figure~\ref{fig:pattern5}. It is a
pipline $A\rightarrow B$ where actor $A$ is stateless and actor $B$ is
stateful. In this pattern, the characteristics of the work required to
execute $A$ and $B$ can lead to different maximal efficiency schedules.
There are two cases. First, if $W_A \le W_B$ then the schedule is
bottlenecked by the stateful actor and the best schedule is the same
as the one derrived for Pattern~4. However, if the bottleneck is the
stateless actor, then there is an opportunity to exploit the data
parallelism in the stateless actor to fully hide the load imbalance
between the two actors.

This fifth pattern (and as will be apparent shortly in the following
pattern) leads to a unique and elegant MESS that has a closed form
solution. Consider the simple case where $A$ produces a single item
per firing, and $B$ consumes a single item per firing, and let $W_A =
2$ units of work and $W_B = 1$ unit of work. Intuitively, to maximize
utilization the two processors, first we require that the total work
done in a steady state by $P_1$ equal the work done by $P_2$. In other
words, it is necessary for the two processors to be properly load
balanced. Furthermore, to miminize synchronization overhead, the
stateful actors should be collocated on the same processor. Lastly,
the schedule should maximize the decoupling between the two processors
by assigning the greatest possible number of stateless actors to the
processor where the stateful actor is not assigned.

In the domain on instruction scheduling, if $A$ and $B$ are
instructions in a loop, then unrolling the loop exposes more
instruction level parallelism that allows a compiler to hide the
latency of either instruction. Coarse grained software
pipelining~\cite{mgordon-aplos06} has a similar effect. Thus
conceptually unrolling the simple pipeline in this example by a factor
of 4 leads to a schedule that is perfectly load balanced as show in
Figure~\ref{fig:pattern5-sched}. 

A generalization of this example that satisfies the MESS desiderata
for this schedule thus maximizes the firings of $A$ on $P_2$ such that
the two processors are load balanced and the total number of $A$ and
$B$ actor firings is the same. It turns out that it is possible to
construct such a schedule readily for any value of $W_A < W_B$ using
the closed form solution we report in
Table~\ref{tab:pattern-sched}. It is trivial to show that both
processors are load balanced for any $W_A < W_B$. The total work done
on $P_1$ is equal to
\[
(W_A - W_B) \cdot W_A + (2 \cdot W_A) \cdot W_B
\]
and similarly the work done on $P_2$ equals
\[
(W_A + W_B) \cdot W_A
\]
and both equations reduce to 
\[
(W_A)^2 + (W_A \cdot W_B)
\]
in terms of total work per steady.

The closed form scheduling solution for this pattern (and similarly
the pattern described next) suggests that with adequate work
estimation for the actor firings, it is feasible to statically
construct a schedule that achieves a utilization that is comparable to
a dynamic scheduler.

\subsection{Pattern 6}

The sixth pattern is also a pipeline $A\rightarrow B$ but in this case
actor $A$ is stateful and actor $B$ is stateless. This pattern is
illustrated in Figure~\ref{fig:pattern6}. There are many similarlities
between this pattern and the previous pattern, and there is a closed
form scheduling solution for this pattern as well. 

When the pipeline is bottlenecked by the stateful actor ($W_A \ge
W_B$), there is not any scheduling freedom and the best possible
schedule is to assign each actor to its own processor. However when
the stateless actor is the bottleneck ($W_A < W_B$), it is possible to
hide its latency by adequately buffering the data between the two
actors. A MESS for this pattern that satisfies the same desiderata as
those in Pattern 5 leads to the schedule illustrated in
Figure~\ref{fig:pattern6-sched} when $W_A = 1$ and $W_B = 2$.  A
generalization of this example leads to the closed form scheduling
solutions shown in Table~\ref{tab:patterns-sched} for any $W_A < W_B$.
