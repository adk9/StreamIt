\section{Tolerating Variance With Static Schedules}

A particular challenge that a static scheduler faces is posed by the
work estimation $W_X$ for an actor $X$. The static scheduling requires
an effective work estimation methodology. Previous work in feedback
directed optimizations has shown that profiling is a practical
approach to characterizing applications. There are numerous examples
of profiling used for various compiler optimizations such as inlining,
unrolling, data prefetching, trace formation, etc. And prior related
work on scheduling of stream graphs has shown that good static
estimation of work is
feasible~\cite{mgordon-asplos04,mgordon-asplos06}.

In the case of the patterns shown in the previous section, variance in
the expected work duration of an actor can lower utilization and
throughput when actor communicate across processor boundaries. It is
however possible to compensate for the variance by simply creating
{\it slack} between the two processors so that they are not tightly
coupled. Slack is essentially the buffering of data produced by actors
on one processor and consumed on another. As the slack increases, or
in other words, the maximum allowed buffer size is increased, the
static schedule becomes more resillient to variance and maintain high
utilization and sustain throughput.

We emperically observed this phenomena by both simulating the patterns
using different variance distributions, and through actual
implementations and measurements on a Cell processor.