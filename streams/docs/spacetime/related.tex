\section{Related Work}
\label{sec:related}

The Imagine stream processor~\cite{rixner98bandwidthefficient}
supports a time multiplexed execution model.  The architecture
contains 48 parallel ALU's organized into 6 VLIW clusters.  The
programming model requires the programmer to write computation filters
in Kernel-C and stitch them together using Stream-C.  Because the
execution unit is data-parallel, the compiler uses time multiplexing
to execute a single filter at a time across all of the parallel
clusters.  While this provides perfect load balancing and high
arithmetic utilization when there is abundant data parallelism, it
suffers when a filter has retained state or data-dependences between
iterations.  Moreover, architectures based solely on
time-multiplexing do not scale spatially, as there are global wires
orchestrating the parallel execution units.  There is also a large
amount of memory traffic due to the prolonged execution of each
time-multiplexed stage; Imagine addresses this challenge by
introducing a compiler-controlled buffer called the ``stream register
file'' to prefetch data streams from memory and arrange them for
processing.

Previous work in compiling StreamIt to Raw has taken a purely space
multiplexed approach~\cite{streamit-asplos}.  In this model, a single
filter was mapped to each execution tile.  To support applications
with more filters than execution tiles, a partitioning algorithm was
employed to adjust the granularity of the graph by fusing adjacent
filters into one.  While filter fusion can be regarded as a restricted
form of time multiplexing (as the fused filter simulates the execution
of the originals, in turn), the approach given in the current paper is
much more general, as a given tile participates in multiple slices
that can be drawn from non-adjacent sections of the stream graph.  The
current paper also advances over previous work in its software
pipelining of slices and template code generation for linear filters.

%Our fine-grained execution model for linear portions of the stream
%graph is closely related to the large body of work on systolic
%architectures~\cite{systolic78,systolic82}.  A systolic architecture
%is arranged as an array of cells with local neighbor-to-neighbor
%communication.  Each cell has limited storage, and computation is
%arranged as data streaming through the array.  A large number of
%algorithms have been recast in a systolic
%context~\cite{leighton-book}, including others that focus on linear
%operations~\cite{wu87systolic}.  The algorithm underlying our template
%code generation is based on Hoffmann's stream
%algorithms~\cite{hoffmann-streams}, which provide a framework for
%decoupling memory accesses from computation for several scientific
%applications on tiled processors.

The Transputer architecture~\cite{transputer88} is an array of
processors, where neighboring processors are connected with unbuffered
point-to-point channels.  The programming language used for the
Transputer is Occam \cite{occammanual}: a streaming language similar
to CSP \cite{Hoare78}.  However, Occam concurrent processes are not
statically load-balanced, scheduled and bound to a processor. Occam
processes are run off a very efficient runtime scheduler implemented
in microcode~\cite{may87communicating}.  

A number of emerging architectures are incorporating a notion of
polymorphic execution modes that include support for
streams~\cite{trips-isca03,smartmemories,m3t}.  Such hardware support
is complementary to space-time multiplexing, as the mode of a given
resource could be toggled on the granularity of a slice to best
accommodate the computation pattern therein.

% \cite{cheops-thesis}
%   http://web.media.mit.edu/~kung/publication/thesis.pdf
%
% other possible things to cite:
%  http://portal.acm.org/citation.cfm?id=801721&dl=ACM&coll=portal
%  http://www.csrl.unt.edu/~kavi/Research/ica3pp156.pdf
%  http://cdmetcalf.home.comcast.net/papers/cop/node1.html#SECTION00010000000000000000
