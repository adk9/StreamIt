\section{Conclusions}
\label{sec:conclusion}

This paper proposes space-time multiplexing, a flexible scheme for
compiling streaming applications to wire-exposed architectures.  Up
until now, the two prevailing compilation strategies have been at odds
with one another: time-multiplexing provides flexible load balancing
at the expense of memory traffic and limited scalability, while space
multiplexing scales with spatial architectures but is difficult to
load balance.  To strike a practical middle ground between these
techniques, we introduce the concept of a {\it slice}: a commonly
executed sequence of filters that benefits from space multiplexing.
Further, we propose using software pipelining for scheduling
time-multiplexed execution of slices on the chip.  The result is a
hybrid algorithm that offers the flexible load balancing of time
multiplexing while preserving the fine-grained benefits of space
multiplexing.

%A second contribution of our work is using program slices as a target
%of domain-specific optimizations.  For slices that compute a linear
%function, we extract the coefficients and generate template assembly
%code that has been hand-tuned for Raw.  This technique yields
%near-optimal floating point utilization and provides a \todo{??}
%speedup over our benchmark suite.  Our vision is that similar
%architecture-specific templates could be generated for a variety of
%filters---not only for those that compute a given function (e.g., FFT)
%but also for filters with a specific property.  For example,
%data-parallel filters can be parallelized by a backend or can admit
%instruction reordering between different instances.  Such properties
%can facilitate a retargetable backend infrastructure, as each backend
%provides its own optimizations for a given class of filters.  Of
%course, domain-specific optimizations such as these rely on space-time
%multiplexing for their generality, as slices that are not subject to
%specific optimizations can can execute in their own time partitions.

Our implementation of space-time multiplexing in the StreamIt-to-Raw
compiler demonstrates an average improvement of \todo{??} over a
purely space-multiplexing scheme.  We believe that hybrid space-time
multiplexing represents a viable technique for harvesting the
computation resources of next-generation machines.
