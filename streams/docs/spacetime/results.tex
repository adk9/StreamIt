\begin{figure}[t]
\centering
\psfig{figure=FMRadio2.bloodgraph.spacetime_8_80_vs_raw.eps,width=3.0in}
\vspace{-6pt}
\caption{Execution trace of FMRadio2 using (a) space multiplexing and (b) space-time multiplexing.
\protect\label{fig:bloodgraph}}
\end{figure}

\section{Results}
\label{sec:results}

We have implemented a fully-automated scheduler that uses space-time
multiplexing in the StreamIt compiler.  The scheduler includes all of
the transformations and optimizations discussed in this paper.  We
evaluate our compiler using the suite of streaming applications shown
in Figure \ref{fig:benchmarks}.  The performance results for running
each application on the Raw architecture is shown in Table
\ref{tab:results}.

The results of this paper were generated using btl, a cycle-accurate
simulator that models arrays of Raw tiles identical to those in the
.15 micron 16-tile Raw prototype ASIC chip.  With a target clock rate
of 450 MHz, the tile employs as compute processor an 8-stage, single
issue, in-order MIPS-style pipeline that has a 32 KB data cache, 32 KB
of instruction memory, and 64 KB of static router memory.



The 

TODO
get tables with static numbers
	filters
	slices
	avg. slice size
	max slice size

get performance numbers for space
	space is unoptimized, un-unrolled, note this.

get performance numbers for spacetime 
	size filter slices
	.2 threshold
	.8 threshold
