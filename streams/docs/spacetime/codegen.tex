\section{Code Generation}
\label{sec:codegen}
In this section we consider low-level code generation for the Raw
microprocessor. Each Raw tile is an amalgam of the filters that
execute on it. To generate the execution code for the entire
application, we traverse the schedule for each phase of computation:
first the peek initialization stage, then prologue stage, and finally
the steady-state.  As we traverse each schedule, we generate the
computation and communication code as we visit each trace.  For the
initialization and prologue stages, we do a traversal of the trace
graph that respects the data-dependencies stipulated by the stream
graph. For the steady-state we traverse the schedule as calculated by
the trace scheduling algorithm of Section \ref{sec:scheduling}.

When we visit a trace, we first generate the DRAM requests and
communication code to gather its input.  Next, we generate computation
and communication code for each filter of the trace.  Finally, we
generate DRAM requests and communication code to scatter its outputs.
For each stage, when we visit a trace, the code generated is appended
to the existing code for the stage, producing the proper ordering of
operations as given by the schedule.  The code produced for the
steady-state is wrapped in an infinite loop. The following sections
investigates the steps in more detail.

\subsection{Trace Data Reordering}
Before the first filter of a trace can execute, its single input
stream must be assembled in the order stipulated by the round-robin
weights on its incoming channels. To achieve this, the various input
streams of the traces are collected into the memory module attached to
the tile to which the first filter of the trace is mapped. Streaming
DRAM requests are generated to read each input channel and to write
the final, reassembled input.  Code is generated for the static
routers, as they enforce the ordering of the final input stream.

After the last filter of a trace executes, its single output stream is
split into memory ports assigned to the output channels of the trace.
The choice of memory port for each inter-trace channel is beyond the
scope of this paper. Streaming DRAM requests are generated to read the
output of the last filter and streaming DRAM requests are generated to
write the reordered (and possible duplicated) output to the memory
ports assigned to the inter-trace channels.

\subsection{Filter Code Generation}
When we visit a trace in the schedule for each execution phase, we
iterate over the filters that compose the trace, generating 
communication and computation code necessary to execute the trace.
The scheduling phase (Section \ref{sec:scheduling}) produces a mapping
of filters to tiles.

\subsubsection{Compute Processor}
%For linear traces, we generate parameterized, template assembly code
%for both computation and communication as described in Section
%\ref{sec:linear}.  
As we iterate over the filters of the trace, we append the code
necessary to execute the current filter to the existing code for the
tile to which the filter is mapped.  We also make sure to execute the
{\tt init} function of each filter before we start the peek
initialization stage. Consequently, each compute processor begins by
calling the {\tt init} function of all the filters mapped to it.
Next, the compute processor executes the {\tt work} function of each
filter assigned to it in sequence.  The sequence is determined by the
traversal of the schedule of each execution phase; first the
peek initialization stage, then the prologue stage, and finally the
steady-state.  Each work function execution is placed in a loop that
executes the number of times the filter fires in the current stage.

For Raw's compute processor, we generate a mixture of C code
and assembly instructions that is compiled using Raw's GCC 3.3
port. The translation of the {\tt init} function, {\tt work}
functions, and any helper function is, for the large part,
straightforward.  Most StreamIt expression have direct analogs in C
except for the channel expressions {\tt push()}, {\tt peek(index)} and
{\tt pop()}.

One major benefit of StreamIt over other streaming languages is that
the StreamIt compiler is responsible for buffer management.  The
programmer is free to call {\tt pop()} or {\tt peek(index)} without
having to worry about the structure of the buffer or updating the
buffer's index variable after each execution of the work function.
Currently, we classify filters as one of three types with respect to
buffer management:
\begin{enumerate}
\item If a filter $(i)$ does not contain any {\tt peek} statements, $(ii)$
does not contain any control-flow dependent channel expressions, and
$(iii)$ has peek rate equals to pop rate, then a buffer is not needed
and {\tt pop()} statements are translated directly into static network
receives.  There exist other initialization considerations which are
beyond the scope of this paper.
\item If a filter contains peek statements, but $peek = pop$ then we can
use a simple linear buffer that is reset after each work function execution.
\item Otherwise, when $peek > pop$, we use a circular buffer to
account for the data remaining on the channel after the work function
invocation.
\end{enumerate}
Each {\tt push()} statement is translated directly into a static
network send.  

\subsubsection{Switch Processor}
As mentioned in Section \ref{sec:implementation}, individual traces
are limited to pipelines, thus we do not have to worry about
complicated patterns for intra-trace communication.  For each tile we
place the switch instructions to receive $pop$ items from the
direction of its upstream neighbor and send $push$ items to the
direction of its downstream neighbor.  The end-points of a trace send
or receive their data to or from off-chip, routing their items to and
from the side of the chip. These instructions are placed in a loop
that executes as many times as the filter fires in the current stage.

