%% \begin{figure}
%% \centering
%% \psfig{figure=beam-graph.eps,width=3.2in}
%% \caption{Stream graph of the Radar Application consisting of 12
%% channels and 4 beams. The non-linear filters are colored black. 
%% \label{fig:beam-graph}}
%% \end{figure}

\section{Illustrating Example}
\label{sec:example}

Use the fm-radio example?


%As a concrete example of space-time multiplexing, we consider the case
%of a simple target detector.  As shown in
%Figure~\ref{fig:target-graph}, the target detector duplicates its
%input using a splitjoin; each branch of the splitjoin contains a
%matched filter (to consider signals in a given range) followed by a
%threshold detector.  Our analysis detects that each matched filter
%computes a linear function, making it eligible for specialized code
%generation in the backend.

%Figure~\ref{fig:target-exec} illustrates the execution sequence that
%is generated by our compiler.  Each stage in the figure shows the
%computation layout for a given time partition; these stages are
%executed cyclically in the steady state.  The first stage implements
%the duplicate splitter by distributing the input data to distinct
%off-chip DRAMs.  Each of the next four stages executes one of the
%matched filters, using linear code generation to spread the
%computation in a systolic fashion across the entire chip.  Finally,
%the last stage executes all of the threshold filters in parallel.

%In our terminology, each filter in the target detector is assigned to
%its own trace.  That is, in this application, there is never a case
%where two filters form a load-balanced pipeline that could execute as
%a single unit on the chip.  However, the compiler recognizes the
%linear filters and implements them as a 16-element pipeline that fully
%utilizes the programmable on-chip network.

%In the case of this example, space-time multiplexing performs 7.8X
%better than a pure space-multiplexing approach~\cite{streamit-asplos}.
%The cause for this dramatic speedup is that each linear trace is
%executed as an independent stage.  In this form, it can utilize
%template assembly code in which the operation is parallelized across
%all tiles, all values are held within the registers and each cycle is
%carefully accounted for.  Space-time multiplexing provides the
%flexibility for such fine-grained codes to interact with general
%application components.

%\begin{figure}[t]
%\vspace{-18pt}
%\centering
%\psfig{figure=target_detect.eps,width=2.5in}
%\vspace{-6pt}
%\caption{Stream graph for a target detector.  Linear filters are indicated in gray.
%\protect\label{fig:target-graph}}
%\vspace{12pt}
%\psfig{figure=td_execute.eps,width=4.5in}
%\vspace{-6pt}
%\caption{Execution sequence for the target detector.  Each stage
%indicates the mapping from filters to Raw tiles during a given time
%partition.  Temporary results are buffered in off-chip DRAMs.
%\protect\label{fig:target-exec}}
%\vspace{-8pt}
%\end{figure}
