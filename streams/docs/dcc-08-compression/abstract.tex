A compressed-domain transformation is one that operates directly on
the compressed format, rather than requiring conversion to an
uncompressed format prior to processing.  Performing operations in the
compressed domain offers large speedups, as it reduces the volume of
data processed and avoids the overhead of re-compression.

While previous researchers have focused on compressed-domain
techniques for lossy data formats, there are few techniques that apply
to lossless formats.  In this paper, we present a general technique
for transforming lossless data as compressed with the sliding-window
Lempel Ziv algorithm (LZ77).  We focus on applications in video
editing, where our technique supports color adjustment, video
compositing, and other operations directly on the Apple Animation
format (a variant of LZ77).

We implemented a subset of our technique as an automatic program
transformation.  Using the StreamIt language, users write a program to
operate on uncompressed data, and our compiler transforms the program
to operate on compressed data.  Experiments show that the technique
offers speedups roughly proportional to the compression factor.  For
our benchmark suite of 12 videos in Apple Animation format, speedups
range from 1.1x to 471x, with a median of 15x.
