\chapter{Conclusions}
\label{chapter:conclusions}

Compression schemes play a key role in the proliferation of 
multimedia applications and the digital media explosion. 
At the same time, applications must be written 
for a plethora of unique parallel architectures. 
This thesis has shown that stream programming is an 
ideal model of computation for realizing image and 
video compression schemes. 
For this domain, stream programming boosts programmer 
productivity and enables scalable parallel execution 
of an application on a variety of architecture targets. 
Streaming language features allow a programmer to 
efficiently express a computation and expose 
parallelism, enabling a compiler to provide scalable performance.

This thesis shows these points through the following contributions: 
$(i)$ clean, malleable, and portable MPEG-2 encoder and 
decoder implementations expressed in a streaming fashion, 
$(ii)$ an analysis showing that a streaming language 
improves programmer productivity, 
$(iii)$ an analysis showing that a streaming language 
enables parallel execution, 
$(iv)$ an enumeration of the language features that are 
needed to cleanly express compression algorithms, 
$(v)$ an enumeration of the language features that 
support large scale application development and promote 
software engineering principles such as portability and reusability. 

This work was performed in the context of the StreamIt 
programming language, for its ability to express streaming 
computations, and the MPEG-2 video compression scheme. 
However the work is relevant to the domain of multimedia
codecs, including JPEG and MPEG-4.
Currently the H.264 
compression scheme is poised to supersede MPEG-2 video 
compression; assuming the language suggestions in this 
paper make their way into streaming languages, the most 
interesting research direction would be scalable, portable, 
and malleable implementations of H.264 codecs, expressed in a 
streaming fashion.

\Section{Concluding Remarks}
\vspace{-11pt}

As computer architectures change from the traditional monolithic
processors, to scalable wire-exposed and multicore processors, there
is a greater need for portable applications that expose parallelism
and communication to enable efficient and high performance
executions---while also boosting programmer productivity. StreamIt is
a programming language and a compilation infrastructure specifically
engineered to naturally expose and leverage stream abstractions that
are embodied in modern streaming applications. We have used StreamIt
to implement DSP codes (e.g., software radio, beamforming), image and
video codecs (e.g., MPEG-2 encoding and decoding), encryption
algorithms (e.g., DES and Serpent), and many other applications. The
language, compiler, and applications are available for download from
the project web page at http://cag.csail.mit.edu/streamit.

The goal of the StreamIt project is to boost productivity for
streaming application developers such that they focus on algorithmic
innovation rather than on performance tuning. The ability to leverage
domain specific language constructs affords optimization opportunities
that can deliver high performance from high levels of abstraction. 

We believe emerging languages such as X10 can provide a framework to
implement domain specific abstractions in a general purpose
programming model. We have designed and implemented a prototype bridge
to run StreamIt code as part of the X10 virtual machine. As a result,
application developers can use the streaming abstractions for the
computation that fit that model of computation, while different
abstractions can be used to describe other aspects of the computation.
A part of our ongoing work is concerned with evaluating the productivity
and performance merit of native interfaces to and from StreamIt codes.