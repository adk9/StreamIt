\SubSection{Video Sampling Rate}

Macroblocks specify colors using a luminance channel to represent
saturation (color intensity), and two chrominance channels to
represent hue. The human eye is more sensitive to changes in
saturation than changes in hue, so the chrominance channels are
frequently compressed by downsampling the chrominance data within a
macroblock. The type of chrominance downsampling an MPEG-2 encoder
uses is its {\it chrominance format}. The most common chrominance
format is 4:2:0, which uses a single block for each of the chrominance
channels, downsampling each of the two channels from 16x16 to 8x8.
The other chrominance format with downsampling is 4:2:2, which uses
two blocks for each chrominance channel, downsampling each of the
channels from 16x16 to 8x16. The possible chrominance formats are
shown in Figure~\ref{fig:chroma-format}.

To support the 4:2:2 chrominance format in our StreamIt decoder, we
modified 31 lines and added 20 new lines. Of the 31 modified lines, 23
were trivial modifications to pass a variable representing the
chrominance format as a stream parameter. The greatest substantial
change was to the decoding splitjoin previously illustrated in
Figure~\ref{fig:decoding-sj}. The stream is reconfigured such that it
can properly deal with the interleaving of chrominance data when the
sampling rate is increased. The more general splitjoin is shown in
Figure~\ref{fig:chroma-stream}.

%% TODO: add figure showing pattern for 4:2:0 and 4:2:2 and expand on
%% description above
\begin{figure*}[t]
  \begin{scriptsize}
    \begin{verbatim}
    int ->int splitjoin(String chromaFormat) {
      if (chromaFormat == ``4:2:0'') {
        split roundrobin(4*N, 2*N);
      } else { // chromaFormat == ``4:2:2''
        split roundrobin(4*N, 4*N);
      }

      add LuminanceChannel(W, H, 4*N, chromaFormat);

      add int->int splitjoin {
        split roundrobin(N, N);
        add ChrominanceChannel(W, H, N, chromaFormat);
        add ChrominanceChannel(W, H, N, chromaFormat);
        join roundrobin(1, 1);
      }

      join roundrobin(1, 2);
  }
  \end{verbatim}
  \end{scriptsize}
  % \vspace{-3pt}
  \caption{Decoding stream to handle 4:2:0 and 4:2:2 chroma formats.}
  \label{fig:chroma-stream}
\end{figure*}
