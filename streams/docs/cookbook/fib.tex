\clearpage
\subsection{Fibonacci}

\begin{lstlisting}{}
void->int feedbackloop Fib {
  join roundrobin(0,1);
  body int->int filter { 
    work pop 1 push 1 peek 2 { push(peek(0) + peek(1)); pop(); }
  };
  loop Identity<int>;
  split duplicate;
  enqueue(0);
  enqueue(1);
}
\end{lstlisting}

Using a feedback loop for a Fibonacci number generator is slightly
unusual but possible.  The joiner reads no items from the stream input
(also declared of type \lstinline|void|), but reads items continuously
from the feedback path.  Within a feedback loop, round-robin splitters
and joiners address the external path first and the feedback path
second.  This loop also uses the special \lstinline|Identity| filter
on the loop path; this is equivalent to an empty filter that copies
its input to its output, but occurs frequently enough that a shorthand
is useful to both the programmer and the compiler.

