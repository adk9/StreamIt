\section{A Moving Average Filter}

\begin{textpic}{\includegraphics{cookbook.2}}
\begin{lstlisting}{}
void->void pipeline MovingAverage {
  add IntSource();
  add Averager(10);
  add IntPrinter();
}
int->int filter Averager(int n) {
  work pop 1 push 1 peek n {
    int sum = 0;
    for (int i = 0; i < n; i++)
      sum += peek(i);
    push(sum/n);
    pop();
  }
}
\end{lstlisting}
\end{textpic}

Most of a typical StreamIt program consists of filters that produce
some output from their input.  The \lstinline|Averager| filter shown
here is such a filter.  Like the filters shown before,
\lstinline|Averager| has a work function with statically declared
input and output rates.

In addition to peeking and popping, \lstinline|Averager| \emph{peeks}
at its input stream.  The \lstinline|peek()| operator returns a
particular item off of the input stream, with \lstinline|peek(0)|
returning the next item that would normally be popped.  The work
function must declare a peek rate if it peeks at all, but this peek
rate is a maximum, rather than an exact, rate; it would be valid for
the \lstinline|Averager| filter to \lstinline|peek(n-1)| and never
\lstinline|peek(n)|, but \lstinline|peek(n+1)| is illegal.  Note that
mixing peeking and popping is valid, but that popping an item shifts
the index of future peeks.

\lstinline|Averager| also has a \emph{stream parameter}.  The number
\lstinline|n| is the number of items to average.  This is passed like
a normal function parameter from the \lstinline|add| statement that
creates the filter.  Within the filter, the parameter is a constant:
it is illegal for code to modify the parameter.  This allows parameter
values to be used in expressions for \emph{e.g.} I/O rates, as in the
peek rate here.

This program also provides a minimal demonstration of StreamIt's
filter scheduler.  There is a guarantee that the \lstinline|Averager|
filter is not run until its input rates can be met, and in particular,
that there are 10 inputs available so peeking can happen.  For this to
happen, the source needs to run nine additional times at the start of
the program; there can then be steady-state exections of source,
averager, printer.  The StreamIt compiler handles this automatically.
While all of the examples so far have had filters with matched I/O
rates, the compiler also automatically schedules the execution of
adjacent filters whose push and pop rates are different.

