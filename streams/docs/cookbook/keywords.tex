\section{Keyword Review}

\noindent Stream object types:

\begin{description}
\item[filter] Declares a filter with a work function
\item[pipeline] Declares a series of stream objects, with the output
  of the first connected to the input of the second, etc.
\item[splitjoin] Declares a parallel set of stream objects, with a
  splitter and a joiner distributing and collecting data
\item[feedbackloop] Declares a feedback loop with two children, with a
  joiner combining input data and the output of the loop and a
  splitter distributing the output of the body to the output and the
  input of the loop
\end{description}

\noindent Filter work functions:

\begin{description}
\item[push] Pushes an item on to the output of the filter.  Must be
  called the exact number of times as in the rate declaration.
\item[pop] Retrieves and removes the first item from the input of the
  filter.  Must be called the exact number of times as in the rate
  declaration.
\item[peek(k)] Retrieves the $k+1$-th item from the input of the
  filter, without removing it.  If $n$ items have been popped, $k+n$
  must be less than the declared peek rate.
\end{description}

\noindent Composite stream declarations:

\begin{description}
\item[add] Adds a child after the existing children.  (pipeline,
  splitjoin)
\item[body] Adds a child as the body part of a feedback loop.
\item[loop] Adds a child as the loop part of a feedback loop.
\item[enqueue] Pushes an item on to the input of the joiner coming
  from the loop part of a feedback loop.
\item[split] Declares the type and weights of the splitter.
  (splitjoin, feedbackloop)
\item[join] Declares the type and weights of the joiner.  (splitjoin,
  feedbackloop)
\item[duplicate] Splitter type that takes each input item and copies
  it to the input of each child.
\item[roundrobin] Splitter or joiner type that takes a specified
  number of items from the input (or output) and copies it to the
  input (or output) of each child.
\end{description}