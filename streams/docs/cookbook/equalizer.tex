\section{An Equalizer}

\begin{lstlisting}{}
float->float pipeline Equalizer(int bands, float[bands] cutoffs,
                                float[bands] gains) {
  add EqSplit(bands, cutoffs, gains);
  add float->float filter {
    work pop bands-1 push 1 {
      float sum = 0;
      for (int i = 0; i < bands-1; i++)
        sum += pop();
      push(sum);
    }
  };
}
float->float splitjoin EqSplit(int bands, float[bands] cutoffs,
                               float[bands] gains) {
  split duplicate;
  for (int i = 1; i < bands; i++)
    add pipeline {
      add BandPassFilter(cutoffs[i-1], cutoffs[i]);
      add float->float filter {
        work pop 1 push 1 { push(pop() * gains[i]); }
      };
    };
  join roundrobin;
}
\end{lstlisting}

\begin{figure}[htbp]
  \begin{center}
    \includegraphics{cookbook.4}
    \caption{Stream graph for an equalizer}
    \label{fig:equalizer}
  \end{center}
\end{figure}

This equalizer works by having a series of band-pass filters running
in parallel, with their outputs added together.  The caller provides
arrays of cutoff frequency and respective gains.

In the implmentation here, the output of \lstinline|EqSplit| is a
series of \lstinline|bands-1| outputs from the respective low-pass
filters.  An \emph{inline filter} is used to sum the results
together.  This is akin to an anonymous class in Java; the filter
declaration does not have an explicit name, but otherwise has syntax
almost identical to a top-level filter.  In general, inline filters
should only be used for very simple filters, such as this or the
inlined amplifier in \lstinline|EqSplit|.

\lstinline|EqSplit| is a normal split-join, as shown previously.  Its
body consists of a set of near-identical inlined pipelines; for
pipelines and split-joins, the input and output type declarations may
be omitted on anonymous streams.  Since the children are so similar,
they are added within a normal \lstinline|for| loop.  The compiler is
able to examine the loop provided that the loop bounds are expressions
of constants and loop parameters.
