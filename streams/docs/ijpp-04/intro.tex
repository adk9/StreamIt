\section{Introduction}

Applications that  are structured around  some notion of  a ``stream''
are increasingly prevalent to common computing practices, and there is
evidence   that  streaming  media   applications  already   consume  a
substantial fraction  of the  computation cycles on  consumer machines
\cite{rixner98bandwidthefficient}. Furthermore, stream processing---of voice and video
data---is  central  to  a  plethora  of  embedded  systems,  including
hand-held computers, cell phones,  and DSPs. The stream abstraction is
also  fundamental  to  high-performance  systems such  as  intelligent
software routers, cell phone base stations, and HDTV editing consoles.

Despite the  prevalence of  these applications, there  is surprisingly
little language and compiler support for practical, large-scale stream
programming.   Of course,  the notion  of  a stream  as a  programming
abstraction was  established decades ago \cite{SICP}, and  a number of
special-purpose stream languages  exist today (see \cite{survey97} for
a review).   Many of these  languages and representations  are elegant
and theoretically sound, but they  are often too inflexible to support
straightforward  development of modern  stream applications,  or their
implementations are too inefficient to use in practice.  Consequently,
most programmers resort to general-purpose  languages such as C or C++
to  implement  stream programs.  Yet  there  are  several reasons  why
general-purpose languages are  inadequate for stream programming. Most
notably, they do not provide  a natural or intuitive representation of
streams,  thereby  reducing  readability, robustness,  and  programmer
productivity.   Moreover,  because   the  widespread  parallelism  and
regular communication  patterns of data  streams are left  implicit in
general-purpose languages,  compilers are not  stream-conscious and do
not   perform    stream-specific   optimizations.    As    a   result,
performance-critical  loops  are   often  hand-coded  in  a  low-level
assembly  language   and  must  be  re-implemented   for  each  target
architecture.  This practice is labor-intensive, error-prone, and very
costly. 

General-purpose  languages are  also  poorly suited  for the  emerging
class of  tile-based architectures \cite{smartmemories,rawshort,trips}
that are  well-geared  for stream processing.   Perhaps the
primary appeal of C is  that it provides a ``common machine language''
for  von-Neumann  architectures.   That  is,  it  abstracts  away  the
idiosyncratic  differences between  machines,  but encapsulates  their
common  properties: a single  program counter,  arithmetic operations,
and a monolithic memory.  However, the von-Neumann model does not hold
in  the   context  of  tiled  architectures  as   there  are  multiple
instruction streams and distributed  memory banks. Consequently, C can
not serve  as a common machine  language, and in fact  it provides the
wrong    abstraction     for    the    underlying     hardware,    and
architecture-specific directives are often needed to obtain reasonable
performance.   Thus   the  responsibilities  of   the  programmer  are
artificially  complicated,   and  the  portability   of  applications  is
hampered.

In  this paper,  we describe  and  justify StreamIt  as a  high-level,
architecture independent  programming language for  stream programming
(Section~\ref{sec:overview}).  The  StreamIt  language is  designed  to
provide  high-level   stream  abstractions  that   improve  programmer
productivity   and    program   robustness   within    the   streaming
domain.  Furthermore, it  is intended  to  serve as  a common  machine
language for  tile-based processors.  At  the same time,  the StreamIt
compiler  aims  to  perform  novel  stream-specific  optimizations  to
achieve     the     performance     of    an     expert     programmer
(Section~\ref{sec:compiler}). 

In  the following  section, we  begin with  a characterization  of the
streaming   domain  and   motivate   the  design   of  StreamIt.    
% In
% Section~\ref{sec:example},  we  present   an  in-depth  example  of  a
% software     radio,     followed     by    relevant     results     in
% Section~\ref{sec:results}.
Section~\ref{sec:related} discusses related
work, and Section~\ref{sec:conc} summarizes and concludes the paper.
