\section{Introduction}

Synchronous dataflow graphs have become a powerful and widespread
programming abstraction for DSP environments.  Originally proposed by
Lee and Messerchmitt in 1987~\cite{LM87-i}, synchronous dataflow has
since evolved into a number of
varieties~\cite{BELP96,Bhatt00,Murthy2002,Buck93} which provide a
natural framework for developing modular processing blocks and
composing them into a concurrent network.  A dataflow graph consists
of a set of nodes with channels running between them; data values are
passed as streams over the channels.  The {\it synchronous} assumption
is that a node will not fire until all of its inputs are ready.  Also,
the input/output rates of the nodes are known at compile time, such
that a static schedule can be computed for the steady-state execution.
Synchronous dataflow graphs have been shown to successfully model a
large class of DSP applications, including speech coding,
auto-correlation, voice band modulation, and a number of filters.  A
significant industry has emerged to provide dataflow design
environments; leading products include COSSAP from Synopsys, SPW from
Cadence, ADS from Hewlett Packard, and DSP Station from Mentor
Graphics.  Academic projects include prototyping environments such as
Ptolemy~\cite{Lee01} and Grape-II~\cite{Lauw95} as well as languages
such as Lustre~\cite{lustre}, Signal~\cite{Gaut87}, and
StreamIt~\cite{Gordo02}.

One of the primary attractions of synchronous dataflow is that it
offers two different models of computation to the programmer.  Within
each processing node, programmers can write fine-grained imperative
code as they would in a scientific application.  However, outside of
the node, they can construct a graph in which all of the scheduling
and buffer management is performed automatically.

Up until now, this two-stage programming model has also implied a
two-stage optimization strategy.  That is, the internal contents of
each node are heavily optimized (often with hand-written C or assembly
code), and then the scheduling and buffer management of the entire
graph is done as a second pass.  Due to the tight resource constraints
in the embedded domain, these graph-level optimizations have been an
intense research focus--e.g., minimizing buffer size while restricting
code size~\cite{murt1997x1}, minimizing buffer size while maintaining
parallelism~\cite{GGD94}, resynchronization~\cite{Bhatta2000}, and
buffer sharing~\cite{murt2001x1}.  However, these optimizations
unilaterally regard each node as a black box that is invisible to the
compiler.  Due to the differing models of computation within the node
and outside the node, they are not able to perform any inter-node
optimizations that leverage the structure of the graph to improve the
computation of the nodes themselves.

In this paper, we give a technique whereby a System of Affine
Recurrence Equations (SARE) can serve as a unifying analysis framework
for inter-node and intra-node optimization in a dataflow graph.  The
SARE is a powerful representation with a mature body of literature
surrounding it.  It has its roots in the systolic community, where
Karp et al. proposed Uniform Recurrence Equations in 1967 as a model
for parallel computation~\cite{Karp67}.  More recently, Feautrier has
shown that a SARE can exactly represent the flow of data in a certain
class of scientific programs~\cite{Feautrier91,Feautrier01}.  This
provides one component of our method: translating each node's internal
computation into a SARE.

Our main contribution is a translation which encapsulates the
inter-node dependences: from a dataflow graph to a SARE.  The input of
our analysis is a ``Phased Computation Graph'' (PCG), which we
formalize as a generalization of several existing dataflow frameworks.
Once a SARE has been constructed that captures both inter-node and
intra-node dependences, we demonstrate how affine techniques could be
employed to perform novel optimizations such as node splitting,
decimation propagation, and steady-state invariant code motion that
exploit both the structure of the dataflow graph as well as the
internal contents of each node.  We also discuss the potential of
affine techniques for scheduling parameterized dataflow graphs, as
well as offering new approaches to classic problems in the dataflow
domain.

The rest of this paper is organized as follows.  We begin by
describing background information and related work on dataflow graphs
and SARE's (Section 2).  Then we define a Phased Computation Graph
(Section 3) and Phased Computation Program (Section 4), and give the
procedure for translating a simple, restricted PCP to a SARE (Section
5).  Next, we give the full details for the general translation
(Section 6), describe a range of applications for the technique
(Section 7), and conclude (Section 8).  A detailed example that
illustrates our technique appears in the Appendix.
