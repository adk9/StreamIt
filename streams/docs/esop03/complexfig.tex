\begin{figure}[t]
\framebox[6.5in]{
\begin{minipage}{6.3in}
\begin{center}
{\bf Helper Functions}
\end{center}
\begin{itemize}
\item The number of integral points in a one-dimensional domain ${\cal
D}$ is denoted by $|{\cal D}|$.  The sign and absolute value of an
integer $i$ are given by $sign(i)$ and $abs(i)$, respectively.
%
\item $\mt{PartialWrite}(c,p) = \sum_{r=0}^{p-1} U(c, r)$
%
\item $\mt{TotalWrite}(n,c) = \sum_{r=0}^{{num}(n)-1} U(c, r)$
%
\item $\mt{PartialRead}(c,p) = \sum_{r=0}^{p-1} O(c, r)$
%
\item $\mt{TotalRead}(n,c) =  \sum_{r=0}^{{num}(n)-1} O(c, r)$
%
\item Given channel $c = (n_a, n_b)$:
\begin{align*}
\mt{Period}(c) \equiv S(n_a) * \mt{TotalWrite}(n_a, c) = 
S(n_b) * \mt{TotalRead}(n_b, c)
\end{align*}
%
\end{itemize}
%
\end{minipage}}
\caption{Helper functions for the PCP to SARE translation.
\protect\label{fig:helper}}
\end{figure}

%%%%%%%%%%%%%%%%%%%%%%%%%%%%%%%%%%%%%%%%%

\begin{figure}[ht]
\framebox[6.5in]{
\begin{minipage}{6.3in}
\begin{center}
{\bf Variables}
\end{center}

For each channel $c = (n_a, n_b)$, do the following:
\begin{itemize}

\item Introduce variable $\mt{BUF}_c$ with the following domain:
\begin{align}
\label{eq:buf}
{\cal D}_{{BUF}_c} = \{ ~(i,j)~|~0 \le i \le N - 1 ~\wedge~ 0 \le j \le \mt{Period}(c) - 1\}
\end{align}

\item For each $p \in [0, num(n_a)-1]$, introduce
variable $\mt{WRITE}_{c, p}$ with this domain:
\begin{align}
\label{eq:write}
{\cal D}_{{WRITE}_{c, p}} = \{~(i,j,k)~|~0 \le i \le N-1 ~\wedge~ 0 \le j \le S(n_a) - 1 ~\wedge~ 0 \le k \le U(c, p) - 1\}
\end{align}
%
\item For each $p \in [0, num(n_b)-1]$, introduce
variable $\mt{READ}_{c, p}$ with this domain:
\begin{align}
\label{eq:read}
{\cal D}_{{READ}_{c, p}} = \{~(i,j,k)~|~0 \le i \le N-1 ~\wedge~ 0 \le j \le S(n_b) - 1 ~\wedge~ 0 \le k \le O(c, p) - 1\}
\end{align}
%
\end{itemize}
%
\end{minipage}}
\caption{Variables for the PCP to SARE translation.
\protect\label{fig:pcptosare1}}
\end{figure}

%%%%%%%%%%%%%%%%%%%%%%%%%%%%%%%%%%%%%%%%%%%%%%%%%%%%%%%%%%%%%%%%%%%%%%%%%%%%%%%%%%%%%%%%%%%%%%%%%%%%%%%%%%%%%%%%%%%%%%%

%\begin{figure}[ht]
%\framebox[6.5in]{
%\begin{minipage}{6.3in}
%\begin{center}
%{\bf Equations (1 of 2)}
%\end{center}
%\noindent For each node $n$, introduce the following equations:
%\begin{itemize}
%
%\item {\bf(I to READ)} For each $c \in \mt{chan\_in}(n)$, and for each phase $p \in [0,
%\mt{num}(n)-1]$:
%\begin{align}
%\label{i2read}
%&\forall (i,j,k) \in {\cal D}_{I \rightarrow SR}(n,c,p):~~
%\mt{READ}_{c, p}(i,j,k) = 
%    \mt{I}(c)[\mt{ReadIndex}(n,c,p,i,j,k)], \mt{where} \nonumber\\
%&\mt{ReadIndex}(n,c,p,i,j,k) = (i*S(n)+j)*\mt{TotalRead}(n,c) \\ \nonumber
%                &~~~~~~+ \mt{PartialRead}(c,p) + k \\\nonumber
%&\mt{and }{\cal D}_{I \rightarrow SR}(n,c,p) = 
%  \{ (i,j,k)~|~\mt{ReadIndex}(n,c,p,i,j,k) \in [0, A(c) - 1] \}
%\end{align}
%
%\end{itemize}
%\end{minipage}}
%\caption{Equations for PCP to SARE translation.
%\protect\label{fig:pcptosare2}}
%\end{figure}

\begin{figure}[ht]
\framebox[6.5in]{
\begin{minipage}{6.3in}
\begin{center}
%{\bf Equations (2 of 2)}
\end{center}
\noindent For each node $n$, introduce the following equations:
\begin{itemize}
%
\item {\bf(READ to WRITE)} For each $c \in \mt{chan\_out}(n)$, and for each $p \in [0,
num(n)-1]$:
%
\begin{align}
\label{read2write}
&\forall (i, j, k) \in {\cal D}_{{WRITE}_{c, p}}:~~\mt{WRITE}_{c,p}(i,j,k) =\nonumber\\
&~~~~~~W(n, p)(\mt{Steady\_Inputs})[\mt{pos\_out}(n, c)][k], \mt{where}\\\nonumber&
\mt{Steady\_Inputs} = [\mt{READ}_{{chan\_in}(n)[0], p}(i, j, *), \dots, \mt{READ}_{{chan\_in}(n)[{num\_in}(n)-1], p}(i, j, *) ]
\end{align}
%
\item {\bf(WRITE to BUF)} For each $c \in \mt{chan\_out}(n)$, for each $p \in [0,
\mt{num}(n)-1]$, and for each $q \in [0,S(n)-1]$:
\begin{align}
\label{write2buf}
&\forall (i,j) \in {\cal D}_{SW \rightarrow SB}(c,p,q):~~
\mt{BUF}_{c}(i,j) = \nonumber\\
&~~~~~~\mt{WRITE}_{c, p}(i, q,
                     j - \mt{Offset}_{SW \rightarrow SB}(q,n,c,p)),
\mt{where } \\&{\cal D}_{SW \rightarrow SB}(c,p,q) = 
  {\cal D}_{{BUF}_{c}} \cap 
  \{ (i,j)~|~\mt{Offset}_{SW \rightarrow SB}(q,n,c,p) \le j
             \le \mt{Offset}_{SW \rightarrow SB}(q,n,c,p+1) - 1 \} \\ \nonumber
&\mt{and } \mt{Offset}_{SW \rightarrow SB}(q,n,c,p') = q*\mt{TotalWrite}(n,c) + \mt{PartialWrite}(c,p')
\end{align}
% here is the simple, base case of the BUF -> READ function
\item {\bf(BUF to READ)} For each $c \in \mt{chan\_in}(n)$, for each phase $p \in [0,
\mt{num}(n)-1]$:
% two possible problems: read has already been written to, or buf has already been read from
\begin{align}
\label{buf2read0}
&\forall (i,j,k) \in {{\cal D}\mt{1}}_{SB \rightarrow SR}(c,p,q):~~
\mt{READ}_{c, p}(i,j,k) = 
    \mt{BUF}_{c}(i,j*\mt{TotalRead}(n,c) + 
                    \mt{PartialRead}(c,p) + k), \\
&\mt{where }{{\cal D}\mt{1}}_{SB \rightarrow SR}(c,p,q) = {\cal D}_{{READ}_{c, p}}\\
%\end{align}
%\begin{align}
%
%
%\label{buf2read1}
&\forall (i,j,k) \in {{\cal D}\mt{2}}_{SB \rightarrow SR}(c,p,q):~~
\mt{READ}_{c, p}(i,j,k) = \nonumber\\
    &~~~~~~\mt{BUF}_{c}(i,
                  j*\mt{TotalRead}(n,c) + \\\nonumber
                    &~~~~~~~~~~~~\mt{PartialRead}(c,p)
		+ k - \mt{Period}(c)),\\ \nonumber&
\mt{where }{{\cal D}\mt{2}}_{SB \rightarrow SR}(c,p,q) = 
  {\cal D}_{{READ}_{c, p}} - {{\cal D}\mt{1}}_{SB \rightarrow SR}(c,p,q)\\ \nonumber \\
%\label{buf2read2}
%\end{align}
%\begin{align}
%
%
%&\mt{where }\mt{PUSHED}(n,c,p) = {\cal D}_{I \rightarrow SR}(n,c,p) \nonumber\\ \nonumber
%&\mt{and }\mt{Num\_Pushed}(n,c,p) = 
% this is a dot product of (i,j,k) with something
%  (max_{\preceq}\mt{PUSHED}(n,c,p)) \cdot (S(n) * \mt{TotalRead}(n,c), \\ \nonumber
%                                        &~~~~~~\mt{TotalRead}(n,c), 1 ) + \mt{PartialRead}(c, p) \\ \nonumber
%&\mt{and }\mt{Offset}(n,c,p) = \mt{Num\_Pushed}(n,c,p) + O(c,p)-O(c,p) \\ \nonumber
%&\mt{and }\mt{Int\_Offset}(n,c,p) = - \lfloor \frac{\mt{Offset}(n,c,p)}{\mt{Period}(c)} \rfloor 
%~~\mt{and }\mt{Mod\_Offset}(n,c,p) = - \mt{Offset}(n,c,p)~\mt{mod}~\mt{Period}(c) \nonumber
%
\end{align}
\vspace{-12pt}
\end{itemize}
\end{minipage}}
\caption{Equations for CSDP to SARE translation.
\protect\label{fig:pcptosare3}}
\end{figure}