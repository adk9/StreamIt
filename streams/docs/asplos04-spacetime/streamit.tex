\begin{figure}[t]
\begin{center}
\begin{minipage}{0.46in}
\centering
\psfig{figure=pipeline.eps,width=0.46in} \\
\end{minipage} 
~
\begin{minipage}{1.3in}
\centering
\psfig{figure=splitjoin.eps,width=1.3in} \\
\end{minipage}
~
\begin{minipage}{1.02in}
\centering
\psfig{figure=feedback.eps,width=1.02in} \\
\end{minipage} 
\\ ~ \\ {\bf \protect\small (a) A pipeline. ~~(b) A splitjoin. ~~(c) A feedbackloop.}
\caption{\protect\small Stream structures supported by StreamIt.
\protect\label{fig:structures}}
\end{center}
\end{figure}

\section{The StreamIt Language}
\label{sec:streamit}

StreamIt is a portable programming language for high-performance
signal processing applications.  The current version of StreamIt is
tailored for static-rate streams: it requires that the input and
output rates of each filter are known at compile time.  In this
section, we provide a very brief overview of the syntax and semantics
of StreamIt, version 2.0.  A more detailed description of the design
and rationale for StreamIt can be found in~\cite{streamitcc}, which
describes version 1.0; the most up-to-date syntax specification can
always be found on our website~\cite{streamitweb}.

\subsection{Language Constructs}

The basic unit of computation in StreamIt is the {\it filter}.  A
filter is a single-input, single-output block with a user-defined
procedure for translating input items to output items.  
%An example of
%a filter is the {\tt FIRFilter}, a component of our software radio
%(see Figure \ref{fig:radiocode}).  
Each filter contains an {\tt init}
function that is called at initialization time; 
%in this case, the {\tt
%FIRFilter} calculates {\tt weights}, which represents its impulse response.  
The {\tt work} function describes the most fine grained
execution step of the filter in the steady state.  Within the {\tt
work} function, the filter can communicate with its neighbors via FIFO
queues, using the intuitive operations of {\tt push(value)}, {\tt
pop()}, and {\tt peek(index)}, where {\tt peek} returns the value at
position {\tt index} without dequeuing the item.  The number of items
that are pushed, popped, and peeked\footnote{We define $peek$ as the
total number of items read, including the items popped.  Thus, we
always have that $peek \ge pop$.} on each invocation are declared with
the {\tt work} function.

In addition to {\tt work}, a filter can contain a {\tt prework}
function that is executed exactly once between initialization and the
steady-state.  Like {\tt work}, {\tt prework} can access the input and
output tapes of the filter; however, the I/O rates of {\tt work} and
{\tt prework} can differ.  In an {\tt FIRFilter}, a {\tt prework}
function is essential for correctly filtering the beginning of the
input stream.  The user never calls the {\tt init}, {\tt prework}, and
{\tt work} functions--they are all called automatically.


The basic construct for composing filters into a communicating network
is a {\it pipeline}.%, such as the FM Radio in Figure \ref{fig:radiocode}.  
A pipeline behaves as the sequential composition
of all its child streams, which are specified with successive calls to
{\tt add} from within the pipeline.  For example, the output of {\tt
DataSource} is implicitly connected to the input of {\tt FIRFilter},
who's output is connected to {\tt FMDemodulator}, and so on.  The {\tt
add} statements can be mixed with regular imperative code to
parameterize the construction of the stream graph.

There are two other stream constructs besides pipeline: {\it
splitjoin} and {\it feedbackloop} (see Figure \ref{fig:structures}).
From now on, we use the word {\it stream} to refer to any instance of
a filter, pipeline, splitjoin, or feedbackloop.

A splitjoin is used to specify independent parallel streams that
diverge from a common {\it splitter} and merge into a common {\it
joiner}.  There are two kinds of splitters: 1) $duplicate$, which
replicates each data item and sends a copy to each parallel stream,
and 2) $roundrobin(w_1, \dots, w_n)$, which sends the first $w_1$
items to the first stream, the next $w_2$ items to the second stream,
and so on.  roundrobin is also the only type of joiner that we
support; its function is analogous to a roundrobin splitter.  If a
roundrobin is written without any weights, we assume that all $w_i =
1$.  The splitter and joiner type are specified with the keywords {\tt
split} and {\tt join}, respectively (see Figure \ref{fig:radiocode});
the parallel streams are specified by successive calls to {\tt add},
with the $i$'th call setting the $i$'th stream in the splitjoin.

The last control construct provides a way to create cycles in the
stream graph: the feedbackloop.  Due to space constraints, we omit a
detailed discussion of the feedbackloop.