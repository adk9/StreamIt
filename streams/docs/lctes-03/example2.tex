\begin{figure}[t]
\begin{minipage}{1.3in}
\begin{center}
\psfig{figure=splitjoin-sample2.eps,width=0.8in} 
\end{center}
\end{minipage}
\begin{minipage}{1.75in}
\caption{\small Example splitjoin to illustrate phased scheduling.
Each node is annotated with its input and output
rates.\protect\label{fig:sjexlabel}}
\end{minipage}
\vspace{-12pt}
\end{figure}

\begin{figure*}[t]
\psfig{figure=splitjoin-sample-tiny2.eps,width=7in}
\caption{{\small Example construction of phased schedules for the
splitjoin of Figure~\ref{fig:sjexlabel}.  First, the splitjoin's
execution is simulated for one steady state according to a push
schedule; the stream graph is labeled with the number of items on each
channel following the firing of a shaded node.  Then, fine-grained
phases are formed that include executions of both the entry (A) and
exit (D) nodes.  Finally, the fine-grained phases are combined into
$\mt{maxPhases}$ phases, each of which is factored into a single
appearance schedule.\protect\label{fig:sjex}}}
\end{figure*}
