\section{{\StreamIt} Language}
\label{chpt:streamit}

This chapter introduces relevant constructs of the {\StreamIt}
language.  Syntax is not explored here, as it is not relevant to
{\StreamIt} scheduling.

Section \ref{sec:streamit:struct} introduces the structured
streaming concept, while Section \ref{sec:streamit:messages}
introduces the low bandwidth messaging semantics of {\StreamIt}.

\subsection{Structure}
\label{sec:streamit:struct}

Perhaps the most distinguishing feature of {\StreamIt} language is
that it introduces structure to the concept of stream computation.
{\StreamIt} concept of structure is conceptually similar to
structured constructs in functional languages such as \C.

In {\StreamIt} programs are composed out of streaming components
called streams.  Each stream is a single-input, single-output
component, possibly made up of a hierarchical composition of other
streams. Streams can only be arranged in a limited number of ways,
using {\pipelines}, {\splitjoins}, and {\feedbackloops}.  Data
passed between {\filters} is read from and written to {{\Channels}}.
Figure \ref{fig:structure} contains examples of various
{\StreamIt} streams.  The restrictions on arrangement of streams
enforces the structure imposed by {\StreamIt}.

\begin{figure}\begin{center}
\begin{minipage}{2in}
\centering
\psfig{figure=filter.eps,width=0.7in} \\
{\protect\small (a) A {\filter}}
\end{minipage}
~
\begin{minipage}{2in}
\centering
\psfig{figure=pipeline.eps,width=0.7in} \\
{\protect\small (b) A {\pipeline} with $n$ children.}
\end{minipage}
~
\begin{minipage}{2in}
\centering
\psfig{figure=splitjoin.eps,width=2in} \\
{\protect\small (c) A {\splitjoin} with $n$ children}
\end{minipage}
~
\begin{minipage}{2in}
\centering
\psfig{figure=feedback.eps,width=1.5in} \\
{\protect\small (d) A {\feedbackloop}}
\end{minipage}
\end{center}
\caption{All {\StreamIt} streams}
\label{fig:structure}
\end{figure}

\subsubsection{\filters}

The basic unit of computation in {\StreamIt} is the {\filter}. The
central aspect of a filter is the {\work} function, which
describes the filter's atomic execution step. Within the {\work}
function, the filter can communicate with its neighbors using the
{\Input} and {\Output} channels, which are typed FIFO queues
declared during initialization of a {\filter}.  Figure
\ref{fig:structure}(a) depicts a {\filter}.

{\filters} also have the restriction of requiring a static amount
of data to be consumed and produced for each execution of a
{\work} function.  The amount of data produced by a {\filter} $F$
upon execution of its {\work} function is called a push amount,
denoted $push$. The amount of data consumed from {\Input}
{{\Channel}} by a {\filter} $F$ upon execution of its {\work}
function is called a pop amount, denoted $pop$.  {\filters} may
require that additional data be available in the {\Input}
{{\Channel}} for the {\filter} to examine.  This data can be read by
the {\filter}'s {\work} function, but it will not be consumed, and
will remain in the {{\Channel}} for the next execution of the
{\work} function.  The amount of data necessary on the {\Input}
{{\Channel}} to execute {\filter}'s {\work} function is called peek
amount, denoted $peek$.  Note, that for all {\filters} $peek
>= pop$.  Extra peek amount is the amount of data required on by
the {\filter} that will be read but will not be consumed, namely
$peek - pop$.  The \emph{peek}, \emph{pop} and \emph{push} values
in Figure \ref{fig:structure}(a) correspond to the $peek$, $pop$
and $push$ amounts of the {\filter}'s {\work} function.

A {\filter} can be a source, if it does not consume any data, but
it produced data.  Namely, a {\filter} is a source if it has $peek
= pop = 0$. Likewise, a {\filter} can be a sink, if it consumes
data, but does not produce any, or $push = 0$.

\subsubsection{\pipelines}

{\pipelines} are used to connect {\StreamIt} structures in a chain
fashion: each child stream's output is the next child stream's
input. {\pipelines} have no {\work} function, as they do not perform
any computation themselves. {\pipelines} are simply containers of
other {\StreamIt} structures.  Figure \ref{fig:structure}(b) depicts
a {\pipeline}.

\subsubsection{\splitjoins}

{\splitjoins} are used to specify independent parallel structures
that diverge from a common {\splitter} and merge into a common
{\joiner}.  There are two types of {\splitters}:
\begin{enumerate}[(a)]
\item %(a)
{\duplicate}, which replicates each data item and sends a copy to
each parallel stream, and

\item %(b)
{\roundrobin} $(w_0,\dots, w_{n-1})$, which sends the first $w_0$
items to the first stream, the next $w_1$ items to the second
stream, and so on.  If all $w_i$ are equal to $0$, all child
streams of the {\splitjoin} must be sources.
\end{enumerate}

{\roundrobin} is also the only type of a {\joiner} supported in
{\StreamIt}; its function is analogous to a {\roundrobin} {\splitter}.

Figure \ref{fig:structure}(c) depicts a
{\splitjoin}.

\subsubsection{\feedbackloops}
\label{sec:explain-fl}

{\feedbackloops} are used to create cycles in the stream graph. A
{\feedbackloop} contains a {\joiner}, a body stream, a {\splitter}, and
a loop stream.  Figure \ref{fig:structure}(d) depicts a
{\feedbackloop}.

A {\feedbackloop} has an additional feature required to allow a
{\feedbackloop} to begin computation: since there is no data on
the feedback path at first, the stream instead inputs data from a
special function defined by the {\feedbackloop}.  The amount of
data pushed onto the feedback path is called delay amount, denoted
$delay_{fl}$, for a {\feedbackloop} $fl$.

\subsection{Messages}
\label{sec:streamit:messages}

In addition to passing data between {\filters} using structured
streams, {\StreamIt} provides a method for low-bandwidth data
passing, similar to a combination of sending messages and function
calls. Messages are sent from within the body of a {\filter}'s
{\work} function, perhaps to change a parameter in another
{\filter}. The sender can continue to execute while the message is
en route. When the message arrives at its destination, a special
message receiver method is called within the destination
{\filter}. Since message delivery is asynchronous, there can be no
return value; only void methods can be message targets. This
allows the send to continue execution while the message is en
route - the sender does not have to wait for the receiver to
receive the message and send a return value back. If the receiver
wants to send a return value to the sender, it can send a message
back to the sender.

Although message delivery in {\StreamIt} is asynchronous in
principle, {\StreamIt} does include semantics to restrict the
latency of delivery of a message. Since {\StreamIt} does not
provide any shared resources to {\filters} (including global
memory, global clock, etc), the timing mechanism uses a concept of
flow of information.

One motivating example for messaging in {\StreamIt} can be found in
cell phone processing application. Modern cellular phone protocols
involve a technique called frequency hopping - the cell phone base
station selects a new frequency or channel for the phone to
communicate with the base station and informs the phone of this
change.  The phone must switch to the new channel within a certain
amount of time, or it risks losing connection with the base
station.

If the phone decoder application is written in {\StreamIt}, the
{\filter} controlling the antenna and the {\filter} which will process
control signals are likely far apart, and may not have a simple
way of communicating data directly with each other. In {\StreamIt},
the {\filter} which decodes control signals can simply send a
message to the {\filter} controlling the antenna. The message can be
sent with a specific latency corresponding to the timing required
by the base station. When the antenna controller receives the
message it can change the appropriate settings in the hardware to
switch to the appropriate new frequency, without having to wait
for the appropriate time. The timing of delivery is taken care of
by {\StreamIt}.

\subsubsection{Information Wavefronts}

When a data item enters a stream, it carries with it some new
information. As execution progresses, this information cascades
through the stream, affecting the state of {\filters} and the values
of new data items which are produced. We refer to an information
wavefront as the set of {\filter} executions that first sees the
effects of a given input item. Thus, although each {\filter}'s {\work}
function is invoked asynchronously without any notion of global
time, two invocations of a work function occur at the same
information-relative time if they operate on the same information
wavefront.

\subsubsection{Message Sending}

Messages can be sent upstream or downstream between any two
{\filters}.  Sending messages across branches of a {\splitjoin} is not
legal.  Timing of message delivery uses the concept of information
wavefront.  The sender specifies that the message is supposed to
be delivered with a certain delay of information wavefront. The
delays are specified as ranges, $[l_0, l_1], l_0 \le l_1$. $l_0$
and $l_1$ specify the information wavefront in executions of the
{\work} function of the sender {\filter}.

If the message is being sent downstream, the sender specifies that
the receiver will receive the data just before it sees the
information wavefront produced by the sender between $l_0$ and
$l_1$ executions of its {\work} function from when it sends the
message. If the message is being sent upstream, the sender
specifies that the receiver must receive the message just before
it produces an information wavefront the sender will see between
$l_0$ and $l_1$ executions of its {\work} function from when it
sends the message.

Message sending is meant to be a low-bandwidth method of
communication between {\filters}. Message sending is not a fast
operation and is intended not to interfere with the high bandwidth
{\StreamIt} communication and processing.  However, depending on how
tight the latency constraints are (both the magnitude of the
latency as well as the range), declaring that messages can be sent
may slow program execution down considerably.

Figure \ref{fig:message-example} presents an example of a
{\pipeline} in which the last {\filter} sends a message to the first
{\filter}.  {\filter}$_3$ sends a message to {\filter}$_0$.  The message
is sent with latency $[3,8]$. This means that after at least 3 and
at most 8 executions of sender's {\work} function, it will see data
produced by the receiver just after receiving the message.

\begin{figure}\begin{center}
\begin{minipage}{4in}
\centering
\psfig{figure=message-example.eps,width=1.5in} \\
\end{minipage}
\end{center}
\caption{Example of a {\pipeline} with a message being sent}
\label{fig:message-example}
\end{figure}
