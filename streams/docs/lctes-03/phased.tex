\section{Phased Scheduling}
\label{chpt:phased}

We now propose Phased Scheduling, a technique which allows to
schedule all valid {\StreamIt} programs, and which allows for
better control of trade-off between schedule size and buffer size.

Section \ref{sec:phased:intro} provides an introduction to and
explanation of Phased Scheduling. Section \ref{sec:min-latency}
presents a Minimal Latency Schedule implementation using Phased
Scheduling.

\subsection{Phased Scheduling}
\label{sec:phased:intro}

The pseudo single-appearance hierarchical scheduling technique
presented in Chapter \ref{chpt:hierarchical}, while quite
effective in scheduling simple applications, cannot schedule a
small number of tight {\feedbackloops}. Furthermore, the technique
is quite inflexible when it comes to attempting to create a
different tradeoff between schedule size and buffer size. Phased
scheduling solves both of these problems.

Phased schedule is a hierarchical schedule, just like the pseudo
single appearance schedule. Every stream uses only the schedule of
its immediate child streams to create its own schedule. Phased
schedules, however, consist of several steps, called phases. All
of the phases must be executed in order to guarantee correctness.
Once all of the phases have been executed, the schedule has been
executed. A parent stream can interleave execution of its
children's phases in its own schedule, to provide any level of
granularity desired.

\begin{comment}
The schedules created using single appearance hierarchical
scheduling tend to be quite small at the expense of larger
buffering requirements.  A quite simple situation when such
tradeoff is not desired, could be if the schedule is being stored
in a large cheap ROM device, while the RAM used for buffering data
is more expensive.  It is also quite possible that latency
constraints cannot be satisfied by a single appearance
hierarchical schedule. Clearly, a more flexible technique is
required for scheduling.

A key observation in hierarchical scheduling is that each
component only needs to worry about the data that enters or leaves
its children.  The amount of buffering done internally in a child
is not noticeable or important to the parent component. This
observation changes slightly if latency constraints are placed on
the computation. Namely, the important information to keep track
of is amount of data that leaves or enters children as well as
amount of data that crosses latency constraint boundaries.

This observation leads to a conclusion that scheduling execution
of the {\StreamIt} programs using hierarchical scheduling can be
simpler than scheduling the entire program all at once (scheduling
the program all at once requires tracking all buffers and latency
constraints at once).  Phased scheduling is a concept that expends
on hierarchical scheduling, but does not require that a stream has
a single or pseudo single appearance schedule.  Each stream is
allowed to have multiple sub-schedules, also called phases. Each
phase consists of phases of the children of the stream that will
be executed to execute the phase. The phases must be executed in
correct order. When all of the initialization phases of a stream
have been executed, the stream has executed its initialization
schedule and is ready to enter steady state execution. When all of
the steady state phases of a stream have been executed in order,
the entire steady state schedule for the stream has been executed.
\end{comment}

The granularity of splitting the steady state schedule into phases
is left up to the specific scheduler.  Different streams can use
different granularities of execution.  In principle, the parent
should not need to know the scheduling granularity of its
children. The only exception to this rule are {\feedbackloops},
which can have children which are not scheduled tightly enough to
allow the {\feedbackloop} to execute. An example of that may be a
pseudo single-appearance hierarchical scheduling algorithm
described in Section \ref{sec:sas} implemented using phase
scheduling.

\begin{comment}
One important observation to make is that it makes little sense to
have phases which do not consume or produce any data, and which do
not have data cross any latency boundaries.  This is because such
phases can easily be merged with preceding or following phases
without any effect on ability to schedule a particular program.
This observation allows to easily bound the size of the resulting
schedules to be the sum of executions of first child, last child
and children with latency boundaries. For example, the {\pipeline}
in Figure \ref{fig:hierarchical-schedule} executes its first
child, {\filter} A, 4 times in steady state execution, and its last
child, {\filter} D, 9 times. Thus a phasing schedule of this
pipeline should at most have $4+9=13$ phases.
\end{comment}

\subsection{Minimal Latency Phased Scheduling}
\label{sec:min-latency}

One of the problems with pseudo single-appearance scheduling is
that it cannot schedule all legal {\StreamIt} programs.  A program
with a {\feedbackloop} can have requirements for tight execution
that cannot be satisfied using a pseudo single-appearance
schedule, leading to deadlock. Phasing scheduling can alleviate
this problem by allowing the program to be scheduled in a more
fine-grained manner. Minimal latency scheduling is an example of a
specific scheduling strategy that solves the problem of deadlock.
Minimal latency schedule is a schedule that requires a minimal
amount of input data in order to output data. In other words, a
minimal latency schedule only buffers as much data as is
absolutely necessary. This means that if a \feedbackloop can be
scheduled, a minimal latency schedule will be able to schedule it.

\begin{comment}
A minimal latency schedule is not necessarily single
appearance. In fact, very few applications can have their minimal
latency schedules expressed as a single appearance schedule.  One
of the consequences of this is that minimal latency schedules
require more space for storage of the schedule. Use of phasing
scheduling facilitates creation of acceptably small minimal
latency schedules.  In spirit of hierarchical scheduling, every
component is scheduled separately, in hierarchical order.

One important consequence of phased scheduling, one that is
highlighted when calculating a minimal latency schedule, is that
every phase is allowed to consume a different amount of data and
produce a different amount of data.

\subsubsection{Peeking}

Phased scheduling has interesting consequence for peeking
calculations.  The reason for this is that not all phases must
consume data, thus not all phases will peek.  The amount of
peeking done by a stream is important for creating an
initialization schedule.  It is thus important to remember that
the amount of peeking done by a stream is not necessarily the
amount of peeking done by that stream in its first phase, because
on first phase, the stream may not consume or peek any data.
\end{comment}

\subsubsection{Notation}

\begin{comment}
We extend the notation for peeking, popping and pushing to
include phases. $u^m_s$ will denote amount of data pushed by the
$m$th phase of stream $s$, $o^m_s$ will denote amount of data
popped by the $m$th phase of stream $s$ and $e^m_s$ will denote
amount of data peeked by $m$th phase of stream $s$.
\end{comment}

A phasing schedule of a stream $s$ is a set $P_s$ of elements,
$P_s = \{T_s, I_s, c_s, c^i_s\}$.  The first element, $T_s$
denotes the phases used for the steady state schedule of $s$.
$I_s$ denotes the phases used for the initialization schedule of
$s$. $c_s$ and $c^i_s$ are defined identically to their
definitions in hierarchical schedules: $c_s$ is the consumption
rate of the stream during its steady state execution and $c^i_s$
is the consumption rate of the initialization schedule.

$T_s$ and $I_s$ are defined by identical structures.  Both are
defined as sets of phases. The only real difference between $T_s$
and $I_s$ is that $T_s$ will be executed indefinitely, while $I_s$
will be executed only once. A phase $A$ is defined as $A = \{E,
c\}$.  $E$ is an ordered list of phases and nodes that are to be
executed in order to execute the phase.  $c$ is the consumption of
the phase, with respect to its stream.

\begin{comment}
As an example, here is a minimal latency schedule for the sample
stream in Figure \ref{fig:sample-sj}.  First, the schedule for the
internal {\splitjoin}:

\begin{displaymath} \small
P_{sj} = \left\{
\begin{array}{c}
T_{sj} = \left\{
\begin{array}{c}
A_{sj,0} = \left\{\{\{6\ split\}\{2C\}\{5D\}\ join\},
\left[\begin{array}{c}12\\12\\12\end{array}\right]\right\}, \\
A_{sj,1} = \left\{\{\{3\ split\}C\{4D\}\ join\},
\left[\begin{array}{c}6\\6\\12\end{array}\right]\right\}
\end{array}\right\}, \\
I_{sj} = \left\{ \right\}, \\
c_{sj} = \left[ \begin{array}{c} 18 \\ 18 \\ 24 \end{array}
\right], c^i_{sj} = \left[ \begin{array}{c} 0 \\ 0 \\ 0
\end{array} \right]
\end{array} \right\}
\end{displaymath}

\noindent And the following is a schedule for the {\pipeline}:

\begin{displaymath} \small
P_p = \left\{
\begin{array}{c}
T_p = \left\{
\begin{array}{c}
A_{p,0} = \left\{ \{\{3A\}A_{sj,0}\ B\}, \left[\begin{array}{c} 18 \\ 18 \\ 10 \end{array}\right]\right\}, \\
A_{p,1} = \left\{ \{\{2A\}A_{sj,1}\ B\}, \left[\begin{array}{c} 12 \\ 12 \\ 10 \end{array}\right]\right\}, \\
A_{p,2} = \left\{ \{\{3A\}A_{sj,0}\ B\}, \left[\begin{array}{c} 18 \\ 18 \\ 10 \end{array}\right]\right\}, \\
A_{p,3} = \left\{ \{A\ A_{sj,1}\ B\}, \left[\begin{array}{c} 6 \\ 6 \\ 10 \end{array}\right]\right\} \\
\end{array}\right\}, \\
I_p = \left\{ \right\}, \\
c_p = \left[ \begin{array}{c} 54 \\ 54 \\ 40 \end{array} \right],
c^i_p = \left[ \begin{array}{c} 0 \\ 0 \\ 0 \end{array} \right]
\end{array}
\right\}
\end{displaymath}
\end{comment}

\subsubsection{\filter}

Since {\filters} have no internal buffering and only one {\work}
function, their schedules are simple.  They contain a single
phase, which in turn contains a single execution of the filter's
{\work} function.  Although in principle, a {\filter} does not need to
be executed to be initialized, it may require some data to be
buffered for its execution.  This means that if $e_f > o_f$, we
insert an artificial initialization phase to phasing schedules of
{\filters}:

\begin{displaymath} \small
P_p = \left\{
\begin{array}{c}
T_p = \left\{
\begin{array}{c}
A_{f,0} = \left\{ \{f\}, \left[\begin{array}{c} e_f \\ o_f \\ u_f \end{array}\right]\right\} \\
\end{array}\right\}, \\
I_p = \left\{ A^i_{f,0} = \left\{ \{ \}, \left[\begin{array}{c}e_f - o_f \\ 0 \\ 0 \end{array}\right]\right\} \right\}, \\
c_p = \left[ \begin{array}{c} e_f \\ o_f \\ u_f \end{array}
\right], c^i_p = \left[ \begin{array}{c} e_f - o_f \\ 0 \\ 0
\end{array} \right]
\end{array}
\right\}
\end{displaymath}

\subsubsection{\pipeline, \splitjoin and \feedbackloop}

Technique used for calculating minimal latency phasing schedule
for a {\pipeline}, {\splitjoin} and {\feedbackloop} is similar to
the technique used to create a pseudo single-appearance
hierarchical schedule for a {\feedbackloop}. Every phase is
computed separately. Every phase knows how much data has been left
in internal buffers by the previous phase.  The goal is to create
a phase that consumes the minimum amount of data from the {\Input}
{{\Channel}} in order to push at least one data item out to the
{\Output} {{\Channel}}. Once the minimum amount of data has been
consumed by the stream, the maximum amount of data possible is
pushed out of the stream without consuming any more data. This is
meant to prevent unnecessary buffering of data internally within
streams, and reduce the number of phases necessary to compute a
complete schedule.

One important technique used for creating phased schedules is
borrowing of data from {\Channels}.  When a child is being
executed, it is allowed to borrow some data from the {\Channel},
and expect that the upstream child will provide the right amount
of data in the {\Channel} for real execution. In the tables, this
means that amount of data can fall below 0.
\begin{comment}
This is obviously illegal during real execution for any
{\Channel}. Some {\Channels}, however, have even stricter
restrictions. If the node reading from a {\Channel} peeks more
than it pops, the amount of data in the {\Channel} during real
execution cannot fall below the $peek-pop$ amount.
\end{comment}
Since some \filters require a positive number of data items in the
\Channel , we also need to keep track of amount of data needed
from a {\Channel}, as can be seen in Table\ref{tbl:min-lat-sj}.

The initialization schedule starts with no internally buffered
data (with exception of {\feedbackloops}) and executes as many
phases as is necessary to ensure that all children have executed
all of their initialization phases. Once that has been achieved,
the steady state schedule is created. The only difference between
computation of an initialization and steady state schedules is
that the steady state schedule stops executing children early, if
they have already executed all the phases allocated to them for
the steady state, while the initialization schedule continues
executing until all initialization phases of all children have
been executed.

The only significant difference between the algorithms used for
minimal latency scheduling of different stream types ({\pipeline},
{\splitjoin} and {\feedbackloop}) is the order with which children
of the stream are considered for execution.

\begin{comment}
For an $i$th child of a stream $s$ (stream $s_n$), the number of
phases that must be executed for its steady state schedule to be
complete is $S_{s,v,i} * |P_{s_i,T}|$.
\end{comment}

For a \pipeline, the order with which children are considered for
execution is as follows.  First all the children are considered
for execution moving from bottom to top.  The last child executes
just enough phases to produce some data.  The child directly above
it executes just enough phases to provide sufficient data for the
child below to execute its child.  This process is repeated until
the top-most child is reached.  At this point the direction of
traversal is reversed. This time, the top-most child is skipped,
and the second top-most child is considered.  It only executes as
many phases as it can, while only using data already buffered
between it and the child above it. Then, the child below it is
executed in the same way. This is repeated until the bottom-most
child is reached. The number of phases executed by each child is
added up, and the phases are inserted in order (all phases of
every child together, in order, iterating from top-most child down
to bottom-most child).  This constitutes one complete phase of the
{\pipeline}.

For a \splitjoin, the process is similar, but the children are
first executed bottom up starting with \joiner, then the stream
children and finally the \splitter. After the children are
executed from top to bottom (excluding the \splitter), consuming
only data already available to them in \Channels.

\begin{comment}
Using the sample {\pipeline} from Figure
\ref{fig:hierarchical-schedule}(a), the following are phasing
schedules for {\filters} A, B, C and D:

\begin{displaymath} \small
P_A = \left\{
\begin{array}{c}
T_A = \left\{
\begin{array}{c}
A_{A,0} = \left\{ \{A\}, \left[\begin{array}{c} 1 \\ 1 \\ 3 \end{array}\right]\right\} \\
\end{array}\right\}, \\
I_A = \left\{ \right\},  c_A = \left[ \begin{array}{c} 1 \\ 1 \\
3 \end{array} \right], c^i_B = \left[ \begin{array}{c} 0 \\ 0 \\ 0
\end{array} \right]
\end{array}
\right\}
\end{displaymath}

\begin{displaymath} \small
P_B = \left\{
\begin{array}{c}
T_B = \left\{
\begin{array}{c}
A_{B,0} = \left\{ \{B\}, \left[\begin{array}{c} 3 \\ 2 \\ 3 \end{array}\right]\right\} \\
\end{array}\right\}, \\
I_B = \left\{ A^i_{B,0} = \left\{ \{ \}, \left[\begin{array}{c}1 \\ 0 \\ 0 \end{array}\right]\right\} \right\}, \\
c_B = \left[ \begin{array}{c} 3 \\ 2 \\ 3 \end{array} \right],
c^i_B = \left[ \begin{array}{c} 1 \\ 0 \\ 0
\end{array} \right]
\end{array}
\right\}
\end{displaymath}

\begin{displaymath} \small
P_C = \left\{
\begin{array}{c}
T_C = \left\{
\begin{array}{c}
A_{C,0} = \left\{ \{C\}, \left[\begin{array}{c} 2 \\ 2 \\ 1 \end{array}\right]\right\} \\
\end{array}\right\}, \\
I_C = \left\{ \right\}, c_C = \left[ \begin{array}{c} 2 \\ 2 \\ 1
\end{array} \right], c^i_C = \left[ \begin{array}{c} 0 \\ 0 \\ 0
\end{array} \right]
\end{array}
\right\}
\end{displaymath}

\begin{displaymath} \small
P_D = \left\{
\begin{array}{c}
T_D = \left\{
\begin{array}{c}
A_{D,0} = \left\{ \{D\}, \left[\begin{array}{c} 5 \\ 3 \\ 1 \end{array}\right]\right\} \\
\end{array}\right\}, \\
I_D = \left\{ A^i_{D,0} = \left\{ \{ \}, \left[\begin{array}{c}2 \\ 0 \\ 0 \end{array}\right]\right\} \right\}, \\
c_D = \left[ \begin{array}{c} 5 \\ 3 \\ 1 \end{array} \right],
c^i_D = \left[ \begin{array}{c} 2 \\ 0 \\ 0
\end{array} \right]
\end{array}
\right\}
\end{displaymath}

\noindent Table \ref{tbl:min-lat-pipe} shows a trace of execution
of the algorithm on the {\pipeline} from Figure
\ref{fig:hierarchical-schedule}(a).

\begin{table} \centering
\small
\begin{tabular}{|c|c|c|c|c|c|c|c|c|c|c|}
\hline
\multicolumn{3}{|c|}{data in {{\Channel}}} & \multicolumn{4}{c|}{\parbox{1in}{\centering phase executions left}} & \parbox{0.5in}{\centering child considered} & \parbox{0.8in}{\centering phases executed} & \parbox{0.8in}{\centering {\pipeline} consumption} \\
\cline{1-7} $in_B$ & $in_C$ & $in_D$ & A & B & C & D & & & \\

\hline 0 (0) & 0 (0) & 0 (-2) & 0 & 1 & 0 & 0 & C & $\{2 A_{C,0}\}$ & $[0\ 0\ 0]$ \\
\hline 0 (0) & -4 (-4) & 2 (0) & 0 & 1 & 0 & 0 & B & $A^i_{B,0}, \{2A_{B,0}\}$ & $[0\ 0\ 0]$ \\
\hline -4 (-5) & 2 (0) & 2 (0) & 0 & 0 & 0 & 0 & A & $\{2A_{A, 0}\}$ & $[2\ 2\ 0]$ \\
\hline 2 (0) & 2 (0) & 2 (0) & 0 & 0 & 0 & 0 & B & - & $[0\ 0\ 0]$ \\
\hline 2 (0) & 2 (0) & 2 (0) & 0 & 0 & 0 & 0 & C & $A_{C,0}$ & $[0\ 0\ 0]$ \\
\hline 2 (0) & 0 (0) & 3 (0) & 0 & 0 & 0 & 0 & D & - & $[0\ 0\ 0]$ \\
\hline 2 (0) &  0 (0) &  3 (0) & \multicolumn{7}{|c|}{init phase 0 done, init done} \\
\hline 2 (0) & 0 (0) & 3 (0) & 4 & 6 & 9 & 3 & D & $A_{D,0}$ & $[0\ 0\ 1]$ \\
\hline 2 (0) & 0 (0) & 0 (-2) & 4 & 6 & 9 & 2 & C & $\{2 A_{C,0}\}$ & $[0\ 0\ 0]$ \\
\hline 2 (0) & -4 (-4) & 2 (0) & 4 & 6 & 7 & 2 & B & $\{2 A_{B,0}\}$ & $[0\ 0\ 0]$ \\
\hline -2 (-3) & 2 (0) & 2 (0) & 4 & 4 & 7 & 2 & A & $A_{A,0}$ & $[1\ 1\ 0]$ \\
\hline 1 (0) & 2 (0) & 2 (0) & 3 & 4 & 7 & 2 & B & - & $[0\ 0\ 0]$ \\
\hline 1 (0) & 2 (0) & 2 (0) & 3 & 4 & 7 & 2 & C & $A_{C,0}$ & $[0\ 0\ 0]$ \\
\hline 1 (0) & 0 (0) & 3 (0) & 3 & 4 & 6 & 2 & D & - & $[0\ 0\ 0]$ \\
\hline 1 (0) &  0 (0) &  3 (0) &  \multicolumn{7}{|c|}{phase 0 done} \\
\hline 1 (0) & 0 (0) & 3 (0) & 3 & 4 & 6 & 2 & D & $A_{D,0}$ & $[0\ 0\ 1]$ \\
\hline 1 (0) & 0 (0) & 0 (-2) & 3 & 4 & 6 & 1 & C & $\{2 A_{C,0}\}$ & $[0\ 0\ 0]$ \\
\hline 1 (0) & -4 (-4) & 2 (0) & 3 & 4 & 4 & 1 & B & $\{2 A_{B,0}\}$ & $[0\ 0\ 0]$ \\
\hline -3 (-4) & 2 (0) & 2 (0) & 3 & 2 & 4 & 1 & A & $\{2 A_{A,0}\}$ & $[2\ 2\ 0]$ \\
\hline 3 (0) & 2 (0) & 2 (0) & 1 & 2 & 4 & 1 & B & $A_{B,0}$ & $[0\ 0\ 0]$ \\
\hline 1 (0) & 5 (0) & 2 (0) & 1 & 1 & 4 & 1 & C & $\{2 A_{C,0}\}$ & $[0\ 0\ 0]$ \\
\hline 1 (0) & 1 (0) & 4 (0) & 1 & 1 & 2 & 1 & D & - & $[0\ 0\ 0]$ \\
\hline 1 (0) &  1 (0) &  4 (0) &  \multicolumn{7}{|c|}{phase 1 done} \\
\hline 1 (0) & 1 (0) & 4 (0) & 1 & 1 & 2 & 1 & D & $A_{D,0}$ & $[0\ 0\ 1]$ \\
\hline 1 (0) & 1 (0) & 1 (-1) & 1 & 1 & 2 & 0 & C & $A_{C,0}$ & $[0\ 0\ 0]$ \\
\hline 1 (0) & -1 (-1) & 2 (0) & 1 & 1 & 1 & 0 & B & $A_{B,0}$ & $[0\ 0\ 0]$ \\
\hline -1 (-2) & 2 (0) & 2 (0) & 1 & 1 & 1 & 0 & A & $A_{A,0}$ & $[1\ 1\ 0]$ \\
\hline 2 (0) & 2 (0) & 2 (0) & 0 & 0 & 1 & 0 & B & - & $[0\ 0\ 0]$ \\
\hline 2 (0) & 2 (0) & 2 (0) & 0 & 0 & 1 & 0 & C & $A_{C,0}$ & $[0\ 0\ 0]$ \\
\hline 2 (0) & 0 (0) & 3 (0) & 0 & 0 & 0 & 0 & D & - & $[0\ 0\ 0]$ \\
\hline 2 (0) &  0 (0) &  3 (0) &  \multicolumn{7}{|c|}{phase 2 done, steady state schedule done} \\
\hline
\end{tabular}
\caption[Trace of execution of Minimal Latency Scheduling on a
{\pipeline}]{Trace of execution of Minimal Latency Scheduling
Algorithm on {\pipeline} from Figure
\ref{fig:hierarchical-schedule}(a). In the "data in {{\Channel}}"
columns the first value represents the actual number of data in
the {{\Channel}}, which can be negative if more data has been popped
from the {{\Channel}} than has been pushed into it.  This is due to
borrowing of data from {\Channels}. The second value represents the
minimal number of data items that the downstream {\filter} has
inspected beyond the 0th data. This value can be higher than the
negative amount of data in the {{\Channel}} because a {\filter} may
peek at data without consuming it.  In general, for a {\filter}
$f$, the amount of data needed on its input {{\Channel}} is $\max(0,
-(in_f - (e_f - o_f)))$. The needed amount is 0 until the
downstream {\filter} is executed for the first time.}
\label{tbl:min-lat-pipe}
\end{table}

\noindent The following is the resulting phasing schedule:

\begin{displaymath} \small
P_p = \left\{
\begin{array}{c}
T_p = \left\{
\begin{array}{c}
A_{p,0} = \left\{ \{A\{2B\}\{3C\}D\}, \left[\begin{array}{c} 1 \\ 1 \\ 1 \end{array}\right]\right\}, \\
A_{p,1} = \left\{ \{\{2A\}\{3B\}\{4C\}D\}, \left[\begin{array}{c} 2 \\ 2 \\ 1 \end{array}\right]\right\}, \\
A_{p,2} = \left\{ \{A\ B\{2C\}D\}, \left[\begin{array}{c} 1 \\ 1 \\ 1 \end{array}\right]\right\} \\
\end{array}\right\}, \\
I_p = \left\{ A^i_{p,0} = \left\{
\{\{2A\}A^i_{B,0}\{2B\}\{3C\}A^i_{D,0}\}, \left[\begin{array}{c} 2  \\2 \\ 0 \\
\end{array}\right]\right\}
\right\}, \\
c_p = \left[ \begin{array}{c} 4 \\ 4 \\ 3 \end{array} \right],
c^i_p = \left[ \begin{array}{c} 2 \\ 2 \\ 0 \end{array} \right]
\end{array}
\right\}
\end{displaymath}

\subsubsection{\splitjoin}

As explained above, the only difference between the algorithm for
a {\pipeline} and a {\splitjoin} is the order in which the children
streams are considered for execution.  In a {\pipeline}, the
children are considered from the bottom-most child to the top
child, and then from second top-most child down to the bottom most
child again.  A {\splitjoin} has only three levels of direct
children in it: the top is a {\splitter}, the middle is formed by
all the child streams of the {\splitjoin} and the bottom is the
{\joiner}.  To schedule a {\splitjoin}, the children are also
considered in the bottom to top and top to bottom order, but the
child streams are also considered from left to right (this choice
is arbitrary - the order does not affect the number of child phase
executions per phase of the {\splitjoin}).

Using the sample {\splitjoin} from Figure
\ref{fig:hierarchical-schedule}(b), the following are phasing
schedules for {\filters} A and B:

\begin{displaymath} \small
P_A = \left\{
\begin{array}{c}
T_A = \left\{
\begin{array}{c}
A_{A,0} = \left\{ \{A\}, \left[\begin{array}{c} 2 \\ 2 \\ 1 \end{array}\right]\right\} \\
\end{array}\right\}, \\
I_A = \left\{ \right\}, c_A = \left[ \begin{array}{c} 2 \\ 2 \\ 1
\end{array} \right], c^i_A = \left[ \begin{array}{c} 0 \\ 0 \\ 0
\end{array} \right]
\end{array}
\right\}
\end{displaymath}

\begin{displaymath} \small
P_B = \left\{
\begin{array}{c}
T_B = \left\{
\begin{array}{c}
A_{B,0} = \left\{ \{B\}, \left[\begin{array}{c} 3 \\ 2 \\ 6 \end{array}\right]\right\} \\
\end{array}\right\}, \\
I_B = \left\{ A^i_{B,0} = \left\{ \{ \}, \left[\begin{array}{c} 1 \\ 0 \\ 0 \end{array}\right]\right\} \right\}, \\
c_B = \left[ \begin{array}{c} 3 \\ 2 \\ 6 \end{array} \right],
c^i_B = \left[ \begin{array}{c} 1 \\ 0 \\ 0
\end{array} \right]
\end{array}
\right\}
\end{displaymath}

\noindent Table \ref{tbl:min-lat-sj} shows execution of the
algorithm on the {\splitjoin} from Figure
\ref{fig:hierarchical-schedule}(b).

\begin{table} \centering \small
\begin{tabular}{|c|c|c|c|c|c|c|c|c|c|c|c|}
\hline
\multicolumn{4}{|c|}{data in {{\Channel}}} & \multicolumn{4}{c|}{\parbox{1in}{\centering phase executions left}} & \parbox{0.5in}{\centering child considered} & \parbox{0.6in}{\centering phases executed} & \parbox{0.6in}{\centering {\pipeline} consumption} \\
\cline{1-8} split & A & B & join & $in_A$ & $out_A$ & $in_B$ & $out_B$ & & & \\
\hline
\hline 0 (0) & 0 (0) & 0 (0) & 0 (0) & 0 & 0 & 1 & 0 & join & - & $[0\ 0\ 0]$ \\
\hline 0 (0) & 0 (0) & 0 (0) & 0 (0) & 0 & 0 & 1 & 0 & A & - & $[0\ 0\ 0]$ \\
\hline 0 (0) & 0 (0) & 0 (0) & 0 (0) & 0 & 0 & 1 & 0 & B & $A^i_{B,0}$ & $[0\ 0\ 0]$ \\
\hline 0 (0) & 0 (0) & 0 (-1) & 0 (0) & 0 & 0 & 0 & 0 & split & split & $[3\ 3\ 0]$ \\
\hline 2 (0) & 0 (0) & 1 (0) & 0 (0) & 0 & 0 & 0 & 0 & A & $A^i_{A,0}$ & $[0\ 0\ 0]$ \\
\hline 0 (0) & 1 (0) & 1 (0) & 0 (0) & 0 & 0 & 0 & 0 & B & - & $[0\ 0\ 0]$ \\
\hline 0 (0) & 1 (0) & 1 (0) & 0 (0) & 0 & 0 & 0 & 0 & join & - & $[0\ 0\ 0]$ \\
\hline 0 (0) &  1 (0) &  1 (0) &  0 (0) & \multicolumn{7}{|c|}{init phase 0 done, init done} \\
\hline 0 (0) & 1 (0) & 1 (0) & 0 (0) & 2 & 2 & 1 & 2 & join & join & $[0\ 0\ 4]$ \\
\hline 0 (0) & 0 (0) & 1 (0) & -3 (-3) & 2 & 2 & 1 & 2 & A & - & $[0\ 0\ 0]$ \\
\hline 0 (0) & 0 (0) & 1 (0) & -3 (-3) & 2 & 2 & 1 & 1 & B & $A_{B,0}$ & $[0\ 0\ 0]$ \\
\hline 0 (0) & 0 (0) & -1 (-2) & 3 (0) & 2 & 2 & 0 & 2 & split & $\{2split\}$ & $[6\ 6\ 0]$ \\
\hline 4 (0) & 0 (0) & 1 (0) & 3 (0) & 0 & 2 & 0 & 2 & A & $\{2A_{A,0}\}$ & $[0\ 0\ 0]$ \\
\hline 0 (0) & 2 (0) & 1 (0) & 3 (0) & 0 & 0 & 0 & 0 & B & - & $[0\ 0\ 0]$ \\
\hline 0 (0) & 2 (0) & 1 (0) & 3 (0) & 0 & 0 & 0 & 1 & join & join  & $[0\ 0\ 4]$ \\
\hline 0 (0) &  1 (0) &  1 (0) &  0 (0) & \multicolumn{7}{|c|}{phase 0 done, steady state schedule done} \\
\hline
\end{tabular}
\caption[Execution of Minimal Latency Scheduling Algorithm on a
{\splitjoin}]{Execution of Minimal Latency Scheduling Algorithm on
{\splitjoin} from Figure \ref{fig:hierarchical-schedule}(b).}
\label{tbl:min-lat-sj}
\end{table}

The trace of the execution shows that even though it is strictly
necessary to traverse the children of the stream second time from
bottom to top, doing so can pay off in reducing the number of
phases necessary to construct a phasing schedule.  Namely, in its
first steady state execution, the {\splitter} needs to push enough
data to execute the {\joiner} again, thus eliminating a need for an
additional phase.

Once all the phases are computed, the phasing schedule is
constructed. For every phase, the number of child phases executed
is added up, and the actual schedule is constructed by
concatenating all the phases of all the children, starting with
the {\splitter}, all stream children (as listed from left to right)
and finally the {\joiner}. The following is the resulting phasing
schedule:

\begin{displaymath} \small
P_{sj} = \left\{
\begin{array}{c}
T_{sj} = \left\{
\begin{array}{c}
A_{sj,0} = \left\{ \{\{2 split\}\{2A\}B\{2 join\}\}, \left[\begin{array}{c} 6 \\ 6 \\ 8 \end{array}\right]\right\} \\
\end{array}\right\}, \\
I_{sj} = \left\{ A^i_{sj,0} = \left\{
\{split\ A^i_{A,0}\ A^i_{B,0}\}, \left[\begin{array}{c} 3 \\ 3 \\ 0 \\
\end{array}\right]\right\}
\right\}, \\
c_{sj} = \left[ \begin{array}{c} 6 \\ 6 \\ 8 \end{array} \right],
c^i_{sj} = \left[ \begin{array}{c} 3 \\ 3 \\ 0 \end{array}
\right],
\end{array}
\right\}
\end{displaymath}

\subsubsection{\feedbackloop}
\end{comment}

Scheduling of {\feedbackloops} is again similar to the above
algorithms.  The children's phases are executed in order of
({\splitter}, body child, {\joiner}, body child, {\splitter}, loop
child).  The {\splitter} tries to execute exactly one time on its
first iteration.  The body child and the {\joiner} execute just
enough times to provide data for the {\splitter} to perform its
first execution.  Then the body child, {\splitter} and the loop
child are executed as many times as possible with the data
available to them on their {\Input} {{\Channels}}.

The one big difference between {\feedbackloop} and the other streams
({\pipeline} and {\splitjoin}) is that in scheduling a {\feedbackloop},
the {\joiner} is {\emph not} allowed to borrow elements from $out_L$
{{\Channel}}.  That is in the trace table, the $out_L$ entry is never
allowed to become negative.  The reason for this is that
{\feedbackloops} are cyclical structures, and allowing the {\joiner}
to borrow elements from $out_L$ would cause a full cycle of
borrowing, leading to deadlock.

\begin{comment}
This one condition does not prevent from scheduling any legal
{\feedbackloops}.  The reason for this is that before the
{\feedbackloop} is initialized, there is data pushed onto the
$out_L$ {{\Channel}}.  At the end of scheduling of any phase, all
available data is pushed through the {\feedbackloop} into the
$out_L$ {{\Channel}}.  Thus any available free data is already always
stored in the $out_L$ {{\Channel}}, and there is no additional data to
borrow from in a {\feedbackloop}.

If the algorithm is unable to schedule an execution of the {\joiner}
in a phase without borrowing data from $out_L$ {{\Channel}}, then the
{\feedbackloop} cannot be scheduled.

\begin{lemma}[{\feedbackloop} Scheduling]
If all children of a {\feedbackloop} are scheduled using minimal
latency scheduling algorithm, then if the {\feedbackloop} cannot
be scheduled using the minimal latency scheduling algorithm then
there is no valid schedule for this {\feedbackloop}.
\end{lemma}

We believe this lemma to be true because minimal latency
scheduling always consumes the minimal amount of data to produce
some data, and produces the maximal amount of data possible given
the amount of data it consumes.  Thus no data is being buffered up
in {{\Channels}} and if the {\feedbackloop} cannot be scheduled, then
the $delay_{fl}$ value is too low and does not provide enough data
to complete a steady state execution. A formal proof is left for
future work.

We will again use the sample {\feedbackloop} from Figure
\ref{fig:hierarchical-schedule}(c).  The following are the phasing
schedules for {\filters} B and L:

\begin{displaymath} \small
P_B = \left\{
\begin{array}{c}
T_B = \left\{
\begin{array}{c}
A_{B,0} = \left\{ \{B\}, \left[\begin{array}{c} 2 \\ 2 \\ 1 \end{array}\right]\right\} \\
\end{array}\right\}, \\
I_B = \left\{ \right\}, \\
c_B = \left[ \begin{array}{c} 2 \\ 2 \\ 1 \end{array} \right],
c^i_B = \left[ \begin{array}{c} 0 \\ 0 \\ 0
\end{array} \right]
\end{array}
\right\}
\end{displaymath}

\begin{displaymath} \small
P_L = \left\{
\begin{array}{c}
T_L = \left\{
\begin{array}{c}
A_{L,0} = \left\{ \{L\}, \left[\begin{array}{c} 9 \\ 5 \\ 6 \end{array}\right]\right\} \\
\end{array}\right\}, \\
I_L = \left\{ A^i_{L,0} = \left\{ \{ \}, \left[\begin{array}{c} 4 \\ 0 \\ 0 \end{array}\right]\right\} \right\}, \\
c_L = \left[ \begin{array}{c} 9 \\ 5 \\ 6 \end{array} \right],
c^i_L = \left[ \begin{array}{c} 4 \\ 0 \\ 0
\end{array} \right]
\end{array}
\right\}
\end{displaymath}

\noindent Table \ref{tbl:min-lat-fl} shows execution of the
algorithm on the {\feedbackloop} from Figure
\ref{fig:hierarchical-schedule}(c).

\begin{table} \centering
\scriptsize
\begin{tabular}{|c|c|c|c|c|c|c|c|c|c|c|c|}
\hline
\multicolumn{4}{|c|}{data in {{\Channel}}} & \multicolumn{4}{c|}{\parbox{1in}{\centering phase executions left}} & \parbox{0.5in}{\centering child considered} & \parbox{0.6in}{\centering phases executed} & \parbox{0.6in}{\centering {\pipeline} consumption} \\
\cline{1-8} $in_B$ & $out_B$ & $in_F$ & $out_F$ & join & B & split & F & & & \\
\hline 0 (0) & 0 (0) & 0 (0) & 15 (0) & 0 & 0 & 0 & 1 & split & split & $[0\ 0\ 3]$ \\
\hline 0 (0) & -3 (3) & 3 (0) & 15 (0) & 0 & 0 & 0 & 1 & B & $\{3A_{B,0}\}$ & $[0\ 0\ 0]$ \\
\hline -6 (6) & 0 (0) & 3 (0) & 15 (0) & 0 & 0 & 0 & 1 & join & $\{2\ join\}$ & $[4\ 4\ 0]$ \\
\hline 4 (0) & 0 (0) & 3 (0) & 9 (0) & 0 & 0 & 0 & 1 & B & $\{2A_{B,0}\}$ & $[0\ 0\ 0]$ \\
\hline 0 (0) & 2 (0) & 3 (0) & 9 (0) & 0 & 0 & 0 & 1 & split & - & $[0\ 0\ 0]$ \\
\hline 0 (0) & 2 (0) & 3 (0) & 9 (0) & 0 & 0 & 0 & 1 & F & - & $[0\ 0\ 0]$ \\

\hline 0 (0) &  2 (0) &  3 (0) &   9 (0) & \multicolumn{7}{|c|}{init phase 0 done} \\

\hline 0 (0) & 2 (0) & 3 (0) & 9 (0) & 0 & 0 & 0 & 1 & split & split & $[0\ 0\ 3]$ \\
\hline 0 (0) & -1 (1) & 6 (0) & 9 (0) & 0 & 0 & 0 & 1 & B & $\{A_{B,0}\}$ & $[0\ 0\ 0]$ \\
\hline -2 (2) & 0 (0) & 6 (0) & 9 (0) & 0 & 0 & 0 & 1 & join & join & $[2\ 2\ 0]$ \\
\hline 3 (0) & 0 (0) & 6 (0) & 6 (0) & 0 & 0 & 0 & 1 & B & $\{A_{B,0}\}$ & $[0\ 0\ 0]$ \\
\hline 1 (0) & 1 (0) & 6 (0) & 6 (0) & 0 & 0 & 0 & 1 & split & - & $[0\ 0\ 0]$ \\
\hline 1 (0) & 1 (0) & 6 (0) & 6 (0) & 0 & 0 & 0 & 1 & F & $\{A^i_{F,0}\}$ & $[0\ 0\ 0]$ \\

\hline 1 (0) &  1 (0) &  6 (0) &   6 (0) & \multicolumn{7}{|c|}{init phase 1 done, init done} \\

\hline 1 (0) & 1 (0) & 6 (0) & 6 (0) & 6 & 15 & 5 & 3 & split & split & $[0\ 0\ 3]$ \\
\hline 1 (0) & -2 (2) & 9 (0) & 6 (0) & 6 & 15 & 4 & 3 & B & $\{3A_{B,0}\}$ & $[0\ 0\ 0]$ \\
\hline -5 (5) & 1 (0) & 9 (0) & 6 (0) & 6 & 12 & 4 & 3 & join & join & $[2\ 2\ 0]$ \\
\hline 0 (0) & 1 (0) & 9 (0) & 3 (0) & 5 & 12 & 4 & 3 & B & - & $[0\ 0\ 0]$ \\
\hline 0 (0) & 1 (0) & 9 (0) & 3 (0) & 5 & 12 & 4 & 3 & split & - & $[0\ 0\ 0]$ \\
\hline 0 (0) & 1 (0) & 4 (0) & 9 (0) & 5 & 12 & 4 & 3 & F & $\{A_{F,0}\}$ & $[0\ 0\ 0]$ \\

\hline 0 (0) &  1 (0) &  4 (0) &   9 (0) & \multicolumn{7}{|c|}{phase 0 done} \\

\hline 0 (0) & 1 (0) & 4 (0) & 9 (0) & 5 & 12 & 4 & 2 & split & split & $[0\ 0\ 3]$ \\
\hline 0 (0) & -2 (2) & 7 (0) & 9 (0) & 5 & 12 & 3 & 2 & B & $\{2A_{B,0}\}$ & $[0\ 0\ 0]$ \\
\hline -4 (4) & 0 (0) & 7 (0) & 9 (0) & 5 & 10 & 3 & 2 & join & join & $[2\ 2\ 0]$ \\
\hline 1 (0) & 0 (0) & 7 (0) & 6 (0) & 4 & 10 & 3 & 2 & B & - & $[0\ 0\ 0]$ \\
\hline 1 (0) & 0 (0) & 7 (0) & 6 (0) & 4 & 10 & 3 & 2 & split & - & $[0\ 0\ 0]$ \\
\hline 1 (0) & 0 (0) & 7 (0) & 6 (0) & 4 & 10 & 3 & 2 & F & - & $[0\ 0\ 0]$ \\

\hline 1 (0) & 0 (0) & 7 (0) & 6 (0) & \multicolumn{7}{|c|}{phase 1 done} \\

\hline 1 (0) & 0 (0) & 7 (0) & 6 (0) & 4 & 10 & 3 & 2 & split & split & $[0\ 0\ 3]$ \\
\hline 1 (0) & -3 (3) & 10 (0) & 6 (0) & 4 & 10 & 2 & 2 & B & $\{3A_{B,0}\}$ & $[0\ 0\ 0]$ \\
\hline -5 (5) & 0 (0) & 10 (0) & 6 (0) & 4 & 7 & 2 & 2 & join & join & $[2\ 2\ 0]$ \\
\hline 0 (0) & 0 (0) & 10 (0) & 3 (0) & 3 & 7 & 2 & 2 & B & - & $[0\ 0\ 0]$ \\
\hline 0 (0) & 0 (0) & 10 (0) & 3 (0) & 3 & 7 & 2 & 2 & split & - & $[0\ 0\ 0]$ \\
\hline 0 (0) & 0 (0) & 10 (0) & 3 (0) & 3 & 7 & 2 & 2 & F & $\{A_{F,0}\}$ & $[0\ 0\ 0]$ \\

\hline 0 (0) & 0 (0) & 5 (0) & 9 (0) & \multicolumn{7}{|c|}{phase 2 done} \\

\hline 0 (0) & 0 (0) & 5 (0) & 9 (0) & 3 & 7 & 2 & 1 & split & split & $[0\ 0\ 3]$ \\
\hline 0 (0) & -3 (3) & 8 (0) & 9 (0) & 3 & 7 & 1 & 1 & B & $\{3A_{B,0}\}$ & $[0\ 0\ 0]$ \\
\hline -6 (6) & 0 (0) & 8 (0) & 9 (0) & 3 & 4 & 1 & 1 & join & $\{2\ join\}$ & $[4\ 4\ 0]$ \\
\hline 4 (0) & 0 (0) & 8 (0) & 3 (0) & 1 & 4 & 1 & 1 & B & $\{2A_{B,0}\}$ & $[0\ 0\ 0]$ \\
\hline 0 (0) & 2 (0) & 8 (0) & 3 (0) & 1 & 2 & 1 & 1 & split & - & $[0\ 0\ 0]$ \\
\hline 0 (0) & 2 (0) & 8 (0) & 3 (0) & 1 & 2 & 1 & 1 & F & - & $[0\ 0\ 0]$ \\

\hline 0 (0) & 2 (0) & 8 (0) & 3 (0) & \multicolumn{7}{|c|}{phase 3 done} \\

\hline 0 (0) & 2 (0) & 8 (0) & 3 (0) & 1 & 2 & 1 & 1 & split & split & $[0\ 0\ 3]$ \\
\hline 0 (0) & -1 (1) & 11 (0) & 3 (0) & 1 & 2 & 0 & 1 & B & $\{A_{B,0}\}$ & $[0\ 0\ 0]$ \\
\hline -2 (2) & 0 (0) & 11 (0) & 3 (0) & 1 & 1 & 0 & 1 & join & join & $[2\ 2\ 0]$ \\
\hline 3 (0) & 0 (0) & 11 (0) & 0 (0) & 0 & 1 & 0 & 1 & B & $\{A_{B,0}\}$ & $[0\ 0\ 0]$ \\
\hline 1 (0) & 1 (0) & 11 (0) & 0 (0) & 0 & 0 & 0 & 1 & split & - & $[0\ 0\ 0]$ \\
\hline 1 (0) & 1 (0) & 11 (0) & 0 (0) & 0 & 0 & 0 & 1 & F & $\{A_{F,0}\}$ & $[0\ 0\ 0]$ \\

\hline 1 (0) & 1 (0) & 6 (0) & 6 (0) & \multicolumn{7}{|c|}{phase 4 done, steady state schedule done} \\
\hline
\end{tabular}
\caption[Execution of Minimal Latency Scheduling Algorithm on a
{\feedbackloop}]{Execution of Minimal Latency Scheduling Algorithm
on {\feedbackloop} from Figure
\ref{fig:hierarchical-schedule}(c).} \label{tbl:min-lat-fl}
\end{table}

Once the number of executions of children's phases is known for
every phase of the {\feedbackloop}'s schedule, the phasing schedule
can be constructed.  For every phase, the children of the
{\feedbackloop} are iterated over in order of ({\joiner}, body child,
{\splitter}, loop child) and for every child the appropriate number
of phases is inserted into the schedule.  Below is the schedule
for {\feedbackloop} in Figure \ref{fig:hierarchical-schedule}(c):

\begin{displaymath} \small
P_{fl} = \left\{
\begin{array}{c}
T_{fl} = \left\{
\begin{array}{c}
A_{fl,0} = \left\{ \{join\ \{3B\}\ split\ F\}, \left[\begin{array}{c} 2 \\ 2 \\ 3 \end{array}\right]\right\}, \\
A_{fl,1} = \left\{ \{join\ \{2B\}\ split\}, \left[\begin{array}{c} 2 \\ 2 \\ 3 \end{array}\right]\right\}, \\
A_{fl,2} = \left\{ \{join\ \{3B\}\ split\ F\}, \left[\begin{array}{c} 2 \\ 2 \\ 3 \end{array}\right]\right\}, \\
A_{fl,3} = \left\{ \{\{2\ join\}\{5B\}\ split\}, \left[\begin{array}{c} 4 \\ 4 \\ 3 \end{array}\right]\right\}, \\
A_{fl,4} = \left\{ \{join\ \{2B\}\ split\ F\}, \left[\begin{array}{c} 2 \\ 2 \\ 3 \end{array}\right]\right\} \\
\end{array}\right\}, \\
T_{fl} = \left\{
\begin{array}{c}
A^i_{fl,0} = \left\{ \{\{2\ join\}\{5B\}\ split\}, \left[\begin{array}{c} 4 \\ 4 \\ 3 \end{array}\right]\right\}, \\
A^i_{fl,1} = \left\{ \{join\ \{2B\}\ split\}, \left[\begin{array}{c} 2 \\ 2 \\ 3 \end{array}\right]\right\} \\
\end{array}\right\}, \\
c_{fl} = \left[ \begin{array}{c} 12 \\ 12 \\ 15 \end{array}
\right], c^i_{fl} = \left[ \begin{array}{c} 6 \\ 6 \\ 6
\end{array} \right],
\end{array}
\right\}
\end{displaymath}
\end{comment}

\begin{table}[t] \centering \small
\begin{tabular}{|c|c|c|c|c|c|c|c|c|c|c|c|}
\hline
\multicolumn{4}{|c|}{data in {{\Channel}}} & \multicolumn{4}{c|}{\parbox{1in}{\centering phase executions left}} & \parbox{0.5in}{\centering child considered} & \parbox{0.6in}{\centering phases executed} & \parbox{0.6in}{\centering {\pipeline} consumption} \\
\cline{1-8} split & A & B & join & $in_A$ & $out_A$ & $in_B$ & $out_B$ & & & \\
\hline 0 (0) & 0 (0) & 0 (0) & 0 (0) & 0 & 0 & 1 & 0 & join & - & $[0\ 0\ 0]$ \\
\hline 0 (0) & 0 (0) & 0 (0) & 0 (0) & 0 & 0 & 1 & 0 & A & - & $[0\ 0\ 0]$ \\
\hline 0 (0) & 0 (0) & 0 (0) & 0 (0) & 0 & 0 & 1 & 0 & B & $A^i_{B,0}$ & $[0\ 0\ 0]$ \\
\hline 0 (0) & 0 (0) & 0 (-1) & 0 (0) & 0 & 0 & 0 & 0 & split & split & $[3\ 3\ 0]$ \\
\hline 2 (0) & 0 (0) & 1 (0) & 0 (0) & 0 & 0 & 0 & 0 & A & $A^i_{A,0}$ & $[0\ 0\ 0]$ \\
\hline 0 (0) & 1 (0) & 1 (0) & 0 (0) & 0 & 0 & 0 & 0 & B & - & $[0\ 0\ 0]$ \\
\hline 0 (0) & 1 (0) & 1 (0) & 0 (0) & 0 & 0 & 0 & 0 & join & - & $[0\ 0\ 0]$ \\
\hline 0 (0) &  1 (0) &  1 (0) &  0 (0) & \multicolumn{7}{|c|}{init phase 0 done, init done} \\
\hline 0 (0) & 1 (0) & 1 (0) & 0 (0) & 2 & 2 & 1 & 2 & join & join & $[0\ 0\ 4]$ \\
\hline 0 (0) & 0 (0) & 1 (0) & -3 (-3) & 2 & 2 & 1 & 2 & A & - & $[0\ 0\ 0]$ \\
\hline 0 (0) & 0 (0) & 1 (0) & -3 (-3) & 2 & 2 & 1 & 1 & B & $A_{B,0}$ & $[0\ 0\ 0]$ \\
\hline 0 (0) & 0 (0) & -1 (-2) & 3 (0) & 2 & 2 & 0 & 2 & split & $\{2split\}$ & $[6\ 6\ 0]$ \\
\hline 4 (0) & 0 (0) & 1 (0) & 3 (0) & 0 & 2 & 0 & 2 & A & $\{2A_{A,0}\}$ & $[0\ 0\ 0]$ \\
\hline 0 (0) & 2 (0) & 1 (0) & 3 (0) & 0 & 0 & 0 & 0 & B & - & $[0\ 0\ 0]$ \\
\hline 0 (0) & 2 (0) & 1 (0) & 3 (0) & 0 & 0 & 0 & 1 & join & join  & $[0\ 0\ 4]$ \\
\hline 0 (0) &  1 (0) &  1 (0) &  0 (0) & \multicolumn{7}{|c|}{phase 0 done, steady state schedule done} \\
\hline
\end{tabular}
\caption[Execution of Minimal Latency Scheduling Algorithm on a
{\splitjoin}]{Execution of Minimal Latency Scheduling Algorithm on
{\splitjoin} from Figure \ref{fig:steady-state}(b). In the "data
in {{\Channel}}" columns, the first value represents the actual
number of data in the {{\Channel}}, which can be negative due to
data borrowing. The second value is the minimal number of
additional data items needed in the \Channel.}
\label{tbl:min-lat-sj}
\end{table}

Table \ref{tbl:min-lat-sj} contains a trace of execution of our
algorithm on the sample \splitjoin from Figure
\ref{fig:steady-state}(b). Below is the phasing schedule for the
\splitjoin. Note that this example does produce a single
appearance schedule.

\begin{displaymath} \scriptsize
P_{sj} = \left\{
\begin{array}{c}
T_{sj} = \left\{
\begin{array}{c}
A_{sj,0} = \left\{ \{\{2 split\}\{2A\}B\{2 join\}\}, \left[\begin{array}{c} 6 \\ 6 \\ 8 \end{array}\right]\right\} \\
\end{array}\right\}, \\
I_{sj} = \left\{ A^i_{sj,0} = \left\{
\{split\ A^i_{A,0}\ A^i_{B,0}\}, \left[\begin{array}{c} 3 \\ 3 \\ 0 \\
\end{array}\right]\right\}
\right\}, \\
c_{sj} = \left[ \begin{array}{c} 6 \\ 6 \\ 8 \end{array} \right],
c^i_{sj} = \left[ \begin{array}{c} 3 \\ 3 \\ 0 \end{array}
\right],
\end{array}
\right\}
\end{displaymath}
