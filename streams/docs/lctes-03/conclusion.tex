\section{Conclusion and Future Work}

This paper presents two techniques used for scheduling Synchronous
Data Flow Graphs as used by the {\StreamIt} language. Unlike other
langauges, {\StreamIt} enforces a structure on the stream graph,
thus allowing a variety of new approaches to stream scheduling.
Algorithms presented here improve current scheduling techniques in
multiple ways.

Hierarchical approach to scheduling of streaming applications
allows for very simple algorithms. Program graphs do not have to
be considered globally, thus less data needs to be kept track of.
In hierarchical approaches presented here, we only need to
consider immediate children of a given stream.

Phasing approach to scheduling allows to schedule arbitrarily
tight {\feedbackloops} and allows for more fine-grained control
of buffering requirements. The fine-grained control of buffering
requirements can provide dramatic reduction of buffer sizes when
scheduling streaming applications, as has been presented here.
Furthermore, phased schedules lend themselves to some easy forms
of compression, thus further reducing the schedule size. Future
work will concentrate on expanding phasing scheduling to implement
schedules that have some real-life constraints put upon them. For
example, a program may need to keep all its data in processor
caches to provide high performance. Adapting buffer sharing to
phased scheduling will also be explored, as it promises further
reduction in buffer requirements.
