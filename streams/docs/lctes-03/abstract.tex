

Applications structured around some notion of a ``stream'' are
becoming increasingly important and widespread. \cite{Rix98} provides
evidence that streaming media applications are already consuming most
of the cycles on consumer machines, and their use is continuing to
grow. The streaming computation model is pervasive and ranges from
small, embedded systems (ex. cell phones) to large, computationally
powerful machines (ex. cell base stations). In this paper, we describe
a novel technique for scheduling execution of synchronous data flow
streaming applications exhibiting hierarchical properties. A vital
aspect in compiling such programs is finding an efficient
schedule. The technique presented here focuses on producing schedules
that are optimized for the amount of space required for buffering and
storing the schedule. A wide variety of real-life applications and a
few synthetic applications are surveyed. Applications benefit from an
average 14.5\% decrease in buffer requirements, with a peak of 93\%
savings in buffer size. No application requires more space than the
most popular technique used today (Single Appearance Scheduling).
