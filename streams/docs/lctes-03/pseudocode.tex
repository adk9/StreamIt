\begin{comment}

\begin{figure}
\begin{center}

\begin{minipage}{1.2in}
\centering \psfig{figure=pipeline-steady-state.eps,width=0.6in} \\
{\protect\small (a) A sample {\pipeline}}
\end{minipage}
~
\begin{minipage}{1.2in}
\centering \psfig{figure=splitjoin-steady-state.eps,width=1.2in} \\
{\protect\small (b) A sample {\splitjoin}}
\end{minipage}
~
\begin{minipage}{1.5in}
\centering \psfig{figure=feedback-steady-state.eps,width=1.0in} \\
{\protect\small (c) A sample {\feedbackloop}.  The $L$ {\filter}
has been flipped upside-down for clarity.\\$peek_L = pop_L = 5,
push_L = 6$}
\end{minipage}

\caption{Sample {\StreamIt} streams} \label{fig:steady-state}

\end{center}
\end{figure}

\subsubsection{\filter}

Since {\filters} do not have any internal buffering, their minimal
steady state is to execute the {\filter}'s {\work} function once.
This is the smallest amount of execution a {\filter} can have.

Thus, for a {\filter} $f$,

\begin{displaymath}
S_f = \left\{[1], \{f\}, { \left[
\begin{array}{c}e_f\\o_f\\u_f
\end{array}
\right]}, [] \right\}
\end{displaymath}

Notice that $S_{f,v}$ is empty, because a {\filter} does not have
any children.

\subsubsection{\pipeline, \splitjoin and \feedbackloop}
\begin{figure}\begin{center}
\begin{minipage}{2in}
\centering \psfig{figure=splitjoin-illegal.eps,width=2in}
\end{minipage}
\end{center}
\caption{An illegal {\splitjoin}} \label{fig:splitjoin-illegal}
\end{figure}


The computation of a steady state for a \pipeline, \splitjoin or a
\feedbackloop is divided into three steps. In the first step we
compute a fractional steady state (a steady state which includes
fractional executions of internal \streams). In the second step we
compute an integral steady state from the fractional steady state.
Finally, we compute a minimal steady state from the integral
steady state.

In order to calculate a fractional steady state of a \stream, we
assign a value of 1 execution for an internal \stream, and iterate
over all the children of the \stream to calculate how many times
they need to execute to consume or produce enough data to conserve
amount of data buffered in the \stream. Here we'll present
specific equations for a \pipeline.

Let the {\pipeline} $p$ have $n$ children and let $p_i$ denote the
$i$th child of the {\pipeline} (counting from {\Input} to
{\Output}, starting with 0, the children may be streams, not
necessarily {\filters}). We must find $S_p$.

We start with calculating all $S_{p_i}, i \in \{0, \dots, n-1\}$.
This task is achieved recursively.

Next we find a fractional vector $v''$ such that executing each
$p_i$ $v_i''$ times will not change the amount of data buffered in
the {\pipeline} and the first child is executed exactly once.
%\begin{comment}
Since the children streams are executed fractional amount of
times, we calculate the amount of data they produce and consume
during this execution by multiplying $S_{p_i,c_o}$ and
$S_{p_i,c_u}$ by $v_i''$.
%\end{comment}
$v''$ must satisfy $v_0'' = 1, \forall i \in \{0,\dots,n-1\},
v_i'' * u_{p_i} = v_{i+1}'' * o_{p_{i-1}}$. Thus
%\begin{comment}
We compute $v''$ as follows.  The first child executes once, thus
$v_0'' = 1$.  The second child must execute $v_1'' = {u_{p_0}
\over {o_{p_1}}}$ times to ensure that all data pushed on the the
first {{\Channel}} is consumed by the second child.  The third
child must execute $v_2'' = v_1'' {u_{p_1} \over o_{p_2}} =
{u_{p_0} \over o_{p_1}} {u_{p_1} \over o_{p_2}}$ times to ensure
that it consumes all the data produced by the second child. Thus,
%\end{comment}
$v_i'' = {\prod_{j = 0}^{i-1} u_{p_j} \over \prod_{j=1}^i
o_{p_j}}$

Next we will find an integral vector $v'$ such that executing each
$p_i$ $v_i'$ times will not change the amount of data buffered in
the {\pipeline}.  $v'$ will be a valid steady state of the
{\pipeline}. In order to calculate $v'$ we multiply $v''$ by
$\prod_{j=1}^{n-1} o_{p_j}$.  Thus

\begin{displaymath}
v'_i = \left({\prod_{j = 0}^{i-1} u_{p_j} \over \prod_{j=1}^i
o_{p_j}} \right) \left(\prod_{j=1}^{n-1} o_{p_j} \right) = \left(
\prod_{j=0}^{i-1} u_{p_j} \right) \left( \prod_{j=i+1}^{n-1}
o_{p_j} \right)
\end{displaymath}

Now we find an integral vector $v$, such that, for some positive
integer $g$, $v' = g * v$, and $\sum_i v_i$ is minimal.  In other
words, we find the greatest integer $g$, such that $v' = g * v$,
with $v$ consisting of integers.  $v$ represents the minimal
steady state for pipeline $p$. This is achieved by finding the
$\gcd$ of all elements in $v'$, and dividing $v'$ by $g$.  Thus $v
= {v' \over \gcd(v')}$

$v$ represents the number of times each child of $p$ will need to
execute its steady state in order to execute the minimal steady
state of $p$, thus $S_{p,v} = v$.  $v$ holds a steady state
because amount of data buffered in $p$ does not change, and it is
a minimal steady state, because $\sum_i v_i$ is minimal.

We construct set $S_p$ as follows:\footnote{Here we use symbol
$\circ$ to denote concatenation of vectors and sets.  Thus $[1\ 2\
3] \circ [4\ 5\ 6] = [1\ 2\ 3\ 4\ 5\ 6]$ and $\{A\ B\ C\} \circ
\{D\ E\ F\} = \{A\ B\ C\ D\ E\ F\}$.}

\begin{displaymath}
S_p = \left\{ \begin{array}{c} v_0 * S_{p_0,m} \circ \dots \circ
v_{n-1}
* S_{p_{n-1}, m}, S_{p_0, N} \circ \dots \circ S_{P_{n-1}, N}, \\
\left[
\begin{array}{c}
e_{p_0} + (v_0 - 1) * o_{p_0} \\
v_0 * o_{p_0} \\
v_{n-1} * u_{p_{n-1}}
\end{array}\right], v \end{array} \right\}
\end{displaymath}

An example is presented in Figure \ref{fig:steady-state} (a). $A$,
$B$, $C$ and $D$ are \filters.
%\begin{comment}For this {\pipeline},
we have the following steady states for all children of the
{\pipeline}:

\begin{displaymath}
\begin{array}{lrlr}
S_A = & \left\{[1], \{A\}, { \left[
\begin{array}{c} 1 \\ 1 \\ 3
\end{array}
\right]}, [] \right\}, &

S_B = & \left\{[1], \{B\}, { \left[
\begin{array}{c} 3 \\ 2 \\ 3
\end{array}
\right]}, [] \right\} \\ \\

S_C = & \left\{[1], \{D\}, { \left[
\begin{array}{c} 2 \\ 2 \\ 1
\end{array}
\right]}, [] \right\}, &

S_D = & \left\{[1], \{D\}, { \left[
\begin{array}{c} 5 \\ 3 \\ 1
\end{array}
\right]}, [] \right\} \\

\end{array}
\end{displaymath}

Using the steady states above, we get the following vector $v'$:

\begin{displaymath}
v' = \left[
\begin{array}{c}
(2 * 2 * 3)\\
(3) (2 * 3) \\
(3 * 3) (3) \\
(3 * 3 * 1)
\end{array}
\right] = \left[
\begin{array}{c}
12\\ 18\\ 27\\ 9
\end{array}
\right]
\end{displaymath}

We now calculate $g = \gcd(v') = \gcd(12,18,27,9) = 3$.  We thus
have

\begin{displaymath}
v = {v' \over 3} = {1 \over 3} \left[
\begin{array}{c}
12\\ 18\\ 27\\ 9
\end{array}
\right] = \left[
\begin{array}{c}
4\\ 6\\ 9\\ 3
\end{array}
\right]
\end{displaymath}

Finally, we construct $S_p$:
%\end{comment}
This \pipeline has the following steady state:

\begin{displaymath}
S_p = \left\{
\begin{array}{c}
4 S_{A,m} \circ 6 S_{B,m} \circ 9 S_{C,m} \circ 3
S_{D,m}, S_{A,N} \circ S_{B,N} \circ S_{C,N} \circ S_{D,N} \\
\left[
\begin{array}{c}
1 + (4-1) * 1 \\
4 * 1 \\
3 * 1
\end{array}\right],
\left[ \begin{array}{c} 4\\ 6\\ 9\\ 3 \end{array} \right]
\end{array}
\right\}
\end{displaymath}

The steady state for \splitjoins and \feedbackloops is computed in
a very similar way. It is important to note that not all
\splitjoins and \feedbackloops have a steady state. It is possible
that one branch of a \splitjoin will produce data at a rate
disproportional to the other branch, thus causing data to
infinitely buffer up inside the \splitjoin. An example of such a
\splitjoin is presented in Figure \ref{fig:splitjoin-illegal}.
Similarly it is possible for a \feedbackloop to consume the data
from the feedback path slower or faster than it is pushed into the
feedback path, thus causing either deadlock or infinite buffering.

%\begin{comment}
\subsubsubsection{\splitjoin}

Let the {\splitjoin} have $n$ children and let $sj_i$ denote the
$i$th child of the {\splitjoin} (counting from left to right,
starting with 0).  Let $sj_s$ and $sj_j$ denote the {\splitter}
and the {\joiner} of the {\splitjoin}, respectively. Let $w_{s,i}$
denote the number of items sent by the {\splitter} to $i$th child
on {\splitter}'s every execution. Let $w_{j,i}$ denote the number
of items consumed by the {\joiner} from the $i$th child on
{\joiner}'s every execution.  We are computing $S_{sj}$.

We start by calculating all $S_{sj_i}, i \in \{0, \dots, n-1\}$.

Next we compute a fraction vector $v''$ and a fraction $a_j''$
such that executing the {\splitter} exactly once, each child
$sj_i$ $v_i''$ times and the {\joiner} $a_j''$ times does not
change the amount of data buffered in the {\splitjoin}. Again,
since $v''$ and $a_j''$ are fractions, we multiply the
steady-state pop and push amounts by appropriate fractions to
obtain the amount of data pushed and popped.  For convenience we
define $a_s''$ to be the number of executions of the {\splitter}
and set it to 1.

%\begin{comment}
\begin{displaymath}
v'', a_j'', a_s'' \ne 0, \forall i \in \{0,\dots,n-1\}, a_s'' *
w_{s, i} = v_i'' * o_{sj_i}, v_i'' * u_{sj_i} = a_j'' * w_{j, i}
\end{displaymath}
\%end{comment}

We thus have that each child $sj_i$ must execute $v_i'' = {w_{s,i}
\over o_{sj_i}}$ times. To compute the number of executions of the
{\joiner}, $a_j''$, we select an arbitrary $k$th child ($0 \le k <
n$) and have that the {\joiner} executes $a_j'' = {{w_{s,k} \over
o_{s_k}}{u_{sj_k} \over w_{j,k}}}$ times.

Next we compute integer vector $v'$ and integers $a_s$ and $a_j$
such that executing the {\splitter} $a_s$ times, each child $sj_i$
$v_i'$ times and the {\joiner} $a_j$ times still does not change
the amount of data buffered in the {\splitjoin}. We do this by
multiplying $a_s''$, $v''$ and $a_j''$ by $w_{j,k}
\left(\prod_{r=0}^{n-1}o_{sj_r}\right)$. Thus we get

\begin{displaymath}
\begin{array}{rl}
a_s' = & w_{j,k} \left(\prod_{r=0}^{n-1}o_{sj_r}\right) \\
v_i' = & w_{j,k} \left(\prod_{r=0}^{n-1}o_{sj_r}\right) * {w_{s,i}
\over o_{sj_i}} = w_{s,i} * w_{j_k} \left( \prod_{r=0}^{i-1}
o_{s_r} \right) \left( \prod_{r=i+1}^{n-1} o_{s_r} \right)
\\
a_j' = & w_{j,k} \left(\prod_{r=0}^{n-1}o_{sj_r}\right) *
{{w_{s,k} \over o_{s_k}}{u_{sj_k} \over w_{j,k}}} = w_{s,k} *
u_{sj_k} * \left( \prod_{r=0}^{k-1} o_{s_r} \right)
\left( \prod_{r=k+1}^{n-1} o_{s_r} \right) \\
\end{array}
\end{displaymath}

Now we use $v'$, $a_s'$ and $a_j'$ to compute minimal steady state
of the {\splitjoin}.  Since $v'$, $a_s'$ and $a_j'$ represent a
steady state, they represent a strict multiple of the minimal
steady state.  Thus we find the multiplier by computing $g$, the
$\gcd$ of all elements in $v'$ and integers $a_s'$ and $a_j'$, and
dividing $v'$, $a_s'$ and $a_j'$ by $g$.  We have that

\begin{displaymath}
\begin{array}{rl}
g = & \gcd(v', a_s', a_j') \\
v = & v' \over g \\
a_s = &  a_s' \over g \\
a_j = & a_j' \over g
\end{array}
\end{displaymath}

Finally, we use $v$, $a_s$ and $a_j$ to construct $S_{sj}$:

\begin{displaymath}
S_{sj} = \left\{
\begin{array}{c}
v_0 * S_{sj_0,m} \circ \dots \circ v_{n-1} * S_{sj_{n-1}, m} \circ
[a_s\ a_j] , \\
S_{sj_0, N} \circ \dots \circ S_{sj_{n-1}, N} \circ \{sj_s,
sj_j\},
\\ \left[
\begin{array}{c}
n_s * o_{s} \\
n_s * o_{s} \\
n_j * u_{j} \\
\end{array}\right], \\
v \circ [a_s] \circ [a_j]
\end{array}\right\}
\end{displaymath}

Figure \ref{fig:steady-state} (b) depicts a sample {\splitjoin}.
The following are the steady states of the {\splitjoin}'s
children: $$
\begin{array}{lrlr} S_A = & \left\{[1], \{A\}, { \left[
\begin{array}{c} 2 \\ 2 \\ 1
\end{array}
\right]}, [] \right\}, & S_B = & \left\{[1], \{B\}, { \left[
\begin{array}{c} 3 \\ 2 \\ 6
\end{array}
\right]}, [] \right\}
\end{array}
$$ For this {\splitjoin}, we select $k = 0$ (we use the left-most child
to compute $a_j'$).  We get the following $v'$, $a_s'$ and $a_j'$

\begin{displaymath}
\begin{array}{rl}
v' = & \left[
\begin{array}{c}
2 * 2 (2)\\
1 * 2 (2)
\end{array}
\right] = \left[
\begin{array}{c}
8 \\ 4
\end{array}
\right] \\
a_s' = & 1 * 2 (2 * 2) = 8 \\
a_j' = & 2 * 1 (2 * 2) = 8
\end{array}
\end{displaymath}

Thus $\gcd(u', a_s', a_j') = \gcd(8,4,8,8) = 4$.  Now we obtain

\begin{displaymath}
\begin{array}{rl}
v = & {v \over 4} = {1 \over 4} \left[
\begin{array}{c}
8 \\ 4
\end{array}
\right] =  \left[
\begin{array}{c}
2 \\ 1
\end{array}
\right]\\
a_s = & {a_s' \over 4} = {8 \over 4} = 2 \\
a_j' = & {a_j' \over 4} = {8 \over 4} = 2
\end{array}
\end{displaymath}

Finally, we construct $S_{sj}$:

\begin{displaymath}
S_{sj} = \left\{
\begin{array}{c}
2 * S_{sj_0, m} \circ 1 * S_{sj_1, m} \circ [2\ 2], \\
S_{sj_0, N} \circ S_{sj_1, N} \circ \{sj_s, sj_j\}, \\
\left[
\begin{array}{c}
2 * 3 \\ 2 * 3 \\ 2 * 4
\end{array}
\right], \left[
\begin{array}{c}
2 \\ 1 \\ 2 \\ 2
\end{array}\right]
\end{array} \right\}
\end{displaymath}

It is important to note, that it is not always possible to compute
a unique $v''$ for all possible {\splitjoins}. The reason is that
unbalanced production/consumption ratios between different
children of a {\splitjoin} can cause data to buffer up infinitely.

\begin{definition}[Valid {\splitjoin}] A {\splitjoin} is valid
{\emph iff} $\forall k, 0 \le k < n-1, a_{j,k}'' = a''_{j,k+1}$,
using notation of $a_{j,k}''$ to indicate that $k$th child of the
{\splitjoin} was used to compute the value of $a_j''$.
\end{definition}

An example of an illegal {\splitjoin} is depicted in Figure
\ref{fig:splitjoin-illegal}.  The rates of throughput of data for
the left child mean that for every execution of the {\splitter},
the {\joiner} needs to be executed exactly once to drain all data
entering the {\splitjoin}.  The rates of throughput of data for
the right child mean that for every execution of the {\splitter},
the {\joiner} needs to be executed exactly twice to drain all data
entering the {\splitjoin}. That means that consumption of data by
the {\joiner} will be relatively slower on the right side, causing
data to buffer up. This means that the given {\splitjoin} does not
have a steady state.

If a {\splitjoin} is such that it does not have a steady state, it
is considered an illegal {\splitjoin}.  It cannot be executed
repeatedly without infinite buffering, so a practical target for
{\StreamIt} cannot execute it.  The calculations presented here
assume that the {\splitjoin} is legal.  In order to check if a
given {\splitjoin} is legal, we test if selecting a different
child for calculation of $a_j''$ yields a different $a_j''$. If it
does, then the two paths tested have different
production/consumption rates, and the {\splitjoin} does not have a
steady state.

\subsubsubsection{\feedbackloop}

Let {\feedbackloop} $fl$ have children $B$ (the body child) and
$L$ (the feedback loop child). Let the {\joiner} and the
{\splitter} of the {\feedbackloop} be denoted $fl_j$ and $fl_s$.
Let $w_{j,I}$ and $w_{j,L}$ denote the number of data items
consumed by the {\joiner} from the {\Input} {{\Channel}} to the
{\feedbackloop} and from $fl_L$, respectively.  Let $w_{s,O}$ and
$w_{s,F}$ denote the number of data items pushed by the
{\splitter} onto the {\feedbackloop}'s {\Input} {{\Channel}} and
to $fl_L$ respectively.  We are computing $S_{fl}$.

First we calculate $S_{B}$ and $S_{L}$.

Now we compute a fractional vector $v'' = [a_B''\ a_L''\ a_s''\
a_j'']$ such that executing the body child $a_B''$ times, the
{\splitter} $a_s''$ times, the loop child $a_F''$ times and the
{\joiner} $a_j''$ times will not change the amount of data
buffered up in the {\feedbackloop}.  Thus

\begin{displaymath}
\begin{array}{rcl}
a_B' * u_B & = & a_s' * o_s \\
a_L' * u_B & = & a_j' * w_{j, L} \\
a_s' * w_{s, F} & = & a_L' * o_B \\
a_j' * u_j & = & a_B' * o_B \\
\end{array}
\end{displaymath}

We begin with setting $a_j'' = 1$. $B$ needs to be executed $a_B''
= u_j \over o_B$ times, the {\splitter} needs to be executed
$a_s'' = {u_j \over o_B}{u_B \over o_s}$ times and $L$ needs to be
executed $a_L'' = {u_j \over o_B}{u_B \over o_s}{w_{s,L} \over
o_L}$ times. Furthermore, in order to assure that the
{\feedbackloop} has a valid steady state, we continue going around
the loop, the {\joiner} must require ${u_j \over o_B}{u_B \over
o_s}{w_{s,L} \over o_L}{u_L \over w_{j,L}} = 1$.  If this
condition is not satisfied, the {\feedbackloop} does not have a
steady state. This is a necessary, but not a sufficient condition
for a {\feedbackloop} to be valid.

Next we compute an integer vector $v' = [a_B'\ a_L'\ a_s'\ a_j']$
such that executing B $a_B'$ times, {\splitter} $a_s'$ times, L
$a_L'$ times and {\joiner} $a_j'$ times will not change the amount
of data buffered in the {\splitjoin}. We do this by multiplying
$v''$ by $o_B * o_s * o_L$.

\begin{displaymath}
\begin{array}{rl}
a_B' = & u_j * o_s * o_L \\
a_L' = & u_j * u_B * w_{s,L} \\
a_j = & o_B * o_s * o_L \\
a_s = & u_j * u_B * o_L
\end{array}
\end{displaymath}

We now use $v'$ to compute $v = [a_B\ a_L\ a_s\ a_j]$, a minimal
steady state for the {\feedbackloop}.  We do this by finding an
integer $g$, the $\gcd$ of all elements in $v'$ and computing $v =
{v' \over g}$.

Finally, we construct $S_{fj}$ as follows:

\begin{displaymath}
S_{fj} = \left\{
\begin{array}{c}
a_B * S_{B,m} \circ a_L * S_{L,m} \circ [a_s \ a_j], \\
S_{B,N} \circ S_{L,N} \circ \{fl_s, fl_j\}, \\
\left[\begin{array}{c}
a_j * w_{j,I} \\
a_j * w_{j,I} \\
a_s * w_{s,O}
\end{array} \right], v
\end{array} \right\}
\end{displaymath}

Figure \ref{fig:steady-state}(c) depicts a sample {\feedbackloop}.
The following are the steady states of the {\splitjoin}'s
children:
$$
\begin{array}{lrlr} S_B = & \left\{[1], \{B\}, { \left[
\begin{array}{c} 2 \\ 2 \\ 1
\end{array}
\right]}, [] \right\}, & S_L = & \left\{[1], \{L\}, { \left[
\begin{array}{c} 5 \\ 5 \\ 6
\end{array}
\right]}, [] \right\}
\end{array}
$$ We compute $v'$ for this {\feedbackloop}:

\begin{displaymath}
v' = \left[
\begin{array}{c}
5 * 3 * 5 \\
5 * 1 * 3 \\
5 * 1 * 5 \\
2 * 3 * 5
\end{array}\right] = \left[
\begin{array}{c}
75 \\
15 \\
25 \\
30
\end{array}\right]
\end{displaymath}

Thus $g = \gcd(75,15,25,30) = 5$ and

\begin{displaymath}
v = {1 \over 5} \left[
\begin{array}{c}
15 \\
3 \\
5 \\
6
\end{array}\right]
\end{displaymath}

Finally, we construct $S_{fl}$

\begin{displaymath}
S_{fl} = \left\{
\begin{array}{c}
15 * S_{B, m} \circ 3 * S_{L, m} \circ [5\ 6], \\
S_{B, N} \circ S_{L, N} \circ \{fl_s, fl_j\}, \\
\left[
\begin{array}{c}
6 * 2 \\ 6 * 2 \\ 5 * 3
\end{array}
\right], \left[
\begin{array}{c}
15 \\ 3 \\ 5 \\ 6
\end{array}\right]
\end{array} \right\}
\end{displaymath}
%\end{comment}

\end{comment}
