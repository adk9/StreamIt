struct Phase {
  Stream str
  Phase[] children
}

Phase pushSchedule (Stream s) {
  result = Phase(s, {})
  child = entrance(s)
  do {
    simulate(child)
    result.children = result.children o child.currentPhase()
    child = most downstream node in s that can fire
  } loop until child==entrance(s) and exit(s) has fired at least once
  return result
}

Phase[] phasedSchedule (Stream s, int maxPhases, int mode) {
  // get maximally fine-grained phases
  phases = {}
  do {
    phases = phases o pushSchedule(s)
  } loop until {each node of s has fired all its init phases    (if mode==INIT)
                each node of s has completed its steady state   (if mode==STEADY) }
  // combine into <maxPhases> groups, with contents sorted by child stream
  size = ceil(phases.length / maxPhases)
  forall i in [0, min(phases.length, maxPhases) - 1] {
    newChildren = phases[i*size].children o ... o phases[(i+1)*size
    result[i] = Phase(s, sortByStream(newChildren))
  }
  return result
}

singleAppSchedule[s]  = <phasedSchedule(s, 1, INIT),
                         phasedSchedule(s, 1, STEADY)>

minLatencySchedule[s] = <phasedSchedule(s, \infty, INIT), 
                         phasedSchedule(s, \infty, STEADY)>

int phasesForFeedback(FeedbackLoop s, int mode) {
  phaseCount = 0
  do {
    phaseCount++
    dataAvail = # items on input tape of joiner(s)
    for i = 1 to floor(dataAvail/joiner(s).weights[1])
      simulate(joiner(s))
    while (exists c \in s, c != joiner(s) and canFire(c) )
      simulate(c)
  } loop until {each node of s has fired all its init phases    (if mode==INIT)
                each node of s has completed its steady state   (if mode==STEADY) }
  return phaseCount
}

