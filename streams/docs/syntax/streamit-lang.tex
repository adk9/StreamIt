\documentclass[11pt]{article}

\usepackage{booktabs}
%\usepackage[margin=0.5in]{geometry}
\usepackage{palatino}
\usepackage{tabularx}
\usepackage{streamit}

\def\note{\trivlist \small\item\relax\textbf{Note:}}
\def\endnote{\endtrivlist}

% Make picture sizes be sane:
\setlength{\unitlength}{\baselineskip}

% Please look the other way, this is a little gross.
\catcode`\$=12
\title{StreamIt Language Specification\thanks{
This is a pre-release version of this document.\hfil\break\ttfamily
\hbox{$Id: streamit-lang.tex,v 1.12 2002-12-06 21:20:58 dmaze Exp $}
}}
\catcode`\$=3

\begin{document}

\maketitle
\tableofcontents

\section{Introduction}

StreamIt is a language intended to simplify coding of
signal-processing and other streaming computations.  The programmer
constructs a stream graph, containing blocks with a single input and a
single output, and describes the function of atomic blocks and the
structure of composite blocks.  The compiler generates code for each
block, and applies optimizations to the stream graph to produce
efficient code for the target architecture.

The current implementation of the StreamIt compiler translates the
syntax described in this document to Java code, which can then be
either run against a Java library or compiled to C code and linked
with a runtime library.  Since the compiler can reconstruct the stream
graph, it can combine adjacent filters, or split computationally
intensive filters into multiple parts, or duplicate filters to have
more parallel computation.

\begin{note}
Java library support will likely soon be deprecated, as the StreamIt
language becomes more expressive than what can be described in Java.
\end{note}

The StreamIt language is vaguely reminiscent of other imperative
languages such as C or Java.  In particular, the bodies of filter work
functions and stream initialization code is generally legal Java
code.

\section{Types}

This section describes the various data and object types used in
StreamIt programs.  Data types are passed along tapes between stream
objects, and can be used as local variables.  Stream objects are only
created in initialization code and form a static stream graph.

\subsection{Data Types}

Data types are always created atomically.  They are of fixed size, and
are generally statically allocated.

\subsubsection{Primitive Types}

The following primitive types exist in StreamIt:

\begin{description}
\item[\lstinline|bit|]  A one-bit unsigned integer type, containing
  the value \lstinline|0| or \lstinline|1|.

\item[\lstinline|int|]  A signed integer type, of unspecified length.
  This length will typically be the native word length on the target
  machine; for RAW and x86 compilation, this is usually 32 bits.

\item[\lstinline|float|]  A floating-point type, of the best precision
  that will give good performance.  On x86 this is a double-precision
  float, since computation with single-precision floats is implicitly
  converted to double-precision; on RAW, this is single-precision,
  since that is all that is supported in hardware.

\item[\lstinline|complex|]  A floating-point complex type, of the same
  precision as \lstinline|float|.  This has real and imaginary parts,
  which can be directly accessed with \lstinline|c.real| and
  \lstinline|c.imag|.
\end{description}

\label{sec:operators}
A number of operations are supported on these primitive types; these
are listed in Table \ref{tab:primitive-operators}.  In a binary or
ternary expression, if two expressions are of different primitive
types, they are promoted to the lowest type on the list above.  Real
expressions converted to \lstinline|complex| have an imaginary part of
zero.  \lstinline/|/, \lstinline|&|, and \lstinline|^| are bitwise or,
and, and exclusive-or operators, respectively.  The boolean logic
operators \lstinline|&&| and \lstinline/||/ behave as in C: they are
short-circuiting, integer values of exactly zero are false, and other
values are true.  \lstinline|==| and other comparison operators return
an integer 1 if the comparison is true and an integer 0 if the
comparison is false.

\begin{table}
\begin{center}
\begin{tabular}{ccl}
\toprule
\textbf{Operators} & \textbf{Types} \\
\midrule
\lstinline|?:| & Any & First part must be \lstinline|int| or \lstinline|bit| \\
\lstinline/||/ & \lstinline|int|, \lstinline|bit| \\
\lstinline|&&| & \lstinline|int|, \lstinline|bit| \\
\lstinline/|/, \lstinline|&|, \lstinline|^| &
  \lstinline|int|, \lstinline|bit| \\
\lstinline|==|, \lstinline|!=| & Any \\
\lstinline|<|, \lstinline|<=|, \lstinline|>|, \lstinline|>=| &
  \lstinline|int|, \lstinline|float| \\
\lstinline|+|, \lstinline|-| &
  \lstinline|int|, \lstinline|float|, \lstinline|complex| \\
\lstinline|*|, \lstinline|/|, \lstinline|%| &
  \lstinline|int|, \lstinline|float|, \lstinline|complex| &
  \lstinline|%| must be \lstinline|int| \\
\lstinline|++|, \lstinline|--| & \lstinline|int| \\
\bottomrule
\end{tabular}
\end{center}
\caption{Operators on primitive types}
\label{tab:primitive-operators}
\end{table}

\subsubsection{Structures}
\label{sec:data-structures}

Named structures of heterogeneous data types are supported.  A
structure must be of a fixed size.  It contains a set of fields, each
of which has a name and a data type.  Any primitive or array type, and
any structure type which has been previously declared in the program,
may be used; recursive structures are not allowed.  The following two
structure definitions are both legal:

\begin{lstlisting}{}
struct A {
  int a;
  int[4] b;
}

struct B {
  A a;
  A[4] b;
}
\end{lstlisting}

A structure definition contains the keyword \lstinline|struct|, the name
of the structure, an open brace, a listing of field (variable)
declarations, and a close brace.  It is not followed by a semicolon.

The only supported operations on structures are field references and
copying.  Field references are of the form \lstinline|a.b|; \lstinline|a| must
be of some structure type \lstinline|A|, and \lstinline|b| must be a field named
in the structure declaration of \lstinline|A|.  The type of this expression
is the declared type of the field.  Note that a similar syntax is used
for referencing the real and imaginary parts of complex numbers; there
is an implicit structure declaration

\begin{lstlisting}{}
struct complex {
  float real;
  float imag;
}
\end{lstlisting}{}

\noindent
along with some syntactic sugar to make complex arithmetic work.

\subsubsection{Arrays}

Arrays of any data type described here are supported.  Arrays must be
of a fixed length.  Array types are written with the base type name,
followed by a single dimension in square brackets.  Multi-dimensional
arrays or matrices are supported as arrays of arrays.

\subsubsection{Variable Declarations}

Variable declarations appear in structure declarations, as well as in
code blocks and \lstinline|filter| declarations.  A variable declaration
always has the form \lstinline|type name;|, where \lstinline|type| is a data
type as described here.  Of note, the syntax for array declarations is
different from C and Java.  A correct declaration is
\lstinline|int[4] rgbi;|, with the entire type stated before the variable name.


\subsection{Stream Types}

Computation in StreamIt is performed within stream objects.  Every
stream object has an input type and an output data type; the object is
connected to ``tapes'' of a hypothetically infinite number of
homogeneous data objects.  \emph{Filters} are atomic, and have
initialization code and steady-state work code; \emph{pipelines},
\emph{split-joins}, and \emph{feedback loops} are all composite
structures that include some number of stream objects as children.

All stream objects have initialization code that runs when the object
is first created.  For filters, this code may be omitted, and
generally just sets up filter-local state.  For composite objects, the
initialization code is responsible for creating the child objects.
Filters also have one or more \emph{work functions}, which are called
each time the filter executes.

\paragraph{I/O types.}  Each stream declared at the top level must
also explicitly declare its input and output types.  This is a
declaration like \lstinline|int->int| or \lstinline|float->complex|; the first
type is the stream's input type, the second its output type.  Either
type may be \lstinline|void|; in this case, the stream has no input or
output, as appropriate.  If it is a filter, it must not declare a
peek/pop or push rate.  The top-level stream in a program has type
\lstinline|void->void|; no other streams of this type are allowed.

\subsubsection{Stream Parameters}

A stream may have a set of parameters provided to it.  These
parameters are available to all functions within the stream type, and
may not be changed.  Values for these parameters must be passed in
when the stream is created.  Stream parameters are declared with types
and names in a parenthesized list after the name of the stream; when a
stream it instantiated, values for the parameters must follow the name
of the stream.  For example:

\begin{lstlisting}{}
float->float filter MatrixMatrixMultiply(int A,
                 int B, int C) { ... }
add MatrixMatrixMultiply(3, 5, 4);
\end{lstlisting}

\paragraph{Compile-time constants.}  Several things in StreamIt must
be \emph{compile-time constant}; for example, the I/O rates of a
filter must be determined at compile-time, and the compiler must be
able to statically create the expanded stream graph.  An expression is
compile-time constant if it is a literal, a stream parameter, or an
expression whose components are entirely compile-time constant.

\subsubsection{Parameterized Stream Types}

Stream types may be \emph{parameterized} on some data type or types,
much like ``template'' types in C++.  This may be used for
data-reordering operations, and is used for several cases of built-in
filters.

Parameterized objects cannot currently be created by the user.  To
instantiate a built-in parameterized type, use the name of the stream
type, followed by the name of the data type in angle brackets, such as
\lstinline|Identity<int>|.

\begin{note}
User-instantiated parameterized types are a future StreamIt
extension.
\end{note}

\subsubsection{Filters}

All computation in StreamIt takes place within filters.  Filters must
explicitly declare their initialization code if present, but also have
the option of omitting it.  A filter may have a single anonymous work
function, or multiple named work functions.  Each work function must
declare its \emph{I/O rates}, the number of items one call to the
function removes, examines, or places on its tape.  A basic filter
declaration looks like this:

\begin{lstlisting}{}
int->int filter IntAvgFilter {
  init {
    // empty
  }
  work pop 1 peek 2 push 1 {
    push(peek(0) + peek(1));
    pop();
  }
}
\end{lstlisting}

This filter is named \lstinline|IntAvgFilter|; it reads integers off of its
input tape, and writes integers to its output tape.  Each execution of
the work function removes exactly one item from the input tape and
examines at most two items; it causes exactly one item to be pushed on
to the output tape.

\paragraph{Init functions.}  Init functions are declared using the
keyword \lstinline|init|.  They take no parameters.  They may initialize
filter fields or do other work, but have no access to the filter's
tapes.

\paragraph{Work functions.}  Work functions are declared using the
keyword \lstinline|work|.  This keyword may be followed by rate
declarations; if these are absent, the filter is a \emph{phased
  filter}.  Rate declarations consist of \lstinline|push expr|,
\lstinline|peek expr|, and \lstinline|pop expr|, in any order.  If the
peek rate is omitted, it is assumed to be equal to the pop rate; if
the pop or push rate is omitted, it is assumed to be zero, and the
corresponding input or output tape type must be \lstinline|void|.  The
function must push and pop exactly the number of items specified, and
may peek up to the number of items specified.  All of the rate
expressions must be compile-time constant.

\paragraph{Phased filters.}  If the work function does not have rate
declarations, it may not perform \lstinline|push| and \lstinline|pop|
operations directly.  However, it may call \emph{phase functions},
which are declared like work functions but begin with the keyword
\lstinline|phase| and a name, and always have rate declarations.  The
runtime system ensures that, when a phase function is called, its
\lstinline|peek| and \lstinline|pop| rates are satisfiable, even if
other filters need to run for later phases to execute.  Beyond this
special use, phase functions may not be directly called.  For example:

\begin{lstlisting}{}
int->int filter BlockReader {
  // Each block has a 4-int header, 24 ints of data,
  // and a 4-int footer.
  work {
    readHeader();
    for (int i = 0; i < 24; i++)
      echoData();
    readFooter();
  }
  phase readHeader pop 4
     { pop(); pop(); pop(); pop(); }
  phase echoData pop 1 push 1 { push(pop()); }
  phase readFooter pop 4
     { pop(); pop(); pop(); pop(); }
}
\end{lstlisting}

\paragraph{Helper functions.}  A helper function performs some
auxiliary bit of computation, and may be called from the init function
or the work function.  A helper function has a list of zero or more
parameters, and returns zero or one values.  A helper function
declaration contains the return type, the name of the function, the
parameter list, and the function body, just like a normal C or Java
function.

\paragraph{Filter state.}  The top level of the filter may include
variable declarations.  These variables are visible to every function
in the filter, and may be used to carry state between iterations of
the work function.


\subsubsection{Pipelines}

The simplest composite filter is a pipeline.  A pipeline contains a
number of child streams; the output of the first stream is connected
to the input of the second, whose output is connected to the input of
the third, and so on.  A pipeline declaration looks like

\begin{lstlisting}{}
float->complex pipeline FloatToComplexAdd {
  add ReImToComplex();
  add ComplexPairAdd();
}
\end{lstlisting}

The body of the pipeline declaration is the initialization code; no
internal function declarations or message handlers are allowed.  In
addition to the internal typing constraints mentioned previously, the
input type of the pipeline must match the input type of the first
filter, and the output type of the pipeline must match the output type
of the last filter.

\begin{figure}[htbp]
  \begin{center}
    \begin{picture}(18,9)
      \put(3,9){\vector(0,-1){1}}
      \put(0,1){\framebox(6,7)[lt]{\lstinline|Pipeline|}}
      \put(7,8){\makebox(9,1)[l]{\lstinline|float->float|}}
      \put(7,7){\makebox(9,1)[l]{\lstinline|pipeline Pipeline \{|}}
      \put(3,8){\vector(0,-1){1}}
      \put(1,6){\framebox(4,1){\lstinline|Child1|}}
      \put(7,6){\makebox(9,1)[l]{\lstinline|\ add Child1();|}}
      \put(3,6){\vector(0,-1){1}}
      \put(1,4){\framebox(4,1){\lstinline|Child2|}}
      \put(7,4){\makebox(9,1)[l]{\lstinline|\ add Child2();|}}
      \put(3,4){\vector(0,-1){1}}
      \put(1,2){\makebox(4,1){$\cdots$}}
      \put(7,2){\makebox(9,1)[l]{\lstinline|\ ...|}}
      \put(3,2){\vector(0,-1){1}}
      \put(7,1){\makebox(9,1)[l]{\lstinline|\}|}}
      \put(3,1){\vector(0,-1){1}}
    \end{picture}
    \caption{Pipelines}
    \label{fig:pipeline}
  \end{center}
\end{figure}


\subsubsection{Split-Joins}

A split-join allows computation to be run in parallel, possibly using
different parts of the input stream.  It is so named because incoming
data passes through a \emph{splitter}, is redistributed to the child
streams for processing, and then is fed through a \emph{joiner} to be
recombined into a single output stream; see section
\ref{sec:expr-splitters-joiners}.

\begin{lstlisting}{}
int->int splitjoin AddAndSub {
  split duplicate;
  add IntAdder();
  add IntSubtractor();
  join roundrobin;
}
\end{lstlisting}

This split-join has two children.  Incoming data is duplicated on to
both streams.  The first branch consumes two elements and adds them;
the second consumes two elements and subtracts them.  The resulting
values are then pushed on to the output stream in a round-robin
fashion, with values from alternating streams.

\begin{figure}[htbp]
  \begin{center}
    \begin{picture}(29,5)
      \put(6,5){\vector(0,-1){1}}
      \put(0,1){\framebox(12,3)[tl]{\lstinline|SplitJoin|}}
      \put(13,4){\makebox(16,1)[l]
        {\lstinline|int->int splitjoin SplitJoin \{|}}
      \put(6,4){\vector(-3,-1){3}}
      \put(6,4){\vector(0,-1){1}}
      \put(6,4){\vector(3,-1){3}}
      \put(13,3){\makebox(16,1)[l]{\lstinline|\ split roundrobin(2);|}}
      \put(1,2){\framebox(4,1){\lstinline|Child1|}}
      \put(5,2){\makebox(2,1){$\cdots$}}
      \put(7,2){\framebox(4,1){\lstinline|ChildN|}}
      \put(13,2){\makebox(16,1)[l]
        {\lstinline|\ add Child1(); ... add ChildN();|}}
      \put(3,2){\vector(3,-1){3}}
      \put(6,2){\vector(0,-1){1}}
      \put(9,2){\vector(-3,-1){3}}
      \put(13,1){\makebox(16,1)[l]{\lstinline|\ join roundrobin;|}}
      \put(6,1){\vector(0,-1){1}}
      \put(13,0){\makebox(16,1)[l]{\lstinline|\}|}}
    \end{picture}
    \caption{Split-Joins}
    \label{fig:splitjoin}
  \end{center}
\end{figure}

\subsubsection{Feedback Loops}

A feedback loop has a body stream.  Its output passes through a
splitter; one branch of the splitter leaves the loop, and the other
goes to the loop stream.  The output of the loop stream and the loop
input then go through a joiner to the body's input.

\begin{lstlisting}{}
float->float feedbackloop AddFeedback(float scaling) {
  join roundrobin;
  body FloatAdder();
  loop FloatScaler(scaling);
  split duplicate;
  enqueue 0.0;
}
\end{lstlisting}

\paragraph{Semantics of splitters and joiners.}  Splitters and joiners
both treat the stream input as the first child, and the loop edge as
the second child.  Splitters and joiners that do not allow exactly two
children are disallowed.

\paragraph{\lstinline|enqueue| statement.}  The \lstinline|enqueue|
statement takes a value and places it in FIFO order on the output tape
from the loop stream.  You will generally need to enqueue enough items
that the input joiner will be able to fire once; enqueuing more items
causes a delay in the feedback loop.

\begin{figure}[htbp]
  \begin{center}
    \begin{picture}(21,9)
      \put(6,9){\vector(0,-1){1}}
      \put(10,8){\makebox(11,1)[l]{\lstinline|float->float|}}
      \put(0,1){\framebox(9,7)[lt]{\lstinline|Feedback|}}
      \put(10,7){\makebox(11,1)[l]{\lstinline|feedbackloop Feedback \{|}}
      \put(6,8){\vector(0,-1){1}}
      \put(10,6){\makebox(11,1)[l]{\lstinline|\ join roundrobin;|}}
      \put(6,7){\vector(0,-1){1}}
      \put(4,5){\framebox(4,1){\lstinline|Body|}}
      \put(10,5){\makebox(11,1)[l]{\lstinline|\ body Body();|}}
      \put(6,5){\vector(0,-1){3}}
      \put(6,2){\line(-1,0){3}}
      \put(3,2){\vector(0,1){1}}
      \put(1,3){\framebox(4,1){\lstinline|Loop|}}
      \put(10,3){\makebox(11,1)[l]{\lstinline|\ loop Loop();|}}
      \put(3,4){\line(0,1){3}}
      \put(3,7){\vector(1,0){3}}
      \put(6,2){\vector(0,-1){1}}
      \put(10,1){\makebox(11,1)[l]{\lstinline|\ split roundrobin(4,1);|}}
      \put(6,1){\vector(0,-1){1}}
      \put(10,0){\makebox(11,1)[l]{\lstinline|\}|}}
    \end{picture}
    \caption{Feedback Loop}
    \label{fig:feedback-loop}
  \end{center}
\end{figure}

\subsubsection{Anonymous Streams}
\label{sec:anonymous-streams}

It is possible to declare a stream object without a name, if it is
used in exactly one place in the program.  In general anonymous stream
declarations look exactly like the corresponding normal stream
declaration, except that the stream type is optional and the name is
omitted.  The semicolon that normally ends the \lstinline|add|,
\lstinline|body|, or \lstinline|loop| statement is optional in this
case.

\begin{lstlisting}{}
complex->float feedbackloop FeedbackMagnitude
    (float scaling) {
  join roundrobin;
  body complex->float filter {
    work pop 2 push 1 {
      push(abs(peek(0) * peek(1)));
      pop(); pop();
    }
  };
  loop pipeline {
    add FloatScaler(scaling);
    add FloatToComplex();
  };
  split duplicate;
}
\end{lstlisting}

\begin{note}
We may add the constraint that anonymous filters consist only of work
code.  In this case, the syntax might look like
\lstinline|body filter pop 2 push 1 { push(...) };|
\end{note}

The input and output types of anonymous streams are determined by the
compiler.

\section{Statements}

This section describes statements legal in StreamIt code.  For the
most part, these are identical to legal statements in C or Java.  Some
statements, particularly those that set up the stream graph, are only
legal in initialization code; other statements are legal anywhere.

\subsection{Initialization-Only Statements}

\subsubsection{\lstinline|add| Statement}

The \lstinline|add| statement adds a child stream to the current stream
object.  It is valid in split-joins and pipelines only.  In a
split-join, it adds a new stream to the end of the current list of
children.  In a pipeline, it adds a new stream to the end of the chain
of children.

This statement takes a \emph{stream constructor}; see section
\ref{sec:expr-stream-constructor}.

\subsubsection{\lstinline|body| Statement}

The \lstinline|body| statement adds a child stream as the body part of
a feedback loop.  This statement must be executed exactly once in the
initialization code of a feedback loop, and is invalid anywhere else.

This statement takes a \emph{stream constructor}; see section
\ref{sec:expr-stream-constructor}.

\subsubsection{\lstinline|enqueue| Statement}

The \lstinline|enqueue| statement is valid only within the
initialization code of a feedback loop.  It causes the specified item
to be placed after all other previously enqueued items on the input to
the feedback loop joiner.  This statement takes a single value
expression.

\subsubsection{\lstinline|join| Statement}

The \lstinline|join| statement declares the joiner type of a feedback
loop or split-join.  It only appears in initialization code, and is
invalid in initialization code for any other construct.  This
statement takes a joiner specification; the only one valid currently
is \lstinline|roundrobin|, see section \ref{sec:expr-round-robin}.

\subsubsection{\lstinline|loop| Statement}

The \lstinline|loop| statement adds a child stream as the loop part of
a feedback loop.  This statement must be executed exactly once in the
initialization code of a feedback loop, and is invalid anywhere else.

This statement takes a \emph{stream constructor}; see section
\ref{sec:expr-stream-constructor}.

\subsubsection{\lstinline|split| Statement}

The \lstinline|split| statement declares the splitter type of a feedback
loop or split-join.  It only appears in initialization code, and is
invalid in initialization code for any other construct.  This
statement takes a splitter specification; only \lstinline|duplicate| and
\lstinline|roundrobin| are currently legal, see sections
\ref{sec:expr-duplicate} and \ref{sec:expr-round-robin}.


\subsection{Work-only Statements}

These statements may only appear within steady-state work code, and in
particular, within a filter's \lstinline|work| and \lstinline|phase|
function(s).

\subsubsection{\lstinline|push| Statement}

The \lstinline|push| statement takes a value expression and pushes its
value on to the output tape of the current filter.  The type of the
expression must match the output type of the filter.

\subsection{General-use Statements}

These statements may appear anywhere code is legal.

\subsubsection{Statements with Java-like Semantics}

\lstinline|do|, \lstinline|for|, and \lstinline|while| loops have the same
syntax and semantics they do in Java.  The \lstinline|continue| and
\lstinline|break| statements are also recognized, though they may not
take  a label reference.

For statements whose syntax match exactly

\begin{lstlisting}{}
for (i = S; i < N; i++)
  /* body */ ;
\end{lstlisting}

\noindent
cause the induction variable \lstinline|i| to be compile-time constant
within the loop if the start and end indices \lstinline|S| and \lstinline|N| are
also compile-time constant.

\subsubsection{Assignment Statements}

Assignment statements have the form \lstinline|lhs = expr;|.  The left-hand
side may be a variable, a field reference, or an array element
reference.  Each of these is evaluated to a location, with the result
being either a primitive value, a structure, or an entire array.  If
it is a primitive value, the right-hand side must be a value
expression.  Otherwise, the right-hand side must be an array or
structure of identical type, and causes all of the elements from the
right-hand-side object to be copied to the left-hand-side object.

\subsubsection{Variable Declarations}

Variable declarations have the form \lstinline|type name;|.  \lstinline|type|
must be a data type; \lstinline|name| may be any legal name not
corresponding to another variable previously declared in the current
block.  A declaration must appear before the variable is used.  The
variable declaration may also contain an initialization, as in
\lstinline|type name = expr;|  The initialization acts exactly as an
initialization statement, above.

Variables declared without initialization are initialized to 0 as best
as possible.  \lstinline|complex| variables are initialized to
\lstinline|0+0i|.  Each element of an array variable is initialized to
0 as described here.  Similarly, each field of a structure variable is
initialized to 0 in the same way.


\section{Expressions}

\subsection{Value Expressions}
\label{sec:expr-value}

A \emph{value expression} carries some value of a data type.  Literal
values are value expressions; these can be signed integer or real
values, e.g. \lstinline|17|, \lstinline|2.45|, or \lstinline|1.4e6|.  A literal value
can also be a pure imaginary number, e.g. \lstinline|17i| or \lstinline|3.4i|.
Unary, binary, and ternary expressions of value expressions are also
value expressions, as described in section \ref{sec:operators}.  Thus,
while \lstinline|3+4i| is not a literal, it is a valid value expression of
complex type.

Field references to structure objects are also value expressions.
These have the form \lstinline|a.b|, where \lstinline|a| is a variable of
structured type \lstinline|A| and \lstinline|b| is a field in \lstinline|A|.
Similarly, array references \lstinline|a[b]| are value expressions;
\lstinline|a| must be an array variable, and \lstinline|b| must be an
integer-valued value expression.

\subsubsection{\lstinline|peek| and \lstinline|pop|}
\label{sec:expr-peek-pop}

The \lstinline|peek| and \lstinline|pop| expressions look at items on
the incoming tape in a filter's work function; they are illegal in any
other context.  \lstinline|peek(n)| examines the n-th item on the tape
without removing it; \lstinline|peek(0)| returns the next incoming
item, \lstinline|peek(1)| the one afterwards, and so on.
\lstinline|pop()| returns the same item \lstinline|peek(0)| returns,
but also removes it from the tape.

\begin{note}
There is not a guaranteed evaluation order for StreamIt expressions,
and the implementation of \lstinline|pop()| in C doesn't give any
guarantees for when the item is actually removed from the tape.  Thus,
an expression should contain no more than one \lstinline|pop()|, and
if there is one, there should be no \lstinline|peek()|s.
\end{note}

Note that \lstinline|pop()| removes an item from the tape, so
\lstinline|peek()|s afterwards use different indices.  A work function
must pop a fixed number of items, and declare the number of items in
its rate declaration.  It may not peek at more items than it declares
in its rate declaration, though it may peek at fewer.  The peek
declaration declares the furthest position on the tape examined from
the point of view of the beginning of the function.

\begin{lstlisting}{}
int->int filter PeekRates {
  work pop 1 peek 3 push 1 {
    int a, b, c, d, e;
    a = peek(0);
    b = peek(1);
    c = pop();   // Same value as a
    d = peek(0); // Same value as b
    e = peek(1);
    push(a + b + c + d + e);
  }
}
\end{lstlisting}

This function pops one item and pushes 1.  It declares a peek rate of
3 because \lstinline|e| peeks at the third item that was on the tape
at the start of the function -- one is popped, and \lstinline|e|
contains the second item beyond that.

\subsubsection{Function Calls}
\label{sec:expr-funcall}

Function calls are value expressions; certain cases of built-in
functions may return values that are compile-time constant if their
parameters are compile-time constant.  The function call syntax
\lstinline|fn(p1, p2)| is identical to that of Java.  All parameters are
always passed by value; this differs from Java's semantics, in which
objects and arrays are passed by reference.

Function calls may be made to declared helper functions, or to
built-in math functions.  The math functions are listed in table
\ref{tab:math-functions}.

\begin{note}
  As an implementation detail, any function name not optimized as a
  built-in function in the front-end or recognized as a helper
  function is treated as a call to a function in the Java package
  \lstinline|java.lang.Math|; see \emph{Java in a Nutshell} or another
  Java reference for details.  These get blindly translated again to
  calls to the C math library.
\end{note}

\begin{note}
  It is conceivable that calls to side-effect-free functions with
  compile-time constant parameters return compile-time constant
  results.  This is not the case for non-built-in functions in the
  current implementation of the compiler; adding this involves
  inlining helper functions, and doing analysis to determine if a
  helper is in fact side-effect free.
\end{note}

\begin{table}[htbp]
  \begin{center}
    \begin{tabular}{cl}
      \toprule
      \texttt{abs(v)} & Real absolute value or complex magnitude \\
      \texttt{arg(v)} & Polar angle of complex value \\
      \texttt{exp(v)} & Real or complex exponent \\
      \texttt{log(v)} & Real or complex natural logarithm \\
      \texttt{sin(v)} & Real or complex sine \\
      \texttt{cos(v)} & Real or complex cosine \\
      \texttt{sqrt(v)} & Real or complex square root \\
      \texttt{csqrt(v)} & Complex square root \\
      \bottomrule
    \end{tabular}
    \caption{StreamIt Math Functions}
    \label{tab:math-functions}
  \end{center}
\end{table}

\subsection{Stream Constructors}
\label{sec:expr-stream-constructor}

A \emph{stream constructor} indicates that a new stream object should
be created.  Stream constructors appear in \lstinline|add|,
\lstinline|body|, and \lstinline|loop| statements in composite stream
initialization code.

Stream constructors may reference an already-declared stream type.  In
this case, the stream constructor form is \lstinline|Name(param, ...)|.
There must be exactly as many parameters in the constructor as
parameters for the stream type.  The parameters must all be
compile-time constant.

An anonymous stream declaration is also a valid stream constructor;
see section \ref{sec:anonymous-streams}.  A statement containing an
anonymous stream declaration does not need to be terminated with a
semicolon, since it ends with a close-brace.

\subsection{Splitters and Joiners}
\label{sec:expr-splitters-joiners}

Splitters have a single input of a given type and multiple outputs of
the same type; joiners have multiple inputs and a single output.  They
appear in feedback loops and split-joins.

\begin{note}
  Other types of splitters and joiners besides duplicate and
  round-robin are conceivable; in particular, structure and array
  packing and unpacking splitters and joiners are obvious.  There also
  may be demand for ``custom'' splitters and joiners, where a
  data-reordering operation is described by the programmer.
\end{note}

\subsubsection{\lstinline|duplicate| Splitters}
\label{sec:expr-duplicate}

\lstinline|duplicate| splitters take each incoming item and push the
same item to each of the outgoing tapes, duplicating data.

\subsubsection{\lstinline|roundrobin| Splitters and Joiners}
\label{sec:expr-round-robin}

\lstinline|roundrobin| splitters take each item and send it to exactly
one of the child streams, in order.  Either \lstinline|roundrobin| or
\lstinline|roundrobin()| causes one item to be sent to each output, in
order; \lstinline|roundrobin(2)| causes two items to be sent to the first
stream, two to the second, and so on.  \lstinline|roundrobin(2, 4, 2)|
requires there to be exactly three children, and sends two to the
first child, four to the second, and two to the third.  A round-robin
weight may be 0; in this case, the child input must be of type
\lstinline|void|.  A \lstinline|roundrobin(0)| splitter on a
split-join means that no children take any input, and the input type
of the entire split-join is \lstinline|void|.

\lstinline|roundrobin| joiners are identical to \lstinline|roundrobin|
splitters, except that they read from the input tapes in the specified
pattern and write data to the output tape.


\section{Built-in Types}

This section describes built in data and stream types in StreamIt.
These types are implicitly declared; code can just use these types
without explicitly importing or otherwise declaring them.

\subsection{Built-in Data Types}

The \lstinline|complex| type is actually implemented as a structure,
though the compiler implements syntactic sugar for complex variables.
The structure definition is shown above in section
\ref{sec:data-structures}.
It is legal to explicitly reference the \lstinline|real| and
\lstinline|imag| fields of complex numbers as though they were the
structure type instead.

\subsection{Built-in Streams}

Several stream types are in common use, and are either simple enough
or require non-standard language functionality that the compiler
should support them directly.  These include parameterized
\lstinline|Identity|, \lstinline|FileReader|, and \lstinline|FileWriter| types, as
shown in figure \ref{fig:built-in-streams}.

\begin{figure}[htbp]
    \begin{lstlisting}{}
template<T> T->T filter Identity {
  work pop 1 push 1 { push(pop()); }
}
template<T> void->T filter FileReader
    (string filename) {
  work push 1 { /* Implementation defined */ }
}
template<T> T->void filter FileWriter
    (string filename) {
  work pop 1 { /* Implementation defined */ }
}
\end{lstlisting}
    \caption{Built-in Stream Types}
    \label{fig:built-in-streams}
\end{figure}

\section{Future Changes}

This section describes features that may appear in future versions of
StreamIt.

\subsection{Additional Primitive Types}

Support for characters and strings is somewhat uncertain.  There is no
particular use for a character primitive type in StreamIt at the
moment, though a byte type may be useful for reading in files.
Strings may be useful for debugging code, and for file name parameters
to file reader and writer objects.

\subsection{Parameterized Types}

StreamIt should allow parameterized types, using a syntax similar to
C++'s \lstinline|template| construct.  For example, one might write:

\begin{lstlisting}{}
template<T> T->T filter Duplicate {
  work pop 1 push 2 {
    T val = pop();
    push(val);
    push(val);
  }
}
\end{lstlisting}

Then \lstinline|Duplicate<int>| would be a stream type that duplicates
integers.

There is some disagreement on how to implement array types.  The
language currently accepts \lstinline|int[N]->int[N] filter Name(int N)| to
have a specified-length array.  It may be better to use a template
construct for this, though then value parameters can appear in two
different places in the code.

\subsection{Helper Packages}

It is entirely conceivable that the programmer might want to write a
set of stateless helper functions that can be used in multiple
filters.  For example, the GSM encoder we have now uses the C
preprocessor to include a set of functions to perform 16-bit
arithmetic.  Syntax for this functionality does not yet exist, though
we may create a \lstinline|static| keyword:

\begin{lstlisting}{}
static MyHelpers {
  int square(int x) { return x * x; }
}
int->int filter SquareIt {
  work pop 1 push 1 { push(MyHelpers.square(pop())); }
}
\end{lstlisting}

\subsection{Messaging}

The early StreamIt papers mentioned the ability to send messages from
one filter to another stream object, and defines semantics for this.
The new StreamIt syntax does not have support for messaging.  However,
there are fundamentals that we clearly will want:

\begin{itemize}

\item \textbf{Portals.}  The template type syntax can be readily used
  to implement portal types.  We'll need to add new syntax to register
  objects with portals and to send messages.

\item \textbf{Message handlers.}  These can use a \lstinline|handler|
  keyword and have a syntax similar to work function declarations.
  The most recent agreement is that message handlers can only be
  installed in filters, avoiding a syntax problem with compound stream
  initialization code.

\item \textbf{Interfaces.}  It is possible multiple stream types may
  implement the same set of functions.  An \lstinline|interface|
  keyword can declare a message interface.  No objects of interface
  types exist, but portals to these types can be created, and any
  object that has message handlers matching the signatures in the
  interface declaration can be added to the portal.

\end{itemize}

This might result in message-handling code that looks like this:

\begin{lstlisting}{}
float->float filter VolumeControl {
  float amplification;
  init { amplification = 1.0; }
  work pop 1 push 1 { push(pop() * amplification); }
  handler setAmplification(float val)
    { amplification = val; }
}

float->float filter VolDetect
    (Portal<VolumeControl> vol) {
  work pop 1 push 1 {
    float val = pop();
    if (/* detected */)
      send vol setAmplification(0.5);
    push(val);
  }
}

float->float pipeline Volume {
  Portal<VolumeControl> vol;
  add VolDetect(vol);
  add_register vol VolumeControl();
}
\end{lstlisting}

\subsection{Reinitialization}

We might occasionally want to change parts of the stream graph at
runtime, or to change the static parameters of the program.  This is a
major effort the way StreamIt code is currently constructed.  The CC
paper proposed sending an \lstinline|init| message through a portal to
perform reinitialization; this would result in the targets'
\lstinline|init| functions being re-run.

This is simple enough for filters, though some state in the filter may
be lost that reinitialization would want to be preserved.  What
happens for composite streams, though?  Are their children destroyed
and reborn, or are individual children added and removed?  What syntax
is necessary for this?

\subsection{Variable Rates}

Certain types of applications, such as those involving compression
algorithms, do not have a fixed ratio of outputs to inputs.  StreamIt
does not currently support variable rates.  The current proposal is
that variable rates will be accomodated by phased filters: one phase
reads the input, and then runs another phase for a certain number of
executions that writes the output, with the number of executions being
variable.

It is still possible that dedicated syntax for variable rates will be
added.  This will likely accomodate bounds on the I/O rates of the
filter, since a filter might read four bytes of input and produce
between two and sixteen bytes of output; having this information can
assist the scheduler.

\subsection{Static Control Flow and Dynamic Phases}

This document should contain a discussion of the notion of
\emph{static control flow}.  This is essentially a requirement that
the path through a given control flow graph is determinable at compile
time; loops can be fully unrolled, and every branch can be
determined.  This extends the concept of compile-time constancy
already covered.

This would allow us to refine the current description of phases.  In
particular, we want the current phase specification to be entirely
specified by static control.  As a future extension, the control flow
can be dynamic, to allow data-dependent phase execution.  This can be
used to implement some forms of dynamic-rate filters.  For example:

\begin{lstlisting}{}
// Format: #reps, #bytes, data
byte->byte filter RLEExpand {
  byte data[256];
  int reps, count;
  work {
    readReps();
    readCount();
    for (int i = 0; i < count; i++)
      readData(i);
    for (int i = 0; i < reps; i++)
      writeData(count);
  }
  phase readReps pop 1 { reps = pop(); }
  phase readCount pop 1 { count = pop(); }
  phase readData(int pos) pop 1 { data[pos] = pop(); }
  phase writeData() push count {
    for (int i = 0; i < count; i++)
      push(data[i]);
  }
}
\end{lstlisting}

Note that the number of executions of the \lstinline|readData| and
\lstinline|writeData| phases are data-dependent, as is the
\lstinline|push| rate of \lstinline|writeData|.

\section{Implementation Limits}

The following limits apply to the current implementation of StreamIt:

\begin{itemize}
\item There is no guaranteed order of evaluation.  In particular, a
  statement in a work function should not contain more than a single
  \lstinline|pop()|, and if contains any \lstinline|pop()|s at all, it
  should contain no \lstinline|peek()|s.
  
\item The names \lstinline|SplitJoin|, \lstinline|FeedbackLoop|,
  \lstinline|Filter|, \lstinline|Pipeline|, \lstinline|StreamIt|, and
  \lstinline|Complex| are reserved.  Certain variable names beginning
  with underscores may cause conflicts if used as local variables.

\end{itemize}

\end{document}