\section{Compiling StreamIt to Raw}
\label{sec:phases}
StreamIt language aims to be portable across communication-exposed
machines.  StreamIt expose the parallelism and communication of
streaming applications without depending on the topology or
granularity of the underlying architecture.  have implemented a
fully-functional prototype of the StreamIt compiler for
Raw~\cite{raw}, a tiled architecture with fine-grained, programmable
communication between processors.  However, the compiler employs three
general techniques that can be applied to compile StreamIt to machines
other than Raw: 1) partitioning, which adjusts the granularity of a
stream graph to match that of a given target, 2) layout, which maps a
partitioned stream graph to a given network topology, and 3)
scheduling, which generates a fine-grained static communication
pattern for each computational element.  We consider this work to be a
first step towards a portable programming model for
communication-exposed architectures.

The front end is built on top of KOPI, an open-source compiler
infrastructure for Java~\cite{kopi}; we use KOPI as our infrastructure
because StreamIt has evolved from a Java-based syntax.  We translate
the StreamIt syntax into the KOPI syntax tree, and then construct the
StreamIt IR (SIR) that encapsulates the hierarchical stream graph.
Since the structure of the graph might be parameterized, we propagate
constants and expand each stream construct to a static structure of
known extent.  At this point, we can calculate an execution schedule
for the nodes of the stream graph.

The StreamIt compiler is composed of the following stages that are
specific for communication-exposed architectures: stream graph
scheduling, stream graph partitioning, layout, and communication
scheduling.  The next four sections provide a brief overview of these
phases. For a detailed explanation see~\cite{Gordo02,Gordon-thesis,Karczma-thesis}.

\subsection{Stream Graph Scheduling}
The automatic scheduling of the stream graph is one of the primary
benefits that StreamIt offers, and the subtleties of scheduling and
buffer management are evident throughout all of the following phases
of the compiler.  The scheduling is complicated by StreamIt's support
for the {\tt peek} operation, which implies that some programs require
a separate schedule for initialization and for the steady state.  The
steady state schedule must be periodic--that is, its execution must
preserve the number of live items on each channel in the graph (since
otherwise a buffer would grow without bound.)  A separate
initialization schedule is needed if there is a filter with $peek >
pop$, by the following reasoning.  If the initialization schedule were
also periodic, then after each firing it would return the graph to its
initial configuration, in which there were zero live items on each
channel.  But a filter with $peek > pop$ leaves $peek-pop$ (a positive
number) of items on its input channel after {\it every} firing, and
thus could not be part of this periodic schedule.  Therefore, the
initialization schedule is separate, and non-periodic.

In the StreamIt compiler, the initialization schedule is constructed
via symbolic execution of the stream graph, until each filter has at
least $peek-pop$ items on its input channel.  For the steady state
schedule, there are many tradeoffs between code size, buffer size, and
latency, and we are developing techniques to optimize different
metrics \cite{streamittech2}.  Currently, we use a simple hierarchical
scheduler that constructs a Single Appearance Schedule (SAS)
\cite{leesdf} for each filter.  We plan to develop better scheduling
heuristics in the future~\cite{Karczma-thesis}.  A SAS is a schedule
where each node appears exactly once in the loop nest denoting the
execution order.  We construct one such loop nest for each
hierarchical stream construct, such that each component is executed a
set number of times for every execution of its parent.  In later
sections, we refer to the ``multiplicity'' of a filter as the number
of times that it executes in one steady state execution of the entire
stream graph.

