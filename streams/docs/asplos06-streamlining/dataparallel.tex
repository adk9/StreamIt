\begin{figure*}[t]
\begin{center}
\psfig{figure=filterbank.eps,width=7in}
\end{center}
\caption{FilterBank benchmark. Only one pipeline of the outer split-join is shown, rest are identical pipelines. \label{fig:filterbank}}
\end{figure*}


\section{Coarse-Grained Data Parallelism}

There is typically widespread data parallelism in a stream graph, as
stateless filters can be applied in parallel to different parts of the
data stream.  For example, Figure~\ref{fig:filterbank-orig} depicts
our Filterbank benchmark, in which all of the filters are stateless.
%Examples of stateless
%filters include FIR filters, FFTs, DCTs, color conversions,
%decimators, or any arithmetic
We detect stateless filters using a simple program analysis that tests
whether there are any filter fields (state variables) that are written
during one iteration of the work function and read during another.  To
leverage the implicit data parallelism of stateless filters, we
convert the filters into an explicit splitjoin of many filters, such
that each filter can be mapped to a separate processor.  This process,
which is called {\it filter fission}, is described in more detail
elsewhere~\cite{gordon02asplos}.

On a distributed memory machine, it can be expensive to distribute
data to and from the parallel products of filter fission.  As a
starting point, our technique only fisses filters in which the
estimated computation to communication ratio is above a given
threshold.  (In our experiments, we use a threshold of 10 compute
instructions to 1 item of communication.)  To further mitigate the
communication cost, we introduce two new techniques: coarsening the
granularity, and complementing task parallelism.

\subsection{Coarsening the Granularity}

Stateless filters are often connected together in a pipeline (see
Figure~\ref{fig:filterbank-orig}).  In such cases, performing fission
independently on each filter leads to undue synchronization, as the
data streams need to be scattered and gathered between each
computation stage.  For this reason, we prefer to fuse pipelines as
much as possible before performing fission.  The only constraint is
that the result of the fusion must also be stateless, as otherwise the
data parallelism is lost.  It can also be beneficial to fuse
splitjoins of stateless components, so long as the fused splitjoin is
later combined as part of a pipeline.  Fusing splitjoins in isolation
does not change the external computation to communication ratio.

If two filters are fused, their result will be stateless under the
following conditions.  Filters that execute in parallel (i.e.,
adjacent children of a splitjoin) will be stateless in fused form if
and only if they were stateless to start with.  However, for filters
connected in a pipeline, fusion may introduce state even for
previously stateless filters due to items buffered on the intervening
data channel.  Due to StreamIt's support for peeking (sliding window
computations) as well as an optional {\tt prework} function
(performing distinct computation on the first firing of a filter), an
{\it initialization schedule} may be needed to prime the buffers prior
to steady-state execution~\cite{karczma02thesis}.  If this schedule
deposits items between two filters, then those items introduce state
into the fused filter and prohibit data parallelism.  For example, in
Figure~\ref{fig:filterbank-orig}, the LowPassFilters and
HighPassFilters cannot be fused with upstream neighbors because their
peek rate exceeds their pop rate.


\begin{algorithm}[t]
\caption{Granularity coarsening algorithm to expose coarse-grained
data parallelism.} \label{fig:coarsening-code}
\textsc{CoarsenGranularity}()
\begin{algorithmic}[1]
\Repeat
	\ForAll{$(f_1, f_2)$ pairs of adjacent stateless filters}
		\If {$f_1$ and $f_2$ in a splitjoin}
			\State Speculatively fuse $f_1$ and $f_2$
		\ElsIf{$f_1$ and $f_2$ in a pipeline and \\
\ \ \ \ \ \ \ \ \ \ \ \ \ \ \ \ \ \ \ \ \ \ no initial buffering between $f_1$ and $f_2$}
			\State Finalize any speculative fusion within
$f_1$ an $f_2$
			\State Fuse $f_1$ and $f_2$
		\EndIf
	\EndFor
\Until{nothing is fused}
\State Undo remaining speculative fuses	
\end{algorithmic}
\end{algorithm}

%\begin{figure}[t]
%\framebox{TODO: typeset this code.}
%\small
%\begin{verbatim}
%CoarsenGranularity() {
%  do { 
%    for each pair of adjacent, stateless filters (f1, f2) do
%      if (f1 and f2 in splitjoin) then
%        speculatively fuse f1 and f2
%      else if (f1 and f2 in pipeline and
%               no initial buffering between f1 and f2) then
%        finalize any speculative fusion within f1 and f2
%        fuse f1 and f2
%      endif
%    endfor
%  } while something is fused
%  undo remaining speculative fuses
%}
%\end{verbatim}
%\caption{Granularity coarsening algorithm to expose coarse-grained
%data parallelism.\label{fig:coarsening-code}}
%\end{figure}

Algorithm~\ref{fig:coarsening-code} gives pseudocode for our granularity
coarsening algorithm.  The code repeatedly considers pairs of
stateless filters that are adjacent in the stream graph.  If the
filters are contained in a splitjoin, they are ``speculatively'' fused
by the algorithm\footnote{Our compiler performs speculative fusion
symbolically rather than actually fusing the filters.}.  
%The fusion is
%speculative because it will only be beneficial if combined with
%further fusion within a higher-level pipeline.  
If the filters are in a pipeline and do not require a buffer at
initialization time, then they are permanently fused, and any
speculative fuses performed in building the filters are finalized.  At
the end of the algorithm, all remaining speculative fuses are
cancelled.  Note that all fused filters produced by the algorithm are
stateless and, due to an increased computation to communication ratio,
are likely to benefit from filter fission.  The output of the
algorithm on the FilterBank benchmark is illustrated in
Figure~\ref{fig:filterbank-coarse}.

Apart from reducing synchronization and increasing the computation to
communication ratio, fusion allows the compiler to perform
optimizations across filter boundaries.  These optimizations, which
include scalar replacement~\cite{sermulins05lctes} and algebraic
simplification of linear filers~\cite{lamb03pldi,agrawal05cases}, can
also have a significant impact on performance.
% mention memory operations vs. network operations?

%% Predicting whether a fusion product will be stateless is not a simple
%% matter of testing the original filters for statelessness.  While
%% pre-existing state in any of the original filters does imply that the
%% fused filter is stateful, there are three additional sources of state
%% in the fusion process.

%% First, filters performing a sliding window computation (i.e., the peek
%% rate is greater than the pop rate) require an internal buffer when
%% they are fused with an upstream neighbor in a pipeline.  For example,
%% in Figure~\ref{fig:filterbank-orig}, each HighPassFilter and
%% LowPassFilter can not be fused with the filters above it without
%% introducing internal state.  In addition, our fusion algorithm
%% introduces state when fusing such filters with neighbors in a
%% roundrobin splitjoin (to optimize access to the input buffers).

%% Second, filters may indicate a {\tt prework} function to indicate
%% distinct functionality on the first execution (for example, reading a
%% file header).  If a filter's {\tt prework} and {\tt work} function
%% declare different input (resp. output) rates, then the filter
%% generally requires an internal buffer when fused with an upstream
%% (resp. downstream) neighbor.

\subsection{Complementing Task Parallelism}

\begin{figure}[t]
\framebox{TODO: typeset this code.}
\small
\begin{verbatim}
// F is the filter to fiss; N is the total number of cores
JudiciousFission(filter F, int N) {
  // estimate work done by F as fraction of 
  // everyone running task-parallel to F
  fraction = 1.0
  Stream parent = getParent(F)
  Stream child = f
  while (parent != null) {
    if (parent is splitjoin)
      totalWork = sum_{c \in children(parent)} avgWorkPerFilter(c)
      myWork = avgWorkPerFilter(child)
      fraction = fraction * myWork / totalWork
    endif
    child = parent
    parent = getParent(parent)
  endwhile
  // fiss F according to its weight in task-parallel unit
  fiss F into ceil(fraction*N) filters
}
\end{verbatim}
\caption{Heuristic algorithm for fissing a filter as little as
possible while filling all cores with task or data-parallel
work.\label{fig:judicious-fission}}
\end{figure}

Even if every filter in an application is data-parallel, it may not be
desirable to fiss each filter across all of the cores.  Doing so would
eliminate all task parallelism from the execution schedule, as only
one filter from the original application could execute at a given
time.  An alternate approach is to preserve the task parallelism in
the original application, and only introduce enough data parallelism
to fill any idle processors.  This serves to reduce the
synchronization imposed by filter fission, as filters are fissed to a
smaller extent and will span a more local area of the chip.  Also, the
task-parallel filters are a natural part of the algorithm and avoid
any computational overhead imposed by filter fission (e.g., fission of
peeking filters introduces a decimation stage on each fission
product).

In order to balance task and data parallelism, we employ a ``judicious
fission'' heuristic that estimates the amount of work that is
task-parallel to a given filter and fisses the filter accordingly.
Depicted in Figure~\ref{fig:judicious-fission}, this algorithm works
by ascending through the hierarchy of the stream graph.  Whenever it
reaches a splitjoin, it calculates the ratio of work done by the
stream containing the filter of interest to the work done by the
entire splitjoin.  Rather than summing the work within a stream, it
considers the average work per filter in each stream so as to mitigate
the effects of imbalanced pipelines.\framebox{Not clear when/why/if
this works.}  After estimating a filter's work as a fraction of
task-parallel filters, the algorithm attempts to fiss the filter the
minimum number of times needed to ensure that none of the fission
products contains more than $1/N$ of the total task-parallel work
(where $N$ is the total number of cores).

Figure~\ref{fig:filterbank-dup} illustrates the outcome of performing
judicious fission on the coarsened-granularity stream graph from
Figure~\ref{fig:filterbank-coarse}.  Because there is 8-way task
parallelism between all of the pipelines, the filters in each pipeline
are fissed a maximum of 2 ways so as not to overwhelm the
communication resources.  As described in the results section, the
combination of granularity coarsening and judicious fission offers
significant speedups over a naive fission policy.
