\section{Implementation} 

To facilitate evaluation of our techniques, we have implemented a
fully-automatic compiler that supports each of the execution models
discussed in this paper.  The compiler targets the Raw
Microprocessor.  In this section we will first give a description of
the Raw Architecture followed by implementation details of the 
execution models.

\subsection{The Raw Architecture}
\label{sec:raw}

\begin{figure}
\centering
\psfig{figure=raw-diagram.eps,width=3in}
\caption{Block diagram of the Raw architecture.
\protect\label{fig:raw-diagram}}
\end{figure}
The Raw Microprocessor \cite{raw10,raw} addresses the wire delay
problem \cite{raw13} by providing direct instruction set architecture
(ISA) analogs to three underlying physical resources of the processor:
gates, wires and pins. Because ISA primitives exist for these
resources, a compiler such as StreamIt has direct control over both
the computation and the communication of values between the functional
units of the microprocessor, as well as across the pins of the
processor.

The architecture exposes the gate resources as a scalable 2-D array of
identical, programmable cores, that are connected to their immediate
neighbors by four on-chip networks.  Values routed through the
networks off of the side of the array appear on the pins, and values
placed on the pins by external devices (wide-word A/Ds, DRAMS, etc.)
appear on the networks.  Each of the cores contains a compute
processor, some memory and two types of routers--one static, one
dynamic--that control the flow of data over the networks as well as
into the compute processor (see Figure \ref{fig:raw-diagram}).  The
compute processor interfaces to the network through a bypassed,
register-mapped interface \cite{raw10} that allows instructions to use
the networks and the register files interchangeably.

Each core's static router has a instruction memory to control the
crossbars of the two static networks. Collectively, the static routers
can reconfigure the communication pattern across these networks every
cycle.  The input and output possibilities for each crossbar are:
North, East, South, West, Processor, to the other crossbar, and into
the static router. The FIFOs are typically four or eight elements
large.  To route a word from one core to another, the compiler inserts
a route instruction on every intermediate static router.  Because the
routers are pipelined and compile-time scheduled, they can deliver a
value from the ALU of one core to the ALU of a neighboring core in 3
cycles, or more generally, 2+N cycles for an inter-core distance of N
hops.

Because we generate bulk DRAM transfers, we do not want these
optimizable accesses to become the bottleneck of the hardware
configuration.  So, we employ a simulation a CL2 PC 3500 DDR DRAM,
which provides enough bandwidth to saturate both directions of a Raw
port \cite{raw_isca}.  Additionally, each chipset contains a streaming
memory controller that supports a number of simple streaming memory
requests. In our configuration, we attach 16 such DRAMs to each of the
16 logical ports of the chip.  The chipset receives request messages
over the dynamic network for bulk transfers to and from the DRAMs.
The transfers themselves can use either the static network or the
general dynamic network (the desired network is encoded in the
request).

The results in this paper were generated using btl, a cycle-accurate
simulator that models arrays of Raw cores identical to those in the
.15 micron 16-core Raw prototype ASIC chip, with a target clock rate
of 450 MHz. The core employs as compute processor an 8-stage, single
issue, in-order MIPS-style pipeline that has a 32 KB data cache, 32 KB
of instruction memory, and 64 KB of static router memory.  We
augmented the simulator to include a 2-way set associative hardware
instruction caching mechanism that is serviced over the dynamic
network, with resource contention modeled accordingly. 

We will now turn to discussing our compiler backends for each of the
execution models.

\subsection{Task + Pipeline Parallel Execution}
\label{sec:space-imp}
To compile for the task plus pipeline parallel execution model, we
leverage our previous work\cite{streamit-asplos}. In this model we
require that the number of filters in the stream graph be less than or
equal to the number of processing cores of the target architecture.
To achieve this, we repeatedly apply fusion and fission
transformations as directed by a dynamic programming algorithm, termed
the {\it partitioner}.  The algorithm fisses individual filters or
fuses local sections of the stream graph in the hope of creating a
load balanced stream graph ready for execution.

The resulting nodes of the stream graph are executed in a
space-multiplexed manner.  We calculate the mapping of nodes to Raw
cores using simulated annealing, with at most one node mapped to each
core.  Communication bypasses the memory system as the nodes
communicate over the Raw's static network.  Because the static network
is blocking and has little buffering, the simulated annealing cost
function attempts to model the synchronization costs of each layout
configuration encountered during the search.  After we anneal to a
final layout, we generate switch assembly instructions to realize the
steady-state communication of the mapped stream graph.  Finally, we
generate code to execute each filter on the compute processor to which
it was assigned.

\subsection{New Backend}
For this paper we have implemented a new backend targeting Raw that
does not solely rely on a space-multiplexed approach.  The new backend
adopts a more traditional approach for its execution model. Multiple
filters can be assigned to each processing core.  Across the
replicated cores, the filters execute in dataflow order, with each
filter reading its input and writing its output to shared memory. Each
core executes a schedule of filters assigned to it and each filter
executes on a single processing core for the entire execution.

Interestingly, for the techniques presented in this paper we do not
leverage Raw's on-chip static network for communication between
filters (although the compiler supports it) nor do we buffer
communication on the data caches of the cores themselves. Each core
executes independently until a split or join point is reached. 

We do not buffer communication on the chip itself because given the
Raw configuration we are simulating and for the regular bulk memory
traffic we generate, it is more expensive to access a core's local
data cache than to fetch the item from one of the streaming memory
controllers. A load hit in the data cache incurs a 3-cycle latency.
So although the networks are register-mapped, two instructions must be
performed to hide the latency of the load, implying a maximum
bandwidth of 1/2 word per cycle.  While each streaming memory
controller has a bandwidth of 1 word per cycle for unit-stride memory
accesses.

Furthermore, we have found that employing Raw's on-chip network for
communication between filters of a pipeline is not necessary
advantageous because of the added synchronization incurred by tightly
coupling the filters.  We experimented with numerous techniques for
forming spatial pipelines of filters that communicate amongst
themselves using only on-chip resources.  Our best techniques afforded
no performance gains.  We have concluded that it is extremely
difficult for a compiler operating at a coarse granularity to manage
the fine-grained synchronization required to map stream communication
to Raw's on-chip static network.  Additionally, our techniques and our
conclusions are more general since nearly all commercial multicores do
not provide fine-grained, low-latency inter-core communication.

Our new backend requires that the target architecture provides an
efficient mechanism for implementing split and join (scatter and
gather) operations.  For Raw, we leverage the static network and the
switch processors of the static network to efficiently implement the
splitting and joining of data streams as described by the stream
graph.  In our previous backend, we devoted a core to buffering and
forwarding joined data streams (split data streams were implemented
completely in the switch).

Our new backend programs the switch processors to form a network to
perform the splitting (including duplication) and the joining of data
streams.  Since Raw's DRAM ports are banked, we must read all the data
participating in the reorganization from off-chip memory and write it
back to off-chip memory.  As opposed to our old backend, we do not
require a core to perform the buffering because each of the joined
data streams are buffered off-chip.  The disadvantages to this scheme
are that all the data required for joining must be available before
the join can commence and the compute processors of the cores involved
are idle during the reorganization.  In our previous backend, we could
forward data as it becomes available, enabling pipeline parallelism.
But as we have discussed, software pipelining the steady-state of the
stream graph removes the dependences between filters allowing us to
perform all the splits and joins of the stream graph in parallel
across the chip.


\subsubsection{Scheduling without Software Pipelining}
Our scheduling techniques for software pipelining of the stream graph
were presented in Section \ref{sec:softpipe}.  To facilitate
evaluation of coarse-grained software pipelining, we implemented a
separate scheduling path for a non-software pipelined schedule that
executes the steady-state respecting the dataflow dependencies of the
stream graph.  This scheduling path ignores the presence of the
encompassing outer loop in the stream graph.    

This scheduling problem is equivalent to static scheduling of a
coarse-grained dataflow DAG to a multiprocessor which has been a
well-studied problem over the last 40 years (see \cite{DAGSched} for a
good review).  After a review of the research, we decided to once
again leverage simulated annealing because recent research
demonstrates that randomized solutions to this static scheduling
problem are superior to other heuristics \cite{kwok99fastest}. 

Briefly, we generate an initial layout by assigning the filters of the
stream graph to processors in dataflow order, assigning a random
processing core to each filter. The simulated annealing perturbation
function randomly selects a new core for a filter, inserting the
filter at the correct slot in the core's schedule of filters based on
the dataflow dependencies of the graph.  The cost function of the
annealer uses our static work estimation to calculate the maximum
critical path length (measured in cycles) from a source to a sink in
the graph.  After the annealer is finished, we use the configuration
that achieved the minimum critical path length over the course of the
search.  Finally, we run a greedy assignment algorithm to assign the
buffers representing the channels of the stream graph to DRAM ports,
attempting to minimize synchronization costs.

\subsubsection{Code Generation}


\subsubsection{Optimizations}

The two backends share a suite of optimizations that target the
computation code of a filter.  For the results detailed in the next
section we ran a host of optimizations including function inlining,
constant propagation, constant folding, array scalarization, and loop
unrolling (with a factor of 4).  These optimizations are especially
important for a fused filter, as we can unroll it, unrolling the
schedule of the constituent filters within the fused filter.  Possibly
scalarizing the buffers within the fused filter used for communication
between the constituent filters.  Finally, the two backends produce a
mix of C and assembly code that is compiled with GCC 3.4 at
optimization level 3.
