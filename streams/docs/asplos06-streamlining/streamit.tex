\subsection{The StreamIt Language}
\label{sec:streamit}
\begin{figure}[t]
\begin{minipage}{3.2in}
\begin{center}
\begin{minipage}{0.46in}
\centering
\psfig{figure=pipeline.eps,width=0.46in} \\
\end{minipage} 
~
\begin{minipage}{1.3in}
\centering
\psfig{figure=splitjoin.eps,width=1.3in} \\
\end{minipage}
~
\begin{minipage}{1.02in}
\centering
\psfig{figure=feedback.eps,width=1.02in} \\
\end{minipage} 
\\ ~ \\ {\protect\small (a) A pipeline. ~~(b) A splitjoin. ~~(c) A feedbackloop.}
\caption{Stream structures supported by StreamIt.
\protect\label{fig:structures}}
\end{center}
\end{minipage}
\end{figure}

StreamIt is a portable programming language for high-performance
streaming applications.  In a stream program, computation is expressed
as a composition of filters that operate over sequences of data. The
current version of StreamIt compiler uses the static-rate version of
StreamIt: we require that the input and output rates of each filter
are known at compile time.  The most up-to-date syntax specification
can always be found on the StreamIt website~\cite{streamitweb}.

The basic unit of computation in StreamIt is the {\it filter}.  There
are three basic constructs for composing filters into a communicating
network: a {\it pipeline}, a {\it splitjoin}, and a {\it
feedbackloop}, see Figure \ref{fig:structures}.  A pipeline behaves as
the sequential composition of all its child streams.  A splitjoin is
used to specify independent parallel streams that diverge from a
common {\it splitter} and merge into a common {\it joiner}. A
feedbackloop provides a way to create cycles in the stream graph.  Due
to space constraints, we omit a detailed discussion of the
feedbackloop. In practice, feedbackloops are rare and currently the
SpaceTime Compiler does not support them. From now on, we use the word
{\it stream} to refer to any instance of a filter, pipeline,
splitjoin, or feedbackloop.


%which are
%specified with successive calls to {\tt add} from within the pipeline.
%The {\tt add} statements can be mixed with regular imperative code to
%parameterize the construction of the stream graph.
% There are two
%kinds of splitters: 1) $duplicate$, which replicates each data item
%and sends a copy to each parallel stream, and 2) $roundrobin(w_1,
%\dots, w_n)$, which sends the first $w_1$ items to the first stream,
%the next $w_2$ items to the second stream, and so on. roundrobin is
%also the only type of joiner that we support; its function is
%analogous to a roundrobin splitter.

%A
%filter is a single-input, single-output block with user-defined
%procedures for translating input items to output items.  Each filter
%contains an {\tt init} function that is called at initialization time
%and a {\tt work} function that defines the most fine grained execution
%step of the filter in the steady state.  Within the {\tt work}
%function, the filter can communicate with its neighbors via FIFO
%queues, termed {\it channels}, using the intuitive operations of {\tt
%push(value)}, {\tt pop()}, and {\tt peek(index)}, where {\tt peek}
%returns the value at position {\tt index} without dequeuing the item.
%The number of items that are pushed, popped, and peeked on each
%invocation are declared with the {\tt work} function.


%\subsection{Linear Filters}
%One far-reaching goal of the StreamIt project is to incorporate the
%knowledge of expert DSP programmers into the compiler.  Toward that
%end, the StreamIt compiler can recognize sections of the stream graph
%that adhere to a certain model, apply targeted transformations and
%optimizations, and generate specialized code for these sections.
%Currently the StreamIt compiler can recognize {\it linear} sections of
%a stream graph. We call a filter linear if its outputs can be
%expressed as an affine combination of its inputs.  Many ubiquitous DSP
%kernels are linear, including FIR filters, expanders, compressors,
%DFTs and DCTs.  In \cite{streamit-linear} a completely automated
%framework for extracting, combining, and transforming linear filters
%in the StreamIt compiler is presented.  In the current paper, linear
%components of the stream graph are targeted for specialized code
%generation (see Section \ref{sec:linear}).
