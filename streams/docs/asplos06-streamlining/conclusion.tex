\section{Conclusions}

As multicore architectures become ubiquitous, it will be critical to
develop a high-level programming model that can automatically exploit
the coarse-grained parallelism of the underlying machine without
requiring heroic efforts on the part of the programmer.  Stream
programming represents a promising approach to this problem, as
high-level descriptions of streaming applications naturally expose
task, data, and pipeline parallelism.  

In this paper, we develop improved techniques for automatically
bridging the gap between the original granularity of the program and
the underlying granularity of the architecture.  To bolster the
benefits of data parallelism on a multicore architecture, we build
coarse-grained data-parallel units that are duplicated as few times as
needed.  And to leverage the benefits of pipeline parallelism, we
employ software pipelining techniques---traditionally applied at the
instruction level---to coarse-grained filters in the program.

A detailed evaluation of these techniques on the Raw microprocessor
offers very favorable results.  Coarse-grained data parallelism offers
a 4.4x speedup over a task-parallel baseline and a 9.9x speedup over a
sequential code.  Without granularity coarsening, these reduce to 0.7x
and 1.4x, respectively.  Coarse-grained software pipelining improves
the generality of the compiler, as it is able to parallelize stateful
filters with dependences from one iteration to the next.  Our two
techniques are complementary and offer a combined speedup of 11.2x
over the baseline (and 1.84x over our previous work).

Though data parallelism is responsible for much of our speedups on a
16-core chip, pipeline parallelism may become more important as
multicore architectures scale.  Data parallelism requires global
communication, and keeps resources sitting idle when it encounters
stateful filters (or feedback loops).  According to our analysis in
Section~\ref{sec:model}, leveraging pipeline parallelism on a 64-core
chip when only 10\% of the filters have state could offer up to a 6.4x
speedup (improvement in load balancing).  Exposing pipeline
parallelism in combination with data parallelism for the stateful
benchmarks in our suite provided a 1.45x speedup over data
parallelism alone.

As our techniques rely on specific features of the StreamIt
programming model, the results suggest that these features are a good
match for multicore architectures.  Of particular importance are the
following two language features:
\begin{enumerate}

\item Exposing producer-consumer relationships between filters.  This
enables us to coarsen the computation to communication ratio via
filter fusion, and also enables pipeline parallelism.

\item Exposing the outer loop around the entire stream graph.  This is
central to the formulation of software pipelining; it also enables
data parallelism, as the products of filter fission may span multiple
steady-state iterations.

\end{enumerate}

While our implementation targets Raw, the techniques developed should
be applicable to other multicore architectures.  As Raw has a
relatively high communication bandwidth, coarsening the granularity of
data parallelism may benefit commodity multicores even more.  In
porting this transformation to a new architecture, one may need to
adjust the threshold computation-to-communication ratio that justifies
filter fission.  As for coarse-grained software pipelining, the
scheduling freedom afforded should benefit many multicore systems.
One should consider the most efficient location for intermediate
buffers (local memory, shared memory, FIFOs, etc.) as well as the best
mechanism for shuffling data (DMA, on-chip network, etc.).  The basic
algorithms for coarsening granularity, judicious fission,
partitioning, and selective fusion are largely
architecture-independent.
