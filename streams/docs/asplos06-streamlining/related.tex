\section{Related Work}
\label{sec:related}

- software pipelining on SDF and instruction-level

% reading group paper: streaming on multithreaded cores
Software pipelining for clustered VLIWs~\cite{qian02}.

Previous work in scheduling computation graphs to parallel targets
have focused on dynamic techniques~\cite{SDFSched, SDFSched2,
may87communicating, DAGSched}. In general, multiple graph nodes are
{\it clustered} onto a single computational node and scheduled
dynamically.

Our work, unlike most previous work in this field, models link
contention and topography.  Furthermore, StreamIt graphs are
implicitly formed of loops so we can apply loop scheduling techniques
such as software pipelining to build our schedules.

%The problem of instruction scheduling for MIMD and VLIW architectures
%is similar to the problem tackled by the space-time compiler.  ILP
%compilers for clustered VLIW architectures~\cite{Bulldog, Multiflow}
%are decomposed into stages that are analogous to the stages of the
%SpaceTime compiler.  These compilers must partition or cluster
%instructions, assign instructions to processors, and then schedule the
%instructions.  

Previous work on software pipelining has focused on scheduling machine
instructions in a loop~\cite{lam-softpipe, rau-softpipe} to a
uniprocessor target.  The algorithms devised must account for tight
resource constraints and complex instruction dependences.  Our
software-pipelining problem is much less constrained; a traditional
modulo scheduling algorithm can not effectively take advantage of this
flexibility.  Previous work on ILP scheduling for the Raw
architectures~\cite{lee98spacetime} also bears similarity.  However,
these compilers schedule graphs of fine-grained instructions. The
partitioning and scheduling heuristics are mindful of a different set
of constraints including different types of dynamism and less regular
communication patterns as compared to StreamIt graphs.

%As far as we know, we are the first to evaluate loop-level scheduling
%techniques as a method for scheduling coarse-grained task graphs.

% \cite{cheops-thesis}
%   http://web.media.mit.edu/~kung/publication/thesis.pdf
%
% other possible things to cite:
%  http://portal.acm.org/citation.cfm?id=801721&dl=ACM&coll=portal
%  http://www.csrl.unt.edu/~kavi/Research/ica3pp156.pdf
%  http://cdmetcalf.home.comcast.net/papers/cop/node1.html#SECTION00010000000000000000

% MIKE
%%%%%%%%%%%%%%%%%%%%%%%%%%%%%%%%%%%%%%%%%%%%%%%%%%%%%%%%%%%%%%%%%%%%%
% BILL

The Imagine stream processor~\cite{rixner98bandwidthefficient}
supports a time-multiplexed execution model.  The architecture
contains 48 parallel ALU's organized into 6 VLIW clusters.  The
programming model requires the programmer to write computation filters
in Kernel-C and stitch them together using Stream-C.  Because the
execution unit is data-parallel, the compiler uses time multiplexing
to execute a single filter at a time across all of the parallel
clusters.  While this provides perfect load balancing and high
arithmetic utilization when there is abundant data parallelism, it
suffers when a filter has retained state or data-dependences between
iterations.  Moreover, architectures based solely on
time-multiplexing do not scale spatially, as there are global wires
orchestrating the parallel execution units. 

- partitioning stuff
- SDF
