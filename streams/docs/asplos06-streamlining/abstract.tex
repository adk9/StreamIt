With the continued miniaturization of the transistor, microprocessor
performance is increasingly dominated by wire delays.  Many
architectures address this problem by replicating processing units.
Furthermore, some architectures expose the communication between units
to a software layer.  For compilers to effectively target multicore
architectures, explicitly parallel programming models are required.
Currently, there are two basic approaches for compiling a parallel
program to a multicore target.  A {\it time multiplexing} approach
utilizes the entire chip for each computation unit, and switches
between units over time.  A {\it space multiplexing} approach distributes
computation units across the entire chip, running them continuously
and in parallel.

In this paper, we describe a hybrid space-time multiplexing scheme
that provides the flexible load-balancing of time multiplexing while
preserving the locality and latency benefits of space multiplexing.
Our work is done in the context of StreamIt, a high-level stream
programming language, with a backend that targets the Raw
architecture.  Our compiler extracts fine-grained {\it slices} from a
concurrent stream graph; it then uses space multiplexing for code
within a slice, but time multiplexing to switch between slices. We
leverage software pipelining, a technique traditionally applied to
scheduling a loop of machine instructions, and apply it to the
scheduling of coarse-grained slices.  We give performance results that
demonstrate the efficacy of space-time multiplexing in compiling
StreamIt to Raw.
