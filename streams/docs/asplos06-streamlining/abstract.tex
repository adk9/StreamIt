As multicore architectures enter the mainstream, there is a pressing
demand for high-level programming models that can effectively utilize
the underlying parallelism.  Stream programming offers an attractive
way to expose coarse-grained parallelism, as streaming applications
(image, video, DSP, etc.) are naturally represented by independent
filters that communicate over explicit data channels.  In order to
obtain the best mapping, a compiler must fully utilize the task, data,
and pipeline parallelism that is present in the application.

In this paper, we describe two new techniques for mapping stream
programs to multicores.  The first technique exploits {\it
coarse-grained data parallelism} by duplicating data-parallel sections
across cores.  However, to lessen the synchronization overhead, we
operate on large sections of the stream graph rather than individual
actors.  The second technique, {\it coarse-grained software
pipelining}, applies traditional instruction-scheduling techniques to
coarse-grained actors in the stream graph.  This affords complete
scheduling freedom and extracts valuable pipeline parallelism from the
application.

We have implemented these techniques in the StreamIt compiler,
targeting the 16-core Raw architecture.  Coarse-grained data
parallelism offers a 9.9x speedup over a single-core baseline, while
coarse-grained software pipelining offers a 7.7x speedup.  Combining
both techniques yields the best results, offering a 11.2x speedup over
a single core and 1.84x over our previous work.
