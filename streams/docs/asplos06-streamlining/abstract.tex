As multicore architectures enter the mainstream, there is a pressing
demand for high-level programming models that can effectively map to
them.  Stream programming offers an attractive way to expose
coarse-grained parallelism, as streaming applications (image, video,
DSP, etc.) are naturally represented by independent filters that
communicate over explicit data channels.
%In order to obtain the best
%mapping, a compiler must fully utilize the task, data, and pipeline
%parallelism that is present in the application.

In this paper, we describe two new techniques for mapping stream
programs to multicores.  The first technique exploits {\it
coarse-grained data parallelism} by duplicating data-parallel sections
across cores.  While traditional data parallelism incurs excessive
communication and synchronization costs on multicores, our technique
utilizes a program analysis to coarsen the granularity of
data-parallel actors and duplicate them the minimum number of times
needed.  The second technique, {\it coarse-grained software
pipelining}, leverages powerful properties of the stream programming
model to apply traditional instruction-scheduling techniques at the
level of coarse-grained actors.  This increases scheduling freedom and
extracts valuable pipeline parallelism from the application.

We have implemented these techniques in the StreamIt compiler,
targeting the 16-core Raw architecture.  Coarse-grained data
parallelism offers a 9.9x mean speedup over a single-core baseline,
while coarse-grained software pipelining offers a 7.7x speedup.
Combining both techniques yields the best results, achieving a 11.2x
speedup over a single core and 1.84x over our previous work.
