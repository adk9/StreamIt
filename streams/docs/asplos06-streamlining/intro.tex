\section{Introduction}

%% Possibly add figure with stream graph that labels task, data, and pipeline parallelism

As centralized microprocessors are ceasing to scale effectively,
multicore architectures are becoming the industry standard.  For
example, the IBM/Toshiba/Sony Cell processor has 9
cores~\cite{Cell-hpca}, the Sun Niagara has 8 cores~\cite{Niagara},
the RMI XLR732 has 8 cores~\cite{RMI-web}, the IBM/Micro\-soft Xbox 360 CPU
has 3 cores~\cite{xbox360}, and most vendors are shipping dual-core
chips.  Cisco has described a next-generation network processor
containing 192 Tensilica Xtensa cores~\cite{etherton05ancs}.
%The Intel IXP 2800 network processor has 1 core and 16 micro-engines.
This trend has pushed the performance burden to the compiler, as
future application-level performance gains depend on effective
parallelization across the cores.  Unfortunately, traditional
programming models such as C, C++ and FORTRAN are ill-suited to
multicore architectures because they assume a single instruction
stream and a monolithic memory.  Extracting coarse-grained parallelism
suitable for multicore execution amounts to a heroic compiler analysis
that remains largely intractable.

The stream programming paradigm offers a promising approach for
exposing parallelism suitable for multicore architectures.  Stream
languages such as StreamIt~\cite{streamitcc}, Brook~\cite{brook04},
SPUR~\cite{spur05samos}, Cg~\cite{cg03}, Baker~\cite{Baker}, and
Spidle~\cite{spidle03} are motivated not only by trends in computer
architecture, but also by trends in the application space, as network,
image, voice, and multimedia programs are becoming only more
prevalent.  In the StreamIt language, a program is represented as a
set of autonomous actors that communicate through FIFO data channels
(see Figure~\ref{fig:vocoder}).  During program execution, actors fire
repeatedly in a periodic schedule.  As each actor has a separate
program counter and an independent address space, all dependences
between actors are made explicit by the communication channels.
Compilers can leverage this dependence information to orchestrate
parallel execution.

Despite the abundance of parallelism in stream programs, it is
nonetheless a challenging problem to obtain an efficient mapping to a
multicore architecture.  Often the gains from parallel execution can
be overshadowed by the costs of communication and synchronization.  In
addition, not all parallelism has equal benefits, as there is
sometimes a critical path that can only be reduced by running certain
actors in parallel.  Due to these concerns, it is critical to leverage
the right combination of task, data, and pipeline parallelism while
avoiding the hazards associated with each.

Task parallelism refers to pairs of actors that are on different
parallel branches of the original stream graph, as written by the
programmer.  That is, the output of each actor never reaches the input
of the other.  In stream programs, task parallelism reflects logical
parallelism in the underlying algorithm.  It is easy to exploit by
mapping each task to an independent processor and splitting or joining
the data stream at the endpoints (see Figure~\ref{fig:exemodel}b).
The hazards associated with task parallelism are the communication and
synchronization associated with the splits and joins.  Also, as the
granularity of task parallelism depends on the application (and the
programmer), it is not sufficient as the only source of parallelism.

Data parallelism refers to any actor that has no dependences between
one execution and the next.  Such ``stateless'' actors\footnote{A
stateless actor may still have read-only state.}  offer unlimited data
parallelism, as different instances of the actor can be spread across
any number of computation units (see Figure~\ref{fig:exemodel}c).
However, while data parallelism is well-suited to vector machines, on
coarse-grained multicore architectures it can introduce excessive
communication overhead.  Previous data-parallel streaming
architectures have focused on designing a special memory hierarchy to
support this communication~\cite{imagine03ieee}.  However, data
parallelism has the hazard of increasing buffering and latency, and
the limitation of being unable to parallelize actors with state.

Pipeline parallelism applies to chains of producers and consumers that
are directly connected in the stream graph.  In our previous
work~\cite{streamit-asplos}, we exploited pipeline parallelism by
mapping clusters of producers and consumers to different cores and
using an on-chip network for direct communication between actors (see
Figure~\ref{fig:exemodel}d).  Compared to data parallelism, this
approach offers reduced latency, reduced buffering, and good locality.
It does not introduce any extraneous communication, and it provides
the ability to execute any pair of stateful actors in parallel.
However, this form of pipelining introduces extra synchronization, as
producers and consumers must stay tightly coupled in their execution.
In addition, effective load balancing is critical, as the throughput
of the stream graph is equal to the minimum throughput across all of
the processors.

In this paper, we describe a robust compiler system that leverages the
right combination of task, data, and pipeline parallelism to achieve
good multicore performance across a wide range of input programs.
Because no single type of parallelism is a perfect fit for all
situations, a unified approach is needed to obtain consistent results.
Using the StreamIt language as our input and targeting the 16-core Raw
architecture, our compiler demonstrates a mean speedup of 11.2x over a
single-core baseline; 7 out of 12 benchmarks speedup by over 12x.
This also represents a 1.84x improvement over our previous
work~\cite{streamit-asplos}.

As part of this effort, we have developed two new compiler techniques
that are generally applicable to any coarse-grained multicore
architecture.  The first technique leverages data parallelism, but
avoids the communication overhead by first increasing the granularity
of the stream graph.  Using a program analysis, we fuse actors in the
graph as much as possible so long as the result is stateless.  Each
fused actor has a significantly higher computation to communication
ratio, and thus incurs significantly reduced communication overhead in
being duplicated across cores.  To further reduce the communication
costs, the technique also leverages task parallelism; for example, two
balanced task-parallel actors need only be split across half of the
cores in order to obtain high utilization.  On Raw, coarse-grained
data parallelism achieves a mean speedup of 9.9x over a single core
and 4.4x over a task-parallel baseline.
%While this technique exploits data parallelism, it also relies on the
%producer-consumer relationships evident in stream programs for
%increasing the granularity of the graph.

The second technique leverages pipeline parallelism.  However, to
avoid the pitfall of synchronization, it employs software pipelining
techniques to execute actors from different iterations in parallel.
While software pipelining is traditionally applied at the instruction
level, we leverage powerful properties of the stream programming model
to apply the same technique at a coarse level of granularity.  This
effectively removes all dependences between actors scheduled in a
steady-state iteration of the stream graph, greatly increasing the
scheduling freedom.  Like hardware-based pipelining, software
pipelining allows stateful actors to execute in parallel.
% and avoids the communication overhead of data parallelism.  
However, it avoids the synchronization overhead because processors are
reading and writing into a buffer rather than directly communicating
with another processor.  On Raw, coarse-grained software pipelining
achieves a 7.7x speedup over a single core and a 3.4x speedup over a
task-parallel baseline.

Combining the techniques yields the most general results, as data
parallelism offers good load balancing for stateless actors while
software pipelining enables stateful actors to execute in parallel.
Any task parallelism in the application is also naturally utilized, or
judiciously collapsed during granularity adjustment.  This integrated
treatment of coarse-grained parallelism leads to an overall speedup of
11.2x over a single core and 5.0x over a task-parallel baseline.

%% Other Multicores (Trademarks?):
%% RMI XLR Family of Processors
%%   (http://razamicroelectronics.com/products/xlr.htm)
%% Cavium OCTEON Processors?  cavium.com
%% Rapport KC256 www.rapportincorporated.com
%% PicoChip's PicoArray

\begin{figure}[t]
\centering
\psfig{figure=vocoder.eps,width=3in}
\vspace{-24pt}
\caption{Stream graph for a simplified subset of our Vocoder
benchmark.  Following a set of sliding DFTs, the signal is converted
to polar coordinates.  Node {\tt S2} sends the magnitude component to
the left and the phase component to the right.  In this simplified
example, no magnitude adjustment is needed.\label{fig:vocoder}}
\vspace{-12pt}
\end{figure}

\begin{figure*}[t]
\psfig{figure=time-proc-4-1.eps,width=7in}
\caption{Parallel execution models for stream programs.  Each block corresponds to a filter in the Vocoder example (Figure~\ref{fig:vocoder}).  The height of the block reflects the amount of work contained in the filter.\label{fig:exemodel}}
\end{figure*}
