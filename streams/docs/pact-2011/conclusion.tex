\section{Conclusion}

% Given the dominance of multicores and the cessation of single-core
% performance scaling, there exists an acute need for programming
% languages and software techniques that enable effective automatic
% management of parallelization.  This paper demonstrates that exposing
% sliding window computation in a programming language is critical to
% achieving robust parallelization scalability in the domain of stream
% programs.  Without such support, the data-parallelization of sliding
% window filters would require heroic effort by a compilation system.
% If data-parallelization of sliding window filters cannot be achieved,
% due to the prevalence of sliding windows, scalability is severely
% curtailed.

This paper presents a framework for data-parallelizing sliding window
computations given that they are exposed to the compilation system.  Our
framework enables scalable parallelization of sliding window
computations for multicores where previous approaches would fail
because they introduce too much communication.  When performing
data-parallelization, our techniques convert the state of the sliding
window into input items shared between the product filters in a
process termed {\it fission}.  Our technique for filter fission
communicates the minimum amount of shared state necessary for the
data-parallelization.  We present a complementary optimization termed
{\it Sharing Reduction} that alters the steady-state of a stream
program to reduce the number of items shared between product filters
of fission.  This has the effect of reducing inter-core communication.

To evaluate our techniques we have implemented them in the context of
the StreamIt compiler infrastructure~\cite{gordon-asplos06}.  We
employ 3 real-world benchmarks from the StreamIt Benchmark
Suite~\cite{streamit-suite} that include sliding window computation.
We show that our techniques are required to achieve scalable
parallelization, achieving a mean parallelization speedups over
single-core of 17x and 62.3x for a 16-core SMP and a 64-core
distributed-memory multicore, respectively.  These results represent a
6.7x mean speedup on the 16-core SMP and a 1.8x mean speedup on the
64-core distributed-memory multicore over a previously published
technique.


