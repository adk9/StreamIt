\section{Introduction}

With the emergence of data-intensive applications such as digital
film, medical imaging and geographic information systems, the
performance of next-generation systems will often depend on their
ability to process huge volumes of data.  For example, each frame of a
digital film requires approximately 2 megabytes, implying that a
fully-edited 90-minute video demands about 300 gigabytes of data for
the imagery alone~\cite{ibm-video}.  Industrial Light and Magic
reports that, in 2003, their processing pipeline output 13.7 million
frames and their internal network processed 9 petabytes of
data~\cite{ilm-interview}.  The U.S. Geological Survey had archived
over 13 million frames of photographic data by the end of 2004, and
estimates that 5 years is needed to digitize 8.6 million additional
images~\cite{usgs}.  In all of these situations, the data is highly
compressed to reduce storage costs.  At the same time, extensive
post-processing of the images and videos is often required.  As such
processing logically operates on the uncompressed format, the usual
practice is to decompress and re-compress the data whenever it needs
to be modified.

In order to accelerate the process of editing compressed data,
researchers have identified specific transformations that can be
mapped into the compressed domain---that is, they can operate directly
on the compressed data format rather than on the uncompressed
format~\cite{surveys-folder}.  In addition to averting the cost of the
decompression and re-compression, such techniques greatly reduce the
total volume of data processed, thereby offering large savings in both
execution time and memory footprint.

However, existing techniques for operating directly on compressed data
are limited to special-purpose transformations on specific data
formats.  The process of mapping a new operation to the compressed
domain is so ad-hoc and painstaking that each mapping warrants a new
publication.  For DCT-based spatial compression formats (JPEG,
Motion-JPEG), researchers have developed algorithms for
resizing\cite{image/dugad01,imag/mukerjee}, edge
detection\cite{11-shen,12-shen}, image
segmentation\cite{image/feng03}, shearing and rotating inner
blocks\cite{image/++shen98}, and arbitrary linear combinations of
pixels~\cite{smith98}.  Techniques extending to DCT-based temporal
compression (MPEG) include captioning\cite{video/nang00}, distortion
detection\cite{video/dorai99}, transcoding~\cite{video/smith98}, and
others~\cite{surveys/wee02}.  For run-length encoded images,
algorithms have been devised for efficient transpose and
rotation~\cite{shoji95,images/99-misra}.  A compressed audio format
has been invented that allows direct modification of pitch and
playback speed~\cite{audio/levine}.

This paper presents the first technique for automatically mapping
flexible, user-defined computations into the compressed domain.  While
this problem has been regarded as too unwieldy for general-purpose
source languages such as C, we make the problem tractable by using a
streaming model of computation.  Stream programming captures the
essential functionality needed by image, video, and signal processing
applications while exposing the flow of data to the compiler.  Our
transformation targets lossless compression formats based on LZ77,
which includes popular standards such as Apple Animation, PNG, and
gzip.  Lossless compression is widely used in computer animation and
digital video editing in order to avoid accumulating compression
artifacts.  By providing a programmable solution, our technique
enables a large class of transformations to be customized by the user
and directly applied to the compressed data.

The key idea behind our technique can be understood in simple terms.
Consider a data stream that is run-length encoded, that is, it
indicates that a given value is repeated $n$ times.  Consider further
that the user program processes only one value at a time---for
example, it applies a uniform adjustment to each pixel.  Since the
input stream contains only one distinct value, the program only needs
to be invoked once; the multiplicity of $n$ can be directly copied
from the input stream to the output stream without performing any
additional computation.  Our technique generalizes this notion in two
important ways.  First, it targets LZ77, a sliding-window compression
algorithm that is much more powerful than run-length encoding.
Second, it supports a broad class of user-defined modules, which may
contain multiple input streams and process multiple values at a time.

We have implemented our transformations as hand-coded plugins for
MEncoder and Blender, two popular video editing tools.  We evaluated
the compressed processing technique for a variety of pixel
transformations (brightness, contrast, color inversion) as well as
video compositing (overlays and mattes).  Using a suite of 12 videos
(screencasts, computer animations, digital television content) in
Apple animation format, computing directly on compressed data offers a
speedup roughly proportional to the compression factor.  For pixel
transformations, speedups range from 3.1x to 235x, with a median of
19x; for video compositing, speedups range from 1.0x to 35x, with a
median of 7.4x.

In the general case, our compressed processing technique may need to
partially decompress the input data, thereby producing an output file
that is larger than the input file.  Even if the output size remains
constant, the output may benefit from an additional re-compression
step if new redundancy is introduced during the transformation (for
example, increasing image brightness can whiteout parts of the image).
This effect turns out to be minor in the case of our experiments.  For
pixel transformations, output sizes are within 0.1\% of input sizes
and usually within 10\% (median) of a full re-compression.  For video
compositing, output files maintain sizable compression ratios (1.0x to
44x) and benefit by 56\% (median) by an additional recompression.

To summarize, this paper makes the following contributions:
\begin{itemize}

\item An automatic mapping from an arbitrary stream program, written
in the synchronous dataflow model, to operate directly on lossless
LZ77-compressed data (Sections 2-3).

\item An analysis of popular lossless compression formats and the
opportunities for direct processing on each (Section 4).

\item An empirical demonstration that computing on compressed data can
speedup realistic operations in popular video editing tools.  Across
all benchmarks, the median speedup is 16x (Section 5).

\end{itemize}

The paper concludes with related work (Section 6) and conclusions
(Section 7).
