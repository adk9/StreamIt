\section{Introduction}

With the emergence of data-intensive applications such as digital
film, medical imaging and geographic information systems, the
performance of next-generation systems will often depend on their
ability to process huge volumes of data.  For example, each frame of a
digital film requires approximately 2 megabytes, implying that a
fully-edited 90-minute video demands about 300 gigabytes of data for
the imagery alone~\cite{ibm-video}.  Industrial Light and Magic
reports that, in 2003, their processing pipeline output 13.7 million
frames and their internal network processed 9 petabytes of
data~\cite{ilm-interview}.  YouTube manages about 45 terabytes of
video data~\cite{wsj-youtube}, with 65,000 videos uploaded
daily~\cite{youtube}.  Microsoft TerraServer holds upwards of 22
terabytes of image data, serving 69 gigabytes per
day~\cite{terraserver}.  In all of these situations, the data is
highly compressed to reduce storage costs.  At the same time,
extensive post-processing is often required for adding captions,
watermarking, resizing, compositing, adjusting colors, converting
formats, and so on.  As such processing logically operates on the
uncompressed format, the usual practice is to decompress and
re-compress the data whenever it needs to be modified.

%The U.S. Geological Survey had archived over 13 million frames of
%photographic data by the end of 2004, and estimates that 5 years is
%needed to digitize 8.6 million additional images~\cite{usgs}.

In order to accelerate the process of editing compressed data,
researchers have identified specific transformations that can be
mapped into the compressed domain---that is, they can operate directly
on the compressed data format rather than on the uncompressed
format~\cite{chang95survey,mandal95survey,smith95survey,wee02survey}.
In addition to averting the cost of the decompression and
re-compression, such techniques greatly reduce the total volume of
data processed, thereby offering large savings in both execution time
and memory footprint.

However, existing techniques for operating directly on compressed data
are limited to special-purpose transformations on specific data
formats.  The process of mapping a new operation to the compressed
domain is so ad-hoc and painstaking that each mapping warrants a new
publication.  For DCT-based spatial compression formats (JPEG,
Motion-JPEG), researchers have developed algorithms for
resizing~\cite{dugad01,mukherjee02}, edge
detection~\cite{shen96,shen96b}, image segmentation~\cite{feng03},
shearing and rotating inner blocks~\cite{shen98}, and arbitrary linear
combinations of pixels~\cite{smith96b}.  Techniques extending to
DCT-based temporal compression (MPEG) include captioning~\cite{nang00},
reversal~\cite{vasudev98}, distortion detection~\cite{dorai00},
transcoding~\cite{smith98}, and others~\cite{wee02survey}.  For
run-length encoded images, algorithms have been devised for efficient
transpose and rotation~\cite{misra99,shoji95}.  A compressed audio
format has been invented that allows direct modification of pitch and
playback speed~\cite{levine98}.

This paper presents the first technique for automatically mapping
flexible, user-defined computations into the compressed domain.  While
this mapping would require a heroic program analysis for arbitrary C
programs, we make the problem tractable by using a streaming model of
computation.  Stream programming captures the essential functionality
needed by image, video, and signal processing applications while
exposing the flow of data to the compiler.  Our transformation targets
LZ77, a lossless encoding utilized by ZIP, and immediately applies to
simpler formats such as Apple Animation, Microsoft RLE, and Targa.
Lossless compression is widely used in computer animation and digital
video editing in order to avoid accumulating compression artifacts.
By providing a programmable solution, our technique enables a large
class of transformations to be customized by the user and directly
applied to the compressed data.

The key idea behind our technique can be understood in simple terms.
Consider a data stream that is run-length encoded, that is, it
indicates that a given value is repeated $n$ times.  Consider further
that the user program processes only one value at a time---for
example, it applies a uniform adjustment to each pixel.  Since the
input stream contains only one distinct value, the program only needs
to be invoked once; the multiplicity of $n$ can be directly copied
from the input stream to the output stream without performing any
additional computation.  Our technique generalizes this notion in two
important ways.  First, it targets LZ77, a popular sliding-window
compression algorithm that subsumes run-length encoding.  Second, it
supports a broad class of user-defined filters, which may contain
multiple input streams and process multiple values at a time.

We manually applied our transformations to produce optimized plugins
for two popular video editing tools, MEncoder and Blender.  We
evaluated the compressed processing technique for a variety of pixel
transformations (brightness, contrast, color inversion) as well as
video compositing (overlays and mattes).  Using a suite of 12 videos
(screencasts, computer animations, digital television content) in
Apple animation format, computing directly on compressed data offers a
speedup roughly proportional to the compression factor.  For pixel
transformations, speedups range from 3.1x to 235x, with a median of
19x; for video compositing, speedups range from 1.0x to 35x, with a
median of 7.4x.

In the general case, our compressed processing technique may need to
partially decompress the input data, thereby producing an output file
that is larger than the input file.  Even if the output size remains
constant, the output may benefit from an additional re-compression
step if new redundancy is introduced during the transformation (for
example, increasing image brightness can whiteout parts of the image).
This effect turns out to be minor in the case of our experiments.  For
pixel transformations, output sizes are within 0.1\% of input sizes
and often within 10\% (median case) of a full re-compression.  For
video compositing, output files maintain a sizable compression ratio
of 8.8x (median) while full re-compression results in a ratio of 13x
(median).

To summarize, this paper makes the following contributions:
\begin{itemize}

\item An algorithm for mapping an arbitrary stream program, written in
the synchronous dataflow model, to operate directly on lossless
LZ77-compressed data (Sections 2-3).

\item An analysis of popular lossless compression formats and the
opportunities for direct processing on each (Section 4).

\item An empirical demonstration that computing on compressed data can
speedup realistic operations in popular video editing tools.  Across
our benchmarks, the median speedup is 16x (Section 5).

\end{itemize}

The paper concludes with related work (Section 6) and conclusions
(Section 7).
