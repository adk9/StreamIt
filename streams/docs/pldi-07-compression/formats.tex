\section{Supported File Formats}
\label{sec:formats}

As LZ77 refers to a compression algorithm rather than a complete
compression format, there are additional factors to consider in
mapping computations to real-world image and video codecs.  Some
codecs are a subset of LZ77, utilizing only run-length encoding or a
fixed window size; these are supported very efficiently by our
technique.  Others are a superset of LZ77, incorporating additional
techniques such as delta coding or Huffman coding; these may incur
additional processing overhead.

In the following sections, we describe the practical considerations
involved in targeting various compression formats with our technique.
Formats are ordered by approximate goodness of the achievable mapping.

\subsection{High-Efficiency Mappings}
\label{sec:formats-good}

All of the formats in this category can be considered to be subsets of
LZ77.

\paragraph{Apple Animation.}  
The Apple Animation codec (which forms the basis for our experimental
evaluation) is supported as part of the Quicktime MOV container
format.  It serves as an industry standard for exchanging computer
animations and digital video content before they are rendered to lossy
formats for final distribution~\cite[p.~106]{adobe-anim}\cite[p.~284]{harrington-anim} \cite[p.~367]{long-anim}\cite[p.~280]{pogue-anim}.

The Animation codec represents a restricted form of LZ77 in which
repeat distances are limited to two values: a full frame or a single
pixel.  A repeat across frames indicates that a stretch of pixels did
not change from one frame to the next, while a repeat across pixels
indicates that a stretch of pixels has the same color within a frame.
% mention bit depths?

\paragraph{Flic Video.}
Flic Video files (FLI/FLC) were originally produced by Autodesk
Animator and are still supported by many animation packages today.
Their compression of frame data is almost identical to Apple
Animation.

\paragraph{Microsoft RLE.}
Microsoft RLE compression can appear in both BMP images and AVI
animations.  Apart from bit-depth and formatting details, its
capabilities are identical to Apple Animation; it can perform
run-length encoding within a frame, and can skip over pixels to
exploit inter-frame redundancy.

\paragraph{Targa.}
The Truevision Targa (TGA) format is a simple image format that is
widely used to render frame sequences in the computer animation and
video industries.  The format includes an optional RLE compression
stage, making it a good target for our technique.

%% \paragraph{PXY.}
%% The pxy format is a research-based image format designed to support
%% efficient transpose and rotation of black-and-white
%% images~\cite{shoji95}.  It consists of the series of $(x,y)$
%% coordinates at which the image changes color during a horizontal scan.
%% As this information can be converted to a run-length encoding, it can
%% also be targetted by our technique.

\subsection{Medium-Efficiency Mappings}
\label{sec:formats-med}

While the formats in this category utilize an encoding that is
compatible with LZ77, they incur extra overhead because the data is
reorganized prior to the compression stage.

\paragraph{Planar RGB.}
The Planar RGB video format is supported by Apple Quicktime files.  It
utilizes run-length encoding for pixels within a frame, with partial
support for expressing inter-frame repeats (only the end of lines can
be skipped).  The red, green, and blue planes are encoded separately
in order to increase compression.  For user transformations that need
to process red, green, and blue values together, this introduces
additional alignment overhead when applying our technique.

\paragraph{OpenEXR.}
OpenEXR is an emerging image format (backed by Industrial Light and
Magic) for use in digital film.  It offers several compression
options, including run-length encoding, zip, and wavelet-based
compression.  However, in run-length encoding mode, the low and high
bytes of the pixels are separated and encoded as separate run-length
sequences; this enables pixels with variations in the low bytes to
nonetheless benefit from compression of the high bytes.  As most user
transformations would utilize the entire bit-width of the pixel, our
technique suffers additional alignment overhead in processing these
files.

\subsection{Low-Efficiency Mappings}
\label{sec:formats-bad}

The formats in this category are supersets of LZ77.  While our
technique could offer some gains in exploiting the LZ77 compression,
it would have to undo any compression sitting on top of LZ77 and
offers limited benefit for filters (as in PNG) applied underneath
LZ77.

\begin{table*}[t]
\vspace{-1\baselineskip}
\psfig{figure=table-benchmarks.eps,width=7.1in}
\caption{Characteristics of the video workloads.
\protect\label{tab:videos}}
\end{table*}

\paragraph{DEFLATE.}
DEFLATE is a general-purpose algorithm that provides all of the
compression for popular formats such as ZIP and GZIP.  The algorithm
consists of a full LZ77 encoder followed by Huffman coding, which
resizes the symbols in the stream to match their usage frequencies.
In targeting ZIP or GZIP with our transformations, we would first
have to undo the Huffman coding (unless the application simply
reordered data, in which case the coding could remain intact).  Though
Huffman decoding is a lightweight lookup operation, it would also
increase the memory footprint.  In addition, as DEFLATE's LZ77
algorithm operates on individual bytes, there may be an exaggerated
alignment cost if the application operates on a larger word size.

\paragraph{TSCC.}
The TechSmith Screen Capture Codec is very similar to Microsoft RLE,
except that the final output is further compressed using DEFLATE.
Thus, any overheads incurred by our technique on DEFLATE also extend
to TSCC.

\paragraph{PNG.}
The PNG image format also relies on DEFLATE to compress the pixels in
the image.  However, before running DEFLATE, the pixels are usually
filtered with a delta encoding; each pixel is replaced with the
difference between its value and a predicted value, where the
prediction is a linear combination of neighboring pixels.  While
program segments that compute a linear function~\cite{aalamb} could
perhaps be mapped to this compressed format, our current technique
only applies if the delta encoding is turned off.  Even in this
scenario, there is a large amount of overhead due to the Huffman
coding in DEFLATE.
