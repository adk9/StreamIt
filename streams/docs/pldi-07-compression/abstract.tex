Due to the high data rates involved in audio, video, and signal
processing applications, it is imperative to compress the data to
decrease the amount of storage used.  Unfortunately, this implies that
any program operating on the data needs to be wrapped by a
decompression and re-compression stage.  Re-compression can incur
significant computational overhead, while decompression swamps the
application with the original volume of data.

In this paper, we present a program transformation that greatly
accelerates the processing of compressible data.  Given a program that
operates on uncompressed data, we output an equivalent program that
operates directly on the compressed format.  Our formulation is based
on LZ77, a lossless compression algorithm utilized by ZIP, and
immediately applies to simpler formats such as Apple Animation,
Microsoft RLE, and Targa.  Our transformations rely on the streaming
model of computation, which exposes the flow of data in the
applications.

To evaluate the impact of our transformations, we implemented plugins
for two digital video editing tools: MEncoder and Blender.  For common
operations such as color adjustment and video compositing, computing
directly on compressed data offers a speedup roughly proportional to
the overall compression ratio.  For our benchmark suite of 12 videos
in Apple Animation format, speedups range from 1.0x to 235x, with a
median of 16x.
