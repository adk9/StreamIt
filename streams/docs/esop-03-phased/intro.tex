\section{Introduction}

Applications structured around some notion of a ``stream'' are
becoming increasingly important and widespread. There is evidence
that streaming media applications are already consuming most of
the cycles on consumer machines \cite{Rix98}, and their use is
continuing to grow.  In the embedded domain, applications for
hand-held computers, cell phones, and DSP's are centered around
stream of voice or video data.  The stream abstraction is also
fundamental to high-performance applications such as intelligent
software routers, cell phone base stations and HDTV editing
consoles.

Despite the prevalence of these applications, there is
surprisingly little language and compiler for practical,
large-scale stream programming.  The notion of a stream as a
programming abstraction has been around for decades \cite{SICP},
and a number of special-purpose stream languages have been
designed (see \cite{survey97} for a review). Many of these
languages and representations are elegant and theoretically sound,
but they often are not designed with neither practical use nor
efficient in mind. Consequently most programmers turn to
general-purpose languages such as C or C++ to implement stream
programs.

\begin{comment}
There are two reasons that general-purpose languages are
inappropriate for stream programming.  Firstly, they are a
mismatch for the application domain.  That is they do not provide
a natural or intuitive representation of streams thereby having a
negative effect on readability, robustness, and programmer
productivity.  Furthermore, general-purpose languages do not
communicate well the inherent parallelism of stream computations.
 Secondly, general-purpose languages are a mismatch for the
emerging class of grid-based architectures
\cite{smartmemories,rawshort,trips}.
\end{comment}

{\StreamIt} is a language and compiler specifically designed for
modern stream programming.  The main goal of the {\StreamIt}
langauge is to provide high-level stream abstractions that improve
programmer productivity and program robustness within the
streaming domain.  At the same time, {\StreamIt} compiler aims to
perform stream-specific optimizations to achieve the performance
of an expert programmer.

In order to achieve these goals, {\StreamIt} provides a number of
features, designed to allow the programmer to easily and naturally
express the required computation, while keeping the program easy
to analyze by a compiler: all {\StreamIt} streaming constructs are
single-input, single-output; all computation happens in
{\filters}; data is passed around between {\filters} using three
streaming constructs: {\pipeline}, which allows stacking of
{\filters} one after another, {\splitjoin}, which allows splitting
and joining of data amongst multiple streams, and
{\feedbackloop}, which constructs cyclic streams. {\StreamIt}
programs are built hierarchically, using {\filters} as the inner
building blocks and {\pipelines}, {\splitjoins} and
{\feedbackloops} as outer building blocks.\footnote{We use the
word {\stream} to refer to any instance of a {\filter},
{\pipeline}, {\splitjoin}, or {\feedbackloop}.} In {\StreamIt},
every {\filter} must declare the rate at which it processes data:
how much data is consumed and produced on every invocation of the
{\filter}'s {\work} function. This model of data passing is called
Synchronous Data Flow ({\SDF}).

\begin{comment}
In addition to {\SDF}, {\StreamIt} allows the programmer to pass
data between {\filters} in an asynchronous manner, similar to a
combination of message passing and function calls. Timing of such
data delivery is expressed in terms of amount of information
wavefronts - the programmer can specify a delay between message
delivery and destination {\filter}'s processing of data currently
being produced or consumed by the source {\filter}.  Such timing
mechanism introduces latency and buffering constraints on
execution of {\StreamIt} programs.
\end{comment}

Using the features present in {\StreamIt}, the programmer can
express complex algorithms and computation models.  One of the
challenges faced in compiling {\StreamIt} programs is scheduling.
Since {\StreamIt} uses {\SDF} computation model it is possible to
schedule the order of execution of {\filters} at compile time.
Scheduling {\SDF} programs presents a difficult challenge to the
compiler:  as the complexity of the program grows, the amount of
memory required to execute the program increases.  This increase
comes from two sources: the schedule size increases, as well as
amount of data needed for buffering increases.  These two sources
are closely coupled. There exist tradeoffs between the schedule
size and the buffer size. The smallest schedules (Single
Appearance schedules) require buffer space may in the worst case
grow exponentially with number of components. The general rule is
that very small schedules result in very large buffering
requirements. On the other hand, reducing buffering requirements
can lead to substantially larger schedule sizes.

\begin{comment}
This problem is further complicated by message latency constraints
placed on the program by the programmer.  While {\StreamIt}
programs are meant to provide relatively lax latency requirements,
it is possible to write programs with latency constraints so tight
that very few valid schedules exist. Finding these schedules is a
challenging task.
\end{comment}

This paper develops a novel technique for scheduling execution of
{\StreamIt} programs. The technique takes advantage of structure
of {\StreamIt} to create compact schedules. These schedules will
be purely hierarchical in nature. The concept of a phased schedule
(which is also hierarchical) will be introduced to reduce the
requirement for buffering data between {\filters} without overly
increasing the size of the schedule. Phased scheduling is meant to
to reduce the severity of tradeoffs the compiler needs to make
between schedule size and buffer requirements, thus allowing
compilation of programs to more resource constrained
architectures. It can also help to allocate resources more
appropriately between resources necessary for execution of the
program and resources necessary to store data.

Hierarchical schedules are also useful, because they allow for
separate compilation of program components. This enables creation
of standardized libraries and their distribution in binary form,
rather than source code. This ability may become important as
streaming languages become more widely used for larger
applications.

\begin{comment}
The contributions of this thesis are:
\begin{itemize} \item
hierarchical scheduling of streaming application, a concept
enabled by {\StreamIt} language, \item first formal handling of
{\SDF} graphs with peeking, \item novel phasing scheduling
technique, \item a minimal latency schedule using hierarchical
phases, \item novel {\SDF} program abstraction called the
information buffering model that simplifies information latency
analysis, \item a solution to scheduling of {\StreamIt} programs
with latency constraints.
\end{itemize}
\end{comment}

The remainder of this paper is organized as follows: Section
\ref{sec:streamit} describes relevant {\StreamIt} constructs;
Section \ref{chpt:sched-basic} explains basic concepts in
scheduling {\StreamIt} graphs; chapter \ref{chpt:hierarchical}
describes a hierarchical single-appearance scheduling technique;
Section \ref{chpt:phased} describes phasing scheduling technique
and presents minimum-latency scheduling algorithm; Section
\ref{chpt:results} presents results of scheduling applications
using the phased scheduling algorithm.
