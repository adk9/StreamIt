\section{Initialization for Peeking}

This section will develop a simple algorithm for constructing an
initialization schedule.  The algorithm developed here will use
hierarchy in a similar way section \ref{sec:calc-min-steady} used
hierarchy to compute steady schedules.  Schedules computed here
are not minimal, but remain reasonably small.

The amount of data required for initialization by stream $s$ will
be denoted by $init^{pop}_s$. The amount of data produced by
initialization schedule will be denoted by $init^{push}_{s}$. This
hierarchical technique will simplify calculations, but may be
unable to compute an initialization schedule for {\feedbackloops}
that would otherwise be possible to schedule.  An algorithm for
computing a valid initialization schedule for any stream that can
be initialized will be presented in Section \ref{sec:min-latency}.

\subsubsection{Notation for Initialization Schedules}

An initialization schedule for a stream $s$ is a set $I_s$ with
elements $I_s = \{c, u\}$.  The set includes a vector $c$ which
holds values $[e_s^{init}, o_s^{init}, u_s^{init}]$ for the
initialization schedule of the stream and a vector $u$ which
stores how many times each of the direct children of $s$ will
execute their steady state.  The elements are denoted $I_{s,S}$,
$I_{s,c}$ and $I_{s,u}$.

\subsubsection{\filter}

Since {\filters} do not buffer any data, they do not require
initialization schedules.  They may, however, require some data
for their initialization.  For a {\filter} $f$, this amount of data
is $e_f - o_f$, as explained above.  {\filters} will not produce any
data during initialization.

An initialization schedule of a {\filter} is thus $I_f = \{[e_f-o_f,
0, 0], \{\}\}$.

\subsubsection{\pipeline}

\begin{algorithm}
\caption{Single Appearance Initialization Counts for a {\pipeline}}
{\bf }($p$). Given a {\pipeline} $p$, calculate how many times each
child needs to execute its steady state to initialize the
{\pipeline} for peeking.
\begin{algorithmic}
\STATE compute $S_{p_{n-1}}$; $u_{n-1} = 0$ \FOR{$i=n-2$ downto 0}
\STATE compute $S_{p_i}$ \STATE $u_i = {I_{p_{i+1}, c_0} + u_{i+1}
* S_{p_{i+1}, c_1} - I_{p_i, c_2} \over I_{p_i, c_2}}$ \ENDFOR
\STATE $I_p = \{[I_{p_0, c_0} + (u_0 - 1) * S_{p_0, c_0}, u_0 *
S_{p_0, c_0}, I_{p_{n-1}, c_2}] ,u\}$
\end{algorithmic}
\end{algorithm}

Let $s_i$ denote $i$th child stream of the {\pipeline}.  Also, let
$m_i$ denote the number of times the $i$th child will execute its
steady state during execution of {\pipeline}'s initialization
schedule.

The $i$th stream needs to provide at least $init^{pop}_{s_{i+1}} +
m_{i+1} * pop_{s_{i+1}}$ data for {\pipeline}'s next child.  For the
last child $m_{n-1} = 0$, thus it only needs
$init^{pop}_{s_{n-1}}$ data to initialize.  Knowing the amount of
data required by the last child, we can compute how much data all
other children require, and thus compute how many times all other
children need to execute their steady state schedules.

In order to provide enough data for $s_{i+1}$, $s_{i}$ is going to
execute its initialization schedule, producing
$init^{push}_{s_{i}}$ data items, and then it is going to execute
its steady state schedule $m_i$ times to provide any additional
required data.  The $i$th child will need to execute its steady
state schedule $m_i = \left\lceil { init^{pop}_{s_{i+1}} + m_{i+1}
* pop_{s_{i+1}} - init^{push}_{s_i} \over{push_{s_i}}}
\right\rceil$ times.

Finally, since the first child of the {\pipeline} is directly
receiving the data that is meant to enter the {\pipeline}, it
follows that $init^{pop}_{p} = init^{pop}_{s_0} + m_0 *
pop_{s_0}$. Similarly, $init^{push}_{pipeline} =
init^{push}_{s_{i-1}}$.

Once all $m_i$s are known, the initialization schedule is
constructed according to the following algorithm:

\begin{singlespace}
\begin{verbatim}
initialization schedule (p) = empty for i = 0 .. n-1
    initialization schedule (p) += initialization schedule (s_i)
    for j = 1 .. m_i
        initialization schedule (p) += steady schedule (s_i)
    end for
end for
\end{verbatim}
\end{singlespace}

Notice, that since the children of the {\pipeline} use their steady
schedules in order to push extra data into the buffers, they may
be pushing more data than required, thus causing the pipeline to
consume more data for its initialization schedule than absolutely
required.  This over-estimation will propagate up with each
{\StreamIt} stream that contains this {\pipeline}.

We again use the example from Figure \ref{fig:steady-state} (a).
Since all children of the {\pipeline} are {\filters}, they do not have
any initialization schedules, and their $init^{push}_{s_i} = 0$.
By inspection we obtain

\begin{displaymath}
\begin{array}{rl}
init^{pop}_A = & 1-1=0 \\
init^{pop}_B = & 3-2 = 1 \\
init^{pop}_C = & 2 - 2 = 0 \\
init^{pop}_D = & 5-3=2 \\
\\
m_3 = & 0 \\
m_2 = & \left \lceil 2 + 3 * 0 - 0 \over 1 \right \rceil = 2\\
m_1 = & \left \lceil 0 + 2 * 2 - 0 \over 3 \right \rceil = 2\\
m_0 = & \left \lceil 1 + 2 * 2 - 0 \over 3 \right \rceil = 2\\
\end{array}
\end{displaymath}

Thus the initialization schedule is $(AABBCC)$.  Also,
$init^{pop}_p = 0 + 2 * 1 = 2$ and $init^{push}_p = 0$.

\subsubsection{\splitjoin}

\begin{algorithm}
\caption{Single Appearance Initialization Counts for a {\pipeline}}
{\bf }($p$). Given a {\pipeline} $p$, calculate how many times each
child needs to execute its steady state to initialize the
{\pipeline} for peeking.

\begin{algorithmic}

\STATE compute $S_{p_{n-1}}$; $u_{n-1} = 0$

\FOR{$i=n-2$ downto 0}

\STATE compute $S_{p_i}$

\STATE $u_i = {I_{p_{i+1}, c_0} + u_{i+1} * S_{p_{i+1}, c_1} -
I_{p_i, c_2} \over I_{p_i, c_2}}$

\ENDFOR

\STATE $I_p = \{[I_{p_0, c_0} + (u_0 - 1) * S_{p_0, c_0}, u_0 *
S_{p_0, c_0}, I_{p_{n-1}, c_2}] ,u\}$

\end{algorithmic}
\end{algorithm}

Initializing a {\splitjoin} is done by executing the {\splitter}
enough times to provide enough data for all the children to
initialize.  The {\joiner} is never run, so there may be some data
buffered between the children streams and the {\joiner}.

Let $s_i$ denote the $i$th child stream of the {\splitjoin},
$w_{s,i}$ denote the amount of data pushed by the {\splitter} of the
{\splitjoin} towards the $i$th child during an execution of the
{\splitter}, and $w_{j,i}$ denote the amount of data popped by the
{\joiner} from the $i$th child during its execution.

The {\splitter} must execute enough times to provide every child of
the {\splitjoin} with enough data to initialize.  For $i$th child,
the required number of executions is $\left\lceil init^{pop}_{s_i}
\over w_{s,i} \right\rceil$.  The {\splitter} needs to execute the
maximum amount of times required by any the children, that is
$m_{split} = \max \left\lceil init^{pop}_{s_i} \over w_{s,i}
\right\rceil, \forall i \in \{0,\dots,n-1\}$.  Thus the amount of
data required for initialization of a {\splitjoin} is
$init^{pop}_{sj} = pop_{split} * m_{split}$. Since the joiner will
never get executed, $init^{push}_{sj} = 0$.

Once $m_{\splitter}$ is has been calculated, the initialization
schedule is constructed according to the following algorithm:

\begin{singlespace}
\begin{verbatim}
initialization schedule (sj) = empty

for i = 1 .. m_{splitter}
  initialization schedule (sj) += {\splitter} execution
end for

for i = 0 .. n-1
    initialization schedule (sj) += initialization schedule (s_i)
end for
\end{verbatim}
\end{singlespace}

The following computes an initialization schedule for the example
{\splitjoin}  from Figure \ref{fig:steady-state} (b). Both children
of the {\splitjoin} are {\filters}, thus they do not have
initialization schedules and their $init^{push}_{s_i} = 0$. By
inspection, we obtain

\begin{displaymath}
\begin{array}{rl}
init^{pop}_A = & 2-2=0 \\
init^{pop}_B = & 3-2 = 1 \\
\\
m_{split} = & \max({0 \over 2}, {1 \over 1}) = \max (0, 1) = 1
\end{array}
\end{displaymath}

Thus the initialization schedule for this {\splitjoin} is simply
$({\splitter})$.  We also get $init^{pop}_{sj} = 1
* 3 = 3$ and $init^{push}_{sj} = 0$.

\subsubsection{\feedbackloop}

The final {\StreamIt} component left to initialize is the
{\feedbackloop}. Let $s_b$ be the body stream of the {\feedbackloop},
and $s_l$ be the feedback path stream of the {\feedbackloop} (thus
$push_{s_b}$ is the amount of data pushed by $s_b$ per steady
state execution, etc).  Let $m_{s_b}$ be the number of times $s_b$
needs to be executed in order to properly initialize the
{\feedbackloop} and $m_{s_l}$ be the number of times $s_l$ needs to
be executed to initialize the {\feedbackloop}.

Initialization for the {\feedbackloop} is calculated in a similar
way to initialization of a {\pipeline}.  Since the initial data is
inserted into the buffer between the loop stream and the {\joiner},
it follows that calculation of initialization requirements should
start from the loop stream as a "last" element - it will be
execute last in the initialization schedule. That means that the
loop stream will execute its initialization schedule, but it will
not execute its steady schedule, namely $m_{s_l} = 0$.  Thus we
have

\begin{displaymath}
\begin{array}{rl}
m_{split} & = \left\lceil {init^{pop}_{s_l} \over w_{s,1}} \right\rceil \\
m_{s_b} & = \left\lceil pop_{split} * m_{split} -
init^{push}_{s_b} \over push_{s_b} \right\rceil \\
m_{join} & = \left\lceil init^{pop}_{s_b} + pop_{s_b} * m_{s_b}
\over { push_{s_b}} \right\rceil
\end{array}
\end{displaymath}

We also calculate the overall consumption and production of data
during initialization of the {\splitjoin}

\begin{displaymath}
\begin{array}{rl}
init^{pop}_{fl} & = w_{j,0} * m_{join} \\
init^{push}_{fl} & = w_{s, 0} * m_{split}
\end{array}
\end{displaymath}

Now the schedule can be constructed by using the following
algorithm:

\begin{singlespace}
\begin{verbatim}
initialization schedule (fl) = empty for i = 1 .. m_{join}
  initialization schedule (fl) += {\joiner} execution
end for initialization schedule (fl) += initialization schedule
(s_b) for i = 0 .. m_{s_b}
    initialization schedule (fl) += steady state schedule (s_b)
end for for i = 1 .. m_{split}
  initialization schedule (fl) += {\splitter} execution
end for initialization schedule (fl) += initialization schedule
(s_l)
\end{verbatim}
\end{singlespace}

It is important to note, that this initialization schedule is not
legal if there isn't enough data in the buffer between the {\joiner}
and the loop stream, that is if $delay_{fl} < w_{j,1} *
m_{joiner}$.  This condition being true, does not mean, however,
that no initialization schedule exists for the particular
{\feedbackloop}.  The reason for this is that executing entire
steady state schedules of the body stream may consume more data
than is actually necessary to provide enough data for the loop
stream to receive $init^{pop}_{s_l}$ data.  Also the
$init^{pop}_{s_b}$ and $init^{pop}_{s_l}$ values may be larger
than necessary, because they also may use their children's steady
schedules in initialization.

An algorithm for initializing any legal stream structure is
presented in Section \ref{sec:min-latency}

The following computes an initialization schedule for the example
{\feedbackloop} from Figure \ref{fig:steady-state} (c). Both
children of the {\feedbackloop} are {\filters}, thus they do not have
initialization schedules and their $init^{push}_{s} = 0$. By
inspection, we obtain

\begin{displaymath}
\begin{array}{rl}
init^{pop}_B = & 3-2 = 1 \\
init^{pop}_L = & 7-5 = 2 \\
\\
m_{split} = & \left\lceil 2 \over 3 \right\rceil = 1 \\
m_{B} = & \left\lceil 3 * 1 - 0 \over 1 \right\rceil = 3 \\
m_{join} = & \left\lceil 1 + 2 * 3 \over 5 \right\rceil = 2 \\
\end{array}
\end{displaymath}

Thus the initialization schedule for this {\feedbackloop} is
$({\joiner}\ {\joiner}\ BBB\ {\splitter})$.  We also get
$init^{pop}_{fl} = 2 * 2 = 4$ and $init^{push}_{fl} = 3 * 1 = 3$.

\section{Steady Schedules}

\subsection{Filter}

The scheduling of a {\filter} is very simple.  Since a {\filter} has
no sub-components (it is an atomic unit), a steady schedule for a
{\filter} is simply an execution of the {\filter}.  Thus, for a
{\filter} $f$, $P_f = \{f, \{f\}, \{[e_f,o_f,u_f]\}\}$

\subsection{Pipeline}

Scheduling a {\pipeline} $p$ first requires calculating the steady
state $S_p$ and phasing schedules for all the children of $p$,
$P_{p_i}$. Once the steady state has been calculated,
multiplicities of execution of each child are known, and the
children are simply scheduled to execute an appropriate number of
times in a row, starting from first child.

\begin{algorithm}
\label{alg:sa-pipeline} \caption{Single Appearance Schedule for a
{\pipeline}} {\bf SASPipeline}($p$).  Given a {\pipeline} $p$,
calculate a phasing Single Appearance Schedule for $p$.
\begin{algorithmic}
\STATE compute $S_p$; $phase = \{\}$ \FOR{$i=0$ to $n_p$}
\FOR{$j=0$ to $S_{p,u,i}$} \FOR{$k=0$ to $|P_{p_i, P}|$} \STATE
$phase = phase + P_{p_i, P, k}$ \ENDFOR \ENDFOR \ENDFOR \STATE
$P_p = \{p, \{phase\}, \{S_{p,c}\}\}$
\end{algorithmic}
\end{algorithm}

This technique works here, because {\pipelines} do not have any
cycles between their children (though their children may have
cycles, ie. {\feedbackloops}), and because a correct initialization
schedule is assumed to have been executed.  Notice, that it is not
necessary to know the amount of data buffered between children of
the {\pipeline} in order to use this algorithm, because once the
{\pipeline} has been initialized, all the needed data will be
provided by the steady state schedule.

\subsection{SplitJoin}

Scheduling a {\splitjoin} is essentially identical to scheduling a
{\pipeline}.  Once steady schedule multiplicities are computed, the
{\splitter} is executed the appropriate number of times, followed by
all the immediate children, and finally the {\joiner}.

\begin{algorithm}
\label{alg:sa-pipeline} \caption{Single Appearance Schedule for a
{\splitjoin}} {\bf SASSplitJoin}($sj$).  Given a {\splitjoin} $sj$,
calculate a phasing Single Appearance Schedule for $sj$.
\begin{algorithmic}
\STATE compute $S_{sj}$; $phase = \{\}$ \FOR{$j=0$ to
$S_{sj,u,n_{sj}}$} \STATE $phase = phase + {\splitter}$ \ENDFOR
\FOR{$i=0$ to $n_p$} \FOR{$j=0$ to $S_{p,u,i}$} \FOR{$k=0$ to
$|P_{p_i, P}|$} \STATE $phase = phase + P_{p_i, P, k}$ \ENDFOR
\ENDFOR \ENDFOR \FOR{$j=0$ to $S_{sj,u,n_{sj} + 1}$} \STATE $phase
= phase + {\joiner}$ \ENDFOR \STATE $P_p = \{p, \{phase\},
\{S_{p,c}\}\}$
\end{algorithmic}
\end{algorithm}

Similarly to {\pipelines}, this technique works because {\splitjoins}
have no cycles, and because the {\splitjoin} is assumed to have been
initialized properly.

\section{Min Latency}

\subsection{Pipeline}

In order to create a steady state schedule for a {\pipeline}, an
initialization schedule must already exist (or at least its
results must have been computed).  This was not the case with
single appearance schedule, because just the assumption that the
{\pipeline} was properly initialized allowed for execution of the
all the child streams from top to bottom.  With minimal latency
scheduling, the amount of data buffered up in the {\pipeline} (or
any other stream, for that matter) makes a big difference in which
child streams need to produce (and thus possibly consume) data,
and which do not.

One consequence of this interaction between the initialization and
steady schedules is that the initialization schedule may affect
the size of the steady schedule.  The difference comes from the
fact, that a steady schedule doesn't necessarily process all the
data buffered up and ready for processing.  This means that it is
possible that the steady schedule will have an additional phase at
the end, that will simply push some data around internally to the
{\pipeline}.  The phase will not produce any data (otherwise, the
phase would be necessary anyway), but it may consume some data (to
complete the amounts required to execute a steady schedule) and/or
push the data in internal buffers lower, in order to return to the
buffering state from the beginning of the steady schedule.

To simplify the calculation of how many times each child needs to
get executed during initialization, initialization is computed
exactly the same way as steady schedule.  Namely, the bottom most
stream is executed enough (minimal number of) times to produce
some data, the stream above is executed just enough times to
provide enough data to the stream below, and so on, until either a
child stream does not need to fire, or the top most child is
reached. If the amount of additional data needed by $n+1$ child is
$m$, and the $n$th child is about to execute $p$th phase of
$phases_n$, then the number of executions of the $n$th child is
computed using following algorithm:

\begin{singlespace}
\begin{verbatim}
while (m > 0)
  fired_n = fired_n + 1
  m = m - push^p_{p_n}
  p = (p + 1) \% phases_n
\end{verbatim}
\end{singlespace}

The result of running this algorithm is that while the
initialization schedule grows a little larger than necessary, the
steady schedule remains as small as possible.  The reason for this
is that all the streams will automatically execute the right
number of times to provide enough data for the bottom most child
to fire enough times.

\subsubsection{\splitjoin}

{\splitjoin} schedule is computed in essentially the same way as
{\pipeline}, except that number of firings of child streams depends
on the {\joiner} and the number of firings of the {\splitter} depends
on the child streams.  Algorithm is omitted here for brevity.

\subsubsection{\feedbackloop}

{\feedbackloop} schedule is also scheduled in a way very similar to
the {\pipeline} schedule.  Since the data is output from the
{\splitter}, the calculation begins by setting number of executions
of the {\splitter} to 1.  Number of firings of the $body$ is
computed, then the {\joiner} and finally the $loop$.

Computing of the schedule for the {\feedbackloop} is guaranteed to
succeed, if the {\feedbackloop} has a valid schedule.  This is
because each component in the {\feedbackloop} is scheduled in a way
that requires minimal amount of data input in order to produce
some output.  Thus, if a deadlock is detected in the
{\feedbackloop}, there genuinly is not enough data in the
{\feedbackloop} to execute the {\splitter}, thus output data.

\section{Notation and equations}

Steady State $T_s = \{s, N, m, c, u\}$.
\begin{itemize}
\myitem $s$ - the stream itself

\myitem $L$ - children of the stream

\myitem $m$ - multiples of execution of the children

\myitem $c$ - peek/push/pop for $s$
\end{itemize}

\noindent Phasing Schedule $P^p_s = \{S, I\}$.
\begin{itemize}
\myitem $S$ - steady state phasing schedule phases

\myitem $I$ - phasing init schedule phases
\end{itemize}

\noindent a set of phases (steady state or init) $S = \{H, c\}$
\begin{itemize}
\myitem $H$ - the actual phases \myitem $c$ - peek/pop/push for
the set
\end{itemize}

\noindent Phase $H_s = \{s, A, c, u\}$.
\begin{itemize}
\myitem $s$ - the stream itself

\myitem $A$ - sub-phases to be executed for this phase

\myitem $c$ - peek/push/pop for the phase

\myitem $u$ - {\bf don't need - so don't use!} multiples for
direct children, {\splitter} and {\joiner}.  {\splitter} and {\joiner} are
the $n$th and the $n+1$st element.
\end{itemize}

\noindent Single Appearance $S_s = \{H, I\}$
\begin{itemize}
\myitem $H$ - steady state phase (single one!) \myitem $I$ -
initialization phase (single one!)
\end{itemize}

\subsection{Steady State}

The steady state is easy to determine.  It's already described,
and I don't want to go into it here.

$c$ vector describes the TRUE peek.

\subsection{Scheduling}

All scheduling is actually done within the framework of phased
scheduling.  Initialization has multiple phases (doh!), so I'll
have to be careful writing this up.


\subsection{Initialization Schedule}

Initialization schedules are done in a very simple way - the child
that is meant to output data is executed at least once.  Because
of feedbackloops, there may be more than one phase of
initialization (for min-latency).

\subsection{Single Appearance}

\subsubsection{Filter}

for {\filter} $f$:

\begin{displaymath}
\begin{array}{rl}
H = & \{f, [f], [1], [e_f, o_f, u_f], [1]\} \\
I = & \{f, [], [], [e_f - o_f], [1]\} \\
S_f = & \{H, I\}
\end{array}
\end{displaymath}

\subsubsection{Pipeline}

for {\pipeline} $p$, children $p_i$, $0 \le i < n_p$, $n_p$ children

\begin{algorithm}
\label{alg:sas-pipeline} \caption{Create a Single Appearance
Schedule for a {\pipeline}} {\bf SASPipeline}($T_p$). Given a steady
state for a {\pipeline}, create a single appearance schedule for it.
\begin{algorithmic}
\STATE no clue how to do this
\end{algorithmic}
\end{algorithm}

\subsection{Min-Latency}

\subsubsection{Peek/Pop/Push from phases}

\begin{algorithm}
\label{alg:ph-consumption} \caption{Computing Consumption of a Set
of Phases} {\bf ConsPhases}($H$).  Given a set of phases $H$,
compute the amount of data peeked/popped/pushed by this set of
phases.
\begin{algorithmic}
\STATE $e = 0, o = 0, u = 0$
\FOR{$i=0$ to $|H|-1$}
\STATE $e = \max(e, o + H_{i, c, e})$
\STATE $o = o + H_{i, c, o}$
\STATE $u = u + H_{i, c, u}$
\ENDFOR
\STATE return $[e, o, u]$
\end{algorithmic}
\end{algorithm}

\begin{algorithm}
\label{alg:init-peek} \caption{Create a Phasing schedule out of
set of phases for Steady State and Initialization Schedules} {\bf
MakePhases}($H^S, H^I$). Given sets of phases to execute for
steady state and initialization schedules, construct a proper
phasing schedule.
\begin{algorithmic}
\STATE $c_S = {\bf ConsPhases} (H^S)$
\STATE $c_I = {\bf ConsPhases} (H^I)$
\STATE $c_{I,e} = \max(c_{I,e}, c_{I,o} + c_{S,e} - c_{S,o})$
\STATE $S = \{H^S, c_S\}$
\STATE $I = \{H^I, c_I\}$
\STATE return $\{S, I\}$
\end{algorithmic}
\end{algorithm}

\begin{algorithm}
\label{alg:get-phase} \caption{Return an appropriate phase of a
schedule.  Phase 0 is the first initialization stage of the
schedule.  Once all initialization stages have been exhausted, all
steady state phases are returned, with wrap-around.} {\bf
GetPhase}(n, $P$)
\begin{algorithmic}
\IF{$n < |P_{I, A}|$}
\STATE return $P_{I, A, n}$
\ELSE
\STATE return $P_{S, A, (n - |P_{I,A}|) {\bf\ mod\ } |P_{S, A}|}$
\ENDIF
\end{algorithmic}
\end{algorithm}

%\subsubsection{\filter}

\begin{algorithm}
\label{alg:min-lat-filter} \caption{Return a min-latency phasing
schedule for a {\filter}} {\bf MLFilter}().
\begin{algorithmic}
\STATE $H = \{f, \{f\}, [e_f, o_f,u_f]\}$
\STATE return ${\bf MakePhases} (\{H\}, \emptyset)$
\end{algorithmic}
\end{algorithm}

%\subsubsection{\pipeline}

\begin{algorithm}
\label{alg:min-lat-init-pipeline} \caption{Return a set of phases
that execute a Minimum-Latency initialization schedule for a
{\pipeline}, the amount of data buffered as a result of
initialization, and how many phases each child has executed.} {\bf
MLInitStagesPipeline} $()$
\begin{algorithmic}
\STATE Let $b$ vector represent amount of data stored between children of
$p$
\STATE Let $d$ vector represent phase/stage being currently executed; 0
represents first initialization stage.


\STATE $b = 0$, $d = 0$

\STATE {\bf Compute how many times each child will get
executed at minimum}
\STATE Let $f$ represent amount of data needed by the $i$th child
to be initialized.  $f$ has $n_p+1$ elements, last child being
fictional and not requiring any initialization.
\STATE $f = 0$

\FOR{$i=n_p-1$ downto $0$}
\WHILE{$d_i < |P_{p_i, I, A}|$ {\bf or} $b_{i+1} < f_{i+1}$}
\STATE $H' = {\bf GetPhase}(P_{p_i}, d_i)$, $d_i = d_i + 1$
\STATE $f_i = \max(f_i, b_i + H'_{c, e})$, $b_i = b_i - H'_{c,
o}$, $b_{i+1} = b_{i+1} + H'_{c, u}$
\ENDWHILE
\ENDFOR

\STATE {\bf Compute initialization stages:}
\STATE Let $u$ represent minimum number of executions of each
child
\STATE $u = d$, $d = 0$, $b = 0$
\STATE $stages = \emptyset$
\WHILE{$\neg \forall i, u_{i} \le 0$}
\STATE{\bf Compute a stage}
\STATE Let $H$ be the stage I'm about to construct
\STATE Let $m$ be the number of phase executions that each child
will perform in this stage
\STATE $H = \{p, \emptyset, [0,0,0]\}$, $b_0 = 0$, $b_{n_p} = 0$, $m = 0$
\STATE $need = 1$
\FOR{$i = n_p-1$ downto $0$}
\STATE $nextNeed = 0$
\WHILE{$need > 0$}
\STATE $H' = {\bf GetPhase}(P_{p_i}, d_i + m_i)$, $m_i = m_i + 1$, $u_i = u_i - 1$
\STATE $need = need - H'_{c, u}$, $nextNeed = \max(nextNeed, H'_{c, e} - b_i)$
\STATE $b_i = b_i - H'_{c, o}$, $b_{i+1} = b_{i+1} + H'_{c, u}$
\ENDWHILE
\STATE $need = nextNeed$
\ENDFOR
\STATE $H_{c,e} = need$
\FOR{$i=0$ to $n_p-1$}
\WHILE{\bf true}
\STATE $H' = {\bf GetPhase}(P_{p_i}, d_i + m_i)$
\IF{$H'_{c, e} \le b_i$}
\STATE $m_i = m_i + 1$, $u_i = u_1 - 1$
\STATE $b_i = b_i - H'_{c, o}$, $b_{i+1} = b_{i+1} + H'_{c, u}$
\ELSE
\STATE {\bf break}
\ENDIF
\ENDWHILE
\ENDFOR
\STATE $H_{c,o} = -b_0$, $H_{c,u} = b_{n_p}$
\STATE{\bf Create the stage:}
\FOR{$i=0$ to $n_p-1$}
\WHILE{$m_i \ne 0$}
\STATE $H_A = H_A \circ {\bf GetPhase}(P_{p_i}, d_i)$
\STATE $d_i = d_i + 1$, $m_i = m_i - 1$
\ENDWHILE
\ENDFOR
\STATE $phases = phases \circ H$
\ENDWHILE
\STATE return $\{phases, b, d\}$
\end{algorithmic}
\end{algorithm}

\begin{algorithm}
\label{alg:min-lat-steady-pipeline} \caption{Return a set of
phases that execute a Minimum-Latency steady state schedule for a
{\pipeline}} {\bf MLSteadyStagesPipeline} $(T_p, b^{init}_p,
d^{init}_p)$. $T_p$ is the steady state for the {\pipeline}.
$b^{init}_p$ is the amount of data stored between children of the
{\pipeline} after initialization, $d^{init}_p$ is the number of
phases and stages executed by the initialization schedule.
\begin{algorithmic}
\STATE Let $b$ vector represent amount of data stored between children of
$p$
\STATE Let $d$ vector represent phase/stage being currently executed; 0
represents first initialization stage.
\STATE Let $u$ be number of phase executions for each child for a full steady state execution of the {\pipeline}.

\STATE $b = b^{init}_p$, $d = d^{init}_p$, $\forall i, u_i = T_{p, m, i} * |P_{p, S, A}|$
\STATE {\bf Compute steady state:}
\STATE $phases = \emptyset$
\WHILE{$u_{n_p -1} \ne 0$}
\STATE{\bf Compute a phase}
\STATE Let $H$ be the phase I'm about to construct
\STATE Let $m$ be the number of phase executions that each child
will perform in this phase
\STATE $H = \{p, \emptyset, [0,0,0]\}$, $b_0 = 0$, $b_{n_p} = 0$, $m = 0$
\STATE $need = 1$
\FOR{$i = n_p-1$ downto $0$}
\STATE $nextNeed = 0$
\WHILE{$need > 0$ {\bf and} $u_i \ne 0$}
\STATE $H' = {\bf GetPhase}(P_{p_i}, d_i + m_i)$, $m_i = m_i + 1$, $u_i = u_i - 1$
\STATE $need = need - H'_{c, u}$, $nextNeed = \max(nextNeed, H'_{c, e} - b_i)$
\STATE $b_i = b_i - H'_{c, o}$, $b_{i+1} = b_{i+1} + H'_{c, u}$
\ENDWHILE
\STATE $need = nextNeed$
\ENDFOR
\STATE $H_{c,e} = need$
\FOR{$i=0$ to $n_p-1$}
\WHILE{$u_i \ne 0$}
\STATE $H' = {\bf GetPhase}(P_{p_i}, d_i + m_i)$
\IF{$H'_{c, e} \le b_i$}
\STATE $m_i = m_i + 1$, $u_i = u_1 - 1$
\STATE $b_i = b_i - H'_{c, o}$, $b_{i+1} = b_{i+1} + H'_{c, u}$
\ELSE
\STATE {\bf break}
\ENDIF
\ENDWHILE
\ENDFOR
\STATE $H_{c,o} = -b_0$, $H_{c,u} = b_{n_p}$
\STATE{\bf Create a phase:}
\FOR{$i=0$ to $n_p-1$}
\WHILE{$m_i \ne 0$}
\STATE $H_A = H_A \circ {\bf GetPhase}(P_{p_i}, d_i)$
\STATE $d_i = d_i + 1$, $m_i = m_i - 1$
\ENDWHILE
\ENDFOR
\STATE $phases = phases \circ H$
\ENDWHILE
\STATE return $phases$
\end{algorithmic}
\end{algorithm}

\begin{algorithm}
\label{alg:min-lat-pipeline} \caption{Return a min-latency phasing
schedule for a {\pipeline}} {\bf MLPipeline}().
\begin{algorithmic}
\STATE $T = {\bf SSPipeline} ()$
\STATE $\{I, b^{init}, d^{init}\} = {\bf MLInitStagesPipeline}()$
\STATE $H = {\bf MLSteadyPhasesPipeline}(T, b^{init}, d^{init})$
\STATE return ${\bf MakePhases} (H, I)$
\end{algorithmic}
\end{algorithm}

\section{Old Latency}

\subsubsection{\filters}

The transfer function for amount of information carried by an item
of data before and after a {\filter} is quite simple.  The amount of
information consumed and produced by the {\filter} needs to remain
constant.  Using notation for a {\pipeline} $p$, the transfer
function for a {\filter} is $info_{p_n} = info_{p_{n-1}} * {o_{p_n}
\over u_{p_n}}$.

\subsubsection{{\splitters} and {\joiners}}

The transfer function for the amount of information carried by an
item of data through a {\splitter} or a {\joiner} is also very simple,
however it is not very intuitive, compared to the {\filter} transfer
function. An intuitive function would preserve the amount of
information carried across a {\splitter} or a {\joiner}, and split or
merge the information appropriately across all branches.  Such an
approach, however, can result in a situation where a {\joiner} is
joining data items with different amount of information carried in
different branches.  This type of a situation is difficult to
handle, because for simplicity all data items in a single buffer
should carry exactly the same amount of information.

The approach that solves the problem above is to not preserve the
amount of information across all branches when data is being split
or joined. Instead, every branch (including the input to a
{\splitter} or output of a {\joiner}) consumes or produces the same
amount of information on every iteration.

Using the {\splitjoin} notation, amount information per data item
transferred to $n$th branch of a {\splitter} is $info_{split_n} =
info_{input} * {o_{split} \over u_{split,n}}$, and the amount of
information per data item transferred from $n$th branch of a
{\joiner} is $info_{output} = info_{join_n} * {o_{join,n} \over
u_{join}}$.  The same equations will work for {\feedbackloops}, with
the property that going through a loop will preserve the amount of
information carried by a data item.

\subsubsection{checking for messages}

 by a {\filter} $f_0$ to {\filter} $f_1$. that consumes
(and thus produces) $info_{f_0}$ amount of information, and
delivered to {\filter} $f_1$ that consumes $info_{f_1}$ amount of
information on every execution of its {\work} function, then the
destination {\filter} must check for delivery of new at least
messages every $\left \lfloor {(l_1-l_0)
* info_{f_0}} \over info_{f_1} \right \rfloor$ firings of its
{\work} function. If this equation yields 0, {\filter} $f_1$ must
check for new messages before every firing of its {\work} function.

%  -- this is minor, and I don't know how to explain it well
%Note, that this frequency of checking for messages to deliver may
%not be sufficient to satisfy the latency requirements.  If the
%schedule enforces very tight information buffering (very close to
%absolute minimums or maximums, as explained below), it is possible
%that the destination {\filter} needs to check for message delivery
%more often.  This effect is schedule specific, and needs to be
%computed on a case-by-case basis.
%

\subsubsection{information and latency}

Next, the amount of information buffered between the source and
the destination of a message needs to be known.  As explained
earlier, there are three types of messages.  The easiest type of
message to handle in terms of buffering is the downstream positive
delay message.  Those messages impose no information buffering
requirements on the schedule.  This is because the information
wavefront that the message needs to be delivered with cannot
possibly have passed the destination filter - it cannot even have
passed the source filter.

The next type of message to consider is the downstream negative
delay message.  A downstream negative delay message will be
delivered to the destination \emph{before} the current source
{\filter} information wavefront reaches the destination {\filter}.
 This type of a message generates a minimal buffering requirement
on the schedule.  There is no maximal requirement, because if the
amount of information buffered up   between the source and
destination is large, the message will not be delivered on the
first iteration of checking for messages to be delivered, but on
some subsequent iteration.  At the minimum, the message must be
delivered just before the minimum latency wavefront reaches the
destination {\filter}.  Just after the destination {\filter} is
executed, the amount of data buffered up is reduced by
$info_{dest}$.  This is the situation when the minimal amount of
information is stored between the source and destination.  Thus,
the amount of information stored between the two {\filters} should
be at least $info_{src} * (-l_0) + info_{dest} {e_{dest} -
o_{dest} \over o_{dest}} - \min (info_{src}, info_{dest})$.

The case of upstream positive delay message is very similar to
downstream negative delay message.  The destination {\filter} must
receive the message before it produces the minimal allowed
information wavefront.  Thus there must be less than $info_{src}
* (l_1 + {e_{src} - o_{src} \over o_{src}})$ information
between the destination and source {\filters}.

The last step required for a complete framework relating latency
and information is to compute the amount of information between
two {\filters}.  This task is actually quite simple.  Given two
{\filters}, $f_0$ and $f_1$, the algorithm selects the upstream
{\filter}, and follows any non-cyclic downstream path towards the
other {\filter}.  Amount of information stored along the path
followed is summed up, and represents the information stored
between the two {\filters}.  Note, that this algorithm will follow
the body path of {\feedbackloops}, and select any branch of a
{\splitjoin}.  The reason this algorithm works is because {\filters}
do not destroy or create information when their {\work} functions
execute, and because amount of data stored along any branch of a
{\splitjoin} is the same.
