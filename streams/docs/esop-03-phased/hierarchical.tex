\section{Pseudo Single Appearance Hierarchical Scheduling}
\label{chpt:hierarchical}

In this section we present Pseudo Single Appearance Hierarchical
Scheduling, a technique which is quite effective for scheduling
{\StreamIt} programs, but which cannot schedule all programs, and
which may require the buffers to be very large.

Section \ref{sec:hierarchical:motivation} provides some motivation
for hierarchical scheduling.  Section
\ref{sec:hierarchical-notation} presents the notation used for
hierarchical notation. Section \ref{sec:sas} provides an algorithm
for computing hierarchical schedules.

\subsection{Motivation}
\label{sec:hierarchical:motivation}

As has been explained in Section \ref{sec:sched-vs-buffer}, the
ordering of execution of nodes in a {\StreamIt} program can have a
significant effect on the amount of resources necessary to execute
the schedule.  The two important factors to consider when creating
the schedule is amount of buffering necessary to execute the
schedule, and the amount of space necessary to store the schedule.
The amount of buffering necessary is controlled by the ordering of
execution of nodes of the {\StreamIt} graph.  The amount of storage
necessary to store the schedule is controlled by the encoding of
the schedule.  As a general rule, ordering which minimizes the
buffering space requirements is fairly irregular and difficult to
encode efficiently.

\begin{figure}
\centering \psfig{figure=hierarchical-sample.eps,width=1in}
\caption[Stream for hierarchical scheduling]{A sample stream used
for hierarchical scheduling.} \label{fig:sample-sj}
\end{figure}

One technique used for encoding schedules is to form loop-nests of
sub-schedules and repeat them multiple times, until a steady-state
schedule is reached.  For example, the stream in Figure
\ref{fig:sample-sj} has a following steady state:

\begin{displaymath}
S_{s} = \left\{ \left[
\begin{array}{c}
9 \\ 6 \\ 18 \\ 18 \\ 4 \\ 4
\end{array} \right], \left[
\begin{array}{c}
A \\ C \\ D \\ split \\ join \\ B
\end{array}
\right], \left[
\begin{array}{c}
54 \\ 54 \\ 40
\end{array}
\right], \left[
\begin{array}{c}
9 \\ 2 \\ 4
\end{array}
\right]\right\}
\end{displaymath}

\noindent Thus one steady state schedule for this stream can be
$$\{9\{A\{2 split\}\{2D\}\}\}\{2\{\{3C\}\{2 split\}\{2B\}\}\}$$
Here, $\{A\{2 split\}\{2D\}\}$ and $\{\{3C\}\{2 split\}\{2B\}\}$
are the inner nests, executed 9 and 2 times respectively.

If, the overall schedule has every {\StreamIt} node appear only
once (as in the example above), the technique is called Single
Appearance Scheduling \cite{bhattacharyya94looped}. One of
difficulties in using Single Appearance Scheduling is finding a
good way to form loop-nests for the sub-schedules, because the
buffering requirements can grow quite large.  An example of this
has been presented in Section \ref{sec:sched-vs-buffer}.

{\StreamIt} provides the scheduler with a pre-existing
hierarchical structure. While it is possible to use techniques
developed for Single Appearance Scheduling to create valid
schedules for {\StreamIt} programs, Single Appearance Scheduling
does not satisfy all requirements of an effective {\StreamIt}
scheduler. This is because some {\feedbackloops} cannot be
scheduled using Single Appearance Scheduling techniques. This
difficulty arises because the amount of data provided to the
{\feedbackloop} by the $delay_{fl}$ variable is not sufficient to
perform a complete steady-state execution of the loop, thus
preventing the schedule for the {\feedbackloop} to be encoded with
only a single appearance of every node in the schedule.

The solution to this problem is to have the same node appear
multiple times in the schedule.  While this solves the problem of
inability to schedule some {\feedbackloops}, it introduces another
problem: which nodes should appear several times, and how many
times should they be executed on each appearance.  The solution
proposed here goes half-way to solve the problem. A more effective
solution will be proposed in Chapter \ref{chpt:phased}.

In hierarchical scheduling we use the pre-existing structure
(hierarchy) to determine the nodes that belong in every loop-nest.
Basically, every stream receives its own loop-nest, and treats
steady-state execution of its children as atomic (even if those
children are streams whose executions can be broken down into more
fine-grained steps). In the example above, the {\pipeline} has a
{\splitjoin} child.  The {\splitjoin} is responsible for
scheduling its children (nodes $C$, $B$, $split$ and $join$).  The
{\pipeline} will use the {\splitjoin}'s schedule to create its
own steady state schedule. Here the {\splitjoin}'s schedule can be
$T_{sj} = \{\{9 split\}\{3C\}\{9D\}\{2 join\}\}$, thus making the
{\pipeline}'s schedule \footnote{Notation for this schedule is
explained in next section (Section
\ref{sec:hierarchical-notation}).} $$T_{pipe} = \{\{9A\}\{2
T_{sj}\}\{4B\}\} = \{\{9A\}\{2\{\{9 split\}\{3C\}\{9D\}\{2
join\}\}\}\{4B\}\}$$


The problem of inability to schedule some {\feedbackloops} is
alleviated by allowing {\feedbackloop} to interleave the execution
of its children (the body, the loop, and the {\splitter} and
{\joiner}).  This results in {\feedbackloop} containing multiple
appearances of its children. All other streams use their
children's schedules in their schedules only once. This technique
is called Pseudo Single Appearance Scheduling, since it results in
schedules that are very similar to proper single appearance
schedules. While it does not allow scheduling of all
{\feedbackloops} (a {\feedbackloop} may have a child which
requires more data for steady state execution then made available
by the $delay_{fl}$ variable) it has been found to be very
effective, and only one application has been found which cannot be
scheduled using this technique.

\subsection{Notation}
\label{sec:hierarchical-notation}

The notation in the above example, is very similar to that
presented in Section \ref{sec:sched-vs-buffer}.  A number in front
of a node represents that the node is meant to be executed a
certain number of times.  The one big difference is that
$\{2T_{sj}\}$ means that the schedule for the {\pipeline} is meant
to be executed twice.  Since $T_{sj} = \{\{9
split\}\{3C\}\{9D\}\{2 join\}\}$, $\{2T_{sj}\}$ is same as
$\{2\{\{9 split\}\{3C\}\{9D\}\{2 join\}\}\}$.

This means that to execute $T_{pipe}$, node $A$ is executed 9
times, schedule $T_{sj}$ is executed twice and node $B$ is
executed twice, in that order. To execute $T_{sj}$, the {\splitter}
is executed 9 times, node $C$ is executed 3 times, node $D$ 9
times and the {\joiner} twice.

Thus, writing the schedule of $T_{pipe}$ into a flat schedule (one
with no loop-nests) results in schedule $\{9A\}\{9
split\}\{3C\}\{9D\}\{2 join\}\{9 split\}\{3C\}\{9D\}\{2
join\}\{4B\}$.

In other words, $T_{sj}$ is a loop-nest, which can be executed
multiple times. When storing a schedule, $T_{sj}$ is stored only
once, and every use of $T_{sj}$ becomes the reference to the
actual schedule.

A steady schedule for a stream $s$ will be denoted by $T_s$, while
an initialization schedule for a stream $s$ will be denoted $I_s$.
A {\splitter} of a {\splitjoin} or a {\feedbackloop} $s$ will be denoted
as $split_s$, while the {\joiner} will be denoted as $join_s$.

This section will continue using the notation for $e$, $o$ and $u$
extended to streams.  That is, for a stream $s$, $e_s$ will
represent the amount of data needed by $s$ on its {\Input} {{\Channel}}
in order to execute its minimal steady state schedule; $o_s$
represents the amount of data consumed by from its {\Input} {{\Channel}}
$s$ during execution of its steady state schedule; and $u_s$
represents the amount of data pushed by $s$ onto its {\Output}
{{\Channel}}.

The notation for $e$, $o$ and $u$ will also be extended to
initialization schedules.  Namely, $e^i_s$ represents the amount
of data required by stream $s$ on its {\Input} {{\Channel}} in order to
execute the initialization schedule for $s$; $o^i_s$ represents
the amount of data consumed by $s$ from its {\Input} {{\Channel}} during
its initialization schedule; and $u^i_s$ denotes the amount of
data pushed by $s$ onto its {\Output} {{\Channel}} during execution of
its initialization schedule.  The initialization schedules are set
up in such a way, that after all streams have executed their
initialization schedules, the program is ready to enter its steady
state execution.

Note, that it is possible that a stream $s$ has $u^i_s \ne 0$.  An
example of this will be presented in Section \ref{sec:sas-fl}.

A hierarchical schedule for a stream $s$ is denoted as $$H_s =
\left\{T_s, I_s, c_s = \left[\begin{array}{c}
e_s\\o_s\\u_s\end{array}\right], c^i_s =
\left[\begin{array}{c}e^i_s\\o^i_s\\u^i_s\end{array}\right]\right\}$$

\subsection{Pseudo Single-Appearance \\ Hierarchical Scheduling}
\label{sec:sas}

\begin{figure}
\begin{minipage}{1.5in}
\centering \psfig{figure=pipeline-steady-state.eps,width=0.6in} \\
{\protect\small (a) A sample {\pipeline}}
\end{minipage}
~
\begin{minipage}{1.5in}
\centering \psfig{figure=splitjoin-steady-state.eps,width=1.2in} \\
{\protect\small (b) A sample {\splitjoin}}
\end{minipage}
~
\begin{minipage}{2.5in}
\centering \psfig{figure=feedback-hierarchical.eps,width=1.0in} \\
{\protect\small (c) A sample {\feedbackloop}.\\ $delay_{fl} = 15$ \\
The $L$ {\filter} has been flipped upside-down for clarity. \\$e_L
= 9, o_L = 5, u_L = 6$ }
\end{minipage}
\caption{Sample {\StreamIt} streams used for Pseudo
Single-Appearance Hierarchical Scheduling}
\label{fig:hierarchical-schedule}
\end{figure}

This section will develop hierarchical scheduling techniques to
create initialization and steady state schedules. A simple
implementation of the hierarchical scheduling creates a
single-appearance schedule.  While single-appearance scheduling is
quite effective in scheduling {\StreamIt} programs, it is also easy
to construct programs that have {\feedbackloops} that are impossible
to schedule.  To alleviate the problem, the single-appearance
scheduling was slightly modified to allow {\feedbackloops} to
schedule programs using hierarchical push scheduling.  This does
not solve the problem altogether (some {\feedbackloops} are still
impossible to schedule using this technique), but this technique
is able to schedule many programs which cannot be scheduled with a
simple single-appearance scheduler.

Sample streams for techniques described here are taken from Figure
\ref{fig:hierarchical-schedule}.  The streams in Figure
\ref{fig:hierarchical-schedule} are identical to those in Figure
\ref{fig:steady-state} with exception of the {\feedbackloop}.

\subsubsection{\filters}

An execution of a {\filter} is an atomic operation.  Thus a steady
state schedule for a {\filter} $f$ is simply $T_f = (f)$.

A {\filter} has no internal buffering.  Thus there is no need to
initialize a {\filter} for its steady state.  {\filters} may, however,
peek data.  That means that in order to enter a steady state,
sufficient amount of data must be pushed onto {\filter}'s {\Input}
{{\Channel}}.  Thus, for a {\filter} $f$, $e^i_f = e_f - o_f$.

Finally, a hierarchical schedule of a {\filter} is

\begin{displaymath}
H_f = \left\{\{f\}, \{\}, \left[
\begin{array}{c}
e_f\\o_f\\u_f
\end{array}\right], \left[
\begin{array}{c}
e_f-o_f\\0\\0\end{array}\right] \right\}
\end{displaymath}

\subsubsection{\pipelines}

Here, for examples, we will use the sample {\pipeline} in Figure
\ref{fig:hierarchical-schedule}(a).  The steady state schedule for
this {\pipeline} is

\begin{displaymath}
S_p = \left\{ \left[
\begin{array}{c} 4 \\ 6 \\ 9 \\ 3 \end{array} \right], \left\{
\begin{array}{c} A \\ B \\ C \\ D \end{array} \right\},\left[
\begin{array}{c} 4 \\ 4 \\ 3 \end{array}\right], \left[
\begin{array}{c} 4\\ 6\\ 9\\ 3 \end{array} \right]\right\}
\end{displaymath}

\subsubsubsection{Initialization}

Calculation of the initialization schedule starts with computing
hierarchical schedules for all children of the {\pipeline}.

In order to create an initialization schedule of a {\pipeline}, all
of {\pipeline}'s children's initialization schedules must be
executed. Every child must execute its initialization schedule
before it can execute its steady-state schedule.  Some children
may require some data in order to execute their initialization
schedules. The upstream children provide this data to them by
first executing their own initialization schedule, and then their
steady-state schedule. Thus, in the final form, the execution of a
{\pipeline}'s initialization schedule first executes the
initialization schedule of the top-most child, then executes the
steady-state schedule this child several times, then the
initialization schedule of the second-from-the-top child, followed
by executing this child's steady-state schedule several, etc,
until the bottom-most child is reached.  Since the bottom-most
child does not need to provide any data {\pipeline}'s downstream
children (there aren't any), the bottom-most child only executes
its initialization schedule.

The initialization schedule is calculated as follows. At every
stream of the {\pipeline}, the amount of data necessary to
initialize all the streams below is calculated. For $k$th stream,
that amount is denoted $init_k$. If the {\pipeline} has $n$
children, then for the bottom-most child, $p_{n-1}$, that amount
is $init_{n-1} = e^i_{p_{n-1}}$. The data to the $k$th child is
provided by the $k-1$ child, during its initialization and
subsequent execution of its steady state schedule.  The
initialization provides $u^i_{p_{k-1}}$ data items. Thus the $k-1$
child must execute its steady state schedule $l_{k-1} =
\left\lceil init_k - u^i_{p_{k-1}} \over u_{p_{k-1}} \right\rceil$
times.  The amount of data required for initialization of the
{\pipeline} by the $k-1$ child is $init_{k-1} = e^i_{p_{k-1}} +
l_{k-1} * o_{p_{k-1}}$.

This calculation is performed for all children of the {\pipeline},
starting at the last (bottom-most) child, and moving up.  For the
sample {\pipeline} in Figure \ref{fig:hierarchical-schedule}(a), the
values computed are:

\begin{displaymath}
\begin{array}{lr}
\begin{array}{rl}
l_3 & = 0 \raisebox{-0.2in}{ } \\
l_2 & = \left\lceil init_3 - u^i_C \over u_C \right\rceil = \left\lceil 2 - 0 \over 1 \right\rceil = 2 \raisebox{-0.2in}{ } \\
l_1 & = \left\lceil init_2 - u^i_B \over u_B \right\rceil =
\left\lceil 4 - 0 \over 3 \right\rceil = 1 \raisebox{-0.2in}{ } \\
l_0 & = \left\lceil init_1 - u^i_A \over u_A \right\rceil =
\left\lceil 4 - 0 \over 3 \right\rceil = 2 \raisebox{-0.2in}{ } \\
\end{array} &
\begin{array}{rl}
init_3 & = e^i_D + l_3 * o_D = 2 + 0 * 1 = 2 \raisebox{-0.2in}{ }\\
init_2 & = e^i_C + l_2 * o_C = 0 + 2 * 2 = 4 \raisebox{-0.2in}{ }\\
init_1 & = e^i_B + l_1 * o_B = 1 + 1 * 3 = 4 \raisebox{-0.2in}{ }\\
init_0 & = e^i_A + l_0 * o_A = 0 + 2 * 1 = 2 \raisebox{-0.2in}{ }\\
\end{array}
\end{array}
\end{displaymath}

Now, the initialization schedule is simply constructed by
iterating over all children of the {\pipeline}, from top to
bottom, and concatenating all initialization and appropriate
steady state schedules.  Thus $I_p = \{I_A\{2T_A\} I_{B}T_B
I_C\{2T_C\} I_D\}$.

Finally, we need to compute the amount of data peeked, popped and
pushed by the {\pipeline} during its initialization.

The amount of data popped is simply the amount of data popped by
the top-most child when executing the {\pipeline}'s initialization
schedule, that is the amount of data popped by the first child
during its own initialization plus the amount of data popped
during its steady-state execution times number of steady state
executions. That is $o^i_p = o^i_{p_0} + l_0 * o_{p_0}$.

Similarly, the amount of data pushed by the {\pipeline} is simply
the amount of data pushed by the bottom-most child during its
initialization. Remember that the bottom-most child never executes
its steady-state schedule.  Thus $u^i_p = u^i_{p_{n-1}}$.

Computing the amount of data peeked by the {\pipeline} during
initialization may be a little more complicated, because unlike
popping and pushing, peeking is not accumulative. Luckily, we can
rely on our knowledge of structure of the {\StreamIt} graph to
calculate the amount of data peeked by a {\pipeline}. We know that a
{\pipeline} is a single-input structure. We also know that this
single input will lead directly into a {\StreamIt} node.  There are
only three possibilities for what this node will be.

\begin{itemize}
\item If {\pipeline}'s first node is a {\filter} $f$ (the first child
of the {\pipeline} is a {\filter} or a {\pipeline} with a {\filter} as its
first node) then the extra amount of data peeked by the {\pipeline}
on initialization will be $e^i_f - o^i_f$. If the first child is a
{\filter}, then $p_0$ is $f$ and the extra amount peeked is also
$e^i_{p_0} - o^i_{p_0}$.  If the first child is a {\pipeline} with a
{\filter} first node, we can show by induction that this {\pipeline}'s
extra peek amount will also be $e^i_{p_0} - o^i_{p_0}$.

\item If {\pipeline}'s first node is a {\splitter} (the first child of
the {\pipeline} is a {\splitjoin} or a {\pipeline} with a {\splitter} as
its first node) then the extra amount of data peeked by the
{\pipeline} on initialization will be 0, because {\splitters} never
peek. Furthermore, for the same reason, the amount of extra data
peeked by the first child on its initialization will also be zero,
or $e^i_{p_0} - o^i_{p_0}= 0$.

\item If {\pipeline}'s first node is a {\joiner} (the first child of
the {\pipeline} is a {\feedbackloop} or a {\pipeline} with a {\joiner} as
its first node) then the amount of extra data peeked by the
{\pipeline} on initialization will be 0, for the same reasons as
above. And again $e^i_{p_0} - o^i_{p_0}= 0$.
\end{itemize}

Thus on initialization, the {\pipeline} will have an extra peek
amount of $e^i_{p_0} - o^i_{p_0}$, and the total amount of data
peeked by the {\pipeline} for initialization is $e^i_p = (e^i_{p_0}
- o^i_{p_0}) + l_0
* o_{p_0}$.

\subsubsubsection{Steady State Schedule}

The steady state state schedule is calculated as a
single-appearance schedule.  Calculation of a single-appearance
schedule starts with computing $S_p$, the steady state for the
{\pipeline}.

Using $S_p$, the steady state schedule for the {\pipeline} is
constructed by concatenating multiple steady state schedules for
all children of the {\pipeline}, from top to bottom.  For $k$th
child, its steady state schedule must be executed $S_{p,v,k}$
times.  The steady state guarantees that every child receives
sufficient amount of data to execute all its steady state
schedules, and since the data is being passed from top to bottom,
all children will have received all their data before executing.

The consumption and production of data for the steady state
schedule is already calculated by the steady state, and is
$S_{p,c}$.

Thus, for our example, $T_p = \{\{4T_A\} \{6T_B\} \{9T_C\}
\{3T_D\}\}$.  The final hierarchical schedule is

\begin{displaymath}
H_p = \left\{\{\{4T_A\} \{6T_B\} \{9T_C\}
\{3T_D\}\},\{I_{A}\{2T_{A}\} I_{B}\{1T_{A}\} I_{C}\{2T_{C}\}
I_{D}\}, \left[
\begin{array}{c}
4\\4\\3
\end{array}\right], \left[
\begin{array}{c}
2\\2\\0
\end{array}\right] \right\}
\end{displaymath}

\subsubsection{\splitjoins}

Creating a schedule for a {\splitjoin} is essentially identical to
scheduling a {\pipeline}.  The initialization schedule only needs to
compute how many times the {\splitter} needs to be executed, and
construct the actual schedule.  The steady state schedule is
constructed by concatenating steady state schedule of {\splitjoin}'s
children, the {\splitter} and {\joiner}.

For our example in Figure \ref{fig:hierarchical-schedule}(b), the
steady state is

\begin{displaymath}
S_{sj} = \left\{ \left[
\begin{array}{c} 2 \\ 1 \\ 2 \\ 2 \end{array}\right], \left\{
\begin{array}{c} A \\ B \\ {\splitter} \\ {\joiner} \end{array} \right\},
\left[ \begin{array}{c} 6 \\ 6 \\ 8
\end{array} \right], \left[
\begin{array}{c}
2 \\ 1 \\ 2 \\ 2
\end{array}\right]\right\}
\end{displaymath}

\subsubsubsection{Initialization}

In order to initialize a {\splitjoin}, all its children must execute
their initialization schedules.  The only requirement for
executing those schedules is that they have been provided with
sufficient data on their {\Input} {{\Channels}}.  Since the {\splitter}
provides data for all the children of a {\splitjoin}, it is the only
element of a {\splitjoin} that must execute its steady state
schedule.

For $k$th child of a {\splitjoin}, the {\splitter} must provide
$e^i_{sj_k}$ data items.  One execution of the {\splitter} causes
it to push $w_{s,k}$ data items toward the $k$th child.  Thus the
{\splitter} must execute at least $l_k = \left\lceil e^i_{sj_k}
\over w_{s,k} \right\rceil$ times.  In order to find out how many
times the {\splitter} needs to execute to initialize all children,
$l_s$, we simply find the maximum $l_k$. Thus $l_s = {\max \atop
k}(l_k)$.

In the sample {\splitjoin} from Figure
\ref{fig:hierarchical-schedule}(b), we get following $l_k$s:

\begin{displaymath}
\begin{array}{rl}
l_0 = & \left\lceil e^i_A \over w_{s,0} \right\rceil = \left\lceil
0 \over 2 \right\rceil = 0 \raisebox{-0.2in}{ }\\
l_1 = & \left\lceil e^i_B \over w_{s,1} \right\rceil = \left\lceil
1 \over 1 \right\rceil = 1 \raisebox{-0.2in}{ }\\
\end{array}
\end{displaymath}

The maximum $l_k$s is 1, thus $l_s = 1$, the {\splitter} must be
executed once for initialization.

The initialization schedule is constructed by concatenating an
appropriate number of executions of the {\splitter} and
initialization schedules of all the children.  Thus in our
example, $I_{sj} = \{split\ I_A\ I_B\}$.

The consumption of an initialization schedule of a  {\splitjoin} is
computed as follows:  $e^i_{sj} = u^i_{sj} = l_s * o_{sj_s}$ and
$u^i_{sj} = 0$. The peeking and popping amounts are simply the
amount of data popped by the {\splitter} for every one of its
executions times the number of times it is executed.  The {\joiner}
is never executed, thus the push amount is 0.

Thus for our example, $e^i_{sj} = u^i_{sj} = 1 * 3 = 3$ and
$u^i_{s} = 0$.

\subsubsubsection{Steady State}

Similarly to the algorithm for {\pipeline}, the steady state is
constructed by using $S_{sj,v}$ to concatenate the executions of
the {\splitter}, all children of the {\splitjoin} and the {\joiner}
together.

For our example, the steady state schedule is simply $$T_{sj} =
\{\{2\ split\}\{2T_A\}T_B\{2\ join\}\}$$

The consumption vector, $c_{sj}$ is the same as $S_{sj,c}$.

Thus the hierarchical schedule for the {\splitjoin} in Figure
\ref{fig:hierarchical-schedule}(b) is

\begin{displaymath}
H_{sj} = \left\{\{\{2\ split\}\{2T_A\}T_B\{2\ join\}\},\{split\
I_A\ I_B\}, \left[
\begin{array}{c}
6\\6\\8
\end{array}\right], \left[
\begin{array}{c}
3\\3\\0
\end{array}\right] \right\}
\end{displaymath}

\subsubsection{\feedbackloops}
\label{sec:sas-fl}

Scheduling of {\feedbackloops} is a task that can be made difficult,
if the amount of data provided for the {\feedbackloop} by the
$delay_{fl}$ value is low. Before a {\StreamIt} program begins
executing, the {\feedbackloop} needs to be provided with some data
in one of the internal {{\Channels}}. Without this data, the {\splitter}
and the {\joiner} of the {\feedbackloop} will not be able to execute,
because they will never have sufficient data on their input
{{\Channels}}. This is a consequence of the {\feedbackloop} having a
cyclical structure.

\begin{figure}
\centering
\psfig{figure=feedback-non-scheduleable.eps,width=1.2in}
\caption[Example of non-shedulable {\feedbackloop}]{Sample
{\feedbackloop}. If this {\feedbackloop} has a $delay_{fl}$ value
set to 7, it does not have a steady state schedule which will
allow it to execute forever. If the $dealy_{fl}$ value is
increased by 1 to 8, the {\feedbackloop} has a steady state
schedule of $\{join\{2B\}\{5 split\}L\ join\{2B\}\{5 split\}L\ $
$join\{2B\}\{5 split\}L\{2\ join\}\{4B\}\{10 split\}\{2L\}\ \}$.}
\label{fig:feedback-non-schedulable}
\end{figure}

The difficulty in scheduling {\feedbackloops} is that if the amount
of data made available to the {\feedbackloop} by the $delay_{fl}$
value (as explained in Section \ref{sec:explain-fl}) is small,
there will be very limited number of ways to execute the
{\feedbackloop}.  In fact, it is possible that the amount of data
available to the {\feedbackloop} is so small, it cannot reach and
complete an execution of a steady state schedule. An example of
such {\feedbackloop} is presented in Figure
\ref{fig:feedback-non-schedulable}.

Here we will use {\feedbackloop} from Figure
\ref{fig:hierarchical-schedule}(c).  The steady state schedule for
this {\feedbackloop} is

\begin{displaymath} S_{fl} = \left\{
\begin{array}{c} \left[
\begin{array}{c}
15 \\ 3 \\ 5 \\ 6 \end{array}\right], \left\{
\begin{array}{c} B \\ L \\ {\splitter} \\ {\joiner} \end{array}\right\}, \left[
\begin{array}{c}
12 \\ 12 \\ 15
\end{array}
\right], \left[
\begin{array}{c}
15 \\ 3 \\ 5 \\ 6
\end{array}\right]
\end{array} \right\}
\end{displaymath}

\subsubsubsection{Initialization Schedule}

Initialization for the {\feedbackloop} is calculated in a similar
way to initialization of a {\pipeline}.  The number of steady
state executions of the children of the {\feedbackloop} is denoted
$l_B$ and $l_L$ for the body child and the loop child,
respectively. The number of executions of the {\splitter} is
denoted $l_s$ and the {\joiner} is denoted $l_j$.

Since the initial data is inserted into the buffer between the
loop child and the {\joiner} (as explained in \ref{sec:explain-fl}),
it follows that the loop child should initialize last - it will be
the last one receive data to initialize. Since the computation of
the initialization schedule is similar to the way it was done for
{\pipeline}, we will start with the child which is executed last,
namely the loop child. Similarly as with {\pipeline}, the which is
initialized last does not execute its steady state schedule for
initialization, thus we set $l_L = 0$. The {\splitter} must provide
the loop child with just enough data to initialize, the body child
must provide the {\splitter} with just enough data for the {\splitter}
to pass enough data to the loop child, etc. Thus,

\begin{displaymath}
\begin{array}{rl}
l_L & = 0 \\
l_{s} & = \left\lceil {o^i_{fl_L} \over w_{s,1}} \right\rceil \raisebox{-0.2in}{ } \\
l_{B} & = \left\lceil o_{s} * l_{s} - u^i_{fl_B} \over u_{fl_B} \right\rceil \raisebox{-0.2in}{ } \\
l_{j} & = \left\lceil o^i_{fl_B} + l_B * o_{fl_B} \over u_{j}
\right\rceil \raisebox{-0.2in}{ }
\end{array}
\end{displaymath}

This initialization schedule will only be valid if there is enough
data provided between the loop child and the {\joiner}, or
$delay_{fl} \ge l_j * w_{j,1}$.  If this condition does not hold,
the {\feedbackloop} cannot be scheduled using pseudo
single-appearance algorithm.

Referring to the example Figure
\ref{fig:hierarchical-schedule}(c), we obtain the following values
for $n$s:

\begin{displaymath}
\begin{array}{rl}
l_{s} & = \left\lceil {4 \over 3} \right\rceil = 2 \raisebox{-0.2in}{ } \\
l_{B} & = \left\lceil 2 * 3 - 0 \over 1 \right\rceil = 6 \raisebox{-0.2in}{ } \\
l_{j} & = \left\lceil 0 + 6 * 2 \over 5 \right\rceil = 3 \raisebox{-0.2in}{ } \\
\end{array}
\end{displaymath}

Furthermore, since $delay_{fl} = 15$, we have $15 \ge 3 * 3$, thus
a valid initialization schedule can be constructed.

The initialization schedule is constructed by concatenating
executions of the {\joiner}, body child, {\splitter} and the loop
child.  The body child will execute both its initialization
schedule as well as its steady state schedule, while the loop
child will only execute its initialization schedule.

Thus for our example we get $I_{fl} = \{\{3\ join\} I_B \{6T_B\}
\{2\ split\} I_L\}$.

We now compute the consumption of data for the initialization
schedule of the {\feedbackloop}: $e^i_{fl} = o^i_{fl} = n_j *
w_{j,0}$ and $u^i_{fl} = n_s * w_{s,0}$. Similarly as in
computation for the {\splitjoin}, these values are simply the
production and consumption of the {\splitter} and {\joiner} from their
appropriate {\Input} and {\Output} channels multiplied by the number
of times the {\splitter} and {\joiner} are executed during
initialization schedule.

In our example, $e^i_{fl} = o^i_{fl} = 3 * 2 = 6$ and $u^i_{fl} =
3 * 3 = 9$.  Note that the {\feedbackloop} pushes data out during
its initialization.

Finally, we compute the amount of data present in {{\Channels}} after
initialization. These amounts are important because they will be
used to compute the steady state schedule of the {\feedbackloop}.
These amounts were not necessary for computation of steady state
schedules of {\pipeline} and {\splitjoin}. These amounts are
calculated by simply subtracting the amount of data popped from a
{{\Channel}} from amount of data pushed into a {{\Channel}}. Here we
adopted the notation for
{\Input} and {\Output} {{\Channel}} from Section \ref{sec:exec-model}.

\begin{displaymath}
\begin{array}{rl}
in^i_B = & l_j * u_j - l_B * o_{fl_B} \\
out^i_B = & u^i_{fl_B} + l_B * u_{fl_B} - l_s * o_s\\
in^i_L = & l_s * w_{s, 1} - l_L * o_{fl_L} \\
out^i_L = & delay_{fl} + u^i_{fl_L} + l_L * u_{fl_L} - l_j * w_{j,1} \\
\end{array}
\end{displaymath}

\subsubsubsection{Steady State Schedule}

Computing the steady state schedule for a {\feedbackloop} is more
complicated than for the other streams.  The reason for this is
(as explained above) that {\feedbackloops} may require a non
single-appearance schedule, while other {\StreamIt} construct can
always be scheduled using single-appearance schedules.

The algorithm used for creating of a steady state schedule will
work in several phases.  The amount of data present in {{\Channels}}
between the children of the {\feedbackloop}, the {\joiner} and the
{\splitter} is kept track of to determine which element is allowed
to execute.

The algorithm for creating a steady state schedule of a
{\feedbackloop} iterates over the elements of the {\feedbackloop} in
order of ({\joiner}, body child, {\splitter}, loop child).  The
algorithm executes each element as many times as possible,
considering the amount of data required to execute the element and
the amount of data available internally within the {\feedbackloop}'s
channels to execute the element and not exceeding the steady state
number of executions for the element.  Each execution of an
element is appended to the steady state schedule.

This iteration is repeated until either all elements have executed
their steady state number of times, or until a complete iteration
has been performed with no element being able to execute. The
first case indicates a successful completion of the algorithm. The
second case indicates a failure - the algorithm is unable to
schedule the {\feedbackloop}.

Table \ref{tab:sas-fl} illustrates the execution of this algorithm
for our sample {\feedbackloop}. Notice that in the table, the first
row and the last row have the same amount of data buffered in
{{\Channels}}, thus indicating that a full steady state schedule has
indeed been computed. Furthermore, the last entry considering
execution of $B$ has sufficient data to execute $B$ 5 times, but
only executes it 4 times to ensure that a steady state schedule is
constructed.

\begin{table} \centering
\begin{tabular}{|l|c|c|c|c|c|c|c|c|c|}
\hline \multicolumn{4}{|c|}{data items in buffer} & \multicolumn{4}{c|}{executions left} & \parbox{0.5in}{element considered} & executions \\
\cline{1-8} $in_B$ & $out_B$ & $in_L$ & $out_L$ & $split$ & B & $join$ & L & & \\
\hline  1   &   6   &   6   &   0   &   5   &   9   &   4   &   3   &   $split$ &   2   \\
\hline  1   &   0   &   12  &   0   &   3   &   9   &   4   &   3   &   $L$ &   1   \\
\hline  1   &   0   &   7   &   6   &   3   &   9   &   4   &   2   &   $join$  &   2   \\
\hline  11  &   0   &   7   &   0   &   3   &   9   &   2   &   2   &   $B$ &   5   \\
\hline  1   &   5   &   7   &   0   &   3   &   4   &   2   &   2   &   $split$ &   1   \\
\hline  1   &   2   &   10  &   0   &   2   &   4   &   2   &   2   &   $L$ &   1   \\
\hline  1   &   2   &   5   &   6   &   2   &   4   &   2   &   1   &   $join$  &   2   \\
\hline  11  &   2   &   5   &   0   &   2   &   4   &   2   &   1   &   $B$ &   4   \\
\hline  3   &   6   &   5   &   0   &   2   &   0   &   2   &   1   &   $split$ &   2   \\
\hline  3   &   0   &   11  &   0   &   0   &   0   &   0   &   1   &   $L$ &   1   \\
\hline  3   &   0   &   6   &   6   &   0   &   0   &   0   &   0   &       &       \\
\hline
\end{tabular}
\caption[Execution of Steady-State algorithm on sample
{\feedbackloop}]{Trace of execution of steady-state algorithm on
sample {\feedbackloop} from Figure
\ref{fig:hierarchical-schedule}(c). The executions left amount is
the number of executions left for a particular child to complete a
steady state execution of the {\feedbackloop}. One this value
reaches 0, the element is not executed anymore, even if it has
data to execute.} \label{tab:sas-fl}
\end{table}

The schedule resulting from the above computation is $$T_{fl} =
\{\{2\ join\} \{6T_B\} \{2\ split\} T_L \{2\ join\} \{5T_B\}\
split\ T_L \{2\ join\} \{4T_B\} \{2\ split\} T_L\}$$ This schedule
is obtained by going through Table \ref{tab:sas-fl} from top to
bottom and concatenating the appropriate number of executions of
every child of the {\feedbackloop}, as listed in the
"executions" column.

The steady state consumption $c_{fl}$ is again simply $S_{fl, c}$.
Thus the hierarchical schedule is:

\begin{displaymath}
H_{fl} = \left\{
\begin{array}{l}
\begin{array}{c}\{\{2\ join\} \{6T_B\} \{2\ split\} T_L \{2\ join\} \{5T_B\},\\
split\ T_L \{2\ join\} \{4T_B\} \{2\ split\} T_L\},\end{array}\\
\{\{3\ join\} I_B \{6T_B\} \{2\ split\} I_L\}, \left[
\begin{array}{c}
12\\12\\15
\end{array}\right], \left[
\begin{array}{c}
6\\6\\9
\end{array}\right] \end{array}\right\}
\end{displaymath}

\begin{comment}
Scheduling {\feedbackloops} requires some extra care, as explained
above.  Once again, steady schedule multiplicities are computed,
but this time, the amount of data buffered between the {\joiner},
$body$, {\splitter} and $loop$ is required in order to perform the
algorithm.

The first step in the algorithm is to execute the {\joiner} as many
times as possible, depending on how much data is available between
the $loop$ and the {\joiner}, up to the number permitted in
executing a steady schedule. Data is transferred between buffers
at known rates, and buffering is adjusted appropriately. Next the
$body$ is executed as much as possible, followed by the {\splitter},
followed by the $loop$.  If, after executing this sequence, the
{\joiner}, $body$, {\splitter} or $loop$ need to be executed more
times in order to complete a steady schedule, this execution is
repeated until the steady schedule is completed.

It is possible, that the algorithm above deadlocks - there is not
enough data for any of the children to advance.  This does not
necessarily mean that the {\feedbackloop} has no legal schedule.
This is because pseudo single-appearance scheduling is a coarse
scheduling technique.  Furthermore, this problem is not caused by
hierarchical scheduling.  Figure
\ref{fig:feedback-non-schedulable} contains an example of a
{\feedbackloop} that cannot be scheduled using any single appearance
technique.
\end{comment}
