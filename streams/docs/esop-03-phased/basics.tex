\section{General {\StreamIt} Scheduling Concepts}
\label{chpt:sched-basic}

\begin{figure}
\begin{center}

\begin{minipage}{1.2in}
\centering \psfig{figure=pipeline-buffers.eps,width=0.6in} \\
{\protect\small (a) A sample {\pipeline}}
\end{minipage}
~
\begin{minipage}{1.2in}
\centering \psfig{figure=splitjoin-steady-state.eps,width=1.2in} \\
{\protect\small (b) A sample {\splitjoin}}
\end{minipage}
~
\begin{minipage}{1.5in}
\centering \psfig{figure=feedback-steady-state.eps,width=1.0in} \\
{\protect\small (c) A sample {\feedbackloop}.  The $L$ {\filter}
has been flipped upside-down for clarity.\\$peek_L = pop_L = 5,
push_L = 6$}
\end{minipage}

\caption{Sample {\StreamIt} streams} \label{fig:steady-state}

\end{center}
\end{figure}

This section introduces the general concepts used for scheduling
{\StreamIt} programs.  Concepts presented here are are common with
other languages \cite{ptolemyoverview} \cite{esterel92}
\cite{lustre}.

Section \ref{sec:exec-model} presents the {\StreamIt} execution
model. Section \ref{sec:steady-state} introduces the concept of a
steady state and shows how to calculate it. Section
\ref{sec:init-peeking} explains the need for initialization of
{\StreamIt} program. Section \ref{sec:general:schedules} introduces
simple notation for expressing schedules while Section
\ref{sec:sched-vs-buffer} presents the tradeoff between schedule
and buffer storage requirements.

\subsection{{\StreamIt} execution model}
\label{sec:exec-model}

A {\StreamIt} program is represented by a directed graph, $G = (N,
E)$.  A node in $G$ is either a {\filter}, a {\splitter} or a
{\joiner}. Edges in $G$ represent data {\Channels}.  Each node in
$G$ takes data from its {\Input} {\Channel}(s), processes this data,
and puts the result on the {\Output} {\Channel}(s).  Each data
{\Channel} is simply a FIFO queue.

In order for a {\filter} $f$ to execute, it must have at least $e_f$
data on its {\Input} {\Channel}. After its execution it will decrease
amount of data on its {\Input} {\Channel} by $o_f$ and increase the
amount of data on its {\Output} {\Channel} by $u_f$. Similarly, a
{\splitter} will consume $o_s$ data from its {\Input} {\Channel} and
push $w_{s,i}$ data into each of its {\Output} {\Channels}, while a
{\joiner} will consume $w_{j,i}$ data from its {\Input} {\Channels} and
push $u_j$ onto its {\Output} {\Channel}.

\begin{comment}
Each {\filter} node $n_f$ has exactly one incoming edge and one
outgoing edge.  The incoming edge is referred to as an {\Input}
{\Channel}, while the outgoing edge is called an {\Output}
{\Channel}. A {\splitter} node $n_s$ has exactly one incoming edge
({\Input} {\Channel}), but has multiple outgoing edges ({\Output}
{\Channels}). A {\joiner} node has multiple incoming edges ({\Input}
{\Channels}) but only one outgoing edge ({\Output} {\Channel}).

Each node of graph $G$ can be executed.  An execution of a node
causes some data to be collected from the node's {\Input}
{\Channel}(s), the data to be processed and the result to be put on
the {\Output} {\Channel}(s).  An execution of a node transfers the
smallest amount of data across the node - it is an atomic
operation.  {\StreamIt} uses a static data flow model, meaning
that every execution of a node $n$ will require the same amount of
data to be present on node's {\Input} {\Channel}(s) for consumption or
inspection, same amount to be consumed from the {\Input} {\Channel}(s)
and same amount of data to be pushed onto its {\Output} {\Channel}(s).

Each {\filter} node $n_f$ is associated with a 3-tuple $(e_f, o_f,
u_f)$. These three values represent the rate of data flow for the
{\filter} for each execution.  The first value represents the
amount of data necessary to be present in its {\Input} {\Channel} in
order to execute the {\filter}.  This is also called the peek
amount of the {\filter}.  The second value represents the amount
of data which will be consumed by the {\filter} from its {\Input}
{\Channel}. This is called the pop amount of the {\filter}.  Note,
that $e_f \ge o_f$. The final value represents the amount of data
that will be put on the {\Output} {\Channel} of the {\filter}. This is
called the push amount of a {\filter}.  The amount of data present
in the {\Input} {\Channel} of a {\filter} node $n_f$ is denoted
$in_f$, while data present in the {\Output} {\Channel} is denoted
$out_f$.

Each {\splitter} node $n_s$ is associated with a tuple $(o_s,
w_s)$. The first value represents the amount amount of data that
will be consumed by $n_s$ from its {\Input} {\Channel}. Thus, in
order to execute $n_s$, there must be at least $o_s$ data in its
{\Input} {\Channel}. $w_s$ is a vector of integers, each
representing the amount of data that will be pushed onto a
corresponding {\Output} {\Channel} of $n_s$.  The amount of data
present in the {\Input} {\Channel} of a {\splitter} node $n_s$
is denoted $in_s$, while data present in the $i$th {\Output}
{\Channel} is denoted $out_{s,j}$.

Each {\joiner} node $n_j$ is associated with a tuple $(w_j, u_j)$.
The first value is a vector of integers, each representing the
amount of data that will be consumed by $n_j$ from its
corresponding {\Input} {\Channels}.  In order to execute $n_j$,
each of its {\Input} {\Channels} must have at least as much data
in it as the corresponding value in $w_j$ indicates.  $u_j$
represents the amount of data that will be pushed by $n_j$ onto
its {\Output} {\Channel}. The amount of data present in the $i$th
{\Input} {\Channel} of a {\joiner} node $n_j$ is denoted
$in_{j,i}$, while data present in the {\Output} {\Channel} is
denoted $in_s$.
\end{comment}

A schedule for a {\StreamIt} program is a list of executions of
nodes of graph $G$.  The list describes the order in which these
nodes are to be executed.  In order for a schedule to be legal, it
must satisfy two conditions: first,for every execution of a node,
a sufficient amount of data must be present on its {\Input}
{\Channel}(s); second, the execution of the schedule must require
a finite amount of memory.

\subsection{Steady State}
\label{sec:steady-state}

A {\StreamIt} schedule is an ordered list of firings of nodes in the
{\StreamIt} graph.  Every firing of a node consumes some data from
{\Input} {\Channel}(s) and pushes data onto the {\Output} {\Channel}(s).

One of the most important concepts in scheduling streaming
applications is the steady state schedule.  A steady state
schedule is a schedule that the program can repeatedly execute
forever.  It has a property that the amount of data buffered up
between any two nodes does not change from before to after the
execution of the steady state schedule.

\begin{comment}
This property is important, because it allows the compiler to
statically schedule the program at compile time, and simply repeat
the schedule forever at runtime.  A schedule without this property
cannot be repeated continuously.  This is because the delta in
amount of data buffered up on between nodes will continue
accumulating, requiring an infinite amount of buffering space.
\end{comment}

A steady state of a program is a collection of number of times
that every node in the program needs to execute in a steady state
schedule.  It does not impose an order of execution of the nodes
in the program.
\begin{comment}
Not every {\StreamIt} program has a steady state schedule.  It is
possible for a program to have unbalanced production and
consumption of data in {\splitjoins} and {\feedbackloops}. The
amount of data buffered continually increases, and cannot be
reduced, thus making it impossible to create a steady state
schedule for them.  It is also possible that a {\feedbackloop}
does not have enough data buffered up internally in order to
complete execution of a full steady state, and thus deadlocks.
Programs without a valid steady state schedule are not considered
valid {\StreamIt} programs. In other words, all valid {\StreamIt}
programs have a steady state schedule.
\end{comment}

\subsubsection{Minimal Steady State}

The size of a steady state is defined as the sum of all executions
of all the nodes in the program per iteration of the steady state.

\begin{definition}
A steady state of stream $s$ is represented by vector $m$ of
non-negative integers. Each of the elements in $m$ represents the
number of times a corresponding node in $s$ must be executed in
the steady state.
\end{definition}

Note that $m$ does not impose an order of execution of nodes. Size
of a steady state is the total number of executions of all the
nodes in the steady state.

Next we will summarize the properties of schedules presented in
\cite{lee87static}. Detailed proofs of these properties are
presented in \cite{karczma-thesis}.

\begin{theorem}[Minimal Steady State Uniqueness]
A {\StreamIt} program that has a valid steady state, has a unique
minimal steady state.
\end{theorem}

This means that every valid {\StreamIt} program has a unique minimal
steady state. Our scheduler will produce schedules that execute
exactly the minimal steady state of a program.

\begin{comment}
\begin{proof}[Minimal Steady State Uniqueness]
Assume that there are two different minimal steady states with
same size.  Let $m$ and $q$ denote vectors representing the two
steady states. Let $\sum_i m_i$ denote size of schedule $m$ and
$\sum_i q_i$ denote size of schedule $q$. Note that since both $m$
and $q$ are minimal steady states, $\sum_i m_i = \sum_i q_i$.
Since the schedules are different, there must be some $j$ for
which $m_j \ne q_j$. Assume without loss of generality that $m_j <
q_j$. Since a steady state does not change the amount of data
buffered between nodes, the node producing data for node $i$ must
also execute less times than corresponding node in $q$. Similarly,
the node consuming data produced by node $j$ also must execute
less times than the corresponding node in schedule $q$. Since a
{\StreamIt} program describes a connected graph, it follows that
$\forall i, m_i < q_i$.  Thus $\sum_i m_i \ne \sum_i q_i$, which
is a contradiction. Thus there cannot be two different minimal
steady state.
\end{proof}

\begin{corollary}[Minimal Steady State Uniqueness]
\label{corollary:minimal-state}
The additional property we have from the above proof is that if
$m$ represents a minimal steady and $q$ any other steady state,
then $\forall i, m_i < q_i$.
\end{corollary}

\begin{lemma}[Composition of Steady Schedules]
\label{lemma:composition}
If $m$ and $q$ are two steady states for a {\StreamIt} program, then
$m + q$ is also a steady state.
\end{lemma}

The above lemma is true because neither $m$ nor $q$ change the
amount of data buffered in the {\Channels}.  Thus a composition of
the steady states does not change the amount of data buffered in
the {\Channels}, which makes the composition also a steady schedule.

\begin{corollary}[Composition of Steady Schedules]
\label{corollary:composition}
If $m$ and $q$ are two steady states, and $\forall i, m_i > q_i$,
then $w = m - q$ is also a steady state.
\end{corollary}

If $q$ is a steady state and $m = w + q$ is a steady state, then
$w$ must not change the amount of data buffered in {\Channels}. Thus
$w$ must be a steady state.
\end{comment}

\begin{theorem}[Multiplicity of Minimal Steady States]
If a {\StreamIt} program has a valid steady state, then all its
steady states are strict multiples of its minimal steady state.
\label{thm:multiplicity}
\end{theorem}

\begin{comment}
\begin{proof}[Multiplicity of Minimal Steady State]
Assume that there exists a steady state that is not a multiple of
the minimal steady state.  Let $m$ denote the minimal steady
state. Let $q$ denote the other steady state.  Note that $w = q -
m$ is still a steady state, as long as all elements of $w$ remain
non-negative (by Corollary \ref{corollary:composition}).  Repeat
subtracting $m$ from $q$ until no more subtractions can be
performed without generating at least one negative element in
vector $w$.  Since $q$ is not a multiple of $m$, $w \ne 0$. But
since we cannot subtract $m$ from $w$ any further, $\exists i, m_i
> w_i$.  Since $m$ is a minimal steady state and $w$ is a steady
state, this is impossible due to Corollary
\ref{corollary:minimal-state}. Thus there are no steady states
that are not multiples of the minimal steady schedule.
\end{proof}
\end{comment}

This property means that in order to find a minimal steady state
schedule of a \stream, we can find any steady state of the
\stream, and divide it by the $gcd$ of executions of all nodes in
the \stream to find the minimal steady state schedule.

\subsubsection{Calculating Minimal Steady State}
\label{sec:calc-min-steady}

Minimal steady states are calculated recursively in a hierarchical
manner. That is, a minimal steady state is calculated for all
children streams of {\pipeline}, {\splitjoin} and {\feedbackloop},
and then the schedule is computed for the actual parent stream
using these minimal states as atomic executions.
\begin{comment}
This yields a minimal steady state
because all child streams must execute their steady states (to
avoid buffering changes), and all steady states are multiples of
the minimal steady states (per Theorem \ref{thm:multiplicity}).
\end{comment}
Executing a full steady state of a stream is referred to as
"executing a stream". The notation for $peek$, $pop$ and $push$ is
is extended to mean entire streams in their minimal steady state
execution.  That is, a {\pipeline} $p$ will consume $o_p$ data,
produce $u_p$ data and peek $e_p$ data on every execution of its
steady state.  Again, in the hierarchical view of {\StreamIt}
programs, a child stream of a {\pipeline} will execute its steady
state atomically.

A steady state of a stream $s$ is represented by a set $S_s$ of
elements, $S_s = \{m, N, c, v\}$. The set includes a vector $m$,
which describes how many times each {\StreamIt} node of the stream
will be executed in the steady state, a corresponding ordered set
$N$ which stores all the nodes of the stream, a vector $c$, which
holds values $[e_s, o_s, u_s]$ for stream $s$, and a vector $v$
which holds number of steady state executions of all direct
children of $s$. $m$ and $v$ are not the same vector, because $m$
refers to nodes in the subgraph, while $v$ refers only to the
direct children, which may be {\filters}, {\pipelines},
{\splitters} and {\feedbackloops}. For a stream $s$, set $S$ is
denoted as $S_s$ and the elements of $S_s$ are denoted as
$S_{s,m}$, $S_{s,N}$, $S_{s,c} and S_{s,v}$.

Note, that a steady state does not say anything about the ordering
of the execution of nodes, only how many times each node needs to
be executed to preserve amount of data buffered by the stream.

We omit the equations for calculating the minimal steady states
for brevity. Details can be found in Appendix \ref{apx:eqs}. As an
example, the {\splitjoin} from Figure \ref{fig:steady-state}(b) has
the following steady state:

\begin{displaymath}
S_{sj} = \left\{
\begin{array}{c}
2 * S_{sj_0, m} \circ 1 * S_{sj_1, m} \circ [2\ 2], \\
S_{sj_0, N} \circ S_{sj_1, N} \circ \{sj_s, sj_j\}, \\
\left[
\begin{array}{c}
2 * 3 \\ 2 * 3 \\ 2 * 4
\end{array}
\right], \left[
\begin{array}{c}
2 \\ 1 \\ 2 \\ 2
\end{array}\right]
\end{array} \right\}
\end{displaymath}

\subsection{Initialization for Peeking}
\label{sec:init-peeking}

Consider a {\filter} $f$, with $e_f = 2$ and a $o = 1$. When a
{\StreamIt} program is first run, there is no data present on any
of the {\Channels}.  This means that for the first execution,
\filter $f$ requires that two data items be pushed onto its
{\Input} {\Channel}.  After the first execution of $f$, it will
have consumed one data item, and left at least one data item on
its {\Input} {\Channel}.  Thus in order to execute $f$ for the
second time, at most one extra data item needs to be pushed onto
$f$'s {\Input} {\Channel}.  The same situation persists for all
subsequent executions of $f$ - at most one additional data item is
required on $f$'s {\Input} {\Channel} in order to execute $f$.

This example illustrates that first execution of a {\filter} may
require special treatment.  Namely, the source for {\filter}'s data
may need to push more data onto {\filter}'s {\Input} {\Channel} for
{\filter}'s first execution.  Due to this condition, a {\StreamIt}
program may need to be initialized before it can enter steady
state execution.

\begin{comment}
There are other constraints (latency constraints) which may
require more complex initialization.  These will be discussed in
Chapter \ref{chpt:constrained}.

After an execution, a {\filter} $f$ must leave at least $e_f - o_f$
data on its {\Input} {\Channel}.  Thus, if the only constraints on
initialization are peek-related, it is a sufficient condition for
entering steady state schedule that $\forall f \in {\filters}, in_f
\ge e_f - o_f$.

Specific strategies for generating initialization schedules for
peeking will be presented in Chapter \ref{chpt:hierarchical} and
Chapter \ref{chpt:phased}.
\end{comment}

\subsection{Schedules}
\label{sec:general:schedules}

Once a program has been initialized, it is ready to execute its
steady state. In order to do this, a steady state schedule needs
to be computed. The steady states computed above do not indicate
the ordering of execution of the nodes, only how many times the
nodes need to be executed.

A schedule is an ordering of nodes in a {\StreamIt} stream. In
order to execute the schedule, we iterate through all of its nodes
in order of appearance and execute them one by one.  For example
in order to execute schedule $\{ABBCCBBBCC\}$ we would execute
node A once, node B twice, node C two times, node B three times
and node C twice again, in that order.

In order to shorten the above schedule we can run-length encode
it.  The schedule becomes $\{A \{2B\}\{2C\}\{3B\}\{3C\}\}$. The
schedule can also be expressed as $\{A \{2 \{BC\}\}\{3\{BC\}\}\}$.
Different levels of loop nests can be used to express the
schedule. The more loop nests are used, the more hierarchical the
schedule becomes.

\subsection{Schedule Size vs. Buffer Size}
\label{sec:sched-vs-buffer}

\begin{comment}
\begin{figure}
\begin{center}

\psfig{figure=pipeline-buffers.eps,width=0.6in} \caption[4 {\filter}
{\pipeline}]{Sample 4 {\filter} {\pipeline}.  This {\pipeline} is
the same as one in Figure \ref{fig:steady-state} (a), except that
its children do not peek extra data} \label{fig:pipeline-buffers}
\end{center}
\end{figure}
\end{comment}

When creating a schedule, two very important properties of it are
schedule size and amount of buffering required.  Schedule size
depends on encoding the schedule in an efficient way, while amount
of space required depends only on order of execution of nodes. The
two are related, however, because order of execution of {\filters}
affects how efficiently the schedule can be encoded.

For example, execution of {\filters} in {\pipeline} depicted in
Figure \ref{fig:steady-state} can be ordered in two simple ways,
one resulting in a large schedule but minimal amount of buffering,
the other resulting in a small schedule but a large amount of
buffering.

The steady schedule of the {\pipeline} in Figure
\ref{fig:steady-state}(a) executes {\filter} $A$ 4 times,
{\filter} $B$ 6 times, {\filter} $C$ 9 times and {\filter} $D$ 3
times. Writing out a schedule that requires minimal buffering
results in schedule $\{AB\{2C\}BCDAB\{2C\}ABCDB\{2C\}ABCD\}$. This
schedule requires a buffer for 4 data items between {\filters} $A$
and $B$, 4 items between $B$ and $C$ and 3 items between $C$ and
$D$, resulting in total buffers size 11, assuming data items in
all buffers require the same amount of space. The schedule itself
has 18 entries.

To compare, writing the schedule in the most compact method we get
$\{4A\}\{6B\}\{9C\}\{3D\}$  This schedule requires a buffer for 12
data items between {\filters} $A$ and $B$, 18 items between $B$
and $C$, and 9 data items between $C$ and $D$, resulting in total
buffers size 39.  The schedule has 4 entries.

Even on such a small example, it is easy to see that the encoding
scheme for a schedule has an immense influence over the amount of
space necessary to store the buffered data and the schedule.

\begin{comment}
We can compare the storage efficiency of these two schedules by
assuming that one data item in a buffer requires $x$ amount of
memory and each entry in a schedule requires $y$ amount of memory.
Thus the two schedules will require the same amount of storage to
store themselves and execute if $11 x + 18 y = 39 x + 4 y$.

\begin{displaymath}
\begin{array}{rcl}
11 x + 18 y & = & 39 x + 4 y \\
14 y & = & 28 x \\
y & = & 2x
\end{array}
\end{displaymath}

Thus the smaller schedule is more efficient if every data item
requires less than twice the amount of storage than every entry in
the schedule.

One of the difficulties in scheduling {\StreamIt} programs lies in
finding a good set of trade-offs between schedule size and
buffering requirements.
\end{comment}

\begin{comment}

\subsection{Minimum Buffer Size between {\filters}}

As illustrated above, the amount of buffering in a {\pipeline} can
be affected greatly by the order of executions of {\filters} in the
{\pipeline}.  The following equation calculates the minimal buffer
size required in order for two {\filters} to be able to push data
between each other indefinitely in the most buffer-efficient way.
Buffers this size cannot always be achieved, because some
components require that data be buffered up for execution (ex.
{\feedbackloops} require data to exist internally in order to
execution to advance) or because extra latency constrains require
additional buffering.

\begin{equation}
buffer_{A \to B} = \left\lceil {{peek_B} \over {\gcd(push_A,
pop_B)}} - 1 \right\rceil \gcd (push_A, pop_B) + push_A
\end{equation}

\emph{I can explain this equation, but I cannot prove it.  what
should I do with this?  it's not necessary for the thesis, but it
is a neat result we never published (PLDI submission), nor have I
seen it in any other papers (nobody does peeking, so it can't be
anywhere else)}

\end{comment}
