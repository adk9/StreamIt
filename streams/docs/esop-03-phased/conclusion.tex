\section{Conclusion and Future Work}

This thesis presents a variety of techniques used for scheduling
Synchronous Data Flow Graphs as used by the {\StreamIt} language.
Unlike other langauge, {\StreamIt} enforces a structure on the
stream graph, thus allowing a variety of new approaches to
scheduling execution. Algorithms presented here improve current
current scheduling techniques in multiple ways.

Hierarchical approach to scheduling execution of streaming
applications allows for a simplification of algorithms. Program
graphs do not have to be considered globally, thus less data needs
to be kept track of. In hierarchical approaches presented here, we
only need to consider immediate children of a given stream.

Phasing approach to scheduling allows to schedule arbitrarily
tight {\feedbackloops} and allows for more fine-grained control of
buffering requirements. The fine-grained control of buffering
requirements can provide dramatic improvements in buffer
requirements when scheduling streaming applications, as has been
presented here. Furthermore, phased schedules lend themselves to
some easy forms of compression, thus reducing the schedule size.
Future work will concentrate on expanding phasing scheduling to
implement schedules that conform to specific buffering
constraints, take advantage of cache sizes, etc. Producing of a
single schedule for many instances of identical streams will also
be explored.

The solution to latency constrained scheduling presented here is
an important contribution to development of {\StreamIt}. It will be
extended to allow for morphing graphs (\cite{thies02streamit}). It
will also be adapted to use phasing scheduling to reduce buffer
and schedule size.
