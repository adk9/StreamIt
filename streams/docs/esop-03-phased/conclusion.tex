\section{Conclusion and Future Work}

This thesis presents two techniques used for scheduling
Synchronous Data Flow Graphs as used by the {\StreamIt} language.
Unlike other langauge, {\StreamIt} enforces a structure on the
stream graph, thus allowing a variety of new approaches to
scheduling execution. Algorithms presented here improve current
current scheduling techniques in multiple ways.

Hierarchical approach to scheduling of streaming applications
allows for very simple algorithms. Program graphs do not have to
be considered globally, thus less data needs to be kept track of.
In hierarchical approaches presented here, we only need to
consider immediate children of a given stream.

Phasing approach to scheduling allows to schedule arbitrarily
tight {{\feedbackloops}} and allows for more fine-grained control of
buffering requirements. The fine-grained control of buffering
requirements can provide dramatic improvements in buffer
requirements when scheduling streaming applications, as has been
presented here. Furthermore, phased schedules lend themselves to
some easy forms of compression, thus reducing the schedule size.
Future work will concentrate on expanding phasing scheduling to
implement schedules that have some real-life constraints put upon
them. For example, a program may want to keep all its data in
processor caches to provide high performance. Adapting buffer
sharing to phased scheduling will also be explored, as it promises
further reduction in buffer requirements.
