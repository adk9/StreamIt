Effective use of the memory hierarchy is critical for achieving high
performance on embedded systems.  We focus on the class of streaming
applications, which is increasingly prevalent in the embedded domain.  We
exploit the widespread parallelism and regular communication patterns
in stream programs to formulate a set of cache aware optimizations
that automatically improve instruction and data locality.  Our work is
in the context of the Synchronous Dataflow model, in which a program
is described as a graph of independent actors that communicate over
channels.  The communication rates between actors is known at compile
time, allowing the compiler to statically model the caching behavior.

We present three cache aware optimizations: 1) execution scaling,
which judiciously repeats actor executions to improve instruction
locality, 2) cache aware fusion, which combines adjacent actors while
respecting instruction cache constraints, and 3) scalar replacement,
which converts certain data buffers into a sequence of scalar
variables that can be register allocated.  The optimizations are
founded upon a simple and intuitive model that quantifies the temporal
locality for a sequence of actor executions.  Our implementation of
cache aware optimizations in the StreamIt compiler yields a 249\%
average speedup (over unoptimized code) for our streaming benchmark
suite on a StrongARM 1110 processor.  The optimizations also yield a
154\% speedup on a Pentium~3 and a 152\% speedup on an Itanium~2.
