\section{Buffer Management}
\label{sec:buffer}

\begin{figure*}[t]
\begin{minipage}{1.7in}
\centering
\psfig{figure=fusion-pipeline.eps,width=0.7in}

\caption{Stream graph for a synthetic buffer test.\protect\label{fig:code-graph}}
\end{minipage}
\hspace{0.3in}
\begin{minipage}{2.2in}
\centering
{\scriptsize
\begin{verbatim}
void->void pipeline BufferTest {
  add Source();
  add FIR();
}

void->float filter Source {
  work push 1 {
    push(random());
  }
}

float->void filter FIR {
  int PEEK = 4;
  work pop 1 peek PEEK {
    float result = 0;
    for (int i=1; i<PEEK; i++) {
      result += i*peek(i);
    }
    pop();
    print(result);
  }
}
\end{verbatim}}

\caption{Original StreamIt code for the buffer test.\protect\label{fig:code-orig}}
\end{minipage}
\hspace{0.3in}
%
\begin{minipage}{2.2in}
\centering
%% modulation
{\scriptsize
\begin{verbatim}
void->void filter BufferTest {
  int PEEK = 4;
  float[4] BUFFER;
  int push_index = 0;
  int pop_index = 0;

  prework {
    for (int i=0; i<PEEK-1; i++) {
      BUFFER[push_index++] = random();
    }
  }

  work {
    // run Source
    BUFFER[push_index] = random();
    push_index = (push_index + 1) & 3;
    
    // run FIR
    float result = 0;
    for (int i=1; i<PEEK; i++) {
      result += i*BUFFER[(pop_index + i) & 3];
    }
    pop_index = (pop_index + 1) & 3;
    print(result);
  }
}
\end{verbatim}}

\caption{Fused buffer test using modulation buffer management strategy.\protect\label{fig:code-modulation}}
\end{minipage}
\vspace{6pt}
\hrule
\end{figure*}

\begin{figure*}
\nocaptionrule
\begin{minipage}{2.2in}
\hspace{-0.1in}\psfig{figure=arm-buf.eps,width=2.4in}
\caption{Performance of buffer management strategies on a StrongARM.\protect\label{fig:buf-arm}}
\end{minipage}
\hspace{0.1in}
\begin{minipage}{2.2in}
\hspace{-0.1in}\psfig{figure=p3-buf.eps,width=2.4in}
\caption{Performance of buffer management strategies on a Pentium~3.\protect\label{fig:buf-p3}}
\end{minipage}
\hspace{0.1in}
\begin{minipage}{2.2in}
\hspace{-0.1in}\psfig{figure=i2-buf.eps,width=2.4in}
\caption{Performance of buffer management strategies on an Itanium~2.\protect\label{fig:buf-itanium}}
\end{minipage}
\vspace{18pt}
\hrule
\begin{minipage}[t]{2.25in}
%% copy-shift
{\FusionFig
\begin{verbatim}


void->void filter BufferTest {
  int PEEK = 4;
  float[3] BUFFER;

  prework {
    for (int i=0; i<PEEK-1; i++) {
      BUFFER[i] = ... ;
    }
  }

  work {
    float[4] TEMP_BUFFER;
    int push_index = 3;
    int pop_index = 0;

    // copy from BUFFER to TEMP_BUFFER
    for (int i=0; i<3; i++) {
      TEMP_BUFFER[i] = BUFFER[i];
    }

    // run Source
    TEMP_BUFFER[push_index++] = ... ;
    
    // run FIR
    float result = 0;
    for (int i=1; i<PEEK; i++) {
      result += i*TEMP_BUFFER[pop_index+i];
    }
    pop_index++;
    print(result);

    // copy from TEMP_BUFFER to BUFFER
    for (int i=0; i<3; i++) {
      BUFFER[i] = TEMP_BUFFER[i+1];
    }
  }
}
\end{verbatim}}

\caption{Copy-shift strategy.\protect\label{fig:copy-shift}}
\end{minipage}
~~\vrule~~
\begin{minipage}[t]{2in}
%% copy-shift + scalar-replacement
{\FusionFig
\begin{verbatim}


void->void filter BufferTest {
  int PEEK = 4;
  float[3] BUFFER;

  prework {
    for (int i=0; i<PEEK-1; i++) {
      BUFFER[i] = ... ; 
    }
  }

  work {
    float TEMP_BUFFER_0;
    float TEMP_BUFFER_1;
    float TEMP_BUFFER_2;
    float TEMP_BUFFER_3;

    // copy from BUFFER to TEMP_BUFFER
    TEMP_BUFFER_0 = BUFFER[0];
    TEMP_BUFFER_1 = BUFFER[1];
    TEMP_BUFFER_2 = BUFFER[2];

    // run Source
    TEMP_BUFFER_3 = ... ;
    
    // run FIR
    float result = 0;
    result += 1*TEMP_BUFFER_1;
    result += 2*TEMP_BUFFER_2;
    result += 3*TEMP_BUFFER_3;
    print(result);

    // copy from TEMP_BUFFER to BUFFER
    BUFFER[0] = TEMP_BUFFER_1;
    BUFFER[1] = TEMP_BUFFER_2;
    BUFFER[2] = TEMP_BUFFER_3;
  }
}
\end{verbatim}}

\caption{Copy-shift with scalar-replacement.\protect\label{fig:code-scalar-replace}}
\end{minipage}
~~\vrule~~
\begin{minipage}[t]{2.25in}
%% copy-shift + scaling
{\FusionFig
\begin{verbatim}


void->void filter BufferTest {
  int PEEK = 4;
  float[3] BUFFER;

  prework {
    for (int i=0; i<PEEK-1; i++) {
      BUFFER[i] = ... ;
    }
  }

  work {
    float[32] TEMP_BUFFER;
    int push_index = 3;
    int pop_index = 0;

    // copy from BUFFER to TEMP_BUFFER
    for (int i=0; i<3; i++) {
      TEMP_BUFFER[i] = BUFFER[i];
    }

    // run Source 16 times
    for (int k=0; k<16; k++) {
      TEMP_BUFFER[push_index++] = ... ;
    }
    
    // run FIR 16 times
    for (int k=0; k<16; k++) {
      float result = 0;
      for (int i=1; i<PEEK; i++) {
        result += i*TEMP_BUFFER[pop_index+i];
      }
      pop_index++;
      print(result);
    }
      
    // copy from TEMP_BUFFER to BUFFER
    for (int i=0; i<3; i++) {
      BUFFER[i] = TEMP_BUFFER[i+16];
    }
  }
}
\end{verbatim}}

\caption{Copy-shift with execution scaling.\protect\label{fig:code-scaling}}
\end{minipage}
\end{figure*}


%% modulation
%% copy-shift
%% scaling
%% scalar replacement

%% - peek scaling amortizes costs of copies
%%   - this is to scale a given filter's execution within a fusion unit

%% - why can't do scalar replacement between filters

%% 2. filter fusion
%%  - for a general class of filters and topologies
%%    - peeking
%%    - parallel composition, not just sequential
%%    --> that was shown in asplos paper

%%  - optimizations
%%    - converting buffers to scalar variables
%%      - also eliminates modulo operations
%%      - not too much new after asplos

%%    - peek scaling
%%      - 

%%    - stack space reuse
%%      - 

%% ----------

%% questions:
%% - have we even compared to a rotating buffer?

%% ------------------------------------------------------------

%% The default implementation of intermediate value buffers
%% in code generated for a StreamIt program is an array.
%% However, array acesses are not very efficient because 
%% of wasted instructions to increment index counter and 
%% to add index counter to the base of the array address 
%% to calculate the memory location.

%% An optimization would be to replace an array of size N with 
%% N scalar variables. In this case we do not need to increment 
%% index variable or add the index variable to the base address, 
%% since target memory location would be fixed for each
%% instruction that accesses the buffer. This optimization
%% allso allows the intermediate variables to be
%% register allocated.

%% Such an optimization is important to improve 
%% performance of fine grained stream programs where each actor
%% performs little amount of work and most of the code just moves 
%% the items betwwen actors (high communication / computation ratio).

%% However, this optimization can not be done if array is accessed in a loop 
%% and array access index is calculated using the loop variable.

%% \begin{verbatim}
%% for (i = 0; i < 5; i++) {
%%    array[index++] = x++;
%% }
%% \end{verbatim}

%% If we fully unroll all such loops for a given array then we 
%% can replace the array with scalar variables.

%% \begin{verbatim}
%% array[0] = x++;          var_0 = x++;
%% array[1] = x++;          var_1 = x++;
%% array[2] = x++;   ====>  var_2 = x++;
%% array[3] = x++;          var_3 = x++;
%% array[4] = x++;          var_4 = x++;
%% \end{verbatim}

%% If we want to replace many arrays then we will have to unroll
%% many loops, this leads to dramatic increase in code size.

%% Despite getting rid of unnecesary instructions we might see
%% an overall performance degradation if we do not schedule
%% actors in a cache aware manner.

%% ------------------------------------

%% \subsection{Peek Optimization}

%% Because of Streamit language having a peek feature, one needs
%% a separate initialization schedule to fill in the peek
%% buffers with data. [ASPLOS'02-Gordon]

%% There are two alternatives for peek implementation. Use a rotating
%% buffer with head and tail pointers or use a peek buffer.

%% Consider a filter that consumes 1 item, peeks 10 (including
%% the one it consumes) and source produces 1.

%% An Example of code generated:

%% \begin{verbatim}
%% item_type PEEK_BUFFER[9];
%% item_type POP_BUFFER[10];

%% 1. Source puts produced item into POP_BUFFER[0];
%% 2. Copy PEEK_BUFFER[0..8] to POP_BUFFER[1..9];
%% 3. Execute the Filter
%% 4. Copy POP_BUFFER[0..8] to PEEK_BUFFER[0..8]

%% \end{verbatim}

%% If $peek rate >> pop$ rate then each item is copied many times.
%% Consider a case where filter has pop rate 1 and peek rate 64,
%% in this case each item has to be copied 128 times to / from PEEK\_BUFFER.

%% We increase filters multiplicity by executing it multiple times.
%% If we execute the above filter $N$ times, then it's pop rate is $N$ 
%% and peek rate is $N+63$ items. If $N=16$ then $poprate=16$ $peekrate=79$
%% and each item is copied to / from PEEK\_BUFFER at most 10 times.

%% Scaling individual filters can result in increase of the buffer 
%% requirements for a single steady state cycyle. 

%% $<$INSERT EXAMPLE$>$

%% Results show that to achieve high performance it is more important to
%% reduce number of item copy operations than reduce buffer size as long
%% as all of the persistent buffers fit into main memory or low speed
%% off-chip cache.

%% \subsection{Reusing Intermediate Storage Variables}

%% Once loops have been unrolled, filters have been fused into a
%% partition and arrays have been replaced with scalar variables we can
%% find approximations of live ranges for the new variables and use this
%% information to reduce stack space required by the partition's work
%% function.

%% We replace the scalar variables that have been created as a result of
%% destroying arrays with a minimal number of variables, where minimal
%% number is the maximum number of overlapping live ranges at any point
%% in the work function of the fused partition.

%% The above optimization improves data access locality.
