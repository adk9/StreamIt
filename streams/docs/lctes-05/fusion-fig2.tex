\begin{figure*}
\nocaptionrule
\begin{minipage}{2.2in}
\hspace{-0.1in}\psfig{figure=arm-buf.eps,width=2.4in}
\caption{Performance of buffer management strategies on a StrongARM.\protect\label{fig:buf-arm}}
\end{minipage}
\hspace{0.1in}
\begin{minipage}{2.2in}
\hspace{-0.1in}\psfig{figure=p3-buf.eps,width=2.4in}
\caption{Performance of buffer management strategies on a Pentium~3.\protect\label{fig:buf-p3}}
\end{minipage}
\hspace{0.1in}
\begin{minipage}{2.2in}
\hspace{-0.1in}\psfig{figure=i2-buf.eps,width=2.4in}
\caption{Performance of buffer management strategies on an Itanium~2.\protect\label{fig:buf-itanium}}
\end{minipage}
\vspace{18pt}
\hrule
\begin{minipage}[t]{2.25in}
%% copy-shift
{\FusionFig
\begin{verbatim}


void->void filter BufferTest {
  int PEEK = 4;
  float[3] BUFFER;

  prework {
    for (int i=0; i<PEEK-1; i++) {
      BUFFER[i] = ... ;
    }
  }

  work {
    float[4] TEMP_BUFFER;
    int push_index = 3;
    int pop_index = 0;

    // copy from BUFFER to TEMP_BUFFER
    for (int i=0; i<3; i++) {
      TEMP_BUFFER[i] = BUFFER[i];
    }

    // run Source
    TEMP_BUFFER[push_index++] = ... ;
    
    // run FIR
    float result = 0;
    for (int i=1; i<PEEK; i++) {
      result += i*TEMP_BUFFER[pop_index+i];
    }
    pop_index++;
    print(result);

    // copy from TEMP_BUFFER to BUFFER
    for (int i=0; i<3; i++) {
      BUFFER[i] = TEMP_BUFFER[i+1];
    }
  }
}
\end{verbatim}}

\caption{Copy-shift strategy.\protect\label{fig:copy-shift}}
\end{minipage}
~~\vrule~~
\begin{minipage}[t]{2in}
%% copy-shift + scalar-replacement
{\FusionFig
\begin{verbatim}


void->void filter BufferTest {
  int PEEK = 4;
  float[3] BUFFER;

  prework {
    for (int i=0; i<PEEK-1; i++) {
      BUFFER[i] = ... ; 
    }
  }

  work {
    float TEMP_BUFFER_0;
    float TEMP_BUFFER_1;
    float TEMP_BUFFER_2;
    float TEMP_BUFFER_3;

    // copy from BUFFER to TEMP_BUFFER
    TEMP_BUFFER_0 = BUFFER[0];
    TEMP_BUFFER_1 = BUFFER[1];
    TEMP_BUFFER_2 = BUFFER[2];

    // run Source
    TEMP_BUFFER_3 = ... ;
    
    // run FIR
    float result = 0;
    result += 1*TEMP_BUFFER_1;
    result += 2*TEMP_BUFFER_2;
    result += 3*TEMP_BUFFER_3;
    print(result);

    // copy from TEMP_BUFFER to BUFFER
    BUFFER[0] = TEMP_BUFFER_1;
    BUFFER[1] = TEMP_BUFFER_2;
    BUFFER[2] = TEMP_BUFFER_3;
  }
}
\end{verbatim}}

\caption{Copy-shift with scalar-replacement.\protect\label{fig:code-scalar-replace}}
\end{minipage}
~~\vrule~~
\begin{minipage}[t]{2.25in}
%% copy-shift + scaling
{\FusionFig
\begin{verbatim}


void->void filter BufferTest {
  int PEEK = 4;
  float[3] BUFFER;

  prework {
    for (int i=0; i<PEEK-1; i++) {
      BUFFER[i] = ... ;
    }
  }

  work {
    float[32] TEMP_BUFFER;
    int push_index = 3;
    int pop_index = 0;

    // copy from BUFFER to TEMP_BUFFER
    for (int i=0; i<3; i++) {
      TEMP_BUFFER[i] = BUFFER[i];
    }

    // run Source 16 times
    for (int k=0; k<16; k++) {
      TEMP_BUFFER[push_index++] = ... ;
    }
    
    // run FIR 16 times
    for (int k=0; k<16; k++) {
      float result = 0;
      for (int i=1; i<PEEK; i++) {
        result += i*TEMP_BUFFER[pop_index+i];
      }
      pop_index++;
      print(result);
    }
      
    // copy from TEMP_BUFFER to BUFFER
    for (int i=0; i<3; i++) {
      BUFFER[i] = TEMP_BUFFER[i+16];
    }
  }
}
\end{verbatim}}

\caption{Copy-shift with execution scaling.\protect\label{fig:code-scaling}}
\end{minipage}
\end{figure*}
