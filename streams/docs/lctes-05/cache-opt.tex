\section{Cache Optimizations}
\label{sec:cache-opt}

In this section we describe a set of cache-aware transformations that
are geared toward improving the instruction and temporal locality of a
stream graph. We first describe our methodology for scaling actor
firings, and subsequently we describe an optimization that is geared
toward reducing the static and dynamic data requirements of
producer-consumer pairs.

% - multiplicity algorithm
% - forward reference to results

% - partitioning algorithm
% - forward reference to results

\subsection{Multiplicity Scaling}

The goal of this optimization is to increase the execution frequency
of the actors in a steady state schedule, such that the temporal
locality in the memory hierarchy is improved. Our approach is to find a
single multiplicity factor \texttt{M} for scaling all actors, such that
at least 90\% of the actors have a data workingset that fits in the cache.
In other words, using our model, we calculate \texttt{M} such that 
$S(f) + D(f) \leq C_D$
% and $DCM(f) = 0$ % this is wrong!
for 90\% of the actors in the stream graph.

The algorithm 
computes the largest scaling factor for every actor such that its data
workingset does not exceed the size of the primary data cache
(16~Kb on a Pentium~3 for example). This yields a set of multiplicities
${\texttt{M}_0, \ldots, \texttt{M}_n}$ for the actors in the stream
graph. For example, the algorithm might calculate
$\texttt{M}_\texttt{A} = 10$,
$\texttt{M}_\texttt{B} = 20$, 
$\texttt{M}_\texttt{C} = 30$, 
$\texttt{M}_\texttt{D} = 40$ for four filters \texttt{A}, \texttt{B},
\texttt{C}, and \texttt{D} in a stream graph.
The algorithm then examines the
distribution and selects the largest $\texttt{M}_i$
for which 90\% of the actors satisfy our data workingset criterion.
In the example, the 90-10 heuristic
selects $\texttt{M} = 10$, otherwise, the workingset of filter
\texttt{A} will exceed the cache. By contrast,
a 75-25 heuristic selects $\texttt{M} = 20$, where 75\% of the
data workingsets are smaller than the cache, and the remaining 25\%
overflowing it.

The 90-10 heuristic was determined empirically. We found that
increasing the threshold beyond 90\% leads to marginal
performance gains. The graphs in Figure~\ref{fig:9010} show the
performance (y-axis) of our benchmarks for different values of the multiplicity
factor (x-axis). In each plot, the diamond represents the \texttt{M} chosen via
the 90-10 rule, and the square represents the largest calculated
\texttt{M} (e.g., in the example above, the largest \texttt{M} is 40).

\subsection{Actor Coarsening}

In StreamIt, the computation boundaries (i.e., the granularity of a
filter) are determined by the application developed according to the
most natural representation of an algorithm. When compiling to a
cache-based architecture, the presence of a  large number of filters
can lead to a high overhead in terms of context swapping between work
functions, and buffer management for the shared data between filters.
It is the compiler's role to adjust the granularity of execution such
that actors do not vary significantly in terms of code  or data
workingset size.

We use a greedy partitioning algorithm to find a set of vertical cuts
in a stream graph such that the resulting partitions are more or less
balanced. The algorithm uses code size estimation to quantify the
instruction footprint of a partition, and analyzes the work function
declarations to estimate the data buffer requirements. The algorithm
also considers the amount of static state that is persistent across
filter executions. This approach is similar to that in
\cite{streamit-asplos} with two distinctions. First, the algorithm
only considers vertical fusion (i.e., within a pipeline): horizontal
fusion is not beneficial from  caching perspective, and may not even
be feasible. Second, since the graph partitioning is limited to
vertical cuts, our actor coarsening optimizations employs a greedy
algorithm instead of a dynamic programing methodology; this reduces
compilation time. 

The cache-aware coarsening of actors, which we will also refer to as
cache-aware fusion, greatly impacts performance as
our results in Section~\ref{sec:evaluation} will demonstrate.