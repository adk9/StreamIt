\section{Cache Model for Streaming}
\label{sec:cache-model}

In this section we describe a simple and intuitve cache model to
estimate the instruction and data cache behavior given an execution
ordering of the filters in the stream graph.

The model is based on the notion of a filter reuse distance which is 
based on LRU stack distance~\cite{mattson-reuse}. We develop the model
first first the instruction cache, and then subsequently evolve it to
account for the data cache.

\subsection{Instruction Cache}

A {\it filter execuion trace} is a sequence of executions of
filter work functions: $T = f_0 f_1 ... f_n$; $f_i$ represents the
execution of filter $f$ at logical time $i$.
The filter {\it instruction reuse distance} ($IRD$) is defined as the
number of unique instructions that are referenced between two
instances of the same work function. Formally, we define $phase(f_i)$
as the sequence $f_i ... f_l$ ($i < l$) of work functions occuring in
$T$ such that $f_i = f_l$ and no where else (i.e., $f_i \neq f_k$
$\forall{k}~~s.t.~~i < k < l$). The $IRD(f_i) = \sum I(f_j)$ over all
distinct filters occuring in $phase(f_i)$, and $I(f)$ equals the code
size of a filter $f$. Based on this reuse metric, we define an
instruction-cache-miss step function $ICM(f_i)$ such that:
\begin{equation}
\label{eq:icm}
  ICM(f_i) =
    \begin{cases}
      0& \text{if $IRD(f_i) \leq C_I$; hit/no cache refill},\\
      1& \text{else; a miss/(some) cache refill}.\\
    \end{cases}
\end{equation}
Note that if $f_i$ is the last occurence of a filter $f$ in the trace,
then $ICM(f_i) = 0$. In Equation~\ref{eq:icm}, $C_I$ represents a
consant proportional to the instruction cache size.

We do not account for the cold start
effects of an execution trace. This does not effect our model, and can
be easily remedied if necessary. To see why this is so, consider the
following two execution traces which represent the executions of work
functions for two filters \texttt{A} and \texttt{B}:
$T_1 = \texttt{A}_1\texttt{A}_2\texttt{B}_3\texttt{B}_4$
and 
$T_2 = \texttt{A}_1\texttt{B}_2\texttt{A}_3\texttt{B}_4$.
In both 
traces, the number of cold starts is the same (i.e., two in all).

What our reuse metric quantifies is the number of misses when a specific work
function is reinvoked. For example, assume that each filter is of the exact
size: $I(\texttt{A}) = I(\texttt{B}) = \lceil{C_I / 2}\rceil + 1$;
that is, the combined instruction workingset of both filters exceeds
the instruction cache, although each filter has a code size that is
smaller than the cache size. Naturally, we expect the trace $T_1$ to
suffer one cold start miss and no other misses for filter \texttt{A}
(and similarly for the other filter). Our model is in accord with
intuition:  $ICM(\texttt{A}_1) = 0$ since
the $IRD(\texttt{A}_1) < C_I$.

In the case of trace $T_2$, we know that since the combined
instruction workingset of the filters exceeds the cache size, when
filter \texttt{B} is invoked following \texttt{A}, it evicts part of
filter \texttt{A}'s instruction workingset. Hence when we transition
back to run filter \texttt{A}, we have to refetch certain
instructions, but in the process, we replace parts of filter
\texttt{B}'s workingset. Hence, we have suffered one cold start miss,
and one capacity miss for filter\texttt{A}, and similarly for
\texttt{B}. Our model is again in accord with intuition: 
$ICM(\texttt{A}_1) = 1$ since
the $IRD(\texttt{A}_1) > C_I$. Note that the
amount of refill is proportional to the number of cache lines that are
replaced when swapping filters, and as such, we may wish to adjust
our cache miss step function. One simple variation is to allow for
some partial replacement without unduly penalizing the overall value
of the metric. Namely, we can allow the constant $C_I$ to be some
fraction greater than the actual cache size. Alternatively, we can use
a more complicated miss function with a more uniform probablity
distribution.

Using the metrics above, we can estimate the instruction-cache miss
rate $(IMR)$ as $\sum ICM(f_i)$ over all $f_i$ in the trace, divided
by the size of the trace. Note that difference execution orderings for
the same stream graph lead to traces of the same size and hence the
comparison are fair. The model allows us to order the quality of an
execution ordering, with scheduling decisions that boost temporal
locality yielding miss rates closer to zero (and schedule that do not
exploit temporal locality yielding traces with miss rates closer to
one).

Our model predicts that scaling is always beneficial from an
instruction-cache point of view. As a reminder, by scaling, we mean
that a filter's work function is run for a greater number of times in
the steady state before transitioning to the next filter in the
graph. Hence for example, a program that would generate a trace $T_2$
might be as follows:
\begin{verbatim}
run_steady_state() {
  for (i = 0; i < 1; i++) A_work();
  for (i = 0; i < 1; i++) B_work();
}
\end{verbatim}
but a scaled program that increases temporal locality is as follows:
\begin{verbatim}
run_steady_state() {
  for (i = 0; i < M * 1; i++) A_work();
  for (i = 0; i < M * 1; i++) B_work();
}
\end{verbatim}
where \texttt{M} (the multiplicity factor) is an integer greater than
one. From our model, it is easy to see that scaling reduces the number
of misses for a given filter $f$ from $K$ to $K/\texttt{M}- 1$, where $K =
\sum ICM(f)$ for all occurences of $f$ in  the execution trace.

\subsection{Data Cache}

As noted earlier, we can not arbitrarily scale the execution frequency
of a filter wihout also considering how scaling might impact
the data buffer sizes between filers. In this regard, a filter's
output buffer size is also constrained by the amount of state that a
filter retains with every execution of its work function.

Clearly if a filter has any static data (e.g., state information or
coeffecient arrays), then it is prudent to maximize their temporal
access locality. We can define a data-cache miss rate equation ($DMR$) based on
a derivation similar to that for the instruction-cache miss rate:
replace $C_I$ with $C_D$ in Equation~\ref{eq:icm}, and $I(f_i)$ with
$S(f_i)$ when calculating the {\it data reuse distance} ($DRD$). 
Here, $C_D$ represent a constant proportional to the data cache size,
and $S(f)$ represents the total size of the static data in the
specified filter.

The presence of static data in a filter implies that we have to limit
the scaling of a filter $f$ such that it does not require more than $C_D -
S(f_i)$ bytes of buffer space for reading and writing data; otherwise
the data workingset of the filter will overflow the cache and we lose
produce-consumer locality. The buffer requirements are representend by
$Y(f_i)$, and the data-cache model accounts for the dynamic data
requirements by adjusting the data reuse distance as follows:
$DRD(f_i) = \sum (S(f_i) + Y(f_i))$ over all distinct filters occuring
in $phase(f_i)$.

Intuitively, the dynamic data accounts for the size of the buffer
required for writing new values, as well as the size of the buffer
necessary for reading values. It also adjusts for any
producer-consumer locality since the output buffer of one filter is
also the input buffer of its sibling in the stream graph. What this
measures tells us is that we can scale the execution of a filter as
long as its buffer requirements (for reading and writing) do not
exceed the cache, and furthermore, as long as the input-output rates
between producer consumer pairs are not grossly mismatched. This later
qualification is important and motivates a series of cache
optimizations discussed in the next section. In the case of rate
mistmatch, the metric tells us that the effective cache size is
reduced, or in other words, the data that is left over after a
producer-consumer firing must be preserved until a future occurence of
the same pair of filters. This translates to lower data locality and
degrade cache performance.

Mathematically. the dynamic data measure is defined as:
\begin{eqnarray}
  \nonumber
  Y(f_i) &=&\sum_{f_k} min(W(f_k), U(f_k) \times A(f_k)) + \\
  \nonumber
	   &&\sum_{f_k} min(R(f_k), O(f_k) \times A(f_k)) - \\
  \nonumber
         &&\sum_{f_s} O(f_s) \times A(f_s)
\end{eqnarray}
with $W(f)$ equal to the output buffer size reserved for writing data, $R(f)$
equal to the input buffer size reserved for reading data, $U(f)$ equal to the
{\tt push} rate (in bytes) of the filter, $O(f)$ equal to the {\tt pop} rate (in
bytes) of the filter, and $A(f)$ equal to the number of occurences of
filter $f$ in $phase(f_i)$. Also note that the $Y(f_i)$ is defined
over all distinct occurences of $f_k$ in the phase, and $f_s$
represents the filter that comsumes the data produced by $f_k$ (i.e.,
it is $f_k$'s successor in the stream graph, and $f_k$-$f_s$
constitute a producer-consumer pair).

The first term in the equation above quantifies the address space
accessed for writing data. It is  equal to the lesser of the buffer size
reserved for output, and the total number of items produced by
the filter (i.e., the push rate multiplied by the number of times the
work function fired in the phase); this avoid over estimating the
address space when an exceedingly large buffer is reserved but only a
portion of it is used for writing in a phase.

The second term in the equation quantified the referenced address space 
for reading the input data. This term is also the lesser of two
values: the input buffer size, and the total number of bytes referenced for
reading data.

The third term avoid double counting since the output buffer of one
filter is also the input buffer of its successor in the stream
graph. The term quantifies the amount of data that is consumed by the
producer's successor, and hence release a portion of the address space
for use by other filters in the phase.

\subsection{Remarks}

Note  that we can  combine the  instruction and  data cache  miss rate
equations to yield one measure that estimates the temporal locality of
a streaming computation.

Also note that while we use  an execution trace to describe our model,
it  is possible  to  calculate  the instruction  and  data cache  miss
metrics at  compile time.  To do so,  we can leverage  the hiearchical
StreamIt   representation  that  dictates   the  ordering   of  filter
executions.
