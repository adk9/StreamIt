\subsection{Instruction Cache}

A steady state execution is a sequence of actor firings
$S=(a_1,\ldots, a_n)$, and a
{\it program execution} corresponds to one or more repetitions of the
steady state. We use the notation $S[i]$ to refer to the
actor $a$ that is fired at logical time $i$, and $|S|$ to
denote the length of the sequence. 

Our cache model is simple in that it considers each actor in the
steady state sequence, and determines whether one or more misses are
bound to occur. The miss determination is based on the 
the {\it instruction reuse distance} ($\mt{IRD}$), which is equal to
the number of unique instructions that are referenced between two
executions of the actor under consideration (as they appear in the
schedule). The steady state is a 
compact representation of the whole program execution, and thus, we
simply account for the misses within a steady state, and generalize
the result to the whole program. Within a steady state, an actor is
charged a miss penalty if and only if the number of referenced
instructions since the last execution (of the same actor)
is greater than the instruction cache capacity.

Formally, let $\mt{phase}(S,i)$ for $1\le i\le|S|$ represent a
subsequence of $k$ elements of $S$:
\[ 
\mt{phase}(S,i) = ( S[i], S[i+1], \ldots, S[i+k-1] )
\]
where $k\in[1,|S|]$ is the smallest integer such that $S[i+k]=S[i]$.
In other words, a phase is a subsequence of $S$ that starts with the
specified actor ($S[i]$) and ends before the next occurrence of the same actor
(i.e., there are no intervening occurrences of $S[i]$ in the phase).
Note that because the steady state
execution is cyclic, the construction of the subsequence is allowed to
wrap around the steady state\footnote{In other words, the subsequence
is formed from a new sequence $S'=S|S$ where $|$ represents
concatenation.}. For example, the steady state $S_1=(\texttt{AABB})$
has 
$\mt{phase}(S_1,1)=(\texttt{A})$,
$\mt{phase}(S_1,2)=(\texttt{ABB})$,
$\mt{phase}(S_1,3)=(\texttt{B})$, and
$\mt{phase}(S_1,4)=(\texttt{BAA})$,


Let $I(a)$ denote the code size of the work function for actor $a$.
Then the instruction reuse distance is
\[
%\mt{IRD}(S,i)=\sum_{j \in \mt{phase}(S,i)} I(S[j])
\mt{IRD}(S,i)=\sum_{a} I(a)
\]
where the sum is over all distinct actors $a$ occurring in
$\mt{phase}(S,i)$. We can then determine if a specific actor will
result in an instruction cache miss (on its next firing) by evaluating
the following step function:
\begin{equation}
\label{eq:imiss}
  \mt{IMISS}(S,i) =
    \begin{cases}
      0& \text{if $\mt{IRD}(S,i) \leq C_I$; hit: no cache refill,}\\
      1& \text{otherwise; miss: (some) cache refill.}
    \end{cases}
\end{equation}
In the equation, $C_I$ represents the instruction cache size. 

Using 
%the instruction miss ($\mt{IMISS}$) metric in
Equation~\ref{eq:imiss}, we can estimate the instruction miss
rate ($\mt{IMR}$) of a steady state as: 
\begin{equation}
\label{eq:imr}
  \mt{IMR}(S) = \frac{1}{|S|}\sum_{i=1}^{|S|} \mt{IMISS}(S,i).
\end{equation}

The cache model allows us to rank the quality of an
execution ordering: schedules that boost temporal locality
result in miss rates closer to zero, and schedules that do not
exploit temporal locality result in miss rates closer to one.

For example, in the steady 
state $S_1=(\texttt{AABB})$, assume that the combined instruction working
sets exceed the instruction cache, i.e., $I(\texttt{A}) + I(\texttt{B}) > C_I$.
Then, we expect to suffer a miss at the start of every
steady state because the phase that precedes the execution of
\texttt{A} (at $S_1[1]$) is $\mt{phase}(S_1,2)$ with 
an instruction reuse distance greater than the cache size 
($\mt{IRD}(S_1,2) > C_I$). Similarly, there is a 
miss predicted for the first occurrence of actor \texttt{B} since
$\mt{phase}(S_1,4)=(\texttt{BAA})$ and 
$\mt{IRD}(S_1, 4)>C_I$. Thus, $\mt{IMR}(S_1)=2/4$ whereas for the
following variant $S_2=(\texttt{ABAB})$, $\mt{IMR}(S_2)=1$.
In the case of $S_2$, we know that since the combined
instruction working sets of the actors exceed the cache size, when
actor \texttt{B} is fired following \texttt{A}, it evicts part of
actor \texttt{A}'s instruction working set. Hence when we transition
back to fire actor \texttt{A}, we have to refetch certain
instructions, but in the process, we replace parts of actor
\texttt{B}'s working set. In terms of our model, $\mt{IRD}(S_2,i)>C_I$
for every actor in the sequence, i.e., $1\le i\le|S_2|$.

Note that the
amount of refill is proportional to the number of cache lines that are
replaced when swapping actors, and as such, we may wish to adjust
our cache miss step function ($\mt{IMISS}$). One simple variation is to allow for
some partial replacement without unduly penalizing the overall value
of the metric. Namely, we can allow the constant $C_I$ to be some
fraction greater than the actual cache size. Alternatively, we can use
a more complicated miss function with a more uniform probability
distribution.

\paragraph*{Temporal Locality} In our model, the concept of improving
temporal instruction locality translates to deriving a steady state where, in the
best case, every actor has only one phase that is longer than unit-length.
For example, a permutation of the actors in $S_2$ (where all
phases are of length two) that improves temporal locality
will result in $S_1$, which we have shown has a relatively lower miss rate.


\paragraph*{Execution Scaling}
Another approach to improving temporal locality is to scale the
execution of the actors in the steady state. Scaling increases the
number of consecutive firings of the same actor. In our model, a
scaled steady state has a greater number of unit-length phases (i.e., a
phase of length one and the shortest possible reuse distance).

We represent a scaled execution of the steady state as
$S^m=(a_1^m,\dots, a_n^m)$: the steady state $S$ is scaled by $m$, which
translates to $m-1$ additional firings of 
every actor. For example, scaling $S_1=(\texttt{AABB})$ by a factor of
two results in
%$S_1^2=(\texttt{\underline{A}A\underline{A}A\underline{B}B\underline{B}B})$
$S_1^2=(\texttt{AAAABBBB})$
and scaling $S_2=(\texttt{ABAB})$ by the same amount results in 
%$S_2^2=(\texttt{\underline{A}A\underline{B}B\underline{A}A\underline{B}B})$.
$S_2^2=(\texttt{AABBAABB})$;
%; the underlined actors constitute the unscaled steady state.

From Equation~\ref{eq:imiss}, we observe that unit-length phases do
not increase the instruction miss rate  as long as the size of the actor's 
instruction working set  is smaller than the cache
size; we assume this is always the case. Therefore, scaling has the
effect of preserving the pattern of 
miss occurrences while also lengthening the steady state. Mathematically,
we can substitute into Equation~\ref{eq:imr}:
\begin{eqnarray}
  \nonumber
  \mt{IMR}(S^m)  &=& \frac{1}{|S^m|}\sum_{i=1}^{|S^m|} \mt{IMISS}(S^m,i) \\
  \nonumber
                 &=& \frac{1}{m \times |S|}\sum_{i=1}^{|S^m|} \mt{IMISS}(S^m,i). \\
  \label{eq:imrM}
                 &=& \frac{1}{m \times |S|}\sum_{i=1}^{|S|} \mt{IMISS}(S,i).
\end{eqnarray}
The last step is possible because $\mt{IMISS}$ is zero for $m-1$ out
of $m$ executions of every scaled actor.  The result is that the miss
rate is inversely proportional to the scaling factor.
\begin{figure}[t]
\begin{center}
  \psfig{figure=fftc-mult2.eps, width=\columnwidth}
% \nocaptionrule
  \caption{Impact of execution scaling on performance.}
 \label{fig:scaling-data}
\end{center}
\end{figure}

In Figure~\ref{fig:scaling-data} we show a
representative curve relating the scaling factor to overall
performance. The data corresponds to a coarse-grained implementation of 
a Fast Fourier Transform (FFT) running on a Pentium~3 architecture. The
x-axis represents the scaling factors (with increasing values from
left to right). The y-axis represents the execution time and is an
indirect indicator of the miss rate (the two measures are positively
correlated). The execution time improves in accord with our model: 
the running time is shortened as the scaling factor grows larger. There
is however an eventual degradation, and as the sequel will show, it is 
attributed to the data cache performance.
