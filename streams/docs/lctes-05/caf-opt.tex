In StreamIt, the granularity of actors is determined by the
application developer, according to 
the most natural representation of an algorithm. When compiling to a
cache-based architecture, the presence of a  large number of actors
exacerbates the transition-overhead between work functions.
It is the role of the compiler  to adjust the granularity of
the stream graph to mitigate the execution overhead.

In this section we describe an actor coarsening technique we refer to
as {\it cache aware fusion} (CAF). When two actors are fused, they
form a new actor whose work function is equivalent to its constituents.
For example, let an actor \texttt{A} fire $n$
times, and an actor \texttt{B} fire $2n$ times per steady:
$S^n=(\texttt{A}^n\texttt{B}^n\texttt{B}^n)$. A new actor \texttt{F}
that is formed by fusing \texttt{A} and \texttt{B} results in
\texttt{F} firing $n$ times per steady state $(S^n=(\texttt{F}^n))$, but
each execution of \texttt{F} is equivalent to  one 
firing of \texttt{A} and two firings of \texttt{B}. If other terms, the work
function for actor \texttt{F} inlines the work functions of
actors \texttt{A} and \texttt{B}.

When two actors are fused, their executions are scaled such that the output
rate of one actor matches the input rate of the next. In the example,
\texttt{A} and \texttt{B} represent a producer-consumer pair of
filters within a pipeline, with filter \texttt{A}  
pushing two items per firing, and \texttt{B} poping one item per
firing. The fusion implicitly scales the execution of \texttt{B} so
that it runs twice for every firing of \texttt{A}.

%Fusion is important because it leads to coarser grained actors with
%I/O rates that are more closely matched to their
%neighbors. Furthermore, the fused actors themselves become the target
%the execution scaling which results in better producer-consumer locality.

Fusion also reduces the overhead of switching between
work functions. In our infrastructure, the steady state is a loop that
invokes the work functions via method calls. Thus, every pair of fused
actors eliminates a method call (per invocation of the actors). The impact on
performance can be significant, but not only because method calls are
removed: the fusion of two actors also enables the compiler to
optimize across function boundaries. In particular, for actors that
exchange only a few data items, the compiler can allocate the data
streams to registers. 
%For example, in the example above, the values
%produced by \texttt{A} may be passed to \texttt{B} via registers, rather
%than through memory. 
The data channel between fused actors is
subject to special buffer management techniques as described in the
next section.

There are, however, downsides to fusion. First, as more and more actors are
fused, the instruction footprint can dramatically increase, possibly
leading to poor use of the instruction cache. Second, fusion also
increases the data footprint when the fused actors maintain state
(e.g., coefficient arrays and lookup tables). Our fusion algorithm is
cache aware specifically because it is cognisant of the instruction
and data factors.

The CAF algorithm uses a greedy fusion heuristic to select the set of
actors to fuse. It continuously fuses actors until the addition of
a new actor causes the fused actor to exceed {\it either} the instruction cache
capacity, or a fraction of the data cache capacity. For the latter, we
allow the state of the new fused actor to occupy up to 50\% of the
data cache capacity.

The algorithm leverages the hierarchical nature of the stream graph,
starting at the leaf nodes and working upwards.  For pipeline streams,
the algorithm identifies the connection in the pipeline with the
highest steady-state I/O rate, i.e., the pair of filters that
communicate the most number of items per steady state.  These filters
are fused if the instruction and data cache constraints are respected.
To prevent fragmentation of the pipeline, each fused filter is further
fused with its upstream and downstream neighbors so long as the
constraints are met.  The algorithm then repeats this process with the
next highest-bandwidth connection in the pipeline, continuing until no
more filters can be fused.  For splitjoin streams, the CAF algorithm
fuses all parallel branches together if the combination satisfies the
instruction and data constraints.  Partial fusion of a splitjoin is
not helpful, as the child streams do not communicate directly with
each other; however, complete combination can enable further fusion in
parent pipelines.

%% This differs from that in
%% \cite{streamit-asplos} in that the algorithm
%% only considers vertical fusion (i.e., within a pipeline): horizontal
%% fusion is not beneficial from caching perspective. \framebox{Expand on this} \framebox{Emphasize order:  fuse then scale.}
%Second, since the 
%graph partitioning is limited to
%horizontal cuts, our actor coarsening optimizations employs a greedy
%algorithm instead of a dynamic programing methodology; this reduces
%compilation time. 

%% \begin{figure*}[t]
%% \begin{verbatim}
%% Pipeline:

%% Calculate number of partitions required for each child.
%% If for any child this is >1 then remember those partitions.

%% For each sequence (i..j) of children where for each child 
%% number of partitions is 1.

%% Interval(i,j) = "
%%     Find maximum bandwidth connection between children. (k, k+1)
%%     Consider partition (k..k+1) 
%%     If partition (k..k+1) does not satisfy the requirement then
%%         Call Interval(i,k) and Interval(k+1,j) to find two sets of partitions.

%%     Else Start with (k, k+1) fused try fusing up or down
%%         until can not fuse up and down let this be (l, m)
%%         Remember (l..m) as a partition.
%%         Use Interval(i,l) and Interval(m,j) to find partitions."

%% SplitJoin:

%% Consider splitter/joiner and all branches in a single partition.
%% If this satisfies requirements then return a single partion
%% else remember partitions corresponding to each branch as final
%% partitions.
%% \end{verbatim}
%% \caption{Pseudocode of a greedy algorithm to find partitions}
%% \label{fig:greedy}
%% \end{figure*}

