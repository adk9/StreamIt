\documentclass[draft]{article}
\usepackage{psfig, verbatim, fullpage}
\title{The StreaMIT Programming Language \\ A Reference Manual \\ Version 1.0}

\begin{document}
\maketitle

\section{Open questions}

\begin{enumerate}

\item If we're building on top of C, then how do we pass a string
between filters without using pointers?  What is the exact restriction
on pointers, then?

\item More generally than above, what's the syntax of stuff that's not
streaming?  Is it C, Java, something subset or combination thereof?

\item Isn't it just a suggestion that {\tt input[0]} and {\tt
output[0]} reference the first unread and unwritten items?  Can't the
programmer manage the indices how he wants to using the {\tt advance}
command?

\item do we have a name for a {\tt FilterBlock} that is both a source
and a sink?  A {\tt SourceSinkBlock}?  Maybe we should make that the
top-level construct instead of {\tt Stream}?

\end{enumerate}

\section{Realizations / Changes}

\begin{enumerate}

\item When we're inlining filters, we don't need to provide the name
of the filter.

\item It's a misnomer to call the toplevel stream a {\tt FilterBlock},
since it has no input and output.  Let's call it a {\tt Stream}
instead?  Also, why name it, since there can be only one of them and
it's not accepting parameters?  Let's just say {\tt Stream} at the
toplevel.

\end{enumerate}

\section{To improve this document...}

\begin{enumerate}

\item Better / more clear distinction between initialization time and
steady state streaming time.  And then scripting time, too.

\item Better names for everything.  Especially control constructs
(graph construction constructs instead?), filters, ``functional
blocks'', etc.

\end{enumerate}

\section{Introduction}
\label{sec:intro}

The StreaMIT programming language is designed to enhance the
programmability and performance of streaming applications.  In
providing high-level constructs for building streams out of a number
of simple blocks, the language provides a clean, expressive, and
modular framework that programmers can use to reason about streaming
computations.  At the same time, the language imposes a structure on
the streams which can be exploited by a compiler to perform
stream-specific optimizations.  Thus, the abstractions provided by the
language serve not only to improve the programming model, but to
facilitate performance improvements that motivate the language in
their own right.

The following example defines a stream that prints a continuous
sequence of ``Hello World!'' messages to the screen.  It illustrates
the main components of StreaMIT:

\begin{verbatim}
Stream {

  Source {
    output String myOutput;

    work() {
      output[0] = ``Hello '';
      output.advance(1);
    }
  }

  Filter {
    input String myInput;
    input String myOutput;

    work() {
      output[0] = input[0] + ``World!'';
      input.advance(1);
      output.advance(1);
  }

  Sink {
    input String myInput;
    
    work() {
      System.out.println(input[0]);
      input.advance(1);
    }
  }
}
\end{verbatim}

In StreaMIT, the basic processing unit is the {\tt Filter}, which
reads items from an input channel and writes items to an output
channel.  A filter that requires no input or output is called a {\tt
Source} and a {\tt Sink}, respectively.  In the example above, a {\tt
Source} generates the ``Hello'' string, a {\tt Filter} appends
``World!'', and a {\tt Sink} prints the result to the screen.

Each filter defines a {\tt work} function that represents the most
fine-grained step of execution in transforming data from the input
channel to the output channel.  This function is called automatically
by the runtime system when the stream is running.  The input and
output channels are referenced as arrays, with {\tt input[0]}
representing the first unread input and {\tt output[0]} representing
the first unwritten output.  The {\tt advance} command is used to
shift the channel indices so that unprocessed elements (at positive
indices) are shifted towards the 0'th position.  In the ``Hello
World'' example, each filter reads or writes a single item before
advancing the tape for the next step.

Filters implicitly are connected together in the order of their
definition.  Thus, in the ``Hello World!'' example, the three filters
are pipelined together and enclosed within a {\tt Stream} construct,
which denotes the top-level stream in the program.

The following sections describe the StreaMIT language in more detail.
First we present the {\it base language}, which encompasses all of the
expressive power of StreaMIT.  Then we discuss the {\it scripting
language}, a layer on top of the base language that simplifies stream
construction via the use of compile-time macros.

\section{The Base Language}

The StreaMIT base language allows the programmer to define processing
units and to arrange them into a stream.  Thus, the base language can
be broken down into two levels: 1) the definition of {\it functional
blocks} to transform one data stream into another, and 2) the use of
{\it control constructs} to arrange the objects into a streaming
structure.  Additionally, the language provides a set of dynamic
constructs for passing messages, re-initializing stream components,
and modifying the structure while the stream is executing.

In the following sections, we describe the StreaMIT base language from
the bottom up: functional blocks (Section \ref{sec:blocks}), control
constructs (Section \ref{sec:control}), and dynamic constructs
(Section \ref{sec:dynamic}).

\subsection{Functional Blocks}
\protect\label{sec:blocks}

There are two kinds of functional blocks in StreaMIT.  The
lowest-level block is the {\tt Filter}, which defines a leaf node;
that is, the code within a filter is organized procedurally instead of
as a stream.  The second construct, the {\tt FilterBlock}, represents
a composition of other filters into a streaming structure.  Though a
filter block does not contain procedural code like a filter, we'll see
that filters and filter blocks can be treated as identical black-box
abstractions during stream construction.

\subsection{Filters}

In many ways, filters correspond to the usual notion of an object in
an object-oriented programming language.  Below we discuss their
declaration and instantiation.

\subsubsection{Filter Declaration}

They are declared with the following syntax:

\begin{tabbing}
\hspace{0.2in} \= \hspace{0.2in} \= \hspace{0.2in} \= \hspace{0.2in} \= \\

{\tt Filter} filter-name (type$_1$ param$_1$, type$_2$ param$_2$,
... , type$_n$ param$_n$) \{ \\ \\

\> {\tt input} input-type input-name; \\
\> {\tt output} output-type output-name; \\ \\

\> {\it data fields} \\ \\

\> {\it initialization code} \\ \\

\> {\tt work()} \{ \\ \\
\> \} \\ \\

\> {\it other functions} \\

\}
\end{tabbing}

We discuss the implications of the above syntax in the order that it
appears:

\begin{itemize}

\item {\it Type declaration.}  The {\tt Filter} type indicates that
there is both an input and output channel for this block.  If there is
only an output channel, the type should be {\tt Source}; if there is
only an input channel, the type should be {\tt Sink}.

\item {\it Naming filters.}  The ``filter-name'' denotes the name for
the type of filter being defined.  This name is used to instantiate
the filter (see below).

\item {\it Fields, parameters, and scoping.}  Data fields can be
declared as in C++ or Java.  All fields are implicitly private--they
are not visible from outside the scope of the filter.  Symmetrically,
no variables defined globally or in an enclosing scope are visible
from within a filter.  However, the filter can reference globally
visible functions that have no side effects and are only pure
functions of their arguments.

The filter can receive any number of parameters upon its
initialization.  These parameters act like data fields that are
initialized to the values that are passed as arguments to the filter.
That is, they are visible to all functions within the filter, they are
part of the filter's persistent state, and they can be modified like
normal fields.

\item {\it Input and output.}  The first field declarations are for
the input and output channels, which represent the means for passing
data through the filter during steady state stream operation.  Each
channel must be typed and given a name for use within the filter.  All
items within a channel are passed by value (with semantics equivalent
to copying each value before writing it to the channel), and no
pointers to global structures can be communicated through a channel.

The input and output channels are referenced as arrays.  The first
time the work function is called, {\tt input[0]} and {\tt output[0]}
represent the front of the input and output queue, resepectively.  One
can read or write the k'th item by referencing the k'th element of the
array.  However, to simplify coding practice, we provide the {\tt
advance} command to shift the indices of a channel.  Thus, after
executing {\tt input.advance(k)}, the contents of {\tt intput[k]}
should now be referenced at {\tt input[0]}.  Of course, the indexing
scheme of the previous block's output is unaffected.

\item {\it Initialization code.}  The initialization code is run when
the filter is first instantiated, just like the constructor to an
object.  The initialization code can modify data fields and
parameters, as well as read input and write output.  We allow I/O from
the initializer in case a filter needs to handle the first few data
items differently than the steady state.

\item {\it Work function.}  The work function holds the code that is
invoked dynamically for a single execution step of the filter.  This
function should be as fine-grained as possible--involving as few input
and output items as possible--to promote program comprehensibility and
to facilitate compiler optimization.  The function accepts no
parameters, although (like every other function in a filter) it has
access to the filter's fields and parameters.  Within the work
function, input and output is done via the channels as described above.

\item {\it Other functions.}  The filter is free to define other
functions, just like objects can define methods in an object-oriented
language.  However, all of these functions are implictly private, and
can't be seen from outside the scope of the filter.

\end{itemize}

\subsubsection{Filter Instantiation}

A filter can be instantiated into the stream graph by calling its name
with some arguments, e.g. {\tt MyFilter(arg$_1$, arg$_2$, ...,
arg$_n$)}.  This is equivalent to inlining the entire filter
definition and supplying the arguments as initial values for the
parameters.  That is, we can convert above call into:

\begin{tabbing}
\hspace{0.2in} \= \hspace{0.2in} \= \hspace{0.2in} \= \hspace{0.2in} \= \\

{\tt Filter MyFilter(type$_1$ param$_1$ = arg$_1$, type$_2$ param$_2$
= arg$_2$, ..., type$_n$ param$_n$ = arg$_n$)} \{ \\ \\

\> {\it filter body} \\ \\
\}
\end{tabbing}

In this inlined form, the name {\tt MyFilter} becomes optional; we
could remove it if desired.  The inlined syntax is thus an equivalent
way of declaring filters.  We could translate any inlined filter into
a named filter that is explicitly instantiated.

\subsection{FilterBlocks}

The second kind of functional block, the {\tt FilterBlock}, is similar
to a filter in that it transforms one data stream into another.
However, a filter block manifests this transformation as an arrangment
of other filters (and/or filter blocks) rather than as procedural code
that deals explicitly with the input and output channels.
FilterBlocks are instantiated just like filters.  However, their
declaration is somewhat different:

\begin{tabbing}
\hspace{0.2in} \= \hspace{0.2in} \= \hspace{0.2in} \= \hspace{0.2in} \= \\

{\tt FilterBlock} filter-block-name (type$_1$ param$_1$, type$_2$ param$_2$,
... , type$_n$ param$_n$) \{ \\ \\

\> {\it data fields} \\ \\

\> {\it stream initialization code} \\ \\

\}
\end{tabbing}

The differences between the {\tt FilterBlock} and {\tt Filter}
declarations are as follows.  Firstly, the filter block lacks the
input and output channels, as well as the work function and the
capacity to define local methods.  However, the crucial difference is
in the initialization code.  Whereas a filter's initialization code is
for modifying fields and doing I/O, a filter block instantiates a
series of filters (an/or filter blocks).  By default, the filters are
connected in the order in which they are instantiated, although more
complex structures can be obtained using control constructs (see
Section {\ref{sec:control}}.  Thus, the filter block contains whatever
graph of filters it initializes.  

There are also {\tt FilterBlock} analogs for the {\tt Source} and {\tt
Sink}.  A {\tt SourceBlock} is a filter block who's first filter is a
{\tt Source}, and a {\tt SinkBlock} is a filter block who's last
filter is a {\tt Sink}.  We modify the ``Hello World!''  program from
Section {\ref{sec:intro}} to give a trivial example of a filter block:

\begin{verbatim}
Stream {

  MakeText();

  Sink {
    input String myInput;
    
    work() {
      System.out.println(input[0]);
      input.advance(1);
    }
  }
}

SourceBlock MakeText() {

  Source {
    output String myOutput;

    work() {
      output[0] = ``Hello '';
      output.advance(1);
    }
  }

  Filter {
    input String myInput;
    input String myOutput;

    work() {
      output[0] = input[0] + ``World!'';
      input.advance(1);
      output.advance(1);
  }
}

\end{verbatim}

\subsection{Control Constructs}
\protect\label{sec:control}

So far we have focussed on building the objects that do the
steady-state processing of stream data.  We now turn our attention to
the constructs that are used to connect the functional blocks into a
streaming structure.  Here it is important to emphasize that the graph
structure is created at {\it initialization time}, whereas the {\tt
work} functions of the filters are executed during the {\it steady
state} operation of the stream.  The initialization code is run only
once for any given period of steady state operation.  Thus, we have
two distinct times of execution in our language (we will introduce yet
another time in Section {\ref{sec:script}}.)

There are three basic control constructs in StreaMIT: 1) Pipe, for
connecting filters into a linear chain, 2) SplitJoin, for splitting a
stream into parallel components, and 3) FeedbackLoop, for
incorporating loops into the graph.  Note that while these constructs
are very expressive, they can not express all arbitrary connections of
filters.  However, the philosophy is that structured streams compare
to arbitrarily connected streams as structured control flow compares
to goto statements: the loss of expressive power will be more than
offset by improved readability and opportunity for optimization.

\subsection{Pipe}

The {\tt Pipe} construct is used for 

\subsection{Dynamic Constructs}
\protect\label{sec:dynamic}

\section{Scripting Language}
\label{sec:script}

\begin{verbatim}

	B. control constructs
		- toplevel is Stream construct
			- first is source, last is sink
		- already seen flow of control is implicitly linear inside
			filterblocks
	
		I. SplitJoin
		- for splitting / merging streams
		- split: RoundRobin, WeightedRoundRobin (+ syntax), Dupliate
		- join: RoundRobin, WeightedRoundRobin
		- for WeightedRoundRobin, must have zero for sources/sinks?

		II. Pipe
		- explicit way of grouping 1-to-1 units to form larger blocks
		- compared to FilterBlock construct, can't be bound to 
			variable, doesn't create a scoping barrier

		III. Feedback
		- specify time in loop (number of elements in delay)
		- need to flesh out rest of details

	C. referencing filters:  FilterRef's

		- whenever a filter/filterblock is instantiated (or inlined),
			the result can be bound to a FilterRef variable
		- FilterRef's can only be assigned once, and (unlike normal
			variables) they can be used before they're defined

		I. Using FilterRef's:  Initialization
			- ref.reset(<dynamic parameters>)
			- legal to initialize from within work function and
				message handlers of Filters.  Not from 
				initialization code anywhere.
			- semantics of flushing streams

		II. Using FilterRef's:  Message Passing
			- ref.handleMessage(<message variable>) @ [min, max];
			- legal to send messages from within work function, 
				not from within message handlers or
				initialization code.
			- message types are typedef'd structures, no pointers
			- filters can receive with ReceiveMessage(Type foo)
			- message handlers can send messages and
				   re-initialize, but not read i/o
			- delivery is from top to bottom

		III. naming FilterRef's
			- ref.getName() returns hierarchical name
			- can assign names after equals sign of
                          assignments to references
			- if no name given, then default name assigned
                          based on name of filter/filterblock and
                          position in parent

	D. advanced control flow:  StreamSelect
		- now that we have construct for re-initialization, might
			want to dynamically change part of stream structure
		- switches on an enumerated type with cases
		- the enumerated variable is dynamic

2) scripting language

	- to ease construction of the filter graph
	- the script that is implemented on top of the basic language can
		always be ``compiled out'' to leave a (possibly larger) 
		graph in the basic language.

	A. script variable type
		- all things of type script can be determined at
                  compile time (i.e.other script variables, constant
                  literals, pure funcs of these)
		- graph structure can depend only on script control
                  flow, but all other constructs can reference script
                  variables, too
		- can only be assigned within scripting control flow
			- sfor, sdo, etc (open to negotiation)
			- can only appear within scripting control flow
		- a function can return a script variable if its parameters
			are script variables and it references no global
			dynamic variables.  

	A. scripting parameters to filters and filterblocks
		- parameters of type script, denoted by < >
		- don't change across all instantiations, for a given 
			compilation.

---------------------------------------------------------------------------

not explicitly going into user's guide:

- four language levels:
	1. code implementing filters
	2. static connection of the filters to form supergraph
	3. scripting to make (2) simple
	4. dynamic invocation / reinitialization

- four levels of semantic checking:
	- incorrect
	- not implemented
	- warning
	- performance problem

x outline

  pipe - inlined
  filter - inlined
  splitjoin
  feedback
  functions
  sources & sinks
  whole program (main, etc)
  script vs dynamic
  filter block
  stream select
  messaging system

x need to define subset of c that we allow

x messages
	- delivery is from top to bottom
	- filters can receive with ReceiveMessage(Type foo)
	- message types are typedef'd structures, no pointers
	- message handlers can send messages and re-initialize, but not
		read i/o

x scripting language
  x procedural; upon instantiating filter, it's put into graph.
  x two means of abstraction: filter & filterblock
  x outermost filterblock is called main, takes no paramters
	x first thing is source, last is sink

  x filterref's
    - can have use before def.
    - require exactly one assignment

  x script variable type
    - all things of type script can be bdetermined at compile time
    - i.e. other script variables, constant literals, pure funcs of these
    - can only be assinged within scripting control flow
    - script control flow is normal c control flow, but predicated by
	an ``s''.  script control flow can't appear within normal 
	control flow.
    - regular constructs can use script variables just like they were normal

x filterblocks
   - also take parameters & have variables scoped
   - no functions, no message handlers, no explicit input/output
   - instantiation equivalent to copying body
   - scope wall between it and enclosing

x pipe
	- somehow implicit in filter blocks
	- for connecting single input, single output blocks

x splitjoin (?? - what's the verdict on joins/splits?)
	- split: roundrobin, weightedroundrobin (+ syntax), duplicate
	- join: roundrobin, weightedroundrobin

	- for w.r.r., must have zero for sources/sinks

x streamselect
	- just goes on an enumerated type with cases
	- the enumerated variable is dynamic

x sources/sinks
	- defined just like filters, but with either no input or no output
	- have special keywords for types, source & sink

x feedback loop
	- specify time in loop (number of elements in delay)
	- need to flesh out rest of details

x naming
	- ref.getName() returns name
	- can assign names after equals sign of assignments to references

x filters 

    x defined as ``Filter FooType<script params>(dynamic params)'' {
	// data members
	// initialization statements
	Work()
	// other functions, message handlers
	}

    x distinction between script and dynamic parameters - script don't
	change across multiple instantiations.

    x single input and output, declared with ``input'', ``output'' with
	name and type.

    x work function
	- no parameters
	- can send messages and re-initialize
	- input/output accessed as an array.  each is advanced with advance
		statement
	- things going through i/o are passed by value.  no pointers to 
		global data structures.

    x initialization
	- no messages, initialization of other blocks
	- but can read i/o to get stuff going

    x re-initialization
	- do ref.reset(dynamic params...).  static params do not need to
		be passed again because they don't change from one 
		instantiation to the next

    x inlined by copying body and having default values for all parameters

\end{verbatim}


\end{document}







