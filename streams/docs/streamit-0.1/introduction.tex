\section{Introduction}
\label{sec:intro}

The StreaMIT programming language is designed to enhance the
programmability and performance of streaming applications.  In
providing high-level constructs for building streams out of a number
of simple blocks, the language provides a clean, expressive, and
modular framework that programmers can use to reason about streaming
computations.  At the same time, the language imposes a structure on
the streams which can be exploited by a compiler to perform
stream-specific optimizations.  Thus, the abstractions provided by the
language serve not only to improve the programming model, but to
facilitate performance improvements that motivate the language in
their own right.

The following example defines a stream that prints a continuous
sequence of ``Hello World!'' messages to the screen.  It illustrates
the main components of StreaMIT:

\begin{verbatim}
Stream {

  Filter {
    output String myOutput;

    work() {
      output.push (``Hello '');
    }
  }

  Filter {
    input String myInput;
    output String myOutput;

    work() {
      output.push (input.pop () + ``World!'');
  }

  Filter {
    input String myInput;
    
    work() {
      System.out.println(input.pop ());
    }
  }
}
\end{verbatim}

In StreaMIT, the basic processing unit is the {\tt Filter}, which
may read items from an input channel and write items to an output
channel.  A filter that requires no input or output is called a
source and a sink, respectively.  In the example above, a 
source generates the ``Hello'' string, a {\tt Filter} appends
``World!'', and a sink prints the result to the screen.

Each filter defines a {\tt work} function that represents the most
fine-grained step of execution in transforming data from the input
channel to the output channel.  This function is called automatically
by the runtime system when the stream is running.  The input and
output channels are accessed as stacks.  Input is popped (or peeked, if
the data in the input is supposed to remain there),
while output is pushed to.  In the ``Hello World'' example, each 
filter reads or writes a single item before advancing the tape for 
the next step.

Filters implicitly are connected together in the order of their
definition.  Thus, in the ``Hello World!'' example, the three filters
are pipelined together and enclosed within a {\tt Stream} construct,
which denotes the top-level stream in the program.

The following sections describe the StreaMIT language in more detail.
First we present the {\it base language}, which encompasses all of the
expressive power of StreaMIT.  Then we discuss the {\it scripting
language}, a layer on top of the base language that simplifies stream
construction via the use of compile-time macros.


