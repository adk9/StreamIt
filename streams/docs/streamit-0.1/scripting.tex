\section{Scripting Language}
\label{sec:script}

The preceding sections completely describe the class of programs that
can be expressed in StreaMIT.  However, the restriction that block
instantiations can not appear within any conditional control flow
makes graph construction tedious and inflexible--for example, by
requiring $n$ repeated filter instantiations instead of having a
single instantiation inside a {\tt for} loop.

To allow such parameterized block instantiations, we introduce special
graph variables, graph control constructs, and graph parameters
whose values can be analyzed at compile time.

\subsection{Graph Type}

The {\tt graph} keyword is a primitive type modifier, with syntax
analogous to {\tt const} in Java or C:

\medskip
{\tt graph }{\it type variable-name};
\medskip

The restriction on graph variables is that their value must be
analyzable at compile-time for all points in the program.  An
expression is of type {\tt graph} if it is: 1) a constant literal, 2)
a graph variable that has been initialized, or 3) a pure function of
these.  A ``pure function'' is either a determistic primitive operator
(e.g., the arithmetic operators) or a user-defined function without
global side effects that returns a {\tt graph} expression.

Control flow that depends only on graph expressions is called {\it
graph control flow}.  To make it clear to the programmer which
control constructs have this property, we designate different keywords
for all graph control constructs.  They have the names of the normal
constructs, but with an ``g'' prepended, e.g., {\tt gfor}, {\tt gif},
and {\tt gwhile}.  Only expressions of type {\tt graph} can appear in
the conditional clause of these constructs.

Thus, with these definitions, we see that for the value of a graph
variable to be statically determined at all points in the program, it
must only be assigned expresseions of type {\tt graph}, and these
assignments must appear only within graph control flow.  By the same
reasoning, the stream structure is statically analyzable if all block
instantiations are within graph control flow.  This is our new
restriction on block instantiations: they can be enclosed in control
flow, but it must be {\it graph} control flow.

\subsection{Graph Parameters}

Parameters of type {\tt graph} can also be passed to filters and
filter blocks.  Since the values of these parameters can be determined
at compile time, they will not change across multiple instantiations
of a block.  Thus, we allow the user to specify graph parameters
enclosed between angled brackets before the regular dynamic
parameters:

{\tt Filter} filter-name $<${\tt graph} type$_{s1}$, {\tt graph} type$_{s2}$, \dots {\tt graph} type$_{sn}>$ (type$_1$ dynamic-param$_1$, type$_2$ dynamic-param$_2$, \dots , type$_n$ dynamic-param$_n$) \{ \dots \} \\

{\tt FilterBlock} filter-block-name $<${\tt graph} type$_{s1}$, {\tt graph} type$_{s2}$, \dots {\tt graph} type$_{sn}>$ (type$_1$ dynamic-param$_1$, type$_2$ dynamic-param$_2$, \dots , type$_n$ dynamic-param$_n$) \{ \dots \} \\

The graph parameters can be used just like the dynamic ones in the
context of the filters, but do not need to be given as arguments to
re-initialization calls.



