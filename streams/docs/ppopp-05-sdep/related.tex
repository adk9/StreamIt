\section{Related Work}

The work most closely related to ours comes from the fields of program
slicing, heterogeneous modeling, and domain-specific languages.

Program slicing identifies the set of statements in a program that a
given statement might depend on.  There is a rich history of work in
program slicing; see Tip~\cite{tip95slice} for a comprehensive review.
Many program slicing techniques rely on the Program Dependence Graph
as described by Horwitz et al.~\cite{hrb88pdg}.  Program slicing has
been applied for debugging, testing, and program analysis.  In many
respects, $\sdep$ analysis can be thought of as a slicing technique
for Synchronous Dataflow graphs.  Because the input domain is
restricted (in particular, because of the absence of control flow and
recursion), the $\sdep$ calculation can make stronger guarantees than
slicing analyses for general procedural languages; $\sdep$ is
decidable, exact, and admits a compact representation in terms of the
steady state schedule.

Pugh and Rosser present an iteration-based slicing~\cite{pugh97slice}
which identifies the dynamic instances of statements (in terms of
their loop iteration) which effect a given value.  This also bears
some similarity to $\sdep$, as we calculate the last iteration of an
actor which affects another.  However,~\cite{pugh97slice} focusses on the
problem of computing the transitive closure of dependences in loops.
We are not interested in this question, as we assume that all actor
invocations depend on their previous invocations; $\sdep$ addresses
the question of finding the last invocation that was relevant.
Moreover, our motivation differs from the slicing community, as we
apply $\sdep$ to enrich the semantics of language features.  To the
best of our knowledge, slicing has not been applied in this way
before.

The models of computation in our system are closely related to those
explored in the Ptolemy project for heterogeneous
design~\cite{ptolemy03overview}.  As part of this effort, Lee et
al. have established the Synchronous Dataflow (SDF)
paradigm~\cite{LM87-i} and have developed hybrid models that
incorporate Dynamic Dataflow (DDF, in which the I/O rates of actors
are fully dynamic).  Boolean Dataflow (BDF)~\cite{ha97profile} is a
compromise between these two extremes; it computes a parameterized
schedule of the graph at compile time, and substitutes runtime
conditions to decide which paths are taken.  The performance is nearly
that of SDF while keeping some flexibility of DDF.  

The messaging system described in this paper shares the motivation of
BDF, but is different in its approach.  We believe that control
messages represent a distinct and well-behaved class of dynamic
communication in which a parameter is ``pushed'' into the receiver
actor in an asynchronous way.  Because the message handlers do not
access the input and output tapes of the receiving filter, their
irregular invocations do not affect the multiplicity of the
steady-state schedule.  Instead, the schedule is constrained only by
the latency of control messages; if a message doesn't show up in the
allotted window, then the receiving actor can go ahead with its
high-bandwidth schedule.  This is the distinction in the computational
model.  In addition, the static/dynamic integration offered by our
system is integrated with language features that support the model.

There are many domain-specific stream languages in addition to
StreamIt.  Streams have a long history in the programming languages
community, with influences from dataflow, CSP, synchronous and
functional languages; see Stephens~\cite{survey97} for a review.
Languages of recent interest include Occam~\cite{occammanual},
Sisal~\cite{sisal}, Parallel Haskell~\cite{ph}, StreamC /
KernelC~\cite{imagine03ieee}, and Spidle~\cite{spidle02}.  The
principle differences between StreamIt and these languages is the
single-input, single-output hierarchical structure that StreamIt
imposes on the stream graph; StreamIt's support for a ``peek''
construct that inspects data without consuming it from a channel; and
the low-bandwidth messaging system described in this paper.
