\section{Conclusions}

This paper makes two contributions.  First, it formulates $\sdep$,
which we believe is a fundamental dependence representation for the
streaming domain.  We outlined several possible applications of
$\sdep$, including latency constraints, debugging, speculation, and
program analysis, and we look forward to pursuing these directions in
the future.

Second, this paper employs $\sdep$ to define a powerful language
construct allowing for precise message delivery between any two nodes
of a stream graph.  In comparison with other methods to implement
messaging functionality in a Synchronous Dataflow model, we believe
that our language construct is more readable, more robust, and easier
to maintain.  In addition, our implementation of messaging in the
StreamIt compiler resulted in a 49\% improvement in performance for a
frequency hopping radio running on a cluster of workstations.  We
achieved this speedup because the messaging construct exposes the true
dependences to the compiler and allows it to optimize the
communication.

Our work can also be viewed as the integration of dynamic behavior
into a static dataflow language.  Our insight is that there is a class
of control messages which only adjust a parameter in the target actor;
they do not otherwise affect the input or output channels upon
delivery.  This model enables a hybrid scheduling scheme in which the
steady-state dataflow is exactly orchestrated at compile time, but
there are windows in which a message could adjust an internal field of
a filter between its execution steps.  We consider this to be a modest
but promising step towards a unified development environment that
captures all aspects of stream application development without
sacrificing either performance or programmability.
