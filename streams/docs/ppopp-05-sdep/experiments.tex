\subsection{Experimental Evaluation}
\label{sec:evaluation}

We have implemented teleport messaging in the StreamIt compiler
infrastructure~\cite{streamit-asplos}, with a backend that targets a
cluster of workstations.  A StreamIt program is compiled to a set of
parallel threads; if two threads are allocated to different machines,
they communicate via dedicated TCP/IP connections.  Teleport messaging
is supported via auxilliary communication channels that transmit two
kinds of signals from a message sender to the receiver: 1) the contents of
a control message, or 2) a {\it credit} that indicates the receiver
can execute some number of iterations before checking for a message
again.

Each actor alternates between normal execution and checking for the
exchange of credits.  This serves to throttle the message receiver as
described in Section~\ref{sec:teleport}, as an actor will block
waiting for credits until the sender has reached a given point in its
execution.  The compiler calculates the $\sdep$ relation and
automatically schedules the exchange of credits to make sure that the
timing constraints are respected.  When a message is sent, it is
tagged with the iteration number during which the receiver should
process it; this is also calculated using $\sdep$ in the compiler.

%% In our implementation, each actor maintains an iteration number which
%% serves to synchronize message delivery. A message from a sender $A$ to
%% a receiver $B$ is tagged with the iteration number in $B$ when the
%% message must be processed. The iteration number is calculated by the
%% sender using the $\sdep$ relation and the specified message latency.

%% The compiler automatically schedules the exchange of credits between
%% actors when it derives an execution of the stream graph. Our
%% communication layer guarantees that messages are received before
%% credits and hence processed during the appropriate execution step.
%% For messages sent downstream with zero or positive latencies, we do
%% not use the credit system, and assume that the messages arrive as fast
%% or faster than the data items exchanged between the two actors. This
%% assumption is safe since, due to the data dependences, a downstream
%% filter cannot get ahead of an upstream actor.

We use a cluster of workstations as our evaluation testbed. Our
StreamIt compiler automatically partitions an input program into a set
of threads that are mapped to the workstations and then run concurrently.
We chose a cluster-based evaluation for two reasons. 
First, a number of streaming applications run on the server side
(e.g., cell phone base stations, radar processing, HDTV editing) and
require large computational resources. Second, clusters provide a
simple abstraction for distributed and parallel computing---multiple program
counters, and distributed memories---which 
is at the heart of emerging multi-core and tile-based architectures
for embedded, desktop, and server computing.  In our experiments, we are primarily interested in quantifying two
performance metrics: throughput and communication overhead.
For the purpose of this paper, throughput is defined
as the number of outputs produced per unit of time, and communication
overhead reflects the quantity of data transmitted between
workstations.

The teleport based implementation of the frequency hopping
radio was compiled into 28 threads
whereas the alternate version using a feedback-loop results in 
32 threads.  Each thread implements a
specific subset of the stream graph  and is assigned to one of
sixteen 750Mhz Pentium~III workstations, each with a
256Kb cache.  The machines are interconnected using a fully switched
high speed network.

Our compiler maps threads to workstations using an algorithm that aims
to reduce the overall application bottleneck while
maximizing the throughput of the output filter (i.e., most downstream
actor) in the stream graph.

\begin{figure}[t]
\psfig{figure=throughput-graph.eps,width=3.3in}
\caption{\small Throughput as a function of the number of workstations
in the cluster. 
\protect\label{fig:fhr-throughput}}
\end{figure}

In Figure~\ref{fig:fhr-throughput}, we report the measured throughput
($y$-axis) for various cluster sizes ranging from one to sixteen
interconnected workstations. Note that due to the limited parallelism
in the two implementations of the frequency hopper, cluster
configurations with more than five workstations lead to negligible
gains in terms of throughput. From the data, we can observe that
teleport messaging achieves a maximal throughput that is 49\% better
than its feedback-loop counterpart.  

There are two factors contributing to this speedup.  First, teleport
messaging exposes more parallelism in the application, as the compiler
understands that the RFtoIF stage can execute ahead of each
Check\_Freq\_Hop stage, so long as the message latency is respected.
In contrast, the manual implementation sends an item along the
feedback path on every iteration, thereby restricting RFtoIF to only
process one frame at a time.  This explains why the speedup improves
as the number of processors increases: the extra parallelism is being
utilized.  A second factor that could contribute to the improved
performance is the reduced communication overhead.  A detailed
analysis of the results shows that teleport messaging reduces the
number of items communicated by 35\%, as a small number of credits are
sent compared to the manual implementation.
