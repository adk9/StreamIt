\section{General Translation: PCP to SARE}
\label{sec:translate}

In this section we give the formal translation of a Phased Computation
Program to a System of Affine Recurrence Equations.  Again, the SARE
will be parameterized by $N$, the number of steady-state cycles that
one wishes to execute in the PCP.  The translation will use helper
functions shown in Figure~\ref{fig:helper}.  We also make use of the
$S(n)$ function giving the number of steady-state executions of node
$n$ (see Section~\ref{sec:balance}).

\begin{figure}[t]
\framebox[6.5in]{
\begin{minipage}{6.3in}
\begin{center}
{\bf Helper Functions}
\end{center}
\begin{itemize}
\item The number of integral points in a one-dimensional domain ${\cal
D}$ is denoted by $|{\cal D}|$.  The sign and absolute value of an
integer $i$ are given by $sign(i)$ and $abs(i)$, respectively.
%
\item $\mt{PartialWrite}(c,t,p) = \sum_{r=0}^{p-1} U(c, t, r)$
%
\item $\mt{TotalWrite}(n,c,t) = \sum_{r=0}^{\mt{num}(n, t)-1} U(c, t, r)$
%
\item $\mt{PartialRead}(c,t,p) = \sum_{r=0}^{p-1} O(c, t, r)$
%
\item $\mt{TotalRead}(n,c,t) =  \sum_{r=0}^{\mt{num}(n, t)-1} O(c, t, r)$
%
\item Given channel $c = (n_a, n_b)$:
\begin{align*}
\mt{Period}(c) \equiv S(n_a) * \mt{TotalWrite}(n_a, c, \mt{steady}) = 
S(n_b) * \mt{TotalRead}(n_b, c, \mt{steady})
\end{align*}
%
\end{itemize}
%
\end{minipage}}
\caption{Helper functions for the PCP to SARE translation.
\protect\label{fig:helper}}
\end{figure}

\begin{figure}[t]
\framebox[6.5in]{
\begin{minipage}{6.3in}
\begin{center}
{\bf Variables}
\end{center}

For each channel $c = (n_a, n_b)$, do the following:
\begin{itemize}

\item Introduce variables $\mt{IBUF}_c$ and $\mt{SBUF}_c$ with the following domains:
\begin{align}
{\cal D}_{{IBUF}_c} = \{~i~|~0 \le i \le A(c) +
\sum_{p=0}^{\mt{num}(n_a, \mt{init})-1} U(c, \mt{init}, p) - 1 \} \\ \nonumber ~ \\
%
{\cal D}_{{SBUF}_c} = \{ ~(i,j)~|~0 \le i \le N - 1 ~\wedge~ 0 \le j \le \mt{Period}(c) - 1\}
\end{align}

\item For each $p \in [0, num(n_a, \mt{init})-1]$, introduce
variable $\mt{IWRITE}_{c, p}$ with this domain:
\begin{align}
{\cal D}_{{IWRITE}_{c, p}} = \{~i~|~0 \le i \le U(c, \mt{init}, p) - 1\}
\end{align}

\item For each $p \in [0, num(n_b, \mt{init})-1]$, introduce
variable $\mt{IREAD}_{c, p}$ with this domain:
\begin{align}
{\cal D}_{{IREAD}_{c, p}} = \{~i~|~0 \le i \le E(c, \mt{init}, p) - 1\}
\end{align}
%
\item For each $p \in [0, num(n_a, \mt{steady})-1]$, introduce
variable $\mt{SWRITE}_{c, p}$ with this domain:
\begin{align}
{\cal D}_{{SWRITE}_{c, p}} = \{~(i,j,k)~|~0 \le i \le N-1 ~\wedge~ 0 \le j \le S(n_a) - 1 ~\wedge~ 0 \le k \le U(c, \mt{steady}, p) - 1\}
\end{align}
%
\item For each $p \in [0, num(n_b, \mt{steady})-1]$, introduce
variable $\mt{SREAD}_{c, p}$ with this domain:
\begin{align}
{\cal D}_{{SREAD}_{c, p}} = \{~(i,j,k)~|~0 \le i \le N-1 ~\wedge~ 0 \le j \le S(n_b) - 1 ~\wedge~ 0 \le k \le E(c, \mt{steady}, p) - 1\}
\end{align}
%
\end{itemize}
%
\end{minipage}}
\caption{Variables for the PCP to SARE translation.
\protect\label{fig:pcptosare1}}
\end{figure}

\begin{figure}[p]
\framebox[6.5in]{
\begin{minipage}{6.3in}
\begin{center}
{\bf Equations (Part 1)}
\end{center}
\noindent For each channel $c = (n_a, n_b)$, introduce the following equations:
\begin{itemize}
%
\item {\bf(I to IBUF)}:
\begin{align}
\forall i \in [0,A(c)-1],~\mt{IBUF}_{c}(i) = I(c)[i]
\label{i2ibuf}
\end{align}

\item {\bf(IWRITE to IBUF)} For each $p \in [0, num(n_a, \mt{init})-1]$:
\begin{align}
\label{iwrite2ibuf}
&\forall i \in [A(c) + \mt{PartialWrite}(c,\mt{init},p),
               A(c) + \mt{PartialWrite}(c,\mt{init},p+1)-1], \\ \nonumber
&~~~~~~\mt{IBUF}_{c}(i) = \mt{IWRITE}_{c, p}(i - A(c) - \mt{PartialWrite}(c,\mt{init},p))\nonumber
\end{align}
%
\item {\bf(IREAD to IWRITE)} For each $c \in \mt{chan\_out}(n)$, and for each $p \in [0,
num(n, \mt{init})-1]$:
%
\begin{align}
\label{iread2iwrite}
&\forall i \in {\cal D}_{{IWRITE}_{c, p}},~\mt{IWRITE}_{c,p}(i) = W(n, \mt{init}, p)(\mt{Init\_Inputs})[\mt{pos\_out}(n, c)][i], \\ \nonumber
&\mt{where Init\_Inputs} = [\mt{IREAD}_{{chan\_in}(n)[1], p}(*), \dots, \mt{IREAD}_{{chan\_in}(n)[\mt{num\_in}(n)], p}(*) ]
\end{align}
%
\item {\bf(SREAD to SWRITE)} For each $c \in \mt{chan\_out}(n)$, and for each $p \in [0,
num(n, \mt{steady})-1]$:
%
\begin{align}
\label{sread2swrite}
&\forall (i, j, k) \in {\cal D}_{{SWRITE}_{c, p}},~\mt{SWRITE}_{c,p}(i,j,k) =\nonumber\\
&~~~~~~W(n, \mt{steady}, p)(\mt{Steady\_Inputs})[\mt{pos\_out}(n, c)][k], \mt{where}\\\nonumber&
\mt{Steady\_Inputs} = [\mt{SREAD}_{{chan\_in}(n)[1], p}(i, j, *), \dots, \mt{SREAD}_{{chan\_in}(n)[\mt{num\_in}(n)], p}(i, j, *) ]
\end{align}
%
\item {\bf(SWRITE to SBUF)} For each $c \in \mt{chan\_out}(n)$, for each $p \in [0,
\mt{num}(n, \mt{steady})-1]$, and for each $q \in [0,S(n)-1]$:
\begin{align}
\label{swrite2sbuf}
&\forall (i,j) \in {\cal D}_{SW {\small \rightarrow} SB}(c,p,q),
\mt{SBUF}_{c}(i,j) = \nonumber\\
&~~~~~~\mt{SWRITE}_{c, p}(i, q,
                     j - \mt{Offset}_{SW {\small \rightarrow} SB}(q,n,c,p)),
\mt{where } \\&{\cal D}_{SW {\small \rightarrow} SB}(c,p,q) = 
  {\cal D}_{{SBUF}_{c}} \cap 
  \{ (i,j) | \mt{Offset}_{SW {\small \rightarrow} SB}(q,n,c,p) \le j
             \le \mt{Offset}_{SW {\small \rightarrow} SB}(q,n,c,p+1) - 1 \} \nonumber\\\nonumber
&\mt{and } \mt{Offset}_{SW {\small \rightarrow} SB}(q,n,c,p') = q*\mt{TotalWrite}(n,c,\mt{steady}) + \mt{PartialWrite}(c,\mt{steady},p'))
\end{align}
%
\item {\bf(IBUF to IREAD)} For each $c \in \mt{chan\_in}(n)$, and for each phase $p \in
[0, \mt{num}(n, \mt{init})-1]$:
\begin{align}
\label{ibuf2iread}
&\forall i \in {\cal D}_{IB {\small \rightarrow} IR}(c,p), 
\mt{IREAD}_{c, p}(i) = \mt{IBUF}_{c}(\mt{PartialRead}(c, \mt{init}, p) + i) \\ \nonumber
&\mt{where }{\cal D}_{IB {\small \rightarrow} IR}(c,p) = 
  \{ i | i \in {\cal D}_{{IREAD}_{c,p}} ~\wedge~ 
         i + \mt{PartialRead}(c, \mt{init}, p) \in {\cal D}_{{IBUF}_{c}} \}
\end{align}
%
\item {\bf(SBUF to IREAD)} For each $c \in \mt{chan\_in}(n)$, for each phase $p \in [0,
\mt{num}(n, \mt{init})-1]$, and for each \\$q \in [0, \lfloor
\frac{\mt{argmax}_{\mt{p}'} \left( \mt{PartialRead}(\mt{c}, \mt{init},
\mt{p}') + \mt{E}(\mt{c}, \mt{init}, \mt{p}') \right)
}{\mt{Period(c)}} \rfloor ]$:
\begin{align}
\label{sbuf2iread}
&\forall i \in {\cal D}_{SB {\small \rightarrow} IR}(c,p), 
\mt{IREAD}_{c,p}(i) = 
    \mt{SBUF}_{c}(q,
      i - \mt{Offset}_{SB {\small \rightarrow} IR}(q,n,c,p)), \mt{where } \\ \nonumber
&{\cal D}_{SB {\small \rightarrow} IR}(c,p) = 
  \{ i | i \in {\cal D}_{{IREAD}_{c,p}} ~\wedge~
         \mt{Offset}_{SB {\small \rightarrow} IR}(q,n,c,p)
         \le i \le
         \mt{Offset}_{SB {\small \rightarrow} IR}(q+1,n,c,p) - 1 \} \\ \nonumber
&\mt{and }\mt{Offset}_{SB {\small \rightarrow} IR}(q',n,c,p) = 
  q' * \mt{Period}(c) + |{\cal D}_{{IBUF}_c}| - \mt{PartialRead}(c,\mt{init},p)
\end{align}
%
\end{itemize}
\end{minipage}}
\caption{Equations for the PCP to SARE translation.
\protect\label{fig:pcptosare2}}
\end{figure}

\begin{figure}[p]
\framebox[6.5in]{
\begin{minipage}{6.3in}
\begin{center}
{\bf Equations (Part 2)}
\end{center}
\noindent For each node $n$, introduce the following equations:
\begin{itemize}
\item {\bf(IBUF to SREAD)} For each $c \in \mt{chan\_in}(n)$, and for each phase $p \in [0,
\mt{num}(n, \mt{steady})-1]$:
\begin{align}
\label{ibuf2sread}
&\forall (i,j,k) \in {\cal D}_{IB {\small \rightarrow} SR}(n,c,p), 
\mt{SREAD}_{c, p}(i,j,k) = 
    \mt{IBUF}_{c}(\mt{ReadIndex}(n,c,p,i,j,k)), \mt{where} \nonumber\\
&\mt{ReadIndex}(n,c,p,i,j,k) = \mt{TotalRead}(n,c,\mt{init})
                + (i*S(n)+j)*\mt{TotalRead}(n,c,\mt{steady}) \\ \nonumber
                &~~~~~~+ \mt{PartialRead}(c, \mt{steady}, p) 
                + k) \\\nonumber
&\mt{and }{\cal D}_{IB {\small \rightarrow} SR}(n,c,p) = 
  \{ (i,j,k) | \mt{ReadIndex}(n,c,p,i,j,k) \in {\cal D}_{{IBUF}_{c}} \}
\end{align}
%
\item {\bf(SBUF to SREAD)} For each $c \in \mt{chan\_in}(n)$, for each phase $p \in [0,
\mt{num}(n, \mt{steady})-1]$, and for each $q \in [0, \lfloor
\frac{\mt{E(c, steady, p)}}{\mt{Period(c)}} \rfloor]$:
% two possible problems: sread has already been written to, or sbuf has already been read from
\begin{align}
\label{sbuf2sread0}
&\forall (i,j,k) \in {{\cal D}\mt{1}}_{SB {\small \rightarrow} SR}(c,p,q),
\mt{SREAD}_{c, p}(i,j,k) = 
    \mt{SBUF}_{c}(i+q+
                  \mt{Int\_Offset}(n,c,p),\nonumber\\ \nonumber
                  &~~~~~~j*\mt{TotalRead}(n,c,\mt{steady}) + 
                    \mt{PartialRead}(c,\mt{steady},p) + k
                   - q * \mt{Period}(c) + \mt{Mod\_Offset}(n,c,p)), \\
&\mt{where }{{\cal D}\mt{1}}_{SB {\small \rightarrow} SR}(c,p,q) = 
  ({\cal D}_{{SREAD}_{c, p}} - \mt{PUSHED}(n,c,p))
                         \cap \\ \nonumber
	&~~~~~~\{ (i,j,k) | q * \mt{Period}(c) - min(0, \mt{Mod\_Offset}(n,c,p))
                                \le k 
                                \le \\ \nonumber
	&~~~~~~~~~~~~min((q+1)*\mt{Period}(c),
         E(c, \mt{steady}, p)) - 1 
         - max(0, \mt{Mod\_Offset}(n,c,p)) \}\\ \nonumber\\
\label{sbuf2sread1}
%\end{align}
%\begin{align}
%
%
&\forall (i,j,k) \in {{\cal D}\mt{2}}_{SB {\small \rightarrow} SR}(c,p,q),
\mt{SREAD}_{c, p}(i,j,k) = \nonumber\\
    &~~~~~~\mt{SBUF}_{c}(i+q+
                  \mt{sign(Offset)} + \mt{Int\_Offset}(n,c,p),
                  j*\mt{TotalRead}(n,c,\mt{steady}) + \\\nonumber
                    &~~~~~~~~~~~~\mt{PartialRead}(c,\mt{steady},p)
		+ k
                   - q * \mt{Period}(c) 
                   - \mt{sign(Offset)} * \mt{Period}(c)
 + \mt{Mod\_Offset}(n,c,p)),\\ \nonumber&
\mt{where }{{\cal D}\mt{2}}_{SB {\small \rightarrow} SR}(c,p,q) = 
  ({\cal D}_{{SREAD}_{c, p}} - \mt{PUSHED}(n,c,p)) \cap \\ \nonumber
                         &~~~~~~( \{ (i,j,k) | q * \mt{Period}(c)
                                \le k 
                                \le min((q+1)*\mt{Period}(c),
                                              E(c, \mt{steady}, p)) - 1 \} - {{\cal D}\mt{1}}_{SB {\small \rightarrow} SR}(c,p,q) )\\ \nonumber\\
\label{sbuf2sread2}
%\end{align}
%\begin{align}
%
%
&\mt{where }\mt{PUSHED}(n,c,p) = {\cal D}_{IB {\small \rightarrow} SR}(n,c,p) \nonumber\\ \nonumber
&\mt{and }\mt{Num\_Pushed}(n,c,p) = 
% this is a dot product of (i,j,k) with something
  (max_{\preceq}\mt{PUSHED}(n,c,p)) \cdot (S(n) * \mt{TotalRead}(n,c,\mt{steady}), \\ \nonumber
                                        &~~~~~~\mt{TotalRead}(n,c,\mt{steady}),
                                        1 ) + \mt{PartialRead}(c, \mt{steady}, p),
\mt{and }\mt{PEEKED}(c,p) = {SB {\small \rightarrow} IR}(c,p) \\ \nonumber
&\mt{and }\mt{Num\_Popped}(c,p) = |\mt{PEEKED}(c,p)| + O(c,\mt{steady},p) - E(c,\mt{steady},p) \\
&\mt{and }\mt{Offset}(n,c,p) = \mt{Num\_Popped}(c,p)-\mt{Num\_Pushed}(n,c,p) \\ \nonumber
&\mt{and }\mt{Int\_Offset}(n,c,p) = \mt{sign(Offset)}*\lfloor \frac{abs(\mt{Offset}(n,c,p))}{\mt{Period}(c)} \rfloor) \\ \nonumber
&\mt{and }\mt{Mod\_Offset}(n,c,p) = \mt{sign(Offset)}*(abs(\mt{Offset}(n,c,p))~\mt{mod}~\mt{Period}(c))
%
\end{align}
%
\end{itemize}
\end{minipage}}
\caption{Equations for the PCP to SARE translation.
\protect\label{fig:pcptosare3}}
\end{figure}

Figure~\ref{fig:pcptosare1} illustrates the variables used for the
translation.  These are very similar to those in the simple case,
except that now there are separate variables for initialization and
steady state, as well as variables for each phase.  The equations for
the translation appear in Figures~\ref{fig:pcptosare2}
and~\ref{fig:pcptosare3}.  A diagram of which variables depend on
others appears in Figure~\ref{fig:communic}.  Due to space
limitations, we discuss only the most interesting parts of the
translation below.

\begin{figure}[t]
\centering
\psfig{figure=communic.eps,width=3.2in}
\parbox{4in}{\caption{Flow of data between SARE variables.  Data flows from the base of each arrow to the tip.  Dotted lines indicate communication that might not occur for a given channel.}
\protect\label{fig:communic}}
\end{figure}

Equation~\ref{i2ibuf} writes the initial values of the initial tokens
on $\mt{IBUF}$, the initial buffer.  Equation~\ref{iwrite2ibuf}
transfers values from the initial epoch of a node to the initial
buffer of a channel.  Equations~\ref{iread2iwrite} and
\ref{sread2swrite} do the work computation of the initial and
steady-state items, respectively, just as in the simple translation of
Section~\ref{sec:simplesare}.  Equation~\ref{swrite2sbuf} copies items
from the node to the channel, using the same slicing technique as
Equation~\ref{eq:w2b}.

Equation~\ref{ibuf2iread} copies values from the initial buffer
($\mt{IBUF}$) of a channel into the initial read array for a node.
However, since the node might read more items in the initial epoch
than exist on the initial buffer, we restrict the domain of the
equation to only those indices that have a corresponding value in
$\mt{IBUF}$.  If it is the case that there are fewer items in the
initial buffer than a node reads in its initial epoch, then
Equation~\ref{sbuf2iread} copies values from the beginning of the
channel's steady-state buffer into the remaining locations on the
node's initial array.  The upper bound for $q$ in
Equation~\ref{sbuf2iread} is calculating the maximum amount that any
phase peeks in the initialization epoch, so that enough equations are
defined to fill the node's initial read array.  Note that for some
phases, the dependence domain of Equation~\ref{sbuf2iread} will be
empty, in which case no equation is introduced.  Also,
Equation~\ref{sbuf2iread} uses the slicing method (see
Section~\ref{sec:simplesare}) because the initial read array is of
lower dimensionality than the steady-state buffer from which it is
reading.

Equation~\ref{ibuf2sread} is dealing with the opposite case as
Equation~\ref{sbuf2iread}: when a node finds items on its channel's
initial buffer when it is entering the steady-state epoch.  In this
case copying over the appropriate items to the beginning of the
steady-state read array is rather straightforward.

Much of the complexity of the entire translation is pushed into
Equations~\ref{sbuf2sread0} and~\ref{sbuf2sread1}.  There are two
distinct complications in these equations.  First is one that we
tackled in Equation~\ref{eq:b2r}: the fact that the read array might
peek beyond the limits of the buffer, which requires us to slice the
domain in increments of one period.  The second complication is that
there could be an offset--an initial buffer might have written into
the beginning of the read array, or an initial read array might have
read from the beginning of the steady-state
buffer\footnote{Fundamentally, this complication ended up in this
equation because we chose to define the initial buffer ($\mt{IBUF}$)
to be the same length as its associated initial write array
($\mt{IWRITE}$).  If we had defined it in terms of the initial read
array ($\mt{IREAD}$) instead, the complexity would be in
Equation~\ref{swrite2sbuf}.}.

Equation~\ref{sbuf2sread2} provides a set of helper functions to
determine the offset for Equations~\ref{sbuf2sread0}
and~\ref{sbuf2sread1}.  The result is a number $\mt{Offset}$ that the
domain needs to shift.  Unfortunately this involves splitting the
original equation into two pieces, since after the shift the original
domain will be split between two different periods (with different $i$
indices in $\mt{SBUF}$).  Each of these equations is clever about the
sign of the offset to adjust the indices in the right direction.
