\section{Applications}

\subsection{Inter-Node Optimization}

\subsection{Scheduling}

Perhaps most attractive feature, as there is a large body of work on
scheduling and automatic parallelization for affine dependences
(lim/lam, feautrier, darte/vivien etc.)  It can identify what filters
can run in parallel, and optimize for different metrics
(synchronization, communication, etc.)  Hasn't been addressed as
thoroughly in the SDF community.

Also scheduling for other optimization metrics is interesting.

\subsection{Inter-Node Optimization}

Attractive because there's a unified mathematical model for expressing
constraints.

\subsubsection{Buffer Minimization}

Data buffer size is important to the DSP community.  There's a large
body of work on minimizing buffer requirements in their community as
well as in the scientific community.  This lets the results of the
whole scientific community apply to the dataflow community.

Not just scheduling - also collapsing, reusing, detecting live ranges,
sharing buffers, etc.

\subsubsection{Code Size}

Cody size is also very important.  While this has perhaps received
less attention in the context of SARE's, there are interesting
advantages that affine scheduling could offer to the DSP community.
For instance, most DSP schedules abide by a ``single appearance
schedule'' where the execution of each filter is in a perfect loop
nest.  However, often times an affine schedule will lead to a case
where a filter is nested in a given loop, but only executes on every
other invocation; it's execution is governed by a conditional.  To our
knowledge, this kind of schedule hasn't been explored by the DSP
community, and could be promising.

Also not just scheduling - how do you arrange things.

\subsubsection{Partitioning}

Provides a good framework for distributing a streaming computation
across an architecture.

\subsection{Intra-Node Optimization}

Unlike other program domains, where the principle aim of compiler
optimization is to shorten the total execution time, there are many
distinct optimization metrics for streaming applications, including
throughput, latency, data size, and code size.  The latter two of these
are especially important in embedded domains, where memory is in short
supply; latency can be critical for real-time applications, and
throughput is always of interest.

\subsection{Fission Transformations}

Just specify the relations within a node, and you have a complete view
of program.

When the machine target is more fine-grained than the stream graph, it
is advantageous to break filters up into smaller pieces so that more
hardware resources can be utilized.  We propose two fission
transformations:

Split in Half
Data Parallel Duplication

\subsection{Fusion Transformations}

e.g. Just doing constant prop on a node.

\subsection{Steady-State Invariant Code Motion}

With the dependence analysis, the compiler can detect items that are
constant during the work epoch, and can move their computation to the
init epoch, letting the data flow through in a feedback loop afterwards.

In the streaming domain, the analog of loop-invariant code motion is
the motion of code from a steady-state work function to an
initialization function if it does not depend on any quantity that is
changing during the steady-state execution of a node.  Quantities that
the compiler detects to be constant during the execution of a work can
be computed from within init, with the results being passed through a
feedback loop for availability on future executions of work.

\subsection{Induction Variable Detection}

The {\tt work} function can also be analyzed as would the body of a loop
to see if there are induction variables from one steady-state execution
to the next.  This analysis is useful both for strength reduction, which
{\it adds} a dependence between invocations by converting an expensive
operation to a cheaper, incremental one, as well as for data
parallelization, which {\it removes} a dependence from one invocation to
the next by changing incremental operations on filter state to
equivalent operations on privatized variables.

\subsection{Decimation Propagation}

Decimation refers to the regular discarding of a fraction of a filter's
input items, perhaps to reduce the sampling rate in a stream.  In the
streaming domain, the analog of dead code elimination is the propagation
of this decimation up through the stream graph, thereby eliminating the
computations that produce the unused items.

\subsection{Parameterized Graphs}

Don't need to duplicate code for nodes that are repeated.  Can also
parameterize the schedule along these dimensions.  Especially
well-tailored to StreamIt.  Compare to PSDF.

\subsection{Semantics}

%% \subsection{Verification}

%% Buffer overrun, no exploding schedules, etc.

%% \subsection{Semantics of Time}

%% \noindent {\bf \dep}

%% \begin{align*}
%% \DEP{X}{Y}(i) = max_{\preceq}~\{~ j ~|~ \exists ~&~ T_0 \dots T_n, ~ k_0 \dots k_n, ~ {\cal E}_1 \dots {\cal E}_n ~ s.t. ~ & ~ \\ 
%% ~ & T_0 = Y ~\wedge~ T_n = X ~\wedge~ k_0 = j ~\wedge~ k_n = i ~\wedge~ & ~ \\
%% ~ & \forall p \in 1 \dots n, ~ T_p(k_p) \equiv f_{{\cal E}_p}(\dots, T_{p-1}(h_{{{\cal E}_p}, T_{p-1}}(k_p)), \dots) ~ \} & ~ \\
%% \end{align*}

%% In StreamIt, this function is composable because of single-input,
%% single-output blocks.  Since every junction is a one-dimensional
%% array, it gives unique point in time for a given filter.  Can use this
%% to think modularly/composably about time in components, too.

%% Describe messaging semantics?  Is there space?
