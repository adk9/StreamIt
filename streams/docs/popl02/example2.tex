\section{Appendix: Detailed Example}

In this section we present an example translation from a PCP to a
SARE.  Our source language is StreamIt, which is a high-level language
designed to offer a simple and portable means of constructing phased
computation programs.  We are developing a StreamIt compiler with
aggressive optimizations and backends for communication-exposed
architectures.  For more information on StreamIt, please
consult~\cite{streamitcc,Gordo02}.

\begin{figure}[t]
\scriptsize
\begin{minipage}{3in}
\begin{verbatim}
void->void pipeline EqualizingRadio {
   add RadioSource();
   add Equalizer(2);
   add Speaker();
}

float->float pipeline Equalizer (int BANDS) {
   add splitjoin {
      split duplicate;
      float centerFreq = 100000;
      for (int i=0; i<BANDS; i++, centerFreq*=2) {
         add BandPassFilter(centerFreq, 50);
      }
      join roundrobin;
   }
   add Adder(BANDS);
}

float->float filter Adder (int N) {
   work push 1 pop N {
      float sum = 0;
      for (int i=0; i<N; i++) {
         sum += pop();
      }
      push(sum);
   }
}

float->float filter BandPassFilter (float centerFreq, int N) {
   float[N] weights;

   init {
      weights = calcImpulseResponse(centerFreq, N);
   }

   prework push N-1 pop 0 peek N-1 {
      for (int i=1; i<N; i++) {
         push(doFIR(i));
      }
   }

   work push 1 pop 1 peek N {
      push(doFIR(N));
      pop();
   }

   float doFIR(int k) {
      float val = 0;
      for (int i=0; i<k; i++) {
         val += weights[i] * peek(k-i-1);
      }
      return val;
   }
}
\end{verbatim}
\vspace{-6pt}
\caption{StreamIt code for a simple Equalizer.
\protect\label{fig:eq-code}}
\end{minipage}
\begin{minipage}{3.5in}
\vspace{2.125in}
\psfig{figure=equalizer.eps,width=3.2in}
\vspace{12pt}
\caption{Stream graph of an Equalizer with 2 bands.  Channels are
annotated with their push (U), pop (O), and peek (E) rates.  Rates
with a value of zero are ommitted. \protect\label{fig:eq-graph}}
\end{minipage}
\end{figure}

\subsection{StreamIt Code for an Equalizer}

Figure~\ref{fig:eq-code} contains a simple example of an equalizing
software radio written in StreamIt.  The phased computation graph
corresponding to this piece of code appears in
Figure~\ref{fig:eq-graph}.  For simplicity, we model the input to the
Equalizer as a DataSource that pushes one item on every invocation.
Likewise, the output of the Equalizer is fed to a Speaker that pops
one item on every invocation.  

The Equalizer itself has two stages: a splitjoin that filters each
frequency band in parallel (using BandPassFilter's), and an Adder that
combines the output from each of the parallel filters.  The splitjoin
uses a duplicate splitter to send a copy of the input stream to each
BandPassFilter.  Then, it contains a roundrobin joiner that inputs an
item from each parallel stream in alternation.  The roundrobin joiner
is an example of a node that has multiple steady-state phases: in each
phase, it copies an item from one of the parallel streams to the
output of the splitjoin.  In StreamIt, splitter and joiner nodes are
compiler-defined primitives.

For an example of a user-defined node, consider the Adder filter.  The
Adder takes an argument, N, indicating the number of items it should
add.  The declaration of its steady-state work block indicates that on
each invocation, the Adder pushes 1 item to the output channel and
pops N items from the input channel.  The code within the work
function performs the summation.

The BandPassFilter is an example of a filter with both an initial and
steady-state epoch.  The code within the init block is executed before
any node fires; in this case, it calculates the weights that should be
used for filtering.  The prework block specifies the behavior of the
filter for the initial epoch.  In the case of an FIR filter, the
initial epoch is essential for dealing with the startup condition when
there are fewer items available on the input channel than there are
taps in the filter.  Given that there are N taps in the filter, the
prework function\footnote{To simplify the equations, we have written
the BandPassFilter to have only one phase in the initial epoch.
However, it would also be possible to give a more fine-grained
description with multiple initial phases.} performs the FIR filter on
the first $i$ items, for $i = 1 \dots N-1$.  Then, the steady state
work block always operates on $N$ items, popping an item from the
input channel after each invocation.

\subsection{Converting to a SARE}

We will generate a SARE corresponding to $N$ steady-state executions
of the above PCP.

\subsubsection{The Steady-State Period}

The first step in the translation is to calculate $S(n)$, the number
of times that a given node $n$ fires in a periodic steady-state
execution (see Section~\ref{sec:balance}).  Using $S(n)$ we can also
derive $\mt{Period}(c)$, the number of items that are transferred over
channel $c$ in one steady-state period.  This can be done using
balance equations on the steady-state I/O rates of the
stream~\cite{leesdf}:
\begin{align*}
\forall c = (n_a, n_b):~~S(n_a) * \mt{TotalWrite}(n_a, c, \mt{steady}) 
  = S(n_b) * \mt{TotalRead}(n_b, c, \mt{steady})
\end{align*}
where $\mt{TotatalRead}$ and $\mt{TotalWrite}$ are defined as in
Figure~\ref{fig:helper} to denote the total number of items that a
node reads and writes to a given channel during one cycle of its
phases.  Expanding these definitions for the stream graph in
Figure~\ref{fig:eq-graph}, we have:
\begin{align*}
S(\mt{RadioSource}) * \mt{TotalWrite}(\mt{RadioSource}, \mt{c1}, \mt{steady}) 
  &= S(\mt{duplicate}) * \mt{TotalRead}(\mt{duplicate}, \mt{c1}, \mt{steady}) \\
S(\mt{duplicate}) * \mt{TotalWrite}(\mt{duplicate}, \mt{c2}, \mt{steady}) 
  &= S(\mt{BPF}_1) * \mt{TotalRead}(\mt{BPF}_1, \mt{c2}, \mt{steady}) \\
S(\mt{duplicate}) * \mt{TotalWrite}(\mt{duplicate}, \mt{c3}, \mt{steady})
  &= S(\mt{BPF}_2) * \mt{TotalRead}(\mt{BPF}_2, \mt{c3}, \mt{steady}) \\
S(\mt{BPF}_1) * \mt{TotalWrite}(\mt{BPF}_1, \mt{c4}, \mt{steady}) 
  &= S(\mt{roundrobin}) * \mt{TotalRead}(\mt{roundrobin}, \mt{c4}, \mt{steady}) \\
S(\mt{BPF}_2) * \mt{TotalWrite}(\mt{BPF}_2, \mt{c5}, \mt{steady}) 
  &= S(\mt{roundrobin}) * \mt{TotalRead}(\mt{roundrobin}, \mt{c5}, \mt{steady}) \\
S(\mt{roundrobin}) * \mt{TotalWrite}(\mt{roudnrobin}, \mt{c6}, \mt{steady})
  &= S(\mt{Adder}) * \mt{TotalRead}(\mt{Adder}, \mt{c6}, \mt{steady}) \\
S(\mt{Adder}) 
  &= S(\mt{Speaker})
\end{align*}
Evaluating the I/O rates, this simplifies to:
\begin{align*}
S(\mt{RadioSource}) &= S(\mt{duplicate}) \\
S(\mt{duplicate})  &= S(\mt{BPF}_2) \\
S(\mt{duplicate})  &= S(\mt{BPF}_2) \\
S(\mt{BPF}_1) &= S(\mt{roundrobin}) \\
S(\mt{BPF}_2) &= S(\mt{roundrobin}) \\
S(\mt{roundrobin}) * 2 &= S(\mt{Adder}) * 2 \\
S(\mt{Adder}) &= S(\mt{Speaker})
\end{align*}
Solving for the minimum integral solution for $S$ gives that $S(n) = 1$ for all nodes $n$:
\begin{align*}
S(\mt{RadioSource}) &= S(\mt{duplicate}) = S(\mt{BPF}_1) = S(\mt{BPF}_2) = 1 \\
S(\mt{roundrobin}) &= S(\mt{Adder}) = S(\mt{Speaker}) = 1
\end{align*}
We can now calculate $\mt{Period}(c)$ for each channel $c$ in the
graph.  Using the definition of $\mt{Period}$ from
Figure~\ref{fig:helper}, we have the following:
\begin{align*}
\mt{Period}(\mt{c1}) &= S(\mt{RadioSource}) * TotalWrite(\mt{RadioSource}, \mt{c1}, \mt{steady}) \\
\mt{Period}(\mt{c2}) &= S(\mt{duplicate}) * TotalWrite(\mt{duplicate}, \mt{c2}, \mt{steady}) \\
\mt{Period}(\mt{c3}) &= S(\mt{duplicate}) * TotalWrite(\mt{duplicate}, \mt{c3}, \mt{steady}) \\
\mt{Period}(\mt{c4}) &= S(\mt{BPF}_1) * TotalWrite(\mt{BPF}_1, \mt{c4}, \mt{steady}) \\
\mt{Period}(\mt{c5}) &= S(\mt{BPF}_2) * TotalWrite(\mt{BPF}_2, \mt{c5}, \mt{steady}) \\
\mt{Period}(\mt{c6}) &= S(\mt{roundrobin}) * TotalWrite(\mt{roundrobin}, \mt{c6}, \mt{steady}) \\
\mt{Period}(\mt{c7}) &= S(\mt{Adder}) * TotalWrite(\mt{Adder}, \mt{c7}, \mt{steady})
\end{align*}
Evaluating each right hand side gives the following:
\begin{align*}
\mt{Period}(\mt{c1}) &= \mt{Period}(\mt{c2}) = \mt{Period}(\mt{c3}) = \mt{Period}(\mt{c4}) = \mt{Period}(\mt{c5}) = 1 \\
\mt{Period}(\mt{c6}) &= 2 \\
\mt{Period}(\mt{c7}) &= 1 
\end{align*}

\subsubsection{The Variables}

Let us consider each variable in turn:

\sssection{IBUF} 

The $\mt{IBUF}$ variables are for holding both initial items and items
that are produced during the initial epoch.  In our example, the only
such nodes are the BandPassFilter's, which push 49 items onto
$\mt{c4}$ and $\mt{c5}$ in the initial epoch:
\begin{align*}
{\cal D}_{{IBUF}_{c4}} &= \{~i~|~0 \le i \le 48\} \\
{\cal D}_{{IBUF}_{c5}} &= \{~i~|~0 \le i \le 48\}
\end{align*}
The domains of all the other $\mt{IBUF}$ variables are empty.

\sssection{IWRITE} 

The $\mt{IWRITE}$ variables are for holding items that are produced
during the initial epoch.  In the general case, $\mt{IWRITE}_c$ might
have a smaller extent than $\mt{IBUF}_c$, since $\mt{IBUF}_c$ also
holds the $A(c)$ items that appear on channel $c$ before the initial
epoch.  However, in our example there are no initial items, and the
domains for $\mt{IWRITE}$ exactly mirror those of $\mt{IBUF}$:
\begin{align*}
{\cal D}_{{IWRITE}_{c4}} &= \{~i~|~0 \le i \le 48\} \\
{\cal D}_{{IWRITE}_{c5}} &= \{~i~|~0 \le i \le 48\}
\end{align*}

\sssection{IREAD} 
The $\mt{IREAD}$ variables are for holding items that are read during
the initial epoch.  In our example, each of the BandPassFilter's read
49 items during the initial epoch, thereby yielding an $\mt{IREAD}$
variable on each of their input channels:
\begin{align*}
{\cal D}_{{IREAD}_{c2}} &= \{~i~|~0 \le i \le 48 \} \\
{\cal D}_{{IREAD}_{c3}} &= \{~i~|~0 \le i \le 48 \}
\end{align*}
For all other channels, the domain of $\mt{IREAD}$ is empty.

\sssection{SBUF} 

The $\mt{SBUF}$ variables represent the steady-state buffer space on a
channel.  They contain two dimensions: the first counts over
steady-state execution cycles, and the second counts over items that
appear on the channel during a given cycle (that is, the $\mt{Period}$
of the channel).  In the case of our example, all channels have a
$\mt{Period}$ of 1, except for $\mt{c6}$, which has a $\mt{Period}$
of 2.  Thus, the equations for $\mt{SBUF}$ are as follows:
\begin{align*}
{\cal D}_{{SBUF}_{c1}} &= \{ ~(i,0)~|~0 \le i \le N - 1\} \\
{\cal D}_{{SBUF}_{c2}} &= \{ ~(i,0)~|~0 \le i \le N - 1 \} \\
{\cal D}_{{SBUF}_{c3}} &= \{ ~(i,0)~|~0 \le i \le N - 1 \} \\
{\cal D}_{{SBUF}_{c4}} &= \{ ~(i,0)~|~0 \le i \le N - 1 \} \\
{\cal D}_{{SBUF}_{c5}} &= \{ ~(i,0)~|~0 \le i \le N - 1 \} \\
{\cal D}_{{SBUF}_{c6}} &= \{ ~(i,j)~|~0 \le i \le N - 1 ~\wedge~ 0 \le j \le 1 \} \\
{\cal D}_{{SBUF}_{c7}} &= \{ ~(i,0)~|~0 \le i \le N - 1\}
\end{align*}

\sssection{SWRITE} 

The $\mt{SWRITE}$ variables represent temporary buffers for the output
of nodes in the steady state.  Here $\mt{c6}$ is distinguished
because it has two separate phases, corresponding to the cyclic
behavior of the $\mt{roundrobin}$ node.  Since each phase outputs an
item, there is a buffer to hold the output of each:
\begin{align*}
{\cal D}_{{SWRITE}_{c1}} &= \{~(i,0,0)~|~0 \le i \le N-1\} \\
{\cal D}_{{SWRITE}_{c2}} &= \{~(i,0,0)~|~0 \le i \le N-1 \} \\
{\cal D}_{{SWRITE}_{c3}} &= \{~(i,0,0)~|~0 \le i \le N-1 \} \\
{\cal D}_{{SWRITE}_{c4}} &= \{~(i,0,0)~|~0 \le i \le N-1 \} \\
{\cal D}_{{SWRITE}_{c5}} &= \{~(i,0,0)~|~0 \le i \le N-1 \} \\
{\cal D}_{{SWRITE}_{{c6}, 0}} &= \{~(i,0,0)~|~0 \le i \le N-1 \} \\
{\cal D}_{{SWRITE}_{{c6}, 1}} &= \{~(i,0,0)~|~0 \le i \le N-1 \} \\
{\cal D}_{{SWRITE}_{c7}} &= \{~(i,0,0)~|~0 \le i \le N-1\} 
\end{align*}

\sssection{SREAD} 

The $\mt{SREAD}$ variables represent temporary buffers for the nodes
to read from in the steady state.  The $k$ dimension of these buffers
represents the number of items that are read at once; this is 49 in
the case of the inputs to the BandPassFilter's, and 1 for all other
channels:
\begin{align*}
{\cal D}_{{SREAD}_{c1}} &= \{~(i,0,0)~|~0 \le i \le N-1\}
{\cal D}_{{SREAD}_{c2}} &= \{~(i,0,k)~|~0 \le i \le N-1 ~\wedge~ 0 \le k \le 48\}
{\cal D}_{{SREAD}_{c3}} &= \{~(i,0,k)~|~0 \le i \le N-1 ~\wedge~ 0 \le k \le 48\}
{\cal D}_{{SREAD}_{{c4}, 0}} &= \{~(i,0,0)~|~0 \le i \le N-1 \}
{\cal D}_{{SREAD}_{{c5}, 1}} &= \{~(i,0,0)~|~0 \le i \le N-1 \}
{\cal D}_{{SREAD}_{c6}} &= \{~(i,0,k)~|~0 \le i \le N-1 ~\wedge~ 0 \le k \le 1 \}
{\cal D}_{{SREAD}_{c7}} &= \{~(i,0,0)~|~0 \le i \le N-1\}
\end{align*}
Note that channels $\mt{c4}$ and $\mt{c5}$ have two separate
phases, corresponding to the cycle of the roundrobin joiner.  However,
since the roundrobin joiner does not read from $\mt{c5}$ during
phase 0, and does not read from $\mt{c4}$ during phase 1, the
domains for these input arrays are empty.  Instead, there is a buffer
only for $\mt{c4}$ at phase 0 and $\mt{c5}$ at phase 1.

%%%%%%%%%%%%%%%%%%%%%%%%%%%%%%%%%%%%%%%%%%%%%%%%%%%%%%%%%%%%%%%%%%%%%
%% 
%% OLD ORGANIZATION OF WRITE-UP

%% \sssection{Channel 1}

%% Since there are no items initially placed on $\mt{c1}$, and there is
%% no writing or reading from $\mt{c1}$ during the initial epoch, the
%% domains for $\mt{IBUF}_{c1}$, $\mt{IWRITE}_{c1}$, and
%% $\mt{IREAD}_{c1}$ are empty.  We obtain the domains for
%% $\mt{SBUF}_{c1}$, $\mt{SWRITE}_{c1}$, and $\mt{SREAD}_{c1}$ using
%% Equations~\ref{eq:sbuf}, \ref{eq:swrite}, and~\ref{eq:sread},
%% respectively.  Since each node connected to $\mt{c1}$ has only one
%% phase, we omit the phase subscript from these variables:
%% \begin{align*}
%% {\cal D}_{{SBUF}_{c1}} &= \{ ~(i,0)~|~0 \le i \le N - 1\} \\
%% {\cal D}_{{SWRITE}_{c1}} &= \{~(i,0,0)~|~0 \le i \le N-1\} \\
%% {\cal D}_{{SREAD}_{c1}} &= \{~(i,0,0)~|~0 \le i \le N-1\}
%% \end{align*}

%% \sssection{Channel 2}

%% As in the case of $\mt{c1}$, $\mt{IBUF}_{c2}$ has an empty domain
%% because there are no items initially on $\mt{c2}$, and
%% $\mt{IWRITE}_{c2}$ has an empty domain because the $\mt{duplicate}$
%% node does not push onto $\mt{c2}$ during the initial epoch.  However,
%% since the BandPassFilter ($\mt{BPF}_1$) has an initial work function,
%% there is an $\mt{IREAD}_{c2}$ variable, with a domain according to
%% Equation~\ref{eq:iread}:
%% \begin{align*}
%% {\cal D}_{{IREAD}_{c2}} &= \{~i~|~0 \le i \le 48 \}
%% \end{align*}
%% The steady-state variables are analogous to $\mt{c1}$, except that
%% the $\mt{SREAD}$ array needs to have space for all of the items peeked
%% by the BandPassFilter:
%% \begin{align*}
%% {\cal D}_{{SBUF}_{c2}} &= \{ ~(i,0)~|~0 \le i \le N - 1 \} \\
%% {\cal D}_{{SWRITE}_{c2}} &= \{~(i,0,0)~|~0 \le i \le N-1 \} \\
%% {\cal D}_{{SREAD}_{c2}} &= \{~(i,0,k)~|~0 \le i \le N-1 ~\wedge~ 0 \le k \le 48\}
%% \end{align*}

%% \sssection{Channel 3}

%% Channel 3 has the same set of variables as Channel 2:
%% \begin{align*}
%% {\cal D}_{{IREAD}_{c3}} &= \{~i~|~0 \le i \le 48 \} \\
%% {\cal D}_{{SBUF}_{c3}} &= \{ ~(i,0)~|~0 \le i \le N - 1 \} \\
%% {\cal D}_{{SWRITE}_{c3}} &= \{~(i,0,0)~|~0 \le i \le N-1 \} \\
%% {\cal D}_{{SREAD}_{c3}} &= \{~(i,0,k)~|~0 \le i \le N-1 ~\wedge~ 0 \le k \le 48\}
%% \end{align*}

%% \sssection{Channel 4}

%% On $\mt{c4}$, items are pushed in both the initial and steady-state
%% epochs, and items are read in two different steady-state phases.
%% However, the second of these phases actually reads zero items from the
%% channel, so the domain for $\mt{SREAD}_{c4, 1}$ is empty.
%% $\mt{IREAD}_{c4}$ has an empty domains by the same reasoning as for
%% $\mt{c1}$.  The other variables have domains as follows:
%% \begin{align*}
%% {\cal D}_{{IWRITE}_{c4}} &= \{~i~|~0 \le i \le 48\} \\
%% {\cal D}_{{IBUF}_{c4}} &= \{~i~|~0 \le i \le 48\} \\
%% {\cal D}_{{SBUF}_{c4}} &= \{ ~(i,0)~|~0 \le i \le N - 1 \} \\
%% {\cal D}_{{SWRITE}_{c4}} &= \{~(i,0,0)~|~0 \le i \le N-1 \} \\
%% {\cal D}_{{SREAD}_{{c4}, 0}} &= \{~(i,0,0)~|~0 \le i \le N-1 \}
%% \end{align*}

%% \sssection{Channel 5}

%% The variables for $\mt{c5}$ are similar to those for $\mt{c4}$,
%% except that the reading phases are reversed: the $\mt{roundrobin}$
%% node reads no items during the first phase, and reads one item during
%% the second phase.  Thus, $\mt{SREAD}_{c5, 0}$ is empty, and
%% $\mt{SREAD}_{c5, 1}$ is as $\mt{SREAD}_{c4, 0}$ above:
%% \begin{align*}
%% {\cal D}_{{IWRITE}_{c5}} &= \{~i~|~0 \le i \le 48\} \\
%% {\cal D}_{{IBUF}_{c5}} &= \{~i~|~0 \le i \le 48\} \\
%% {\cal D}_{{SBUF}_{c5}} &= \{ ~(i,0)~|~0 \le i \le N - 1 \} \\
%% {\cal D}_{{SWRITE}_{c5}} &= \{~(i,0,0)~|~0 \le i \le N-1 \} \\
%% {\cal D}_{{SREAD}_{{c5}, 1}} &= \{~(i,0,0)~|~0 \le i \le N-1 \}
%% \end{align*}

%% \sssection{Channel 6}

%% Channel 6 is distinguished by having two phases that write to it.  As
%% above, $\mt{IBUF}_{c6}$ is empty due to a lack of initial items, and
%% $\mt{IWRITE}_{c6}$ and $\mt{IREAD}_{c6}$ is empty because there is no
%% activity during the initial epoch.  The other variables have domains
%% as follows:
%% \begin{align*}
%% {\cal D}_{{SBUF}_{c6}} &= \{ ~(i,j)~|~0 \le i \le N - 1 ~\wedge~ 0 \le j \le 1 \} \\
%% {\cal D}_{{SWRITE}_{{c6}, 0}} &= \{~(i,0,0)~|~0 \le i \le N-1 \} \\
%% {\cal D}_{{SWRITE}_{{c6}, 1}} &= \{~(i,0,0)~|~0 \le i \le N-1 \} \\
%% {\cal D}_{{SREAD}_{c6}} &= \{~(i,0,k)~|~0 \le i \le N-1 ~\wedge~ 0 \le k \le 1 \}
%% \end{align*}

%% \sssection{Channel 7}

%% Channel 7 has identical I/O characteristics as channel 1, and thus its variables are identical:
%% \begin{align*}
%% {\cal D}_{{SBUF}_{c7}} &= \{ ~(i,0)~|~0 \le i \le N - 1\} \\
%% {\cal D}_{{SWRITE}_{c7}} &= \{~(i,0,0)~|~0 \le i \le N-1\} \\
%% {\cal D}_{{SREAD}_{c7}} &= \{~(i,0,0)~|~0 \le i \le N-1\}
%% \end{align*}

\subsubsection{The Equations}

We consider each of the equations in turn.

\sssection{I to IBUF}

In our example, this equation always has an empty domain, as there are
no channels with initial items.  (That is, for all $c$, $A(c) = 0$.)

\sssection{IWRITE to IBUF}

This equation has a non-empty domain for channels that are written to
during the initialization epoch, which in the case of our example is
$\mt{c4}$ and $\mt{c5}$.  Since both of these channels only have
one phase of writing during this epoch, these equations specify a
simple copy from $\mt{IWRITE}$ to $\mt{IBUF}$:
\begin{align*}
\forall i \in [0, 48]:~~\mt{IBUF}_{c4}(i) &= \mt{IWRITE}_{c4}(i) \\
\forall i \in [0, 48]:~~\mt{IBUF}_{c5}(i) &= \mt{IWRITE}_{c5}(i)
\end{align*}

\sssection{IREAD to IWRITE}

These equations represent the computation of nodes that fire during
the initialization epoch.  In our example, the only such nodes are the
BandPassFilter's, and there is one equation for each of them:
\begin{align*}
&\forall i \in [0, 48]:~~\mt{IWRITE}_{c4}(i) &= W(\mt{BPF}_1, \mt{init})(\mt{IREAD}_{c2}(*))[0][i] \\
&\forall i \in [0, 48]:~~\mt{IWRITE}_{c5}(i) &= W(\mt{BPF}_2, \mt{init})(\mt{IREAD}_{c3}(*))[0][i]
\end{align*}

\sssection{SREAD to SWRITE}

These equations represent the steady-state computation of the nodes.
For the BandPassFilter's and Adder nodes, the computation is according
to the user-defined work function $W$:
\begin{align*}
\forall i \in [0, N-1]:~~\mt{SWRITE}_{c4}(i,0,0) &= W(\mt{BPF}_1, \mt{steady})(\mt{SREAD}_{c2}(i, 0, *))[0][0] \\
\forall i \in [0, N-1]:~~\mt{SWRITE}_{c5}(i,0,0) &= W(\mt{BPF}_2, \mt{steady})(\mt{SREAD}_{c3}(i, 0, *))[0][0] \\
\forall i \in [0, N-1]:~~\mt{SWRITE}_{c7}(i,0,0) &= W(\mt{Adder}, \mt{steady})(\mt{SREAD}_{c6}(i, 0, *))[0][0]
\end{align*}
For the $\mt{dupliacte}$ and $\mt{roundrobin}$ nodes, the work
function is simple and compiler-defined, so we give the equations
directly without appealing to a work function:
\begin{align*}
\forall i \in [0, N-1]:~~\mt{SWRITE}_{c2}(i,0,0) &= \mt{SREAD}_{c1}(i, 0, 0)[0][0]~~~~~~~(\mt{duplicate c1} \rightarrow \mt{c2}) \\
\forall i \in [0, N-1]:~~\mt{SWRITE}_{c3}(i,0,0) &= \mt{SREAD}_{c1}(i, 0, 0)[0][0]~~~~~~~(\mt{duplicate c1} \rightarrow \mt{c3}) \\
\forall i \in [0, N-1]:~~\mt{SWRITE}_{{c6}, 0}(i,0,0) &= \mt{SREAD}_{{c4}, 0}(i, 0, 0)[0][0]~~~~~~~(\mt{roundrobin c4} \rightarrow \mt{c6}) \\
\forall i \in [0, N-1]:~~\mt{SWRITE}_{{c6}, 1}(i,0,0) &= \mt{SREAD}_{{c5}, 1}(i, 0, 0)[0][0]~~~~~~~(\mt{roundrobin c5} \rightarrow \mt{c6})
\end{align*}

\sssection{SWRITE to SBUF}

This equation transfers items into the index space in which they were
written to the index space of the buffer for a given channel.  In the
general case (Equation~\ref{swrite2sbuf}) we need to ``slice'' the
domain into $S(n)$ pieces in order to maintain a uniform left-hand
side (see Section~\ref{sec:simplesare}).  However, in this example,
$S(n) = 1$ for all $n$, and this equation becomes a direct copy
between the $\mt{SWRITE}$ and $\mt{SBUF}$ variables.  There is one
such copy for each channel in the program:
\begin{align*}
\forall i \in [0, N-1]:~~\mt{SBUF}_{c1}(i, 0) = \mt{SWRITE}_{c1}(i, 0, 0) \\
\forall i \in [0, N-1]:~~\mt{SBUF}_{c2}(i, 0) = \mt{SWRITE}_{c2}(i, 0, 0) \\
\forall i \in [0, N-1]:~~\mt{SBUF}_{c3}(i, 0) = \mt{SWRITE}_{c3}(i, 0, 0) \\
\forall i \in [0, N-1]:~~\mt{SBUF}_{c4}(i, 0) = \mt{SWRITE}_{c4}(i, 0, 0) \\
\forall i \in [0, N-1]:~~\mt{SBUF}_{c5}(i, 0) = \mt{SWRITE}_{c5}(i, 0, 0) \\
\forall i \in [0, N-1]:~~\mt{SBUF}_{c6}(i, 0) = \mt{SWRITE}_{{c6}, 0}(i, 0, 0) \\
\forall i \in [0, N-1]:~~\mt{SBUF}_{c6}(i, 1) = \mt{SWRITE}_{{c6}, 1}(i, 0, 0) \\
\forall i \in [0, N-1]:~~\mt{SBUF}_{c7}(i, 0) = \mt{SWRITE}_{c7}(i, 0, 0)
\end{align*}
Note that $\mt{c6}$ is slightly different because there is one
equation required for each phase of writing.

\sssection{IBUF to IREAD}

This equation has a non-empty domain for channels which contain items
that are both written and read during the initial epoch.  In the case
of our example, none such channels exist, because the only nodes with
an initial epoch (the BandPassFilter's) do not communicate with each
other.  Mathematically, one can see that the domain of this equation
is empty because there is no channel with both a non-empty
$\mt{IREAD}$ variable and a non-empty $\mt{IBUF}$ variable, as would
be required for a non-empty ${\cal D}_{IB \rightarrow IR}$ in
Equation~\ref{ibuf2iread}.

\sssection{SBUF to IREAD}

This equation has a non-empty domain for channels which contain items
that are written during the steady-state epoch but read during the
initial epoch.  In the case of our example, channels $\mt{c2}$ and
$\mt{c3}$ satisfy this criterion.  According to
Equation~\ref{sbuf2iread}, we need to introduce 49 equations for
each channel ($q$ ranges from $0$ to $49$, but when $q = 49$ the
domain is empty).  Doing so would give the following:
\begin{align*}
\forall q \in [0, 48]:~~\forall i \in \{q\}, \mt{IREAD}_{c2}(i) &= \mt{SBUF}_{c2}(i, 0) \\
\forall q \in [0, 48]:~~\forall i \in \{q\}, \mt{IREAD}_{c3}(i) &= \mt{SBUF}_{c3}(i, 0)
\end{align*}
In these equations, we have encountered a special case where
$\mt{Period}(c1)$ $=$ $\mt{Period}(c2)$ $=$ $1$, and it is not
necessary to slice the domain (as the index to the second dimension of
$\mt{SBUF}$ is constant.)  Thus, we can reduce these 98 equations into
just 2, where the domain of each equation is the domain of $q$ above:
\begin{align*}
\forall i \in [0, 48]:~~\mt{IREAD}_{c2}(i) &= \mt{SBUF}_{c2}(i, 0) \\
\forall i \in [0, 48]:~~\mt{IREAD}_{c3}(i) &= \mt{SBUF}_{c3}(i, 0)
\end{align*}
However, in the general case, it might be necessary to introduce a
large number of equations in this step.  The number of equations will
grow with the extent of a node's peeking during the initial epoch.
Note that the number of equations will {\it not} grow with the number
of steady-state cycles that the SARE is simulating.

\sssection{IBUF to SREAD} 

This equation has a non-empty domain for channels which contain items
that are written during the initial epoch but read during the
steady-state epoch.  In the case of our example, channels $\mt{c4}$
and $\mt{c5}$ satisfy this criterion.  The following equations
simply copy the items from the initial buffer into the steady-state
buffer:
\begin{align*}
\forall i \in [0, 48]:~~\mt{SREAD}_{{c4}, 0}(i,0,0) &= \mt{IBUF}_{c4}(i) \\
\forall i \in [0, 48]:~~\mt{SREAD}_{{c5}, 1}(i,0,0) &= \mt{IBUF}_{c5}(i) 
\end{align*}
Note that these equations are somewhat simpler than those in
Equation~\ref{ibuf2sread} because no items are read from
$\mt{c4}$ or $\mt{c5}$ during the initial epoch.  If this were the
case, then there would be offsets in the domains above.

\sssection{SBUF to SREAD}

This equation represents the copying of items from a channel's buffer
to the read array that a node will access.  As
Equation~\ref{sbuf2sread} is rather complicated, we will consider each
case in turn.  

First let us examine $\mt{c1}$ and $\mt{c2}$, which are simple due to
their lack of both peeking and epochs.  In this case, the equation
copies items directly from the $\mt{SBUF}$ array to the $\mt{SREAD}$
array:
\begin{align*}
\forall i \in [0,N-1]:~~\mt{SREAD}_{c1}(i,0,0) = \mt{SBUF}(i,0) \\
\forall i \in [0,N-1]:~~\mt{SREAD}_{c7}(i,0,0) = \mt{SBUF}(i,0)
\end{align*}
Only slightly more complex is the equation for $\mt{c6}$, in which two
items are read from the channel for each invocation of the Adder.
This requires an extra subscript $q$ on $\mt{SREAD}$, that happens to
coincide with $\mt{SBUF}$ since the Adder executes once per
steady-state execution of the graph ($S(\mt{Adder}) = 1$):
\begin{align*}
\forall q \in [0,1]:~~\forall i \in [0,N-1]:~~\mt{SREAD}_{c6}(i,0,q) = \mt{SBUF}_{c6}(i,q)
\end{align*}
Note that the above translation is naively specified as two separate
equations (one for each value of $q$); however, as we saw above, we
can combine these equations into a single one by considering $q$ to be
a domain quantifier rather than an enumerator over equations.

Next, let us consider the equations for $\mt{c2}$ and $\mt{c3}$, which
encapsulates the peeking behavior of the BandPassFilter's.  In this
case, $\mt{SREAD}$ is viewing the items on the channel in overlapping
segments of 50 items each, whereas $\mt{SBUF}$ is viewing the channel
as a continuous sequence where each item appears only once.  Thus, the
following equations duplicate the data by a factor of 50 as they copy
it into the $\mt{SREAD}$ array:
\begin{align*}
\forall q \in [0,49]:~~\forall i \in [0,N-1]:~~\mt{SREAD}_{c2}(i,0,q) = \mt{SBUF}_{c2}(i+q,0) \\
\forall q \in [0,49]:~~\forall i \in [0,N-1]:~~\mt{SREAD}_{c3}(i,0,q) = \mt{SBUF}_{c3}(i+q,0)
\end{align*}

Finally, consider the equations for $\mt{c4}$ and $\mt{c5}$.  On these
channels, the beginning of the $\mt{SREAD}$ array was filled by the
initial ouput of the BandPassFilter's (see IBUF to SREAD above).
Thus, the steady-state output in the $\mt{SBUF}$ array needs to be
copied at an offset into the $\mt{SREAD}$ array so that it does not
overwrite the initial output.  Following Equation~\ref{sbuf2sread2},
we calculate this offset as the number of items that were pushed onto
the channel in the initial epoch; this is given by the size of the
domain of the IBUF to SREAD equation: $|{\cal D}_{IB \rightarrow
SR}(\mt{roundrobin}, c4)|$ $=$ $|{\cal D}_{IB \rightarrow
SR}(\mt{roundrobin}, c4)|$ $=$ $49$.  Since the $\mt{c4}$ and
$\mt{c5}$ both have a $\mt{Period}$ of 1, this offset does not fall on
the boundary between two different $i$ indices in $\mt{SBUF}$, and
thus we don't need to worry about the $\mt{Int\_Offset}$ and
$\mt{Mod\_Offset}$ defined in Equation~\ref{sbuf2sread2}.  The
resulting equations represent a simple shifted copy operation, as
follows:
\begin{align*}
\forall i \in [49,N-1]:~~\mt{SREAD}_{{c4}, 0}(i,0,0) = \mt{SBUF}_{c4}(i-49,0) \\
\forall i \in [49,N-1]:~~\mt{SREAD}_{{c5}, 1}(i,0,0) = \mt{SBUF}_{c5}(i-49,0)
\end{align*}

This concludes the translation of the StreamIt-based PCP to a SARE.
The equations and variables defined in the previous sections are
exactly equivalent to an execution of the original StreamIt program
for $N$ steady-state cycles.
