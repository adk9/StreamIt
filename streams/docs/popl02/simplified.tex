\section{Basic Translation: SDF to SARE}
\label{sec:simple}

In order to familiarize the reader with the basics of our technique,
we present in this section the translation procedure for a simplified
input domain.  The translation rules for the general case can be found
in \todo{Next section??}.

Our basic translation operates on synchronous dataflow graphs with all
of the channels initially empty.  That is, we assume that:
\begin{enumerate}

\item No items appear on the channels at the beginning:  $\forall c, A(c) = 0$.

\item There are no initialization phases: $\forall n, \mt{num}(n, \mt{init}) = 0$.

\item Each node has only one steady-state phase: $\forall n, \mt{num}(n, \mt{steady}) = 1$.

\item There is no peeking: $\forall c, \forall t, \forall p, E(c, t, p) = O(c, t, p)$.

\end{enumerate}

Given these restrictions, we can simplify our notation considerably.
Since we are only concerned with the steady-state epoch and the 0'th
phase of each node, we can omit some arguments to any function that
requires an epoch $t$ or a phase $p$.  For instance, the push and pop
rates of a channel $c$ are now just $U(c)$ and $O(c)$, respectively;
the work function for a node $n$ is simply $W(n)$.

\subsection{Calculating the Steady-State Period}

Let $S(n)$ denote the number of times that node $n$ fires its 0'th
phase for a balanced steady-state execution cycle of the entire
graph. \todo{Say more about this.}

We will use the following helper function.  Given channel $c = (n_a, n_b)$:
\begin{align*}
\mt{Period}(c) \equiv S(n_a) * U(n_a) = S(n_b) * U(n_b)
\end{align*}

\subsection{Generating a SARE}

The SARE will be parameterized by $N$, the number of steady-state
cycles that one wishes to execute in the PCP.

\subsubsection{Variables}

For each channel $c = (n_a, n_b)$, do the following:
\begin{itemize}

\item Introduce variable $\mt{BUF}_c$ with the following domain:
\begin{align*}
{\cal D}_{{BUF}_c} = \{ ~(i,j)~|~0 \le i \le N - 1 ~\wedge~ 0 \le j \le \mt{Period}(c) - 1\}
\end{align*}

\item Introduce variable $\mt{WRITE}_{c}$ with this domain:
\begin{align*}
{\cal D}_{{WRITE}_{c}} = \{~(i,j,k)~|~0 \le i \le N-1 ~\wedge~ 
                                       0 \le j \le S(n_a) - 1 ~\wedge~ 0 \le k \le U(c) - 1\}
\end{align*}

\item Introduce variable $\mt{READ}_{c, p}$ with this domain:
\begin{align*}
{\cal D}_{{READ}_{c}} = \{~(i,j,k)~|~0 \le i \le N-1 ~\wedge~ 
                                      0 \le j \le S(n_b) - 1 ~\wedge~ 
                                      0 \le k \le E(c) - 1\} \\
\end{align*}

\end{itemize}

\subsubsection{Equations}

\noindent For each node $n$, do the following:
\begin{itemize}

\item For each $c \in \mt{chan\_out}(n)$, introduce this equation
(READ to WRITE):
%
\begin{align*}
\forall (i, j, k) \in {\cal D}_{{WRITE}_{c}},~\mt{WRITE}_{c}(i,j,k) = W(n)(\mt{Steady\_Inputs})[\mt{pos\_out}(n, c)][j] \\
\mt{where Steady\_Inputs} = [\mt{READ}_{{chan\_in}(n)[1]}(i, j, *), \dots, 
                             \mt{READ}_{{chan\_in}(n)[\mt{num\_in}(n)]}(i, j, *) ]
\end{align*}
%
\item For each $c \in \mt{chan\_out}(n)$, and for each $q \in
[0,\frac{\mt{Period(c)}}{\mt{U(n)}}-\mt{1}]$,
introduce this equation (WRITE to BUF):
\begin{align*}
\forall (i,j) \in {\cal D}_{SW {\small \rightarrow} SB}(c,q), \\
\mt{BUF}_{c}(i,j) = 
  \mt{WRITE}_{c}(i * \frac{\mt{Period}(c)}{O(c)} + \lfloor \frac{q}{S(n)} \rfloor, q~mod~S(n),
                     j - q * U(c) \\
\mt{where } {\cal D}_{SW {\small \rightarrow} SB}(c,q) = 
  {\cal D}_{{BUF}_{c}} \cap 
  \{ (i,j) | q*U(c) \le j \le (q+1)*U(c) - 1 \}
\end{align*}

% here is the simple, base case of the BUF -> READ function
\item For each $c \in \mt{chan\_in}(n)$, introduce this equation (BUF to
READ):
\begin{align*}
\forall (i,j,k) \in {\cal D}_{{READ}_{c}},
\mt{READ}_{c}(i,j,k) = j*O(n) + k
\end{align*}
\end{itemize}

\begin{figure}[t]
\framebox[6.5in]{
\begin{minipage}{6.5in}
\end{minipage}
}
\end{figure}


