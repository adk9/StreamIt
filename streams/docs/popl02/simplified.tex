\section{Basic Translation: CG to SARE}
\label{sec:simple}

In order to familiarize the reader with the basics of our technique,
we present in this section the translation procedure for a simplified
input domain.  The translation rules for the general case can be found
in Section~\ref{sec:translate}.

Our basic translation operates on computation graphs with all of the
channels initially empty.  That is, we assume that:
\begin{enumerate}

\item No items appear on the channels at the beginning:  $\forall c,~A(c) = 0$.

\item There are no initialization phases: $\forall n,~\mt{num}(n, \mt{init}) = 0$.

\item Each node has only one steady-state phase: $\forall n,~\mt{num}(n, \mt{steady}) = 1$.

\end{enumerate}

Given these restrictions, we can simplify our notation considerably.
Since we are only concerned with the steady-state epoch and the 0'th
phase of each node, we can omit some arguments to any function that
requires an epoch $t$ or a phase $p$.  For instance, the push and pop
rates of a channel $c$ are now just $U(c)$ and $O(c)$, respectively;
the work function for a node $n$ is simply $W(n)$.

\begin{figure}[t]
\framebox[6.5in]{
\begin{minipage}{6.3in}
\begin{center}
{\bf Variables}
\end{center}
\noindent For each channel $c = (n_a, n_b)$, do the following:
\begin{itemize}
%
\item Introduce variable $\mt{BUF}_c$ with the following domain:
\begin{align}
{\cal D}_{{BUF}_c} = \{ ~(i,j)~|~0 \le i \le N - 1 ~\wedge~ 0 \le j \le \mt{Period}(c) - 1\}
\end{align}
%
\item Introduce variable $\mt{WRITE}_{c}$ with this domain:
\begin{align}
{\cal D}_{{WRITE}_{c}} = \{~(i,j,k)~|~0 \le i \le N-1 ~\wedge~ 
                                       0 \le j \le S(n_a) - 1 ~\wedge~ 0 \le k \le U(c) - 1\}
\end{align}
%
\item Introduce variable $\mt{READ}_{c, p}$ with this domain:
\begin{align}
{\cal D}_{{READ}_{c}} = \{~(i,j,k)~|~0 \le i \le N-1 ~\wedge~ 
                                      0 \le j \le S(n_b) - 1 ~\wedge~ 
                                      0 \le k \le E(c) - 1\}
\end{align}
\end{itemize}
%
\begin{center}
{\bf Equations}
\end{center}
\noindent For each node $n$, do the following:
\begin{itemize}
%
\item For each $c \in \mt{chan\_out}(n)$, introduce this equation
(READ to WRITE):
%
\begin{align}
\forall (i, j, k) \in {\cal D}_{{WRITE}_{c}},~\mt{WRITE}_{c}(i,j,k) = W(n)(\mt{Steady\_Inputs})[\mt{pos\_out}(n, c)][k] \\ \nonumber
\mt{where Steady\_Inputs} = [\mt{READ}_{{chan\_in}(n)[1]}(i, j, *), \dots, 
                             \mt{READ}_{{chan\_in}(n)[\mt{num\_in}(n)]}(i, j, *) ]
\end{align}
%
\item For each $c \in \mt{chan\_out}(n)$, and for each $q \in
[0,S(n)-\mt{1}]$,
introduce this equation (WRITE to BUF):
\begin{align}
\forall (i,j) \in {\cal D}_{SW {\small \rightarrow} SB}(c,q), \\ \nonumber
\mt{BUF}_{c}(i,j) = 
  \mt{WRITE}_{c}(i, q, j - q * U(c) \\ \nonumber
\mt{where } {\cal D}_{SW {\small \rightarrow} SB}(c,q) = 
  {\cal D}_{{BUF}_{c}} \cap 
  \{ (i,j) | q*U(c) \le j \le (q+1)*U(c) - 1 \}
\end{align}
% here is the simple, base case of the BUF -> READ function
\item For each $c \in \mt{chan\_in}(n)$, introduce this equation (BUF to
READ):
\begin{align}
\forall (i,j,k) \in {\cal D}_{{READ}_{c}},
\mt{READ}_{c}(i,j,k) = \mt{BUF}_{c}(i,j*O(n) + k)
\end{align}
\end{itemize}

\end{minipage}}
\caption{Basic Procedure for generating a SARE from an SDF graph with no initial tokens.
\protect\label{fig:sdftosare}}
\end{figure}


\subsection{Calculating the Steady-State Period}
\label{sec:balance}

Let $S(n)$ denote the number of times that node $n$ fires its first
phase for a periodic steady-state execution of the entire graph. A
{\it periodic} schedule is one that does not change the number of
items on the channels in the graph after executing; in other words, it
is legal to execute in the steady state.  For an SDF graph, there is a
unique and minimal periodic schedule, of which all other periodic
schedules are a multiple~\cite{leesdf}.  It is straightforward to use
a set of balance equations to solve for $S(n)$ given the declared
rates of filters in the graph.  If the graph is invalid (i.e., it
would lead to deadlock or an unbounded buffer size) then the balance
equations will have no solution.  See~\cite{leesdf} for details.

We will use the following helper function in our analysis.  Given
channel $c = (n_a, n_b)$:
\begin{align*}
\mt{Period}(c) \equiv S(n_a) * U(c) = S(n_b) * O(c)
\end{align*}

That is, $\mt{Period}(c)$ denotes the number of items that are passed
over channel $c$ during a single steady-state execution of the graph.

\clearpage
\subsection{Generating a SARE}
\label{sec:simplesare}

We now generate a SARE for the simplified instance of a PCP.  The {\it
domains} of the SARE will be parameterized by $N$, the number of
steady-state cycles that one wishes to execute in the PCP.  However,
it is important to note that $N$ does not affect the number of
variables or equations in the SARE, as that would require a separate
compilation and analysis for each value of $N$.  It is a key benefit
of the SARE representation that $N$ can be incorporated as a symbolic
parameter in the schedule.

The procedure for generating the SARE appears in
Figure~\ref{fig:sdftosare}.  The basic idea is to introduce a variable
$\mt{BUF}_c$ for each channel $c$ which keeps track of the entire
history of values that are transmitted over $c$ (we keep track of the
entire history since all elements in a SARE must be assigned only
once.)  The $i$ dimension of $\mt{BUF}$ counts over steady-state
periods, while the $j$ dimension holds elements that were written
during a given period.  We also introduce variables $\mt{WRITE}_c$ and
$\mt{READ}_c$ that write and read from channel $c$.  For reasons that
become clear below, these variables have a $k$ dimension to index the
values that are pushed on a given firing, while the $j$ dimension
counts firings of a node and the $i$ dimension counts steady-state
periods, as before.

Equation~\ref{eq:r2w} expresses the computation internal to each node,
whereas Equations~\ref{eq:w2b} and~\ref{eq:b2r} expresses the
communication of the nodes with the channels.  In
Equation~\ref{eq:r2w}, the node's work function is used to calculate
all of the $\mt{WRITE}$ values from all of the $\mt{READ}$ values,
during a given firing $(i,j)$. 
%The $k$ dimension was added to
%$\mt{READ}$ and $\mt{WRITE}$ so that a given $(i,j)$ index pair would
%uniquely identify the read and write operations of a given firing.

Equation~\ref{eq:w2b} shows the communication from the $\mt{WRITE}$
array to the $\mt{BUF}$ array for a given channel.  To understand this
equation, it might help to consider a simpler version which expresses
the correct relationship but is not a legal SARE:
\begin{align*}
&\forall (i,j,k) \in {\cal D}_{\mt{WRITE}_c}:~~
\mt{BUF}_{c}(i,j*U(c)+k) = \mt{WRITE}_{c}(i, j, k)
\end{align*}
The equation above is simply copying the contents of $\mt{WRITE}$ into
$\mt{BUF}$ while accounting for the differing array dimensions.
However, this equation is not valid for a SARE, since there is an
affine expression indexing the array on the left hand side.

To deal with this issue, we split up the equation into several pieces,
each of which assigns to a different portion of the $\mt{BUF}$ array.
In Equation~\ref{eq:w2b}, we introduce a variable $q$ that counts up
to $S(n)$, which is the extent of the $j$ dimension of $\mt{WRITE}_c$.
For each of these $q$, we copy over $U(c)$ values from the appropriate
section of the $\mt{WRITE}$ array; the domain ${\cal D}_{W \rightarrow
B}(c,q)$ represents the $q$'th slice of $\mt{BUF}$ into which the
section should be written.  This equation is a valid component of our
SARE, as the boundaries on each equation's domain are fully resolvable
at compile time.

Equation~\ref{eq:b2r} is similar to Equation~\ref{eq:w2b} regarding
the slicing technique that is used to parameterize the domain of the
equation.  However, here the motivation is different: the extent of
the $k$ dimension of $\mt{READ}$ might exceed the length of the period
in $\mt{BUF}$ (this is because nodes are allowed to peek more than
they pop.)  Thus, the $q$ variable is used to slice the peek amount
$E(c)$ into $\mt{Period}(c)$-sized chunks so as to resolve the proper
source location in $\mt{BUF}$.  Again, all the variables in the index
expressions are compile-time constants (except, of course, for the
index quantifiers $i$, $j$, and $k$.)
