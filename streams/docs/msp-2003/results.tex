\section{Results}

We implemented the dynamic programming partitioner in our StreamIt
compiler, which targets Raw: a scalable tiled
architecture~\cite{raw-micro}. We evaluate the partitioner by
comparing it to a greedy partitioner (which was used
in~\cite{streamit-asplos}). The greedy partitioner simply identifies the
two nodes with the least work, and fuses them together before
proceeding. Both partitioning algorithms output a partition of filters
for each tile on the Raw machine.

Results appear in Table 1. We measured our benchmarks across a range
of rectangular Raw configurations (2x2, 2x4, 3x4, and 4x4) with both
the greedy and dynamic programming partitioner. The dynamic
programming partitioner achieves an average speedup of 63\%.

One reason that the dynamic programming partitioner performs better
than greedy is that it can perform both filter fission and filter
fusion in the same partitioning operation, since it knows the extent
to which it should fuse or split things. In fact, we believe that this
is a characteristic which separates our partitioner from general
scientific partitioners, and could hold the greatest promise in its
applications to other problems.

In some cases, the greedy algorithm performs better than dynamic
programming. This is due to communication overhead on the Raw
machine. Currently, the dynamic programming algorithm is not conscious
of communication costs, and can sometimes introduce extra
communication when restructuring the graph.  A future version of the
algorithm could incorporate communication costs as part of a more
accurate performance model. Also, on the FIR benchmark (which is
simply a long pipeline), the greedy algorithm has an advantage because
it re-evaluates the overhead in fusing filters after each step of the
transformation, while the dynamic programming algorithm calculates a
global solution beforehand and then carries it out without the benefit
of monitoring the overhead incurred. This effect is most pronounced
for FIR because it has the longest cascade of adjacent fusion
operations.

We have also implemented this partitioning algorithm to decide when it
is beneficial to combine linear sections of a stream graph, and/or to
them to the frequency domain~\cite{lamb03}. The partitioner
demonstrates significant savings over a naive replacement strategy.
