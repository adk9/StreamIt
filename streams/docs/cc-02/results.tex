\section{Results}
\label{sec:results}

We have implemented a fully-functional prototype of the StreamIt
compiler as an extension to the Kopi Java Compiler, a component of the
open-source Kopi Project \cite{kopi}.  At this time, our compiler is a
proof-of-concept and does not yet include the stream-specific
optimizations that we are working on; we generate C code that is
compiled with a StreamIt runtime library to produce the final
executable.  We have also developed a library in Java that allows
StreamIt code to be executed as pure Java, thereby providing a
verification mechanism for the output of the compiler.

The compilation process for streaming programs contains many novel
aspects because the basic unit of computation is a stream rather than a
procedure.  In order to compile stream modules separately, we have
developed a runtime interface--analogous to that of a procedure call for
traditional codes--that specifies how one can interact with a black box
of streaming computation.  The stream interface contains separate phases
for initialization and steady-state execution; in the execution phase,
the interface includes a contract for input items, output items, and
possible message production and consumption.

Though we have yet to add optimizations to our compiler, it is
nonetheless interesting to evaluate its baseline performance.  For this
purpose, we developed StreamIt implementations of four applications: 1)
A GSM Decoder, which takes GSM-encoded parameters as inputs, and uses
these to synthesize audible speech\cite{gsm}, 2) A system from the
Polymorphic Computing Architecture (PCA) \cite{pca} which encapsulates
the core functionality of modern radar, sonar, and communications signal
processors, 3) A software-based FM Radio with equalizer, and 4) A
performance test from the SpectrumWare system that implements an
Orthogonal Frequency Division Multiplexor (OFDM)
\cite{spectrumware}.  Table \ref{tab:benchmarks} gives characteristics
of the above applications including the number of filters implemented
and the size of the stream graph as coded.

Table~\ref{tab:performance} gives the performance of our compiler by
comparing the StreamIt implementation against either the SpectrumWare
implementation or (in the case of GSM) a hand-optimized C version.
SpectrumWare \cite{spectrumware}~is a high-performance runtime library
for streaming programs, implemented in C++.  The StreamIt language
offers a higher level of abstraction than SpectrumWare (see Section
\ref{sec:oo-rat}), and yet the StreamIt compiler is able to beat the
SpectrumWare performance by a factor of two for the PCA Demo and FM
Radio.

\begin{table}[t]
\begin{minipage}{2.05in}
\centering
\scriptsize
\begin{tabular}{|l|r|r|r|} \hline
Benchmark & Lines & Filters & Graph Size\\
\hline \hline
PCA Demo & 484 & 5 & 7\\
\hline
FM Radio & 411 & 5 & 27\\
\hline
perftest4 & 347 & 5 & 20\\
\hline
GSM Decoder & 3050 & 11 & 21 \\
\hline
\end{tabular} \\
\vspace{6pt}
\caption{\protect\small Application Characteristics
\label{tab:benchmarks}}
\vspace{-24pt}
\end{minipage}
\hspace{0.2in}
\begin{minipage}{2.5in}
\centering
\scriptsize
\begin{tabular}{|l|r|r|r|} \hline
%& \multicolumn{2}{|c|}{StreamIt} &  \multicolumn{2}{|c|}{Hand Coded}\\
%\hline 
Benchmark & StreamIt & SpectrumWare & C \\
\hline \hline
PCA Demo & 1.3 & 3.4 & N/A\\
\hline
FM Radio & 4.9 & 9.9 & N/A\\
\hline
perftest4 & 330 & 330 & N/A\\
\hline
GSM Decoder & 4.88 & N/A & .47\\
\hline
\end{tabular} \\
\vspace{6pt}
\caption{\protect\small \mbox{Performance Results (in $\mu$sec/item)}
\label{tab:performance}}
\vspace{-24pt}
\end{minipage}
\end{table}

%% \subsection{GSM Decoder}

%%   The decoder portion of the StreamIt GSM Vocoder takes GSM encoded
%% parameters as inputs, and uses these to synthesize audible speech.  This
%% is accomplished by processing the parameters through four main filters.
%% The RPE decoder filter produces some "pink noise" that very loosely
%% estimates the speech waveform, using quantized bit sequences and a
%% maximum value parameter from the encoded input.  This "pink noise" is
%% fed to the Long Term Prediction portion, which applies long-term
%% characteristics to the sequence through a delay filter within a feedback
%% loop.  The resulting signal is then sent to the Short Term Synthesis
%% filter, which decodes high frequency voice characteristics from the
%% encoded parameters and applies these to the signal.  Finally, the
%% Post-processing filter identifies peaks in the signal to make it audible.

%% \subsection{PCA Demo}

%% This application is representative of the core functionality needed by
%% modern radar, sonar, and communications signal processors.

For the GSM application, the extensively hand-optimized C version
incorporates many transformations that rely on a high-level knowledge of
the algorithm, and StreamIt performs an order of magnitude slower.
However, this version of the compiler is only a prototype, and is not
yet intended to compete with hand-coded C.  Our code generation strategy
currently has many inefficiencies, and in the future we plan to generate
optimized assembly code by interfacing with a code generator.  We
believe that stream-conscious optimizations can improve the performance
by an order of magnitude on uniprocessors; moreover, we have yet to
consider parallel targets, and this is where we expect to find the most
pronounced benefits of the abundant parallelism and regular
communication patterns exposed by StreamIt.
