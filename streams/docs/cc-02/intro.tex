\section{Introduction}

Applications that are structured around some notion of a ``stream''
are becoming increasingly important and widespread.  There is evidence
that streaming media applications are already consuming most of the
cycles on consumer machines \cite{Rix98}, and their use is continuing
to grow.  In the embedded domain, applications for hand-held
computers, cell phones, and DSP's are centered around a stream of
voice or video data.  The stream abstraction is also fundamental to
high-performance applications such as intelligent software routers,
cell phone base stations, and HDTV editing consoles.

Despite the prevalence of these applications, there is surprisingly
little language and compiler support for practical, large-scale stream
programming.  Of course, the notion of a stream as a programming
abstraction has been around for decades \cite{SICP}, and a number of
special-purpose stream languages have been designed (see
\cite{survey97} for a review).  Many of these languages and
representations are elegant and theoretically sound, but they often
lack features and are too inflexible to support straightforward
development of modern stream applications, or their implementations
are too inefficient to use in practice.  Consequently, most
programmers turn to general-purpose languages such as C or C++ to
implement stream programs.

There are two reasons that general-purpose languages are inadequate for
stream programming.  Firstly, they are a mismatch for the application
domain.  That is, they do not provide a natural or intuitive
representation of streams, thereby having a negative effect on
readability, robustness, and programmer productivity.  Moreover, because
the widespread parallelism and regular communication patterns of data
streams are left implicit in general-purpose languages, compilers are
not stream-conscious and do not perform stream-specific optimizations.
As a result, performance-critical loops are often hand-coded in a
low-level assembly language and must be re-implemented for each target
architecture.  This practice is labor-intensive, error-prone, and very
costly.

Secondly, general-purpose languages are a mismatch for the emerging
class of grid-based architectures \cite{smartmemories,rawshort,trips} that
are especially well-suited for stream processing.  Perhaps the primary
appeal of C is that it provides a ``common machine language'' for
von-Neumann architectures.  That is, it abstracts away the
idiosyncratic differences between machines, but encapsulates their
common properties: a single program counter, arithmetic operations,
and a monolithic memory.  However, for grid-based architectures, the
von-Neumann model no longer holds, as there are multiple instruction
streams and distributed memory banks.  Thus, C no longer serves as a
common machine language--in fact, it provides the wrong abstraction
for the underlying hardware, and architecture-specific directives are
often needed to obtain reasonable performance.  Again, this greatly
complicates the job of the programmer and hampers portability.

StreamIt is a language and compiler specifically designed for modern
stream programming.  The StreamIt language has two goals: first, to
provide high-level stream abstractions that improve programmer
productivity and program robustness within the streaming domain, and
second, to serve as a common machine language for grid-based
processors.  At the same time, the StreamIt compiler aims to perform
stream-specific optimizations to achieve the performance of an expert
programmer.

This paper motivates, describes, and justifies the high-level language
features of StreamIt, version 1.0.  The major limitation of StreamIt
1.0 is that all flow rates in the streams must be static; applications
such as compression that have dynamically varying flow rates will be
the subject of future work.  A large set of applications can be
implemented with static rates, and while dynamic rates will require a
different runtime model, it will still be essential to fully analyse
and optimize static sub-sections in order to obtain high performance.

The paper is organized as follows. In Section {\ref{sec:domain}}, we
characterize the domain of streaming programs that motivates the
design of StreamIt, and in Section~\ref{sec:overview} we describe the
language features in detail.  We present an in-depth example of a
software radio in Section~\ref{sec:example}, preliminary results in
Section~\ref{sec:results}, related work in Section~\ref{sec:related},
and conclusions in Section~\ref{sec:conc}.

