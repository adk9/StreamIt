\section{Partitioning}

StreamIt provides the Filter construct as the basic abstract unit of
autonomous stream computation.  The programmer should decide the
boundaries of each Filter according to what is most natural for the
algorithm under consideration.  While one could envision each Filter
running on a separate machine in a parallel system, StreamIt hides the
granularity of the target machine from the programmer.  Thus, it is
the responsibility of the compiler to adapt the granularity of the
stream graph for efficient execution on a particular architecture.

We use the word {\it partitioning} to refer to the process of dividing
a stream program into a set of balanced computation units.  Given a
number $N$, which represents the maximum number of computation units
that can be supported, the partitioning stage transforms a stream
graph into a set of no more than $N$ filters, each of which performs
approximately the same amount of work during the execution of the
program.  Following this stage, each filter can be run on a separate
machine to obtain a load-balanced executable.  

Load balancing is particularly important in the streaming domain,
since the throughput of a stream graph is equal to the {\it minimum}
throughput of each of its stages.  This is in contrast to scientific
programs, which often contain a number of stages which process a given
data set; the running time is the {\it sum} of the running times of
the phases, such that a high-performance, parallel phase can partially
compensate for an inefficient phase.  However, in a stream program,
the most burdened node is the bottleneck for the entire computation,
and the other nodes sit idle while the low-throughput node is working.

\subsection{Overview}

\begin{figure}[t]
\vspace{-6pt}
\begin{minipage}{3.1in}
\psfig{figure=orig-blood.eps,width=3in}
\vspace{-12pt}
\caption{\protect\small Blocking diagram for the naive partitioning of
 FM Radio.
\protect\label{fig:fmblood1}}
\end{minipage}
\hspace{0.3in}
\vspace{-12pt}
\begin{minipage}{3.1in}
\psfig{figure=adjust-2-blood.eps,width=3in}
\caption{\protect\small Blocking diagram for the optimized
partitioning of FM Radio.
\protect\label{fig:fmblood2}}
\end{minipage}
\vspace{-6pt}
\end{figure}

Our partitioner employs a set of fusion, fission, and reordering
transformations to incrementally adjust the stream graph to the
desired granularity.  To achieve load balancing, the compiler
estimates the number of instructions that are executed by each filter
in one steady-state cycle of the entire program; then, computationally
intensive filters can be split, and less demanding filters can be
fused.  Currently, the decision of which transformations to apply is
done by hand, but the transformations themselves are fully automated.

We return to our software radio example to illustrate the partitioning
process.  Figure~\ref{fig:fmblood1} illustrates a blocking diagram for
the original partitioning of the radio (which appears in
Figure~\ref{fig:??}).  The dark bands indicate where a processor is
waiting to send or receive an item, while the light areas indicate
periods of useful work.  With the filters partitioned in the original
configuration, most of the processors spend their time waiting for the
input from the first LowPassFilter.

Figure~\ref{fig:??} illustrates a sequence of transformations that
improve the load balancing for the radio, resulting in the blocking
diagram shown in Figure~\ref{fig:fmblood2}.  \todo{Describe the
transformations} In the following sections, we describe these
transformations in more detail.

% note that sometimes you have to fuse and then later fizz, if you
% want partial multiplication of a given work function

\subsection{Fusion Transformations}

\begin{verbatim}

  - motivation
    - decreases granularity of stream graph to fit on a given target
    - opens up opportunities for further optimization
      . comparably to loop fusion
      . like keeping working set in cache the whole way through a set of
        filters, even though each stage would overflow cache (or registers)
	if run to completion as a unit.

  - examples
    . pipeline fusion algorithm
      - could use Filter Fusion ala Proebsting
        . but code blowup possible
      - we use simple hierarchical scheduling
    . split-join fusion algorithm
      - for taking something with explicit parallelism and start/end
        synchronization and sequentializing it
      - determining the order of interleaving will have effect on
        memory requirements of fused region.

  - implementation
    - two-stage filters ??
    - involves some simulation of the scheduling, then serializing
      code to match

\end{verbatim}

\subsection{Fission Transformations}

\begin{verbatim}

  - motivation
    - fission not typically considered as an ``optimization'', but
           it is the analog of parallelization for the stream domain
      . can split the heaviest filters to distribute the load

  - examples
    . stateless filter into splitjoin
    . stateless filter into pipeline
       - communicate partial results across the pipe
       - can be stateful if all state is written on top and read on bottom
    . feedback loop into splitjoin (?? is there a case this really helps?)
       - if all components of the loop are stateless, then can extract
         parallel streams
       - ?? need to define real transformation here
    . a stateless feedback loop can be ``unrolled'' into a splitjoin with
       multiple instances of the the feedbackloop on different parts of
       the splitjoin.  how realistic are stateless feedback loops?  well,
       you could apply this transformation to do stateful fission.  just
       put all the state in the loop, with delay one, then unroll... and
       you ... maybe this doesn't work after all.

  -> using induction variable analysis to eliminate some state

  - implementation
    . stateless duplication algorithm

\end{verbatim}

\subsection{Reordering Transformations}

\begin{verbatim}

- re-arrangement ('hoisting'? 'simplification'? 'refactoring'?)

  - can sometimes change granularity of graph by re-arranging nodes
     . eg to remove duplicate / redundant operations

  - examples
     . pushing a stateless filter through a round-robin and adjusting weights
     . making a big split-join into a hierarchical split-join, or vice versa
     . synchronization removal at split-joins

  - can eliminate pipelines that are used only for packaging
     . pipeline that contains a pipeline
     . pipeline that contains only one filter

\end{verbatim}



