\section{Introduction}

This is the intro.  I will cite some people \cite{pca}.

---

- as wire delay comes to dominate the performance of architectures, a
	number of machines have software-exposed communication between
	replicated processing units (imagine, raw, trips, grid,
	smartmemories).  

These processors have replicated processing units, and are
especially good for regular streaming applications.

For these architectures to gain wide-scale acceptance, there needs to
be a high-level portable language that can be efficiently compiled to
differing targets and differing granularities.

- we present streamit as the language, and demonstrate a compiler that
	maps to differing granularities of the RAW architecture, which
	is a scalable tiled architecture.  

StreamIt presents the computation as a set of communicating filters in
a hierarchical stream graph.

- streamit as language we've developed for exposing parallelism and
	communication, especially good for static stream programs

	streamit is high-level language and should be independent of
	any architecture.  the language expresses a computation that
	can be mapped to any number of communication-exposed architctures.

- suite of fission/fusion transformations to adjust the granularity of
	a stream graph to match the granularity of a target, and for
	load balancing between different components of the architecture.

- this load balancing is especially important in streaming
	computations since the overall throughput depends on the
	slowest stage; the program is only as fast as its weakest link

- then, a layout generator that maps the layout to an architecture,
	avoiding deadlock.

- then, a communication scheduler that schedules the network inteface
	between the computational elements.  relies on a simulation of
	the execution of the graph in order to know how to schedule
	the communication code.

- results - show we can effectively utilize parallel computation
	resources on the RAW architecture for a different application
	granularities and architecture granularities
