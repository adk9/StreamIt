\begin{figure}
\centering
\psfig{figure=raw-diagram.eps,width=6in}
\caption{A block diagram of the Raw architecture.
\protect\label{fig:raw-diagram}}
\end{figure}

\section{The Raw Architecture}
\label{sec:raw}
The Raw Processor is a general-purpose microprocessor being developed
in the Computer Architecture Group at The Massachusetts Institute of
Technology.

The general organization of the Raw Processor is as a chip
multiprocessor with multiple fine-grain, first-class, register mapped
communication networks \cite{raw}.  The processor contains a 2-D mesh
of identical tiles, see Figure \ref{fig:raw-diagram}.  A tile consists
of a tile processor, memory, two dynamic network routers, two static
switch crossbars and a static switch processor.  Tiles are connected
to each of their four nearest neighbors by the two sets of static
network interconnect and two sets of dynamic network interconnect.
The tile processor uses a 32-bit MIPS-like instruction set with 32K
SRAM data memory and 32K SRAM instruction memory.

The StreamIt Compiler maps the infinite, FIFO channels of the language
to Raw's static networks.  Each static network routes single-word
quantities of data (with no header) between the switch processor of
nearest neighbors.  The tile processor communicates with the switch
processor using buffered, blocking sends and receives.  The switch
processors communicate using the same mechanism. Each tile must know
in advance to whom it is sending data and from whom it is receiving
data.  It is the task of the compiler to generate the appropriate
route instructions at compiler time.  The static network allows
3-cycle nearest neighbor ALU to ALU communication latency.

The switch processor controls the static networks of the chip.  The
switch processor uses a stripped down MIPS-like instruction set
containing only moves and branches.  The switch processor has 4
registers, an 8096-instruction instruction memory, and no data
memory. Each switch instruction has a ROUTE component, executed in
parallel, that specifies the transfer of values on the static network
between the switch and its neighboring switches.  Each switch
instruction can execute multiple moves in parallel using a VLIW-like
instruction encoding for the ROUTE component.  The Raw instruction set
architecture (ISA) works together with this parallel architecture by
exposing both the computational and communication resources up to the
software.

Currently, the StreamIt Compiler generates code that executes on Raw's
cycle accurate simulator. The simulator can model Raw configurations
of up to 8 tiles per side.  During the summer of 2002, a prototype 4x4
tile Raw chip will be available.  With a target clock rate of 225MHz,
the chip will support 16 OPS/FLOPS per cycle, 3.6 GLOPS per second,
and a bisection bandwidth of 230 Gb/sec.
