\section{Communication Scheduler}
\label{sec:communic}

With the nodes of the stream graph assigned to computation nodes of
the target, the next phase of the compiler must map the communication
explicit in the stream graph to the interconnect of the target.  This
is the task of the communication scheduler.  The communication
scheduler maps the infinite FIFO abstraction of the stream channels to
the limited resources of the target.  The communication scheduler must
avoid deadlock and starvation while trying to utilize the parallelism
explicit in the stream graph.

The exact implementation of the communication scheduler is tied to the
communication model of the target.  For targets implementing an
end-to-end, infinite FIFO abstraction [are there any???], the
scheduler only needs to decide to whom to send an item and from whom
to receive an item.  As the communication model becomes more
constrained, the communication scheduler becomes more complex,
requiring analysis of the stream graph. For targets implementing a
finite, blocking near-neighbor communication model, the exact ordering
of tile execution must be specified.  Raw's communication model falls
under this category.

Due to the static nature of StreamIt, the compiler can manage the
communication resources statically.  We can create an initialization
schedule and a steady-state schedule that fully describe the execution
of the stream graph.  The schedules can give us an order for execution
of the graph if necessary.  We can generate ordering to minimize
buffer length, maximize parallelism, or minimize latency.  We can
disregard the ordering and just use the execution counts of each node
in the stream graph to decide when the job of the communication
scheduler is complete.  Thus, we can create a communication scheduler
of arbitrary detail.  If the architecture must statically orchestrate
all aspects of communication, the StreamIt language provides this
facility.

Deadlock must be carefully avoided in the communication
scheduler. Each architecture requires a different deadlock avoidance
mechanism and we will not go into a detailed explanation of deadlock
here.  In general, deadlock occurs when there is a circular dependence
on resources.  A circular dependence can surface in the stream graph
or in the routing pattern of the layout.  If the architecture does not
provide sufficient buffering, the scheduler must serialize all
potentially deadlocking dependencies.

Two StreamIt language constructs that can lead to deadlock are
FeedbackLoops and Joiners. Figure ?? describes the potential for
deadlock in a FeedbackLoop.  One potential solution is to forbid
filters of the FeedbackLoop to interleave sends and receives.  This
has the effect of serializing the FeedbackLoop. In figure ?? we
descibe the potential for deadlock introduced by Joiner nodes.  A
solution to this problem is described below.
