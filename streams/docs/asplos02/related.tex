The Transputer architecture shares many similarities with Raw.  A
Transputer system is either an array or a grid of tiles, where
neighbors are interconnected with unbuffered point-to-point links. The
programming language used for the Transputer, Occam, is a streaming
language similar to CSP.
%However, Occam programs were exposed to the
%low-level details of the machine. Each sequential computation unit was
%explicitly mapped to a single node. Since no buffering was provided,
%the users had to hand compute the schedule and perform load balancing
%and buffer management.

The Imagine architecture is specifically designed for the streaming
application domain.  It operates on streams by applying a computation
kernel to multiple data items off the stream register file.  The
compute kernels are written in Kernel-C while the applications
stitching the kernels are written in Stream-C.  Unlike StreamIt, with
Imagine the user has to manually extract the computation kernels that
fit the machine resources in order to get good steady state
performance for the execution of the kernel.  On the other hand,
StreamIt uses fission and fusion transformations to create
load-balanced computation units and filters are replicated to create
more data parallelism when needed.  Furthermore, the StreamIt compiler
is able to use global knowledge of the program for layout and
transformations at compile-time while Stream-C interprets each basic
block at runtime and performs local optimizations such as stream
register allocation in order to map the current set of stream
computations onto Imagine.
