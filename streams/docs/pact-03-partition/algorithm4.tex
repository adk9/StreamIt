// Refactors children <begin>...<start> of <sj> into their own 
// sub-splitjoin.
splitjoin lowerChildren(splitjoin sj, int begin, int end)
------------------------------------------------------------------
// make hierarchical sub-splitjoin
subChild = { sj.child[begin], ... , sj.child[end] }
if (sj.splitter is duplicate)
  subSplit = duplicate
else
  subJoin = roundrobin( sj.split.w[begin], ... , sj.split.w[end] )
endif
subJoin = roundrobin( sj.join.w[begin], ... , sj.join.w[end] )
subSJ = splitjoin( subSplit, subJoin, subChild )

// make new splitjoin to return
newChild = {sj.child[0], ... , sj.child[begin-1], subSJ,
            sj.child[end+1], ... , sj.child[sj.size-1]}
if (sj.split is duplicate)
  newSplit = duplicate
else
  newSplit = roundrobin ( sj.split.w[0], ... , sj.split.w[begin-1],
               sj.split.w[begin] + ... + sj.split.w[end],
               sj.split.w[end+1], ... , sj.split.w[sj.size-1] )
endif
newJoin = roundrobin( sj.join.w[0], ... , sj.join.w[begin-1],
            sj.join.w[begin] + ... + sj.join.w[end],
            sj.join.w[end+1], ... , sj.join.w[sj.size-1] )
return splitjoin ( newSplit, newJoin, newChild )


// Opposite trasnformation as lowerChildren.  Given a splitjoin
// <sj> and a  position <pos> of a child with aligned joiner weights
// and matching split type, raise the children of the child into <sj>.
splitjoin raiseChildren(splitjoin sj, int pos)
------------------------------------------------------------------
sj2 = sj.child[pos]
// must have "aligned rates" as follows
sumJoin = sj2.join.w[0] + ... + sj2.join.w[sj2.size-1]
if (count != sj.join.w[pos]) 
  return sj
endif
// build new new split... split types/weights must match
if (sj.split is duplicate)
  if (!sj2.split is duplicate)
    return sj
  else 
    newSplit = duplicate
  endif
else
  sumSplit = sj2.split.w[0] + ... + sj2.split.w[sj2.size-1]
  if (sumSplit!=sj.split.w[pos])
    return sj
  else
    newSplit = roundrobin( sj.split.w[0], ... , sj.split.w[pos-1], 
                 sj2.split.w[0], ... , sj2.split.w[sj2.size-1], 
                 sj.split.w[pos+1], ... , sj.split.w[sj.size-1] )
  endif
endif
// otherwise we can definately raise the child...
newChild = {  sj.child[0], ... , sj.child[pos-1], 
              sj2.child[0], ... , sj2.child[sj2.size-1], 
              sj.child[pos+1], ... , sj.child[sj.size-1] );
newJoin = roundrobin( sj.join.w[0], ... , sj.join.w[pos-1], 
            sj2.join.w[0], ... , sj2.join.w[sj2.size-1], 
            sj.join.w[pos+1], ... , sj.join.w[sj.size-1] )
return splitjoin( duplicate, newJoin, newChild )