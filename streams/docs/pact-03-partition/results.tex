\section{Results}
\label{sec:results}
\begin{table*}[th]
\begin{center}
\scriptsize
\begin{tabular}{|l|l||r||r|r|r|r||r||c||} \hline
 & & {\bf lines of} & \multicolumn{4}{|c||}{\bf \# of constructs in the program} & {\bf \# of filters in the} & {\bf Ave Throughput}\\ \cline{4-7}
{\bf Benchmark} & {\bf Description} & {\bf code} & filters & pipelines & splitjoins & feedbackloops & {\bf expanded graph} & {\bf Improvement}
\\
\hline \hline
FIR & 64 tap FIR & 
125 & 5 & 1 & 0 & 0 & 132 & 1.55x
\\ \hline
Radar & Radar array front-end\cite{pca} & 
549 & 8 & 3 & 6 & 0 & 52 & .811x
\\ \hline
Sort & 32 element Bitonic Sort & 
419 & 4 & 5 & 6 & 0 & 242 & 1.34x
\\  \hline
FFT & 64 element FFT & 
200 & 3 & 3 & 2 & 0 & 24 & .915x
\\  \hline
Filterbank & 8 channel Filterbank & 
650 & 9 & 3 & 1 & 1 & 51 & 1.77x
\\  \hline
\end{tabular}
\vspace{-6pt}
\caption{\protect\small Results}
\label{tab:benchmarks}
\end{center}
\end{table*}

Our results appear in Table 1.  We implemented and evaluated the
partitioner compared to an improved version of our greedy partitioner
(which was used in~\cite{streamit-asplos}).  The greedy partitioner
simply identifies the two nodes with the least work, and fuses them
together before proceeding.  An interesting single case comparison
appears in Figure~\ref{fig:trans}.  

One reason that the dynamic programming partitioner performs better
than greedy is that it can perform both filter fission and filter
fusion in the same partitioning operation, since it knows the extent
to which it should fuse or split things.  In fact, we believe that
this is a characteristic which separates our partitioner from general
scientific partitioners, and could hold the greatest promise in its
applications to other problems.

We have also implemented this partitioning algorithm to decide when it
is beneficial to combine linear sections of a stream graph, and/or to
them to the frequency domain~\cite{lamb03}.  The partitioner
demonstrates significant savings over a naive replacement strategy.

