  \clearpage
  \begin{figure*}[t]
\begin{center}
\begin{minipage}{6.5in}
\begin{center}
    \begin{tabular}{cc}
%      \begin{minipage}{6in}
	\psfig{figure=filterbank1.eps,width=2.5in} %dp-before
	(a) &
	\psfig{figure=filterbank2.eps,width=1.9in} %dp-during
	(b) \\
%	\psfig{figure=filterbank3.eps,width=2in} %dp-partitions
%	(c)
%      \end{minipage}
    \end{tabular}

\vspace{0.3in}

    \begin{tabular}{cc}

%      \begin{minipage}{6in}
	\psfig{figure=filterbank4.eps,width=2.5in} %dp-after
	(c) &
	\psfig{figure=filterbank5.eps,width=1.9in} %greedy-after
	(d)
%      \end{minipage}
      %    \psfig{figure=filterbank6.eps,width=2.5in} %work-before
    \end{tabular}
\end{center}
\end{minipage}
\end{center}

    \caption{While compiling FilterBank to 16 partitions, this shows
    the transitions from (a) the original program, as written by the
    programmer, to (b) the representation used by the partitioner, to
    (c) the final partitioning.  On this benchmark, the dynamic
    programming partitioner demonstrates a 176\% improvement over the
    greedy partitioner, whose final partitioning appears in
    (d). \protect\label{fig:trans}}
  \end{figure*}

  \clearpage
  \begin{figure*}[t]
    \begin{tabular}{ll}
    \psfig{figure=algorithm.eps,width=3.5in}
    &
    \psfig{figure=algorithm2.eps,width=3.5in}
    \end{tabular}
    \caption{Search phase of the partitioning algorithm.  The traceback phase appears in Figure~\ref{code:trace}.
      \protect\label{code:partition}}
  \end{figure*}

  \clearpage
  \begin{figure}[t]
    \psfig{figure=algorithm3.eps,width=3.9in}
    \caption{Traceback phase of the partitioning algorithm.
      \protect\label{code:trace}}
  \end{figure}
  
  \clearpage
  \begin{figure}[t]
    \psfig{figure=algorithm4.eps,width=3.3in}
    \caption{Vertical cut algorithms.
    \protect\label{code:vert}}
  \end{figure}
  \begin{figure}[t]
\centering
    \psfig{figure=refactor2.eps,width=2in}
    \psfig{figure=refactor1.eps,width=2in}
    \caption{Vertical cut example.
    \protect\label{ex:vert}}
  \end{figure}
  
  \clearpage
  \begin{figure}[t]
    \psfig{figure=algorithm5.eps,width=3.5in}
    \caption{Horizontal cut algorithms.
    \protect\label{code:horiz}}
  \end{figure}
  \begin{figure}[h]
\centering
\begin{minipage}{3in}
\begin{center}
\begin{tabular}{c}
    \psfig{figure=matchsync2.eps,width=2in}
\end{tabular}

\vspace{0.2in}

\begin{tabular}{c}
    \psfig{figure=matchsync1.eps,width=2in}
\end{tabular}
\end{center}
\end{minipage}
    \caption{Horizontal cut example.
      \protect\label{ex:horiz}}
  \end{figure}

\clearpage
  \begin{figure}[t]
    \psfig{figure=algorithm6.eps,width=3.5in}
    \caption{Aggressive synchronization removal.
    \protect\label{code:sync}}
  \end{figure}
  \begin{figure}[h]
\centering
\begin{minipage}{3in}
\begin{center}
\begin{tabular}{c}
    \psfig{figure=sync2.eps,width=2in}
\end{tabular}

\vspace{0.2in}

\begin{tabular}{c}
    \psfig{figure=sync1.eps,width=2in}
\end{tabular}
\end{center}
\end{minipage}
    \caption{Example of synchronization removal.
    \protect\label{ex:sync}}
  \end{figure}
