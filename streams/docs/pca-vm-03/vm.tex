\section{Streaming Virtual Machine API}

The Streaming Virtual Machine API consists of a set of streaming
computation kernels that are connected in a stream graph.
General-purpose threaded code controls graph construction and
execution.  We describe the API for stream kernels in
Section~\ref{sec:kernel} and the API for stream control in
Section~\ref{sec:processor}.

\subsection{Stream Kernel API}
\label{sec:kernel}

\subsubsection{Overview}

Each kernel is represented as a C++ object, with the following
components:

\begin{enumerate}

\item A constructor, which receives the following:

\begin{itemize}
\item The input and output streams for the kernel.
\item The architecture resource where the kernel will execute.
\item A block of memory for spilling local variables.
\item Any other kernel-specific initialization data.
\end{itemize}

\item A {\it work} function that represents the steady-state execution
step.

\item A {\it workInfo} function that describes the properties of {\tt
work} to the low-level compiler.

\item (Optional) A {\it prework} function that is called instead of
{\it work} on the first execution, as well as an associated {\it
preworkInfo} function.

\item (Optional) Data members, which represent local kernel data that
are preserved between invocations of {\it work}.

\end{enumerate}

\noindent For example, a kernel for a simple amplifier could be as follows:
{\small
\begin{verbatim}
    class AmplifierKernel : Kernel_IStreamOutOfOrder_OStreamOutOfOrder <float, float> {
      int N;

    public:
      AmplifierKernel(IStreamOutOfOrder<float> in, OStreamOutOfOrder<float> out, VM_NODE location, IOStreamRandom<byte> scratch, int _N) : 
         Kernel_IStreamOutOfOrder_OStreamOutOfOrder<float, float> (in, out, location, scratch) {
        N = _N
      }

    protected:
      boolean work(IStreamOutOfOrder<float> in, OStreamOutOfOrder<float> out) {
        out.push(in.pop() * N);
        return false;
      }

      void workInfo(IStreamOutOfOrder<float> in, OStreamOutOfOrder<float> out) {
        KernelInfo.setPop(in, 1);
        KernelInfo.setPush(out, 1);
        KernelInfo.isSIMD();
      }
    }  
\end{verbatim}}
As evident in the example, input and output streams are represented by
objects that support {\tt push} and {\tt pop} operations.  In the
following sections, we describe the API for streams
(Section~\ref{sec:kerstreams}) and the API for kernels
(Section~\ref{sec:kernels}).

\newcommand{\lefttab}[1]{\begin{minipage}{0.6in}\begin{center}#1\end{center}\end{minipage}}
\begin{figure}[t]
\hspace{-24pt}
\begin{minipage}{6in}
\begin{tabular}{c|c|c|c}
~ & {\small Input} & {\small Output} & {\small Input and Output}
\\ \hline
\lefttab{{\small Unordered \\ Stream}}
&
\begin{minipage}{1.95in}
  \scriptsize
  \begin{verbatim}

template <class T>
class IStreamUnordered :
      public BaseStream {
public:
  IStreamUnordered(int capacity,
          VM_NODE_TYPE_MEM loc,
          int addr,
          int initLength=0,
          boolean aliased=true,
          boolean wraps=true);
  IStreamUnordered(int capacity,
          VM_NODE_TYPE_PROC loc,
          int addr,
          int initLength=0,
          boolean aliased=true,
          boolean wraps=true);

  T pop();
  boolean canPop(int n);
}
  \end{verbatim}
\end{minipage}
&
\begin{minipage}{2.0in}
  \scriptsize
  \begin{verbatim}

template <class T>
class OStreamUnordered :
      public BaseStream {
public:
  OStreamUnordered(int capacity,
          VM_NODE_TYPE_MEM loc,
          int addr,
          int initLength=0,
          boolean aliased=true,
          boolean wraps=true);
  OStreamUnordered(int capacity,
          VM_NODE_TYPE_PROC loc,
          int addr,
          int initLength=0,
          boolean aliased=true,
          boolean wraps=true);

  void push(T& val);
  boolean canPush(int n);
}
  \end{verbatim}
\end{minipage}
&
\begin{minipage}{1.8in}
  \scriptsize
  \begin{verbatim}

template <class T>
class StreamUnordered : 
      public IStreamUnordered<T>, 
      public OStreamUnordered<T> {
public:
  StreamUnordered(int capacity,
          VM_NODE_TYPE_MEM loc,
          int addr,
          int initLength=0,
          boolean aliased=true,
          boolean wraps=true);
  StreamUnordered(int capacity,
          VM_NODE_TYPE_PROC loc,
          int addr,
          int initLength=0,
          boolean aliased=true,
          boolean wraps=true);
}
  \end{verbatim}
\end{minipage}
\\ \hline
\lefttab{{\small Ordered \\ Stream}}
&
\begin{minipage}{1.95in}
  \scriptsize
  \begin{verbatim}

template <class T>
class IStream : 
      public IStreamUnordered<T> {
public:
  IStream(int capacity,
          VM_NODE_TYPE_MEM loc,
          int addr,
          int initLength=0,
          boolean aliased=true,
          boolean wraps=true);
  IStream(int capacity,
          VM_NODE_TYPE_PROC loc,
          int addr,
          int initLength=0,
          boolean aliased=true,
          boolean wraps=true);

  T peek(int index);
}
  \end{verbatim}
\end{minipage}
&
\begin{minipage}{2.0in}
  \scriptsize
  \begin{verbatim}

template <class T>
class OStream : 
      public OStreamUnordered<T> {
public:
  OStream(int capacity,
          VM_NODE_TYPE_MEM loc,
          int addr,
          int initLength=0,
          boolean aliased=true,
          boolean wraps=true);
  OStream(int capacity,
          VM_NODE_TYPE_PROC loc,
          int addr,
          int initLength=0,
          boolean aliased=true,
          boolean wraps=true);
}
  \end{verbatim}
\end{minipage}
&
\begin{minipage}{1.8in}
  \scriptsize
  \begin{verbatim}

template <class T>
class Stream : 
      public IStream<T>, 
      public OStream<T> {
public:
  Stream(int capacity,
         VM_NODE_TYPE_MEM loc,
         int addr,
         int initLength=0,
         boolean aliased=true,
         boolean wraps=true);
  Stream(int capacity,
         VM_NODE_TYPE_PROC loc,
         int addr,
         int initLength=0,
         boolean aliased=true,
         boolean wraps=true);
}
  \end{verbatim}
\end{minipage}
\\ \hline
\lefttab{\small Block}
&
\begin{minipage}{1.95in}
  \scriptsize
  \begin{verbatim}

template <class T>
class IBlock : 
      public BaseBlock {
public:
  IBlock(int capacity, 
         VM_NODE_TYPE_MEM loc, 
         int addr);
  IBlock(int capacity, 
         VM_NODE_TYPE_PROC loc);

  T read(int index);
}
  \end{verbatim}
\end{minipage}
&
\begin{minipage}{2.0in}
  \scriptsize
  \begin{verbatim}

template <class T>
class OBlock :
      public BaseBlock {
public:
  OBlock(int capacity, 
         VM_NODE_TYPE_MEM loc, 
         int addr);
  OBlock(int capacity, 
         VM_NODE_TYPE_PROC loc);

  void write(int index, T& val);
}
  \end{verbatim}
\end{minipage}
&
\begin{minipage}{1.8in}
  \scriptsize
  \begin{verbatim}

template <class T>
class Block : 
      public IBlock<T>, 
      public OBlock<T> {
public:
  Block(int capacity, 
        VM_NODE_TYPE_MEM loc, 
        int addr);
  Block(int capacity, 
        VM_NODE_TYPE_PROC loc);
}
  \end{verbatim}
\end{minipage}
\end{tabular}
\end{minipage}
\caption{Partial class declarations for streams and blocks.  Each stream and block also contains constructors with the same signature as those in {\tt BaseStream} and {\tt BaseBlock}, respectively, but these are ommitted for the sake of brevity.\protect\label{fig:declgrid}}
\vspace{-6pt}
\end{figure}


\begin{figure}[t]
\begin{center}
\psfig{figure=inherit.eps,width=5in}
\end{center}
\vspace{-12pt}
\caption{Class hierarchy diagram for stream objects.\protect\label{fig:inherit}}
\end{figure}

\subsubsection{Stream Objects}
\label{sec:kerstreams}

A {\it stream} is a C++ data type that represents a sequence of items
of a given type.  The kernel API supports several kinds of streams,
which vary in two respects: 1) whether the stream is used for input,
output, or both, and 2) whether the stream supports in order, out of
order, or random access to its elements.  All combinations of these
attributes represent a valid kind of stream, and thus there are nine
stream types (see Figure~\ref{fig:declgrid}).  For the sake of
clarifying the specification and the implementation of the runtime
system, this proposal uses inheritance to describe these
types\footnote{However, nothing in the specification relies on the
inheritance relationships for correctness, so these could be expanded
into nine stand-alone types if desired.}(see
Figure~\ref{fig:inherit}).

The rest of this section gives details for each of the functions
supported by a stream.  It refers to the {\tt StreamBuffer} class,
which is described in Section~\ref{sec:procstreams} as a structure
that is allocated in the stream control code in order to store the
items that appear in streams.  The pseudocode below gives a reference
implementation of each stream method, assuming that each {\tt
StreamBuffer} {\it s} is implemented as a circular buffer with an
array {\it data} of length {\it capacity}, a {\it start} pointer, and
an {\it end} pointer.

\ssss{Constructors} Each stream provides a constructor with a single
argument of type {\tt StreamBuffer}.  The constructor saves this {\tt
StreamBuffer} in a field {\tt s} for other functions to use as the
interface to the storage space for the stream.

\ssss{push} For output streams, the {\tt push} method enqueues a value
onto the end of the stream.  If the stream is full, its behavior is
undefined.

{\small
\begin{verbatim}
  template <class T>
  void OStreamOutOfOrder<T> :: push(T& val) {
    s->data[s->end++] = val;
    if (s->end == s->capacity) {
      s->end = 0;
    }
    // keep count of items pushed for use by StreamBuffer.getTotalLength()
    s->totalLength++;
  }
\end{verbatim}}

\ssss{pop} For input streams, the {\tt pop} method dequeues a value
from the front of the stream.  If the stream is empty, its behavior is
undefined.

{\small
\begin{verbatim}
  template <class T>
  T IStreamOutOfOrder<T> :: pop() {
    T result = s->data[s->start++];
    if (s->start == s->capacity) {
      start = 0;
    }
    return result;
  }
\end{verbatim}}

\ssss{canPush} For output streams, the {\tt canPush} method returns
whether or not it will be possible to push $n$ additional items onto
the output tape.  Note that for this could return {\tt true} even if
there are currently less than $n$ free spaces in the buffer, since
further execution of the stream graph could consume some items.  The
function must take this into account, and only return {\tt false} if
it will never be possible to push $n$ items at any point in the
future.

Note that {\tt canPush} is allowed to cause a change in the number of
free spaces in the stream.  In the reference implementation, we drain
the stream graph as much as possible in the event that fewer than $n$
items are currently available.

{\small
\begin{verbatim}
  template <class T>
  boolean OStreamOutOfOrder<T> :: canPush(int n) {
    if (s->getCapacity() - s->getLength() >= n) {
      return true;
    } else {
      if (graphCanDrain()) {
        waitForDrain();
        return (s->getCapacity() - s->getLength() >= n);
      } else {
        return false;
      }
    }
  }
\end{verbatim}}

\ssss{canPop} For input streams, the {\tt canPop} method returns
whether or not it will be possible to pop $n$ additional items from
the input tape.  Just as with {\tt canPush} above, {\tt canPop}
returns false only if it will never be possible to pop an item from
the buffer at some point in the future.

{\small
\begin{verbatim}
  template <class T>
  boolean IStreamOutOfOrder<T> :: canPop(int n) {
    if (s->getLength() >= n) {
      return true;
    } else {
      if (graphCanFill()) {
        waitForFill();
        return (s->getLength() >= n);
      } else {
        return false;
      }
    }
  }
\end{verbatim}}

\ssss{peek} For ordered input streams, the {\tt peek} method returns
the element at position {\it index}, where {\it index} is zero-indexed
(such that {\tt peek(0)} gives the same value as {\tt pop()}).  If
there are fewer than $\mt{index}+1$ items in the stream, or if
$\mt{index}+1$ exceeds the capacity of the buffer, then the return
value is undefined.

{\small
\begin{verbatim}
  template <class T>
  T IStream<T> :: peek(int index) {
    int i = s->start+index;
    if (i >= s->capacity) {
      i -= s->capacity;
    }
    return s->data[i];
  }
\end{verbatim}}

\ssss{poke} For ordered output streams, the {\tt poke} method modifies
elements that have already been pushed onto the output tape.  It sets
the element at position {\it index}, where {\it index} is zero-indexed
(such that {\tt push(val1); poke(0, val2)} would overwrite {\tt val1}
with {\tt val2} on the output tape).  If there are fewer than
$\mt{index}+1$ items in the stream, or if $\mt{index}+1$ exceeds the
capacity of the buffer, then the effect is undefined.

{\small
\begin{verbatim}
  template <class T>
  void OStream<T> :: poke(int index, T& val) {
    int i = s->end-index-1;
    if (i < 0) {
      i += s->capacity;
    }
    s->data[i] = val;
  }
\end{verbatim}}

\ssss{read} For random access input streams, the {\tt read} method
returns the element at position {\it index}.  The {\it index} must be
less than the capacity of the buffer, or the result is undefined.

{\small
\begin{verbatim}
  template <class T>
  void IStreamRandom<T> :: read(int index) {
    return(s->data[index]);
  }
\end{verbatim}}

\ssss{write} For random access output streams, the {\tt write} method
sets the element at position {\it index}.  The {\it index} must be
less than the capacity of the buffer, or the result is undefined.

{\small
\begin{verbatim}
  template <class T>
  void OStreamRandom<T> :: write(int index, T& val) {
    s->data[index] = val;
  }
\end{verbatim}}

\subsubsection{Kernel Objects}
\label{sec:kernels}

Each kernel is described as a subclass of a basic kernel class, such
as the following:
{\small
\begin{verbatim}
    template <class I1, class I2, class O1>
    class Kernel_IStream_IStream_OStream {
    protected:
      // construct a kernel with the input and output streams <in1>, <in2>, and <out1> that
      // executes on <location> and uses <scratch> to spill local variables
      Kernel(IStream<I1> in1, IStream<I2> in2, OStream<O1> out1, 
             VM_NODE location, IORandomStream<byte> scratch);

      // steady-state execution step.  Returns whether or not the kernel is finished with its
      // computation (i.e., if kernel returns true, then it should not be invoked again by the
      // runtime system.)
      virtual boolean work(IStream<I1> in1, IStream<I2> in2, OStream<O1> out1);

      // (optional) execution step for first invocation.  Returns whether or not the kernel is
      // finished with its computation, as in "work".
      virtual boolean prework(IStream<I1> in1, IStream<I2> in2, OStream<O1> out1);

      // annotations for work function
      virtual void workInfo(IStream<I1> in1, IStream<I2> in2, OStream<O1> out1);

      // annotations for prework function
      virtual void preworkInfo(IStream<I1> in1, IStream<I2> in2, OStream<O1> out1);
    }  
\end{verbatim}}

The {\tt Kernel\_IStream\_IStream\_OStream} base class is named as
such because it is connected to two {\tt IStream}'s and one {\tt
OStream}.  The API includes the definition of class {\tt
Kernel\_S1\_S2\_...\_SN} for every sequence of stream types $S_1 \dots
S_N$; for a given kernel, the declared types of streams are used for
the first $N$ parameters of the constructor, and for the only
parameters of the other member functions.  Though it might be
infeasible to manually define all such base classes, the low-level
compiler can identify each instance of a {\tt Kernel} definition and
interpret it accordingly.  This mechanism is important for preserving
type information of the input and output streams.

The input and output streams are specified in the constructor.  These
streams are then made available as arguments to {\tt work} and {\tt
prework}.  This is supported automatically by the low-level compiler;
no call to {\tt work} or {\tt prework} appears in the output of the
high-level compiler.

The constructor also receives two special arguments that are signals to the low-level compiler.
The first is the {\tt location}, which indicates which processor resource should be used to
execute the kernel.  The second is a {\tt scratch} space, which is a block of memory in which
the low-level compiler can store local variables if they overflow the registers of the kernel
processor.  The scratch space is always of type {\tt IOStreamRandom<byte>}.

Within a kernel, standard {\tt public}, {\tt protected}, and {\tt
private} modifiers are used to indicate what is visible outside the
class--in particular, to the stream control code.  As described in
Section~\ref{sec:streamgraph}, the stream control code can read and
write the public fields of a kernel, and invoke the public methods of
a kernel.  Protected and private methods and data can only be accessed
from within the kernel.

\sss{Annotations}

There are several pieces of information that are available to the high-level compiler which
should be transferred to the low-level compiler in the form of annotations.  Each annotation
takes the form of a function call to the {\tt KernelInfo} class.  Calls from the {\tt workInfo}
function apply to {\tt work}, while calls from {\tt preworkInfo} apply to {\tt prework}.
{\small
\begin{verbatim}
    class KernelInfo {
    public:
      // these functions declare the push, pop, peek, and poke rate of a
      // work or prework function with respect to a given stream.
      static void setPush(BaseStream str, int push);
      static void setPop(BaseStream str, int pop);
      static void setPeek(BaseStream str, int peek);
      static void setPoke(BaseStream str, int poke);

      // these functions indicate a dynamic rate that is unknown at compile
      // time.  However, they provide an optional estimate of the average
      // runtime rate to help the low-level compiler.
      static void setDynamicPush(BaseStream str, float pushEstimate=UNKNOWN);
      static void setDynamicPop(BaseStream str, float popEstimate=UNKNOWN);
      static void setDynamicPeek(BaseStream str, float peekEstimate=UNKNOWN);
      static void setDynamicPoke(BaseStream str, float pokeEstimate=UNKNOWN);

      // indicates that a work or prework function is data-parallel 
      // and fit for SIMD execution
      static void isSIMD();
    }  
\end{verbatim}}

All arguments to annotations must be compile-time constants.  For
example, in a kernel from a MergeSort, the input rates are marked
dynamic (with an estimate of 0.5 items per stream on average) because
they depend on values from the input streams:
{\small
\begin{verbatim}
    class MergeKernel : Kernel_IStream_IStream_OStream <int, int, int> {
    public:
      MergeKernel(IStream<int> in1, IStream<int> in2, OStream<int> out, 
                  VM_NODE location, IOStreamRandom<byte> scratch) : 
                  Kernel_IStream_IStream_OStream<int, int, int>(in1, in2, out, location, scratch) {}

    protected:
      boolean work(IStream<int> in1, IStream<int> in2, OStream<int> out) {
        // if in1 is empty, draw from in2
        if (!in1.canPop(1)) {          
          out.push(in2.pop());
          return false;
        }

        // if in2 is empty, draw from in1
        if (!in2.canPop(1)) {
          out.push(in1.pop());
          return false;
        } 

        // otherwise, compare elements from in1 and in2
        if (in1.peek(0) < in2.peek(0)) {
          out.push(in1.pop());
        } else {
          out.push(in2.pop());
        }
        return false;
      }

      void workInfo(IStream<int> in1, IStream<int> in2, OStream<int> out) {
        KernelInfo.setDynamicPush(in1, 0.5);
        KernelInfo.setDynamicPop(in2, 0.5);
        KernelInfo.setPush(out, 1);
      }
    }  
\end{verbatim}}

All instances of {\tt IStream} and {\tt IStreamOutOfOrder} must be
annotated with their pop and peek rates, and all instances of {\tt
OStream} and {\tt OStreamOutOfOrder} must be annotated with their push
rates.

\sss{Kernel Restrictions}

Only a subset of C++ is supported from within a kernel; restrictions
are listed in Figure~\ref{fig:restrict}.

\begin{figure}[t]
\framebox[6.5in]{
\begin{minipage}{6in}

\begin{enumerate}

\item No pointers.

\item No dynamic memory allocation.

\item No accesses to global memory.

\item No GOTO statements (all control flow is structured).

\item No recursive functions (all function calls have inline
semantics).

\item No references to templates or objects besides the classes
described in this document.  Further, no creation or casting of
objects within kernels.

\item Each {\tt work} and {\tt prework} function must
return\footnote{Though termination is impossible to verify in user
code, the idea is that the high-level compiler should not produce code
that polls a stream property (e.g., canPush or canPop), waiting for a
change that might never occur.  Instead, it should return and wait to
be fired again.}.

\item All kernel functions (including the constructor) must receive their arguments by value
(not by reference.)  Also, the kernel constructor cannot invoke any member functions of a
stream object.

\item Supported opcodes are only the logical, arithmetic, and boolean
operations found in C (no special-purpose DSP operations).

\item Supported types include 64-bit signed and unsigned {\tt long},
32-bit {\tt float}, 32-bit signed and unsigned {\tt int}, 16-bit
signed and unsigned {\tt short}, 8-bit {\tt byte}, {\tt boolean},
arrays with a fixed (int literal) length, and {\tt struct}'s
containing members of any other type.

\end{enumerate}

\caption{Restrictions on C++ code within kernels.\protect\label{fig:restrict}}
\end{minipage}}
\end{figure}

\subsection{Stream Control API}
\label{sec:processor}

\subsubsection{Overview}
\label{sec:streamover}

A control thread in a restricted subset of C++ is used to manage stream memory and to supervise
the execution of stream kernels.  All static stream operations are represented by explicit
stream graphs, in which kernel objects are the nodes and stream buffers are the edges.  The
network model is also integrated into the graph representation: each stream buffer is annotated
with the memory bank in which it resides, and pre-defined network kernels are used to
communicate between streams in different memories.  Dynamic operations and dynamic control flow
are fully supported in the code that constructs stream graphs and transitions between them.

An example of the stream control API is as follows:
{\small
\begin{verbatim}
    // read file into memory location of s0
    int length1 = readFile("input.dat", 1024, MEM1, 0x1000);

    // declare streams with their size and the location in which they are held
    StreamBuffer<float> s0(length1, MEM1, 0x1000, length1), 
                        s1(128, SRF1, 0x200), 
                        s2(1024, SRF1, 0x280);
    StreamBuffer<byte> scratch2(32, SRF1, 0x680);

    // set up a graph to do some audio filtering and compression
    Graph compress(new Copy(s0, s1),
                   new RunLengthEncode(s1, s2, PROC1, scratch2));

    // run the graph
    compress.run();
    compress.wait();

    // if the output is still too large, run additional compression, 
    // overwriting s2 in place
    int length2 = s2.getTotalLength();
    if (length2 > SIZE_THRESHOLD) {
      StreamBuffer<float> s3(1024, SRF1, 0x280);
      StreamBuffer<byte> scratch3(32, SRF1, 0x680);
      Graph compressMore(new ZipCompress(s2, s3, PROC2, scratch3));
      compressMore.run();
      compressMore.wait();
      length2 = s3.getTotalLength();
    }

    // store the result from memory to "output.dat"
    writeFile("output.dat", length2, MEM1, 0x280);
\end{verbatim}}

The above code fragment illustrates several aspects of the stream
processor API.  In the following, Section~\ref{sec:procstreams}
describes the {\tt StreamBuffer} class; Section~\ref{sec:streamgraph}
explains the construction of graphs from kernels and stream buffers;
Section~\ref{sec:library} describes support for architectural-specific
black-box kernels, and Section~\ref{sec:predef} describes a
pre-defined set of kernels for dealing with memory and network
operations (such as the {\tt Copy} kernel above).

\subsubsection{Stream Buffers}
\label{sec:procstreams}

The control API uses the {\tt StreamBuffer} class to allocate storage
space for the contents of a stream:
{\small
\begin{verbatim}
    template<class T>
    class StreamBuffer {
    public:
      // make a stream buffer that can hold at most <capacity> elements,
      // allocated at <address> of memory node <location>, with <initLength>
      // items already in place.
      StreamBuffer(int capacity, VM_NODE_TYPE_MEM memLocation, int address, int initLength=0);
      // same as above, for stream held in registers of a processor
      StreamBuffer(int capacity, VM_NODE_TYPE_PROC procLocation, int initLength=0);

      // returns capacity of this
      int getCapacity();

      // returns number of live items in this (given reference implementation
      // of stream types, returns (end-start+capacity)%capacity).
      int getLength();

      // returns total number of items that were pushed onto this since
      // construction or last reset
      int getTotalLength();

      // clears the elements of the buffer
      void reset();

      // default copy operation, making element-wise copy of data array
      StreamBuffer(StreamBuffer& _s);

      // private data and friends included only for benefit of reference implementation (see text)
    private:
      T* data;
      int capacity;
      int start;
      int end;
      int totalLength;
      friend IStream; friend IStreamOutOfOrder; friend IStreamRandomAccess;
      friend OStream; friend OStreamOutOfOrder; friend OStreamRandomAccess;
      friend IOStream; friend IOStreamOutOfOrder; friend IOStreamRandomAccess;
    }
\end{verbatim}}

A {\tt StreamBuffer} is constructed with a {\tt capacity}, which
indicates the maximum number of items it can hold at a given time.  It
is also constructed with a {\tt location}, which indicates the memory
bank in which the buffer is held for its entire lifetime.  The {\tt
address} indicates where the buffer begins in the memory node.  Using
the second constructor, a buffer can also be assigned to a processor,
which means that it is buffered in the processor's registers.  The
{\tt initLength} parameter specifies how many items from memory it
should push into its own representation initially (thereby
incrementing {\tt length} and {\tt totalLength} by {\tt initLength}
during construction).

The {\tt getTotalLength} method returns how many items have been
pushed onto the buffer since it was last added to a graph.  In other
words, {\tt getTotalLength} returns how many elements would be in the
buffer if no storage was ever reused, and it was implemented as a
acyclic list instead of a wrap-around array.  Note that this measure
is unrelated to the capacity of the buffer.  It is also unrelated to
how many items will be written to the buffer in the future; if this
quantity is predictable, then it can be calculated as a function of
the {\tt totalLength} of other buffers.  The reference implementation
of streams in Section~\ref{sec:kerstreams} maintains a totalLength
field that could be directly returned from {\tt StreamBuffer}{\tt
.}{\tt getTotalLength()}.

\sss{Moving Data In and Out of Stream Buffers}
\label{sec:movestream}

For stream buffers to be useful, they need to be initialized with
(possibly non-streaming) data from general-purpose threaded code.
Likewise, the results of a streaming computation need to be used in
the threaded code.  Both of these transfers are done through memory,
via the interface defined by the threaded API.  In particular, device
I/O such as file handling is done using the threaded API, and then
made available to streams through memory.

To enable the threaded-streaming interface to work via memory, each
memory-based stream buffer {\tt s} makes an important guarantee on its
data layout: if {\tt s.getTotalLength()} $\le$ {\tt s.getCapacity()},
then all data items are stored consecutively from the {\tt address} of
the {\tt memLocation} with which the buffer was constructed.  If {\tt
s.getTotalLength()} $>$ {\tt s.getCapacity()}, then there must have
been some reuse of storage, and there is no contract on the data
layout.

\subsubsection{Stream Graphs}
\label{sec:streamgraph}

A stream graph represents a static unit of streaming computation.  It
supports the following interface:
{\small
\begin{verbatim}
    class Graph {
    public:
      // constants for use by graph methods
      static const UNSTARTED = 0;
      static const RUNNING = 1;
      static const SUSPENDED = 2;
      static const FINISHED = 3;

      // construct graph out of set of kernels.  Sets status to Graph.UNSTARTED.
      Graph(void* kernel1, void* kernel2, ...);

      // indicates that this graph cannot start executing until <g> has finished.
      void addDependence(Graph g);

      // non-blocking call to start or resume graph execution.  Graph execution will not begin
      // until all dependent graphs (specified with addDependence) have a status of FINISHED.
      // Graph will run "as long as possible".  Sets status to Graph.RUNNING; when the run is
      // finished, the status is modified to Graph.FINISHED.
      void run();
      
      // like run(), except graph will run until <kernel> has fired an additional <iters>
      // times, or until graph has run "as long as possible".  Each call to prework or work counts
      // as a single firing.  Sets status to Graph.RUNNING; when the run is finished, the
      // status is modified to Graph.FINISHED or Graph.SUSPENDED, depending on the outcome.
      voit run(void* kernel, int iters);

      // non-blocking call that marks the graph as having finished.  If the graph is currently
      // executing, then it will proceed until it naturally leaves the RUNNING state, at which
      // point the status of this will be set to Graph.FINISHED for all subsequent calls to
      // status() or wait().
      void markAsFinished();

      // Resets this graph for a fresh execution by clearing all counts on kernel executions
      // and calling s.reset() on any stream buffer s that is written to by the graph.  Sets
      // status to Graph.UNSTARTED.
      void reset();      

      // non-blocking call to inspect the status of graph, as maintained by other methods
      int status();

      // blocking call that is the same as status(), except that if the result is
      // Graph.RUNNING, it blocks until the status is something else.
      int wait();

    }
\end{verbatim}}

A {\tt Graph} is constructed as a set of kernels.  The connectivity of these kernels is
implicit in the stream buffers that are shared between the input of one kernel and the output
of another.  The graph does not need to be structured (as in StreamIt).  However, no two
kernels in a graph may read from (or write to) the same stream buffer.

Graphs are executed using the {\tt run} methods.  The no-argument {\tt run} method runs each
kernel ``as long as possible''.  The second {\tt run} method includes an explicit iteration
count, but even in this case it might not be possible to complete the requested number of
iterations.  The specification requires that a kernel stops running (it is {\it finished}) when
one of these conditions is met:
\begin{enumerate}

\item The kernel returns {\tt true} from {\tt prework} or {\tt work}.

\item The kernel's downstream neighbors have finished.  That is, a
kernel finishes if all of its output streams are being read by other
kernels in the same graph, and all of those kernels (excluding itself,
in the case of a self-loop) have finished.

\item The kernel declares a static peek rate of {\tt e} for an input
stream {\tt s}, but there are fewer than {\tt e} items left in {\tt
s}.  That is, {\tt s.canPop(e)} is false.

\item The kernel declares a static push rate of $u$ for an output
stream {\tt s}, but {\tt s} does not have space for an additional $u$
items.  That is, {\tt s.canPush(u)} is false.

\end{enumerate}
Once a kernel finishes, each of its output streams ends with the last item (if any) that the
kernel pushed onto it.  A graph is finished executing only when all of its component kernels
have finished.

Note that the execution of graphs is formulated without reference to
an end-of-stream (EOS) token.  Indeed, EOS tokens are below the level
of the streaming API.  Control code should abstractly know the length
of a stream or (if it is being produced/received) the number of
elements available and if more are expected.  There are multiple ways
to implement this abstraction for different kinds of hardware.

\sss{Transitioning Between Graphs}

A {\tt Graph} represents only the static sections of the stream graph.
As in the example of Section~\ref{sec:streamover}, dynamic control
flow can surround the construction of graphs and can predicate their
execution.  

There are two ways to transfer the outputs of one stream graph to the
inputs of another.  Perhaps the most natural way is to reuse the same
stream buffer in both graphs, thereby carrying over the results; in
our example, stream buffer {\tt s2} is used in multiple graphs.  The
other way to transfer items is by allocating a new input stream that
overlaps with the output stream in memory.  This could be desirable
if, for example, a single output stream is being split between
multiple input streams, all of which are in the local memory of some
processor.  To support this case, the {\tt Graph.addDependence} method
can be used to ensure that subsequent graphs will not start to execute
until previous graphs have finished executing.  This method saves the
low-level compiler from doing location-based memory analysis to
discover dependences between graphs.

It is also possible for the control processor to inspect and modify
the public fields of a kernel, and to invoke public methods on the
kernel.  However, these accesses (as well as the {\tt
StreamBuffer}.{\tt length()} and {\tt StreamBuffer}.{\tt
getTotalLength()} functions) have undefined behavior while the graph
containing the kernel is running; rather, they can only be done when
the status of the graph is {\tt UNSTARTED}, {\tt SUSPENDED}, or {\tt
FINISHED}.  The {\tt run(void*, int)} method allows the control
processor to repeatedly run the graph for a limited number of
iterations and poll a value of interest.

Accesses to kernel fields are useful for passing parameters to
subsequent stream graphs, or for retrieving a reduction value from
inside a kernel.  For example: 
{\small
\begin{verbatim}
    // --- kernel code ---
    class SumKernel : Kernel_IStreamOutOfOrder<int> {
    public:
      int sum;

      SumKernel(IStreamOutOfOrder<int> in, VM_NODE location, IOStreamRandom<byte> scratch) : 
                Kernel_IStreamOutOfOrder<int> (in, location, scratch) {
        sum = 0;
      }

    protected:
      boolean work(IStreamOutOfOrder<int> in) {
        sum += in.pop();
        return false;
      }

      void workInfo(IStreamOutOfOrder<int> in) {
        KernelInfo.setPop(in, 1);
      }
    }

    // --- control code ---
    StreamBuffer<int> s1(128, SRF, 0x100);
    StreamBuffer<byte> scratch1(16, SRF, 0x180);

    SumKernel sk(s1, PROC1, scratch1);
    Graph g(sk);
    g.run();
    g.wait();

    int finalSum = sk.sum;
\end{verbatim}}

\sss{Annotations}

Just as {\tt KernelInfo} provides annotations for stream kernels, there is information about
graphs that can be conveyed from the high-level compiler to the low-level compiler.  These
annotations take the form of function calls to the {\tt GraphInfo} class from within the stream
control code:
{\small
\begin{verbatim}
    class GraphInfo {
      public:
        // gives a hint to the low-level compiler that it should loading the instructions for
        //  <kernel> at the time of this call, as it is likely to be used in the near future
        static void loadKernel(void* kernel);
    }
\end{verbatim}}

\sss{Restrictions on Stream Control}

A stream graph can be constructed and executed from general-purpose threaded code.  In order to
facilitate static analysis in the low-level compiler, there are a number of restrictions on the
statements dealing with stream buffers, kernels, and graphs.  However, there are no
restrictions on non-streaming statements, which can be finely interleaved with the stream
statements; these statements can be arbitrarily complex threaded code (although, in accordance
with the threaded API, they must adhere to C instead of C++).

The restrictions on stream control code are as follows:

\begin{enumerate}

\item In a given graph, at most one kernel can be assigned to a given processor resource.  To
meet this restriction, the high level compiler will either have to merge neighboring kernels
into one, or spread them across multiple graphs.

\item All stream buffer, kernel, and graph variables have a static single assignment in the
program.  This exposes exactly which constructor is invoked for a given variable name.

\item For each stream buffer that a kernel writes to, there must be a
connection in the architecture graph from the location of the kernel
to the location of the stream buffer.  For each stream buffer that a
kernel reads from, there must be a connection in the architecture
graph from the location of the stream buffer to the location of the
kernel.

\item If one or more graphs are executing, then any ordering of kernel
execution is valid so long as stream buffers do not overflow or underflow.

In other words, the low-level compiler can select any schedule
respecting the data dependences of the graphs; no memory analysis is
needed.  If there are location-based dependences due to overlapping
streams or parallel writes, the high-level compiler should insert
synchronization (for inter-graph dependences) or verify that other
graph dependences will ensure a consistent execution order (for
intra-graph dependences).

\item All arguments to stream constructors must be literals.

\item Processor resources must be specified by literals when they are
passed to stream, kernel, or graph functions.  That is, constants
denoting processor nodes, memory nodes, connections, and channels must
be passed to API functions directly, rather than passing the value in
a variable.

\item All arguments to {\tt Graph.addDependence} must be the identifier of
another graph.

\item {\tt StreamBuffer}'s can only be templated on types that have a
fixed size.  Variable sized stream records can be supported at the
language-level and compiled into the fixed size scheme using a data
encoding.

\item $(*)$ An instance of a kernel can appear in only one graph.
That is, on every execution path, a given kernel object is used as an
argument to at most one {\tt Graph} constructor.

\item No references to templates or objects besides the classes described in this document.

\item Any function that contains a reference to a stream, kernel, or graph object must NOT be
recursive.

\item Supported types are the same as those of stream kernels (see Figure~\ref{fig:restrict}).

\end{enumerate}

The $(*)$ items represent dynamic properties that are not statically
verifiable by the low-level compiler.  Thus, they represent a contract
that the high-level compiler must respect for the sake of
correctness\footnote{In the future, these restrictions could perhaps
be lifted to allow stream graphs to be interrupted, modified, and
resumed}.

\subsubsection{Black-Box Kernels:  Architecture Specific}
\label{sec:library}

The API supports architecture specific library routines via black-box kernels that are defined
in the metadata and then instantiated by name in the stream control code.  These kernels are
identical to user-defined kernels, except that they have no {\tt work} and {\tt prework}
functions; instead, they are recognized by the low-level compiler for the architecture and are
translated into hand-optimized library code or architecture-specific directives.  The kernels
should include {\tt workInfo} and {\tt preworkInfo}, however, so that the high-level compiler
can judge how many items will be consumed in each firing (in case it wants to specify an
iteration count for {\tt Graph.run}).

To ensure that the low-level compiler can recognize an
architecture-specific kernel, the high-level compiler guarantees that
it will not merge architecture-specific kernels with other kernels, or
rename the kernel so that it would become unrecognizable to the
low-level compiler.

\subsubsection{Black-Box Kernels:  Architecture Independent}
\label{sec:predef}

All memory management and network support for streams are integrated
into the graph model of Section~\ref{sec:streamgraph}.  This is done
using pre-defined architecture independent kernels that can connect
streams in different memories, or route streams to network channels.

For the sake of the {\tt Graph.run} method, all the kernels described
in this section can be considered to have a fine-grained work
function.  That is, in a single iteration, they produce at most one
item on a given output stream, and consume at most one item from a
given input stream.  In addition, the pre-defined kernels do not
require a location or a scratch space since they can be directly
implemented by the low-level compiler.  A location, however, can be
optionally provided for architectures where it is appropriate.

\subsubsection*{Memory Management Kernels}

\ssss{Copy} The primary use of this kernel is for copying items
between streams in different memory banks, though it can also be used
for copying between streams in a single memory.  However, it can only
copy across one connection--in the architectural graph, there must be
an edge from the location of the input stream to the location of the
output stream.
{\small
\begin{verbatim}
    template <class I1, class O1> 
    class Copy : Kernel_IStream_OStream <I1, O1> {
    public:
      Copy(IStream<I1> srcStr, OStream<O1> destStr, VM_NODE location = 0, 
           int length = ENTIRE_STREAM, int srcOffset = 0, int destOffset = 0, 
           int recordSize = 1, int srcStride = 1, int destStride = 1);
    }
\end{verbatim}}

In this declaration, {\tt recordSize} indicates the number of words in
each data record, and {\tt srcStride} and {\tt destStride} indicate
the separation between records in each stream.  The {\tt length}
indicates how many words should be transferred; by default, all
elements in the {\tt srcStr} are copied.  The {\tt srcOffset} and {\tt
destOffset} arguments can be used for shifting all accesses by an
offset in the source or destination streams.

\ssss{Scatter/Gather} The {\tt Scatter} and {\tt Gather} kernels allow
indexed accesses to memory:
{\small
\begin{verbatim}
    template <class I1, class I2, class O1>
    class Scatter : Kernel_IStream_IStream_OStream <I1, I2, O2> {
    public:
      Scatter(IStream<I1> srcStr, IStream<I2> indexStr, OStream<O1> destStr, VM_NODE location = 0,
              int srcOffset = 0, int destOffset = 0, int recordSize = 1);
    }

    template <class I1, class I2, class O1>
    class Gather : Kernel_IStream_IStream_OStream <I1, I2, O2> {
    public:
      Gather(IStream<I1> srcStr, IStream<I2> indexStr, OStream<O1> destStr, VM_NODE location = 0,
             int srcOffset = 0, int destOffset = 0, int recordSize = 1);
    }  
\end{verbatim}}

These kernels copy items from a source stream to a destination stream,
in chunks of {\tt recordSize} words.  In the {\tt Scatter} kernel,
the {\tt indexStr} indicates the positions in the output stream at
which the records should be written; in the {\tt Gather} kernel, the
{\tt indexStr} indicates the positions in the input stream at which
the records should be read.  The {\tt srcOffset} and {\tt
destOffset} arguments can be used for shifting all accesses by an
offset in the source or destination streams.

Like the {\tt Copy} kernel, these kernels assume that the
architectural graph contains an edge from the location of the input
stream to the location of the output stream.  The location of the
index stream can be on either of the two nodes.

\subsubsection*{Network Kernels}

The network kernels are for processor-processor communication.

\ssss{Send} The {\tt Send} kernel sends a stream from one processor to
another, subject to the connection protocol described below.
{\small
\begin{verbatim}
    template <class I1>
    class Send : Kernel_IStream <I1> {
    public:
      // send <srcStr> on <channel> of <connection>
      Send(IStream<I1> srcStr,  VM_EDGE connection, int channel);
    }
\end{verbatim}}

Given that the kernel is executing on processor $P$, we require that
{\tt srcStr} is located in a memory connected to $P$ (or located
within $P$ itself), and that {\tt connection} is an edge from $P$ to a
neighboring processor.

\ssss{Receive} The {\tt Receive} kernel receives a stream from a
neighboring processor, subject to the connection protocol described
below.  
{\small
\begin{verbatim}
    template <class O1>
    class Receive : Kernel_OStream <O1> {
    public:
      // receive <destStr> from <channel> of <connection>.
      Receive(OStream<O1> destStr,  VM_EDGE connection, int channel);
    }  
\end{verbatim}}

Given that the kernel is executing on processor $P$, we require that
{\tt destStr} is located in a memory connected to $P$ (or located
within $P$ itself), and that {\tt connection} is an edge into $P$ from
a a neighboring processor.

\ssss{Send/Receive Protocol} We refer to channel number $n$ of connection $c$ as the pair $(c,
n)$.  Note that $n$ is a virtual channel identifier; $n$ does not need to fall within $[0,
\mbox{VM\_PROP\_CHAN\_NUM}]$ for connection $c$.  Rather, the communication protocol will
ensure that there are less than VM\_PROP\_CHAN\_NUM active channels at a time.

The protocol maintains a queue of {\tt Send} and {\tt Receive} kernels that are waiting to
communicate across each $(c, n)$; let them be $\mt{SendQ}(c, n)$ and $\mt{ReceiveQ}(c, n)$,
respectively.  Kernels are pushed onto these queues in the same order that their containing
graphs are executed from the stream processor API.  We disallow the case where multiple kernels
in a given graph are targeting the same queue.  Thus, the order of the kernels in the queues is
well-defined\footnote{Unless there are multiple threads executing on the control processor, in
which case synchronization should be used to ensure a deterministic ordering of the
send/receive kernels across threads.}.

To open a new session of data transfer across $(c, n)$, the following conditions must be met:
\begin{enumerate}

\item Channel $n$ is {\it free} on connection $c$.  That is, no other
session is open on $(c, n)$, and $c$ has room for another active
channel.

\item $\mt{SendQ}(c, n)$ and $\mt{ReceiveQ}(c, n)$ are non-empty.

\end{enumerate}
If these conditions are satisfied, then a new session is opened between the kernels at the
front of $\mt{SendQ}(c, n)$ and $\mt{ReceiveQ}(c, n)$.  Items are transmitted across the
channel until either the {\tt Send} kernel or the {\tt Receive} kernel finishes its execution.
At this point, both kernels are caused to finish, and they are removed from the respective
queues for $(c, n)$.  The session on $(c, n)$ is finished.

Note that until a session is opened, all pending kernels are blocked.  The graphs that contain
these kernels could possibly execute other nodes, but the {\tt Send} or {\tt Receive} nodes
must wait until the channel is ready.

\subsubsection*{Example}

We consider one more example to illustrate the use of the above
kernels.  In this example, there are two processors that each contain
their own memory:

\begin{figure}[h]
\begin{center}
\psfig{figure=ex1.eps,width=2in}
\end{center}
\vspace{-12pt}
\end{figure}

The application does audio segmentation on a series of 10 input files
and plays a summary of each file on a speaker.  The first processor
does the segmentation itself, while the second processor filters the
extracted segments to provide a smooth transition between them.

{\small
\begin{verbatim}
    // --- code for PROC1 ---

    for (int i=0; i<10; i++) {
      // read file and put in memory
      int fileLength = readFile(filename[i], 10000, MEM1, 0x1000);

      // allocate overlapping stream in memory for raw data
      StreamBuffer<float> rawData(fileLength, MEM1, 0x1000, fileLength), 

      // allocate other streams
      StreamBuffer<float> segIndices(10, PROC1), sumData(10, PROC1);
      StreamBuffer<byte> scratch1(128, MEM1, 0x4000);

      Graph g(new ExtractSegments(rawData, segIndices, 
                                  PROC1, scratch1),       // make indices of summary segments
              new Gather(rawData, segIndices, sumData),   // gather summary audio in sumData
              new Send(sumData, c1, 3));                  // send over connection c1, channel 3

      g.run();                                            // run for whole length of file
      g.wait();
    }

    // --- code for PROC2 ---

    for (int i=0; i<10; i++) {
      stream<float> sumData(128, MEM2, 0x2000), smoothData(100, MEM2, 0x2080);
      stream<byte> scratch2(128, MEM2, 0x2160);

      Graph g(new Receive(sumData, c1, 3),                // receive summaries over channel
              new FIRFilter(sumData, smoothData,          // filter summaries
                            PROC2, scratch));        
              new Speaker(smoothData));                   // send to speaker

      g.run();
      g.wait();
    }   
\end{verbatim}}
In processor 1, a {\tt Gather} kernel is used to load the audio file at the indices where the
summary segments appear.  The {\tt Gather} kernel is directly connected to a {\tt Send} kernel
which sends the summary segments across virtual channel 3 of connection {\tt c1}.  Processor 2
uses a {\tt Receive} kernel to receive the summary segments before filtering them and sending
them to a speaker.  Note that there are 10 sessions of data transfer between the processors,
and the amount of data transferred during each session depends on the length of the audio file;
a session is finished when processor 1 finishes executing its stream graph.
