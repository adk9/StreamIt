\section{Multicore Streaming Layer}\label{ch:lib}

Emerging multicore architectures provides an excellent target for
streaming language compilers for a number of reasons:
\begin{itemize}
\item Individual cores are optimized for computation, often supporting
short vector operations in the form of SIMD instructions.
\item Limited memory capacity on a core is not a severe
limitation for streaming actors. In a stream program, actors typically
embody computation, are independent of each other, have extremely
local data-access patterns, and generally have small code sizes.
\item The availability of high-bandwidth and low-latency on-chip
communication network enables a large number of scheduling options
which would not be feasible for other targets, such as computing
clusters.
\end{itemize}

In a multicore setting, a streaming language compiler (or programmer)
must address the following challenges:
\begin{enumerate}
\item Generating code that explicitly manages data communication
  (e.g., DMA operations). Architectures that provide an asynchronous
  communication model also require pipelining the data transfers (e.g.,
  double-buffering) to increase efficieny and throughput.
\item For architectures with a finite local store on a core, the code,
  input and output buffers, and the state required by the computation
  must be tightly packaged to fit into the local memory. This
  consideration is akin to locality enhancing optimizations for
  architectures with caches.
\item Performing high-level optimizations and scheduling to achieve a
  balanced distribution of work among the cores, avoiding excess
  communication, transforming code to improve efficiency, and
  ultimately delivering high processing throughput.
\end{enumerate}

The purpose of the multicore streaming layer (MSL) is to abstract
\textsf{(1)} and provide facilities that simplify \textsf{(2)} and
\textsf{(3)}. The MSL frees a compiler or programmer from needing to
deal with the details of the architecture communication model,
allowing it to focus on exploring high-level optimization and
scheduling choices. The main goal of the MSL is to provide a generic
framework for controlling and dispatching computation to multicores
that simplifies scheduling operations. The low level details that are
specific to individual platforms are embedded in the MSL library
implementation, and hidden from the programmer or the compiler. As a
result, the MSL can provide a common platform for mapping streaming
computation to multicore and enhance portability.
