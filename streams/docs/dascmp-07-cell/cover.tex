% -*-latex-*-
% $Log: not supported by cvs2svn $
% Revision 1.1  2007/09/07 20:47:38  dxzhang
% Added thesis
%
% Revision 1.7  2001/02/08 18:53:16  boojum
% changed some \newpages to \cleardoublepages
%
% Revision 1.6  1999/10/21 14:49:31  boojum
% changed comment referring to documentstyle
%
% Revision 1.5  1999/10/21 14:39:04  boojum
% *** empty log message ***
%
% Revision 1.4  1997/04/18  17:54:10  othomas
% added page numbers on abstract and cover, and made 1 abstract
% page the default rather than 2.  (anne hunter tells me this
% is the new institute standard.)
%
% Revision 1.4  1997/04/18  17:54:10  othomas
% added page numbers on abstract and cover, and made 1 abstract
% page the default rather than 2.  (anne hunter tells me this
% is the new institute standard.)
%
% Revision 1.3  93/05/17  17:06:29  starflt
% Added acknowledgements section (suggested by tompalka)
% 
% Revision 1.2  92/04/22  13:13:13  epeisach
% Fixes for 1991 course 6 requirements
% Phrase "and to grant others the right to do so" has been added to 
% permission clause
% Second copy of abstract is not counted as separate pages so numbering works
% out
% 
% Revision 1.1  92/04/22  13:08:20  epeisach
\title{A Streaming Computation Framework for the Cell Processor}

\authorinfo{Xin David Zhang}
% Make the titlepage based on the above information.  If you need
% something special and can't use the standard form, you can specify
% the exact text of the titlepage yourself.  Put it in a titlepage
% environment and leave blank lines where you want vertical space.
% The spaces will be adjusted to fill the entire page.  The dotted
% lines for the signatures are made with the \signature command.
\maketitle

% The abstractpage environment sets up everything on the page except
% the text itself.  The title and other header material are put at the
% top of the page, and the supervisors are listed at the bottom.  A
% new page is begun both before and after.  Of course, an abstract may
% be more than one page itself.  If you need more control over the
% format of the page, you can use the abstract environment, which puts
% the word "Abstract" at the beginning and single spaces its text.

%% You can either \input (*not* \include) your abstract file, or you can put
%% the text of the abstract directly between the \begin{abstractpage} and
%% \end{abstractpage} commands.

% First copy: start a new page, and save the page number.

% Uncomment the next line if you do NOT want a page number on your
% abstract and acknowledgments pages.
% \pagestyle{empty}

\begin{abstract}
Due to the high data rates involved in audio, video, and signal
processing applications, it is imperative to compress the data to
decrease the amount of storage used.  Unfortunately, this implies that
any program operating on the data needs to be wrapped by a
decompression and re-compression stage.  Re-compression can incur
significant computational overhead, while decompression swamps the
application with the original volume of data.

In this paper, we present a program transformation that greatly
accelerates the processing of compressible data.  Given a program that
operates on uncompressed data, we output an equivalent program that
operates directly on the compressed format.  Our transformation
applies to stream programs, a restricted but useful class of
applications with regular communication and computation patterns.  Our
formulation is based on LZ77, a lossless compression algorithm
utilized by ZIP, and immediately applies to simpler formats such as
Apple Animation, Microsoft RLE, and Targa.

We implemented a simple subset of our techniques in the StreamIt
compiler, which emits executable plugins for two popular video editing
tools: MEncoder and Blender.  For common operations such as color
adjustment and video compositing, computing directly on compressed
data offers a speedup roughly proportional to the overall compression
ratio.  For our benchmark suite of 12 videos in Apple Animation
format, speedups range from 1.1x to 471x, with a median of 15x.

\end{abstract}

% Additional copy: start a new page, and reset the page number.  This way,
% the second copy of the abstract is not counted as separate pages.
% Uncomment the next 6 lines if you need two copies of the abstract
% page.
% \setcounter{page}{\thesavepage}
% \begin{abstractpage}
% Due to the high data rates involved in audio, video, and signal
processing applications, it is imperative to compress the data to
decrease the amount of storage used.  Unfortunately, this implies that
any program operating on the data needs to be wrapped by a
decompression and re-compression stage.  Re-compression can incur
significant computational overhead, while decompression swamps the
application with the original volume of data.

In this paper, we present a program transformation that greatly
accelerates the processing of compressible data.  Given a program that
operates on uncompressed data, we output an equivalent program that
operates directly on the compressed format.  Our transformation
applies to stream programs, a restricted but useful class of
applications with regular communication and computation patterns.  Our
formulation is based on LZ77, a lossless compression algorithm
utilized by ZIP, and immediately applies to simpler formats such as
Apple Animation, Microsoft RLE, and Targa.

We implemented a simple subset of our techniques in the StreamIt
compiler, which emits executable plugins for two popular video editing
tools: MEncoder and Blender.  For common operations such as color
adjustment and video compositing, computing directly on compressed
data offers a speedup roughly proportional to the overall compression
ratio.  For our benchmark suite of 12 videos in Apple Animation
format, speedups range from 1.1x to 471x, with a median of 15x.

% \end{abstractpage}

