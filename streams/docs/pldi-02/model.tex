%% \begin{figure}
%% \centering
%% \psfig{figure=tapes.eps,width=2.5in}
%% \caption{A filter's input and output tapes during an execution step.
%% With each step, the filter pushes two items, pops two items, and peeks
%% at three additional items.  The initial state of the input tape is
%% shown at left.  The center shows the filter with both input and output
%% tapes during the invocation of {\tt work}.  The final state of the
%% output tape is shown at right.}
%% \label{fig:tapes}
%% \end{figure}

\begin{figure}
\centering
\psfig{figure=pipeline.eps,width=1.64in}

(a) A Pipeline. \\
\vspace{6pt}
\psfig{figure=splitjoin.eps,width=2.75in}

(b) A SplitJoin. \\
\vspace{6pt}
\psfig{figure=feedback.eps,width=2.17in}

(c) A FeedbackLoop. \\
\vspace{-6pt}
\caption{StreamIt structures with labeling.}
\vspace{-12pt}
\label{fig:tapelabels}
\end{figure}

\section{Streaming Model of Computation}

In this section, we develop an abstract model of streaming computation
to serve as a basis for reasoning about program transformations and
compilation techniques within the streaming domain.  A stream graph
differs from a traditional, sequential program in that all of the
filters of the graph are implicitly running in parallel, with the
execution order constrained only by the availability of data on
channels between the filters.  Further, filters communicate only with
their immediate neighbors, thereby removing any notion of global time
or non-local dependences of one filter on another.  These properties
merit the development of a new model of computation, in which the
notions of timing, scheduling, and dependence analysis are in terms
that are relative to a given filter in the graph, instead of being
global characteristics of a program.

We will arrive at this notion of relative timing and dependence via a
{\bf stream dependence function}, \sdep, that is defined for a given
stream graph. In Section \ref{sec:notation} we provide a definition of
\sdep~along with some notation.  We then motivate the \sdep~function
in \ref{sec:informationflow} by deriving a concept of relative time
and a meaning for StreamIt's messaging system.  Only then, in Section
\ref{sec:derivemin}, do we turn to deriving expressions for the
\sdep~function itself; Sections \ref{sec:prog-verif},
\ref{sec:denotational} and \ref{sec:optimization} further employ the
function in the respective contexts of program verification,
denotational semantics, and program optimization.

\subsection{Notation}
\label{sec:notation}

We use the following notation:

\begin{itemize}

\item A {\it tape} is an infinite history of the values that have been
  pushed onto a channel between two filters. We use $I_S$ and $O_S$ to
  denote the input and output tapes of stream $S$, respectively, with
  numbering used to distinguish between multiple input or output tapes
  (see Figure \ref{fig:tapelabels}).  Finally, $n(T)$ represents the
  number of items on tape $T$ at a given point of execution.

\item We say that a filter $A$ is {\it upstream} of filter $B$ (or,
  equivalently, $B$ is {\it downstream} of $A$) if there is a directed
  path in the stream graph from $O_A$ to $I_B$.  We use this
  terminology for tapes as well as filters.

\item The number of items that are pushed, popped, and peeked by
  filter $A$ during a single execution of its work function are
  denoted by $push_A$, $pop_A$, and $peek_A$, respectively.  Note that
  $peek_A$ includes the items that are popped, such that $pop_A \le
  peek_A$.

\end{itemize}

Now, we are ready to give a definition of the stream dependence
function, \sdep:
\begin{definition}
$\mi{a}{b}(x)$ is the minimum number of items that must appear on tape
 $a$ given that there are $x$ items on tape $b$, where $b$ is
downstream of $a$.
\end{definition}
Thus, one can think of \sdep~as an inter-filter data dependence
mapping.  Though the actual data references in a stream program appear
from within the work functions, there is an aggregate dependence that
restricts a filter from firing until it has enough items on its input
tape to satisfy all its internal references (we assume that each
firing is atomic).  The \sdep~function generalizes this dependence to
answer a different question: how many items are needed on {\it
another} filter's input tape before my filter can fire?  The following
sections will provide some additional intuition as to the meaning and
applications of the \sdep~function.

\subsection{Information Flow}
\label{sec:informationflow}

Above, the \sdep~function is described in terms of data dependences.
However, we can also think of this function as defining a common
timing mechanism that asynchronous filters can use to synchronize
events.  We present this timing mechanism in terms of ``information
flow'', which we believe is a central concept of the streaming domain.

\subsubsection{Information Wavefronts}

When an item enters a stream, it carries with it some new information.
As execution progresses, this information cascades through the stream,
affecting the state of filters and the values of new data items which
are produced.  We refer to an ``information wavefront'' as the set of
filter executions that first sees the effects of a given input item.
Thus, although each filter's {\tt work} function is invoked
asynchronously without any notion of global time, two invocations of a
work function occur at the same ``information-relative time'' if they
operate on the same information wavefront.

The \sdep~function can be used to give a precise definition to an
information wavefront.  One interpretation of $y = \mi{a}{b}(x)$ is
that the item at position $y$ of tape $a$ is the the latest item on
tape $a$ to {\it affect} the item at position $x$ of tape $b$.  This
is because item $x$ on tape $b$ can be produced if and only if tape
$a$ contains at least $y$ items.  Note that this effect might be via a
control dependence rather than a data dependence--for instance, if
item $y$ needs to pass through a round robin joiner before some data
from another stream can be routed to tape $b$.  This is why we choose
``information flow'' instead of ``data flow'' to describe the timing
concept.

\subsubsection{Message Timing}
\label{sec:messagesemantics}

We can also use the \sdep~function to give a precise meaning to the
message delivery guarantees in StreamIt.  Though we cannot give the
details here due to space constraints (see \cite{streamittech} for a
careful treatment), the general idea is as follows.

A filter $A$ can send a message to filter $B$ to communicate
low-bandwidth, asynchronous data.  To send a message, there needs to
be an upstream or downstream path from $A$ to $B$ in the stream graph
(the filters need not be directly connected.)  The message statement
appears in $A$'s {\tt work} function and includes a specified latency
$n$ that indicates ``when'' the target filter $B$ should receive the
message.  The StreamIt language specification measures the latency $n$
in terms of information wavefronts: if $A$ is upstream of $B$, then
$B$ will receive the message immediately preceding the first
invocation of its own {\tt work} function which reads items that were
affected by some output of the $n$'th invocation of $A$'s {\tt work}
function.  That is, the message handler in $B$ is invoked when $B$
sees the information wavefront that $A$ sees in $n$ execution steps.

In some cases, the ability to synchronize a message with an
information wavefront can be very useful.  For instance, if the input
port of a hand-held computer detects a change in the networking
protocol, it can send a reconfiguration message to all downstream
filters with latency 0.  This guarantees that each filter will
reconfigure just in time to understand the new protocol, but will
still process previous elements in the pipeline according to the old
protocol.

\subsection{The Stream Dependence Function}
\label{sec:derivemin}

We now turn to deriving $\mi{a}{b}$ for all pairs of tapes $a$ and $b$
in a filter graph where $a$ is upstream of $b$.

\subsubsection{Filters}

Let us derive $\mi{I_A}{O_A}(x)$, which represents the time shift
across a single filter $A$.  Since the filter produces $push_A$ items
on every invocation, it must be invoked
$\left\lceil\frac{x}{push_A}\right\rceil$ to produce the $x$'th item.
On each invocation, it consumes $pop_A$ items, and peeks at an
additional $peek_A-pop_A$ items.  Thus, the total number of items that
must be present on the input is:
\begin{align}
\label{eq:minfilter}
\mi{I_A}{O_A}(x) = \left\lceil\frac{x}{push_A}\right\rceil*pop_A+(peek_A-pop_A)
\end{align}

\subsubsection{Pipelines}
\label{sec:timepipe}

Let us now derive an expression for \sdep~in the case of a pipeline.
In the base case, consider that two filters are connected, with the
output of $A$ feeding into the input of $B$ (see
Figure~\ref{fig:tapelabels}).  We are seeking $\mi{I_A}{O_B}(x)$: the
minimum number of items that must appear on tape $I_A$ given that
there are $x$ items on tape $O_B$.  Observing that a minimum of
$\mi{I_B}{O_B}(x)$ items must appear on tape $I_B$, and that $I_B$
must equal $O_A$ since the filters are connected, we see that a
minimum of $\mi{I_A}{O_A}((\mi{I_B}{O_B})(x))$ items must appear
on $I_A$.  Using $\circ$ to denote function composition, we have:
\begin{align*}
\mi{I_A}{O_B} = \mi{I_A}{O_A} \circ \mi{I_B}{O_B}
\end{align*}
By identical reasoning, this composition law holds for pipelined
streams as well as filters.  That is, a Pipeline of streams $S1 \dots
Sn$ has the following \sdep~function:
\begin{align}
\label{eq:composepipe}
\mi{S1}{Sn} &= \mi{I_{S1}}{O_{S1}} \circ \dots \circ \mi{I_{Sn}}{O_{Sn}}
\end{align}
One might be tempted to define the \sdep~function for any pair of
connected tapes as the composition of functions for the operators
connecting those tapes.  However, such a definition turns out to be
problematic for the SplitJoin and FeedbackLoop constructs, which
require a slightly different composition law for their components (as
shown below).  Instead, we can further extend our notation to include
the {\it components} of streams that are connected in a pipeline.
That is, if tapes $t_i$ and $t_j$ are contained within stream
constructs $S_i$ and $S_j$, respectively, and $S_i$ and $S_j$ belong
to a pipeline of streams $S_1 \dots S_n$, then:
\begin{align}
\label{eq:composetape}
\mi{t_i}{t_j} &= \mi{t_i}{O_{S_i}} \circ \mi{I_{S_{i+1}}}{O_{S_{i+1}}}
\circ \dots \circ \mi{I_{S_j}}{t_j}
\end{align}

\subsubsection{SplitJoins}
\label{sec:timesj}

We now derive \sdep~expressions for the components of a SplitJoin, and
for the SplitJoin construct as a whole.  We denote the $n$ output
tapes of the splitter $S$ by $O_{1,S} \dots O_{n,S}$, and the $n$
input tapes of the joiner $J$ by $I_{1,J} \dots I_{n,J}$ (see Figure
\ref{fig:tapelabels}).

{\bf Duplicate splitter.}  We consider the $i$'th output tape of an
$n$-way duplicating splitter.  Since the splitter duplicates each
input item onto each output tape, there must be at least $x$ items on
$I_S$ if there are $x$ items on $O_{i,S}$.  This yields a simple
expression for \sdep:
\begin{align*}
\mi{I_S}{O_{i,S}}(x) = x
\end{align*}

\begin{table}[t]
\begin{center}
\scriptsize
\begin{tabular}{|c|c|} \hline
$n(O_{i,S})$ & $\mi{I_S}{O_{i,S}}(n(O_{i,S}))$ \\
\hline \hline
$0$  & $\le 0$ \\ \hline
$1$  & $X + 1$ \\ \hline
$2$  & $X + 2$ \\ \hline
$\dots$ & $\dots$ \\ \hline
$w_i$ & $X + w_i$ \\ \hline
$w_i + 1$ & $X + W + 1$ \\ \hline
$w_i + 2$ & $X + W + 2$ \\ \hline
$\dots$ & $\dots$ \\ \hline
$2 * w_i$ & $X + W + w_i$ \\ \hline
$2 * w_i + 1$ & $X + 2*W + 1$ \\ \hline
$2 * w_i + 2$ & $X + 2*W + 2$ \\ \hline
$\dots$ & $\dots$ \\ \hline
\end{tabular}
\vspace{-6pt}
\caption{\protect\small SDEP values for a round robin splitter, where
$X = \sum_{j=0}^{i-1} w_j$ and $W = \sum_{j=0}^n w_j$.  A closed-form
solution appears in Equation~\ref{eq:sdepsplit}.
\label{tab:sdepsplit}}
\vspace{-12pt}
\end{center}
\end{table}

{\bf Round robin splitter.}  We consider an $n$-way splitter with
weights $w_1 \dots w_n$.  Let $X$ denote the sum of the weights before
the $i$'th output tape, and $W$ denote the sum of all the weights.
That is, for a given output tape $i$, $X = \sum_{j=0}^{i-1} w_j$ and
$W = \sum_{j=0}^n w_j$.  To derive $\mi{I_S}{O_{i,S}}$, the reader is
referred to Table \ref{tab:sdepsplit}, which lists \sdep~for a number
of representative inputs.  Generally speaking, the table illustrates
that for each set of $w_i$ items appearing on tape $O_{i,S}$, the
splitter must have executed a full round.  Additionally, $X$ items
must have gone through the splitter before the first item could appear
on $O_{i,S}$.  The reader can verify that the following expression
gives the values of \sdep~shown in the table:
\begin{align}
\protect\label{eq:sdepsplit}
&\mi{I_S}{O_{i,S}}(x) = \\ &~~~~~~X +
W*\left\lfloor\frac{n(O_{i,S})-1}{w_i}\right\rfloor + 1 +
(n(O_{i,S})-1)~mod~w_i \nonumber
\end{align}
Note that this equation does give a negative value for \sdep~in the
case when $n(O_{i,S}) = 0$.  However, this is not problematic, since
an output tape can contain zero items regardless of the number of
items on the input.  In general, \sdep~is only meaningful within the
domain of strictly positive tape values.

{\bf Round robin joiner.}  Let us consider an $n$-way joiner with
weights $w_1 \dots w_n$, with $X = \sum_{j=0}^{i-1} w_j$ and $W =
\sum_{j=0}^n w_j$ for a given input tape $i$.  The values of
$\mi{I_{i,J}}{O_J}$ for representative inputs is shown in Table
\ref{tab:sdepjoin}.  No items are required on tape $I_{i,J}$ until
there are $X+1$ items on the output of the joiner, since the first $X$
items are drawn from tapes $0$ to $i-1$.  Then, the joiner reads from
tape $I_{i,J}$ as it outputs items $X+1$ through $X+w_i$, thereby
requiring $w_i$ items on $I_{i,J}$ to produce output $X+w_i$.  The
joiner does not read from the $i$'th input tape until the next round,
when it is producing item $X + W + 1$.  From here, the cycle repeats.
The reader can verify that the following expression gives the correct
values for \sdep:
\begin{align}
\protect\label{eq:sdepjoin}
&\mi{I_{i,J}}{O_J}(x) = \\ &~~~~~~w_i *
\left\lfloor\frac{x}{W}\right\rfloor + MIN(MAX(0, x~mod~W - X), w_i) \nonumber
\end{align}

{\bf SplitJoin construct.}  As with the Pipeline construct, we can
derive the \sdep~function across an entire SplitJoin as a composition
of the component functions.  However, a SplitJoin differs from a
Pipeline in that the joiner imposes a control dependence between the
parallel streams.  That is, for there to be $x$ items on the output of
the joiner, there must be at least $\mi{I_{i,J}}{O_J}(x)$ items on {\it
every} input tape $I_{i,J}$.  Applying the composition law for pipelines
(Equation \ref{eq:composepipe}), it follows that there must be at
least at least $\mi{O_{i,S}}{I_{i,J}} \circ \mi{I_{i,J}}{O_J}(x)$ items on
every output tape $O_{i,S}$ of the splitter.  Finally, the minimum number
of items appearing on the input tape $I_S$ of the splitter is the {\it
maximum} of the item requirement from any output tape $O_{i,S}$.  By this
reasoning, the \sdep~function for a SplitJoin is as follows:
\begin{align*}
&\mi{I_S}{O_J}(x) = \\ &~~~~~\fn{MAX}_{i \in [1,n]}(\mi{I_S}{O_{i,S}} \circ
\mi{O_{i,S}}{I_{i,J}} \circ \mi{I_{i,J}}{O_J})(x)
\end{align*}

\subsubsection{FeedbackLoops}
\label{sec:timefl}

\begin{table}[t]
\begin{center}
\scriptsize
\begin{tabular}{|c|c|} \hline
$n(O_J)$ & $\mi{I_{i,J}}{O_J}(n(O_J))$ \\
\hline \hline
$0$ & $0$ \\ \hline
$\dots$ & $\dots$ \\ \hline
$X$ & $0$ \\ \hline
$X + 1$ & $1$ \\ \hline
$X + 2$ & $2$ \\ \hline
$\dots$ & $\dots$ \\ \hline
$X + w_i$ & $w_i$ \\ \hline
$X + w_i + 1$ & $w_i$ \\ \hline
$\dots$ & $\dots$ \\ \hline
$X + W$ & $w_i$ \\ \hline
$X + W + 1$ & $w_i + 1$ \\ \hline
$X + W + 2$ & $w_i + 2$ \\ \hline
$\dots$ & $\dots$ \\ \hline
$X + W + w_i$ & $2 * w_i$ \\ \hline
$X + W + w_i + 1$ & $2 * w_i$ \\ \hline
$\dots$ & $\dots$ \\ \hline
$X + 2 * W$ & $2 * w_i$ \\ \hline
$X + 2 * W + 1$ & $2 * w_i + 1$ \\ \hline
$X + 2 * W + 2$ & $2 * w_i + 2$ \\ \hline
$\dots$ & $\dots$ \\ \hline
\end{tabular}
\vspace{-6pt}
\caption{\protect\small SDEP values for a round robin joiner, where $X
= \sum_{j=0}^{i-1} w_j$ and $W = \sum_{j=0}^n w_j$.  A closed-form
solution appears in Equation~\ref{eq:sdepjoin}.
\label{tab:sdepjoin}}
\vspace{-12pt}
\end{center}
\end{table}

The \sdep~function for a feedback loop requires extra care. Although
the feedback splitter $FS$ serves as a normal splitter, with the same
\sdep~function as derived above, the feedback joiner $FJ$ is slightly
different due to the initialization phase of the loop.  Also, the
\sdep~function does not compose across all components of the loop,
since otherwise there would be conflicting definitions for paths that
circle the loop several times.

{\bf Feedback joiner}.  For a feedback loop with delay $d$, the
feedback joiner must fabricate its first $d$ input values, since no
items have yet been pushed onto the loop tape $I_{2,FJ}$.  This means
that there must be an offset of $d$ in the \sdep~function, since the
first $d$ items are direct inputs to the joiner instead of appearing
as items on the input tape.  Using $J$ to denote a round robin joiner
as considered above, we thus have the following expression for the
\sdep~function across the feedback path:
\begin{align*}
\mi{I_{2,FJ}}{O_{FJ}}(x) = \mi{I_{2,J}}{O_J}(x) - d
\end{align*}
However, the \sdep function remains unchanged with respect to the
input from the main stream:
\begin{align*}
\mi{I_{1,FJ}}{O_{FJ}}(x) = \mi{I_{1,J}}{O_J}(x)
\end{align*}

{\bf Feedback components}.  Within a feedback loop, the \sdep function
between tape $a$ and any downstream tape $b$ can be uniquely defined
by composing the \sdep functions along the directed acyclic path
between $a$ and $b$.  We require an acyclic path to avoid successive
passes around the loop, which would prevent a unique definition of the
function.  Denoting this path of tapes by $(a, t_1, \dots , t_n, b)$,
the composition follows the form of Equation \ref{eq:composepipe}:
\begin{align*}
\mi{a}{b}(x) = \mi{a}{t_1} \circ \mi{t_1}{t_2} \circ \dots \mi{t_n}{b}
\end{align*}
Note that these functions can then be composed with those of
constructs neighboring the feedback loop to obtain, for instance, the
relation between the loop tape $I_{2,FJ}$ and a downstream pipeline (by
application of Equation \ref{eq:composetape}).

{\bf Feedback loop construct.}  As a special case of the equation
above, we can see that the \sdep function for the feedback loop as a
whole is the composition of the \sdep functions along the main path:
\begin{align*}
\mi{I_{1,FJ}}{O_{FS}}(x) = \mi{I_{1,J}}{O_J}(x) \circ \mi{I_{1,J}}{O_J}(x) 
\end{align*}
Intuitively, this is because--in any semantically correct stream
program--the loop itself is guaranteed to have enough inputs to feed
the joiner, such that the output tape of the feedback loop places a
restriction only on the input tape of the feedback loop.

\subsubsection{Summary}

In the preceding sections, we have derived a \sdep function for the
components of each stream construct, as well as for the stream
construct as a whole.  By application of Equation
\ref{eq:composetape}, this yields a function $\mi{a}{b}$ for every
pair of tapes $a$ and $b$ where $b$ is downstream of $a$.

\subsection{Program Verification}
\label{sec:prog-verif}

A number of program analysis techniques are immediately afforded by
the \sdep function.  In particular, it is very simple to compute 1)
whether or not the program will deadlock as a result of a starved
input channel, and 2) whether or not any buffer will grow without
bound during the steady-state execution of the program.

{\bf Deadlock detection.}  The deadlock detection algorithm takes
advantage of the fact that the only loops in our stream graph are part
of a FeedbackLoop construct.  A stream graph will be deadlock-free if
and only if every feedback loop produces enough data to satisfy its
own feedback joiner.  This can be formulated in terms of the \sdep
function by considering $\mi{t}{t}$, the data that a tape $t$ in a
feedback loop requires of itself.  However, since we didn't define
\sdep across circular paths in the stream graph, we will denote this
function by \loopdep and define it at the loop input to the feedback
joiner:
\begin{align*}
\loopdep(x) \equiv \mi{I_{2,FJ}}{O_{FJ}} \circ \mi{O_{FJ}}{I_{2,FJ}}
\end{align*}
Now, the loop will be deadlock-free if and only if $\forall x \in
\mathcal{N}, x - \loopdep(x) > 0$.  This condition follows directly
from causality--the $x$'th item can be produced if and only if its
production depends only on some subset of the $x-1$ items that are
already on the channel.

{\bf Overflow detection.}  There are two places that a buffer can grow
to an unbounded size in the stream graph.  The first is in a feedback
loop, when\footnote{$f(x) = \omega(g(x))$ if $lim_{x \rightarrow
\infty}\frac{f(x)}{g(x)} = \infty$}~$x - \loopdep(x) = \omega(1)$.
That is, if $\loopdep(x)$ items on the feedback tape enables the
production of an additional $x - \loopdep(x)$ items that grows
asymptotically with the position $x$ on the tape, then the constant
consumption rate will not keep up with the growing production rate,
and the buffer will overflow.

The second case of buffer overflow is when the parallel streams of a
SplitJoin have asymptotically different production rates.  For a given
stream $i$ in a SplitJoin construct, the buffer corresponding to the
joiner input tape $I_{i,J}$ will overflow if and only if there is a
stream $j$ in the SplitJoin for which:
\begin{align*}
(\mi{I_S}{O_{i,S}}& \circ \mi{O_{i,S}}{I_{i,J}} \circ \mi{I_{i,J}}{O_J})(x) - \\
(\mi{I_S}{O_{j,S}}& \circ \mi{O_{j,S}}{I_{j,J}} \circ \mi{I_{j,J}}{O_J})(x) = \omega(1)
\end{align*}
Both of these cases could be detected by a compiler to verify that no
buffers will overflow during steady-state execution.

\subsection{Denotational Semantics}
\label{sec:denotational}

In this section, we develop a denotational semantics for obtaining the
meaning of an entire stream graph.  In Section \ref{sec:optimization},
this semantics is used to show that a optimizing transformation on the
stream graph preserves the meaning of the entire program.

Our denotational semantics contains three algebras: one for literal
StreamIt syntax, one for an intermediate abstract syntax, and one for
the semantic analysis.  The purpose of the intermediate algebra is to
provide a simplified syntax for developing stream transformations, and
to abstract away the StreamIt-specific aspects of the program.  We
provide an informal description of how to translate back and forth
between StreamIt programs and the abstract syntax, and then consider
more formal valuation functions for determining the meaning of the
abstract syntax within the semantic algebra.  Throughout the analysis,
we assume that filters are stateless and that the stream program is
semantically correct.

\subsubsection{Intermediate Algebra}
\label{sec:intalgebra}

The intermediate algebra provides a common mathematical representation
for manipulating stream programs.  Though we have referred to this
algebra as providing an abstract syntax for stream programs, the
representation is strictly a mathematical framework within semantic
domains rather than a program that is fit for execution.  Nonetheless,
the LISP-like syntax allows us to think of the representation as a
program that is amenable to straightforward transformation techniques.

\begin{figure}[t]
\scriptsize
\begin{align*}
&Item = \mathcal{R} \\
i~\in~&Index = \mathcal{N} \\
g~\in~&IndexTransform = Index \ra Index \\
t~\in~&Tape = Index \ra Item \\
&Pop, Peek, Push = \mathcal{N} \\ 
f~\in~&WorkStatement = IndexTransform \ra (Tape \ra Item) \\ 
&WorkFunction = WorkStatement^{+} \\
S~\in~&SplitType = \{Duplicate, RoundRobin\} \\ 
J~\in~&JoinType = \{RoundRobin\}
\end{align*}
\vspace{-18pt}
\caption{Semantic domains that are shared between the intermediate and
  transform algebras.
\protect\label{fig:shareddom}}
\vspace{3pt}
%\vspace{-6pt}
\begin{align*}
s~\in~&Stream = Filter + Pipeline + SplitJoin + FeedbackLoop \\
&Filter = Push \times Pop \times Peek \times WorkFunction \\
&Pipeline = Stream^{+} \\
&SplitJoin = SplitType \times Stream^{+} \times JoinType \\
&InitFeedback = Int^{+} \\
&BodyStream, LoopStream = Stream \\
&FeedbackLoop = JoinType \times BodyStream \times \\
& \hspace{0.5in} LoopStream \times SplitType \times InitFeedback
\end{align*}
\vspace{-18pt}
\caption{Semantic domains specific to the intermediate algebra.
\protect\label{fig:interdom}}
\vspace{3pt}
%\vspace{-6pt}
\begin{align*}
&TapeTransform = Tape \ra Tape \\
&StreamTransform =  IndexTransform \ra TapeTransform
\end{align*}
\vspace{-18pt}
\caption{Semantic domains specific to the transform algebra.
\protect\label{fig:transformdom}}
\end{figure}

The domains of the intermediate algebra are shown in
Figures~\ref{fig:shareddom}~and~\ref{fig:interdom}.  The algebra
represents a tape as an infinite mapping from indices to items.
Generally, stream constructs are represented as lists of their
component streams, and filters' work functions are encoded as lists of
push statements that--given the transform from their local indexing to
the global tape position--return a mapping from a tape to an output
item.

{\bf Converting to the intermediate algebra}.  It is straightforward
to generate an expression in the intermediate algebra that reflects
the meaning of a given StreamIt program.  Due to space limitations, we
consider here only the translation of the work functions.

The translation of a filter's work function contains two steps.
First, the function is arranged in a {\it canonical form}, in which
each pushed item is given as a direct function of the peeked items,
and all of the pop statements are at the end of the function.  Let us
consider a work function with I/O rates PUSH, POP, and PEEK.  The
canonical form of this work function gives us an element {\tt w} of
the syntactic domain {\tt StreamItWorkFunction}: \\ \\
\begin{scriptsize}
\begin{tabular}{l}
{\tt w = void work()}~\{ \\
\hspace{12pt} {\tt output.push(} ($f_{1}$ {\tt input.peek(0) ... input.peek(PEEK-1)} ) \\
\hspace{12pt} $\dots$ \\
\hspace{12pt} {\tt output.push(} ($f_{PUSH}$ {\tt input.peek(0) ... input.peek(PEEK-1)} ) \\
\hspace{12pt} {\tt for (int i=0; i$<$POP; i++) \{ input.pop(); \}} \\
\}
\end{tabular}
\end{scriptsize}
\\ \\
Above, we model the computation of the work function as pure
mathematical functions that can be injected into the semantic domain.
To simplify our notation, we define the valuation ${\cal W}: {\tt
StreamItWorkFunction} \ra WorkFunction$ in terms of {\tt w}, the example
syntactic work function from above.  The valuation, then, is the
alternate application of each push statement's function $f$, with the
index expressions transformed from their local index $x$ to a global
index $g(x)$ on the input tape:
\begin{align*}
{\cal W}[{\tt w}] =& [h_1 \dots h_{push}] \\
where~h_i =& [~\la~g~t~.~f_i(t(g(0)), \dots, t(g(\fn(PEEK-1))))]
\end{align*}

{\bf Converting from the intermediate algebra}.  To convert back to
StreamIt, we can perform the inverse of the translation shown above,
with a push statement for each function and a local index expression
$x$ in place of the global index $g(x)$.  Common sub-expression
elimination can be used to eliminate duplicate peek statements or
shared portions of the $f_i$'s.

\subsubsection{Transform Algebra}

The transform algebra is designed to express the meaning of a stream
graph as a transformation from an input tape to an output tape.  Its
semantic domains are given in Figures \ref{fig:shareddom} and
\ref{fig:transformdom}.  Our goal is to express the meaning of a
$Stream$ in the intermediate algebra as a $TapeTransform$ in the
transform algebra.  

To do this, we introduce the $StreamTransform$ domain, each element of
which maps an $IndexTransform$ (let us call it $g$) to a
$TapeTransform$.  The intuition is that $g$ represents a relative
indexing function that is imposed by an enclosing stream construct.
For instance, in a two-way SplitJoin with a RoundRobin splitter, the
SplitJoin construct imposes a $g = \la~i~.~2*i$ on the second parallel
stream component.  That is, index $i$ on the component stream's input
tape corresponds to index $g(i)$ on the input tape of the SplitJoin.
Thus, if the component stream is transforming the input tape of the
entire SplitJoin, it must apply $g$ to its original index references.

Let us denote our valuation functions as ${\cal M}: Stream \ra
TapeTransform$ and ${\cal S}: Stream \ra StreamTransform$ .  Then the
meaning of a top-level stream $s$ is as follows:
\begin{align*}
&{\cal M} [s] = {\cal S}[s]({\cal I}) \\
where~{\cal I}~den&otes~the~identity~function
\end{align*}
That is, the meaning of an entire stream program is simply the
$Stream$-$Transform$ for that program applied to the identity function
as the $Index$-$Transform$, since at the top level there is no
enclosing stream constructs and the the tape transformation is
relative to the input tape of the stream itself.  We now turn our
attention to deriving ${\cal S}$ for Filters, Pipelines, and
SplitJoins.  Letting $\%$ denote the $mod$ function, for a filter we
have that:
\begin{align*}
&{\cal S} [(\mbox{in~Filter}~push~pop~peek~(h_1~\dots~h_{push}))] =
  \la~g~t~i~.~\\ &(h_{i~\%~push})(\la~i_{local}~.~g((\mi{I_F}{O_F}(i) - peek + 1 + i_{local})))(t)
\end{align*}
That is, the value that a filter pushes onto the $i$'th position of its
output tape is calculated with its function at index $i~\%~push$.  By
the definition of \sdep, the index offset to the last value the filter
peeks is $\mi{I_F}{O_F}(i)$, where $I_F$ and $O_F$ denote the input and
output tapes of the filter (as shown in Equation \ref{eq:minfilter},
this is a pure function of $push$, $pop$, and $peek$).  Thus, the offset
to the first value the filter peeks is $\mi{I_F}{O_F}(i) - peek + 1$,
and we obtain the global index by adding this offset to the local index
$i_{local}$.

For a pipeline, the transform function is simply the composition of the
transforms of component streams.  At the internal connections of the
pipeline, the index transform is the identity function, but at the start
of the pipeline we apply the transform $g$ to interface the pipeline to
its outside connection.
\begin{align*}
{\cal S} [(\mbox{in~Pipeline}~s_1~s_2~\dots~s_n)] = \\
\la~g~.~({\cal S}[s_n]({\cal I}) \circ \dots \circ {\cal S}[s_2]({\cal I}) \circ {\cal S}[s_1](g)) \\
where~{\cal I}~denotes~the~identity~function
\end{align*}
The valuation function for a SplitJoin follows the same idea, but the
notation is slightly heavier.  Given that we have a round robin joiner
with weights $w_1 \dots w_n$ and $W = \sum{w}$, we first represent the
parallel stream $p(i)$ which computes the $i$'th output of the
joiner:
\begin{align}
\label{eq:p}
p(i) = MIN(j~s.t.~\sum_{k=0}^{j-1}{w_i} \le i~mod~W)
\end{align}
Now, the $i$'th tape position assumes the value that is produced along
stream $p(i)$ in the SplitJoin, and the value of interest appears at
position $\mi{I_{p(i),J}}{O_J}(i)$ on the output tape of stream $p(i)$.
The indexing function transforms the stream's local index $i_{local}$
for its own input tape to the corresponding index
$\mi{I_S}{O_{p(i),S}}(i_{local})$ for the input tape of the splitter:
\begin{align*}
&{\cal S} [(\mbox{in~SplitJoin}~S~s_1~s_2~\dots~s_n~J)] = \la~g~t~i~.~ \\
&(({\cal S}[s_{p(i)}])(\la~i_{local}~.~g(\mi{I_S}{O_{p(i),S}}(i_{local}))))(\mi{I_{p(i),J}}{O_J}(i))
\end{align*}
This completes our description of the transform algebra, as we have
not yet formulated the valuation function for FeedbackLoops.  Given
the valuation functions above, however, we express the meaning of any
combination of Pipelines and SplitJoins as a mathematical
transformation between infinite tapes.  We will utilize this
formulation to prove that certain transformations of the stream graph
preserve the meaning of the program.
%% $\cal{S}$ 
%%   [(FeedbackLoop (. S) $s_body$ $s_loop$ (. J) ($init_1 \dots
%%   init_n$))] = ??????