\section{Scheduling}

\begin{figure}
\centering
\psfig{figure=sched_diag.eps,width=3.0in}
\caption{The three different scheduling schemes.}
\label{fig:sched}
\end{figure}

The tradeoffs between different optimization criteria are particularly
pronounced in the scheduling stage of a streaming compiler.  As shown
in Figure \ref{fig:sched}, at the extreme ends of the optimization
space are schedules which minimize code size (at the expense of
latency and buffer size) and which minimize buffer size and latency
(at the expense of code size).  We give an overview of this scheduling
space, and present a new phased scheduling technique that takes
advantage of the structured streams in StreamIt to obtain a minimum
latency schedule without a large increase in code size.

\subsection{Initialization vs. Steady State}

Firstly, one must note that StreamIt programs can require a separate
schedule for initialization and for the steady state.  The steady
state schedule must be periodic--that is, its execution must preserve
the number of live items on each channel in the graph.  We need a
separate initialization schedule if there is a filter with $peek >
pop$, since no periodic schedule could eliminate all of the live items
on the filter's input channel (which would be needed to return the
graph to its initial configuration).  In the StreamIt compiler, this
initialization schedule is constructed via symbolic execution of the
stream graph, until each filter has $peek-pop$ items on its input
channel.

For graphs without peeking, one can find a unique and minimal set of
multiplicities for a periodic schedule, and all other periodic
schedules will be a multiple of these \cite{leesdf}.  Thus, the
challenge in scheduling is to impart an {\it order} on the steady
state execution set so that a given metric is optimized.  In what
follows, we consider three approaches to this problem.

\subsection{Minimizing Code Size}

A schedule with minimal code size is a Single Appearance Schedule
(SAS): one where each node appears exactly once in the loop nest
denoting the schedule (e.g., (4A)(6B)(9C)(3D) in Figure
\ref{fig:sched}).  There has been a lot of attention (e.g.,
\cite{bhattamurtlee97, leesdf}) on SAS's because their minimal code
size allows extensive function inlining, which enables compiler
optimizations and improves performance.  In the StreamIt compiler, we
compute a simple SAS with hierarchical ordering according to the
original stream structure.  The problem with this and other SAS's is
that the data buffer size can grow quite large, which motivates other
techniques.  Moreover, the inlining benefits afforded by SAS's are
less important in StreamIt, where the compiler itself can consider
inter-procedural optimizations.

\subsection{Minimizing Buffer Size}

On the other end of the spectrum, one can minimize buffer size by
implementing a ``pull schedule'', in which filters are executed in
demand-driven order to fire the output node of the stream.  A pull
schedule guarantees the minimal static buffer size (assuming each
filter has its own input buffer), with each channel not exceeding
$\frac{push_A + pop_B}{gcd~(push_A, pop_B)}$.  However, a pull
schedule is very irregular, and could require an exponential number of
instructions to encode.

\subsection{Minimizing Latency}

The pull schedule also minimizes the average latency of the stream,
which could be important for real-time applications.  We define the
latency of an output item as the number of work functions that were
executed within the stream before the item was output; the stream's
average latency is taken over all of its output items.  While the pull
schedule is sufficient to minimize latency, it is possible to factor
more of the schedule into shared loop nests.  For this we present the
notion of a ``phased schedule''.

\subsection{Phased Schedules}

We invented phased schedules--which rely heavily on the structured
streams of StreamIt--to achieve a minimum-latency schedule without
risking the code explosion of a pull schedule (see Figure
\ref{fig:sched}).  A phase is a (possibly non-periodic) schedule for a
stream structure in which the bottom-most filter in that structure
fires exactly once.  There could be several phases for a given stream
component, and each phase has an associated push, pop, and peek count.
In the base case, a filter has just one phase with its own push, pop,
and peek.  For stream constructs, the list of phases is determined by
simulating a ``phased pull''--that is, just like a pull schedule,
except that child streams must execute in steps of their own phases.

Due to space limitations, we cannot give a more detailed description
of the phased scheduling algorithm.  However, it is the case that
phased schedules have minimum latency because they invoke the same set
of filters as the pull model for a given output item; only the {\it
ordering} of those filter executions can be rearranged to improve the
code size.

\subsection{Respecting Message Constraints}

Another responsibility of the scheduler in StreamIt is to satisfy the
message delivery guarantees.  Each downstream message with a negative
latency imposes a lower bound on the buffer size between the source
and target filter.  Likewise, an upstream message with a positive
latency imposes an upper bound on this buffer size.  Generally
speaking, these constraints can be satisfied by adopting a
hierarchical scheduling technique in which an initial coarse-grained
schedule (such as our SAS) is incrementally refined at increasing
depths until the message constraints are satisfid.  The
refinement--which could be either a simulated pull model or a phased
schedule--can be targetted to sub-structures of the graph in which a
message constraint is violated.
