% uncomment this to use pubform
%\documentstyle[psfig, amsmath]{pubform}

\documentclass{sig-alternate}
\usepackage{gregs-defs}

\begin{document}

\title{\makebox{A Common Machine Language for Grid-Based Architectures}}

% uncomment this to use pubform
%% \author{Bill Thies, Michal Karczmarek, Michael Gordon, David Maze, Jeremy Wong, \\ Henry Hoffmann, Matthew Brown, and Saman Amarasinghe \\
%% \parbox{6in}{\centering{Laboratory For Computer Science\\
%%     Massachusetts Institute of Technology\\
%%     Cambridge, MA  02139\\
%%     \tt{\{thies, karczma, mgordon, dmaze, jnwong, hank, morris, saman\}@lcs.mit.edu}}}}
%% \date{\today}

\numberofauthors{1}
\author{
\alignauthor \vspace{-28pt}
William Thies, 
Michal Karczmarek, 
Michael Gordon, 
David Maze, 
Jeremy Wong,
Henry Hoffmann, 
Matthew Brown, 
and Saman Amarasinghe\\
	\vspace{6pt}
	{\large \{thies, karczma, mgordon, dmaze, jnwong, hank, morris,
	saman\}@lcs.mit.edu} \\ 
	\vspace{6pt}
	Laboratory for Computer Science, Massachusetts Institute of
	Technology\\ \vspace{-12pt}
%	\vspace{12pt}
%        \today
}

% for the arrow of a function def, etc.
\newcommand{\ma}[2]{max_{#1 \rightarrow #2}}
\newcommand{\mi}[2]{\textsc{sdep}_{#2 \small{\rightarrow} #1}}
\newcommand{\floor}[2]{\left\lfloor\frac{#1}{#2}\right\rfloor}
\newcommand{\ceil}[2]{\left\lceil\frac{#1}{#2}\right\rceil}
\newcommand{\ra}[0]{\rightarrow}
\newcommand{\la}[0]{\lambda}

\def\fn#1{\mathop{\mbox{\it #1}}}
\def\fun#1#2{\ensuremath{\mathop{\mbox{\it #1}}(#2)}} % function call

\newtheorem{definition}{Definition}
\newtheorem{theorem}{Theorem}

\maketitle

%%%%%%%%%%%%%%%%%%%%%%%%%%%%%%%%%%%%%%%%%%%%%%%%%%%%%%%%%%%%%%%%%%%%%%%%

\section{Introduction}

A {\it common machine language} is an essential abstraction that
allows programmers to express an algorithm in a way that can be
efficiently executed on a variety of architectures.  The key
properties of a common machine language (CML) are: 1) it abstracts
away the idiosyncratic differences between one architecture and
another so that a programmer doesn't have to worry about them, and 2)
it encapsulates the common properties of the architectures such that a
compiler for any given target can still produce an efficient
executable.

For von-Neumann architectures, the canonical CML is C: instructions
consist of basic arithmetic operations, executed sequentially, which
operate on either local variables or values drawn from a global block
of memory.  C has been implemented efficiently on a wide range of
architectures, and it saves the programmer from having to adapt to
each kind of register layout, cache configuration, and instruction
set.

However, recent years have seen the emergence of a class of grid-based
architectures \cite{smartmemories, raw, trips} for which the
von-Neumann model no longer holds, and for which C is no longer an
adequate CML.  The design of these processors is fundamentally
different in that they are conscious of wire delays--instead of just
arithmetic computations--as the barriers to performance.  Accordingly,
grid-based architectures support fine-grained, reconfigurable
communication between replicated processing units.  Rather than a
single instruction stream with a monolithic memory, these machines
contain multiple instruction streams with distributed memory banks.

Though C can still be used to write efficient programs on these
machines, doing so either requires architecture-specific directives or
a very smart compiler that can extract the parallelism and
communication from the C semantics.  Both of these options renders C
obsolete as a CML, since it fails to hide the architectural details
from the programmer and it imposes abstractions which are a mismatch
for the domain.

To bridge this gap, we propose a new common machine language for
grid-based processors: StreamIt.  The StreamIt language makes explicit
the large-scale parallelism and regular communication patterns that
these architectures were designed to exploit.  A program is
represented not as a monolithic memory and instruction stream, but
rather as a composition of autonomous filters, each of which contains
its own memory and can only communicate with its immediate neighbors
via high-bandwidth data channels.  In addition, StreamIt provides a
low-bandwidth messaging system that filters can use for non-local
communication.  We believe that StreamIt abstracts away the variations
in grid-based processors while encapsulating their common properties,
thereby enabling compilers to efficiently map a single source program
to a variety of modern processors.

\section{The StreamIt Language}

In this section we provide a brief overview of the StreamIt language;
please see \cite{streamittech} for a more detailed description.  The
current version of StreamIt is legal Java syntax to simplify our
presentation and implementation, and it is designed to support only
streams with static input and output rates.

\subsection{The Stream Graph}

The basic unit of computation in StreamIt is the {\tt Filter}.  An
example of a Filter is the {\tt Adder}, a component of our software
radio (see Figure \ref{fig:radiocode}).  Each {\tt Filter} contains an
{\tt init} function that is called at initialization time; in this
case, the {\tt Adder} records {\tt N}, the number of items it should
add at once.  The {\tt work} function describes the most fine grained
execution step of the filter in the steady state.  Within the {\tt
work} function, the filter can communicate with neighboring blocks
using the {\tt input} and {\tt output} channels, which are typed FIFO
queues declared within the {\tt init} function.  These high-volume
channels support the intuitive operations of {\tt push(value)}, {\tt
pop()}, and {\tt peek(index)}, where {\tt peek} returns the value at
position $index$ without dequeuing an item.

The basic construct for composing filters into a communicating network
is a {\tt Pipeline}, such as the {\tt Radio} in Figure
\ref{fig:radiocode}.  Like a {\tt Filter}, a {\tt Pipeline} has an
{\tt init} function that is called upon its instantiation.  However,
there is no {\tt work} function, and all input and output channels are
implicit; the stream behaves as the sequential composition of filters
that are specified with successive calls to {\tt add} from within {\tt
init}.

\begin{figure}
\psfig{figure=radio.ps,height=1.4in}
\vspace{-6pt}
\caption{Block diagram of a software radio.
\protect\label{fig:radiodiagram}}
%\vspace{-6pt}
%\end{figure}

%\begin{figure}
\scriptsize
\begin{verbatim}
class Adder extends Filter {
  int N;

  void init(int N) {
    input = new Channel(Float.TYPE, N);
    output = new Channel(Float.TYPE, 1);
    this.N = N;
  }
  
  void work() {
    float sum = 0;
    for (int i=0; i<N; i++) {
      sum += input.pop();
    }
    output.push(sum);
  }
}

public class Equalizer extends Pipeline {
  void init(int BANDS) {
    add(new SplitJoin() {
      void init() {
        int bottom = 2500;
        int top = 5000;
        setSplitter(DUPLICATE());
        for (int i=0; i<N; i++, bottom*=2, top*=2) {
          add(new BandPassFilter(bottom, top));
        }
        setJoiner(ROUND_ROBIN());
    }});
    add(new Adder(BANDS));
  }
}
  
class CheckHop extends Filter {
   int DELAY;
   RFtoIFPortal portal;

   void init(RFtoIF rf2if, int N) {
      input = new Channel(Float.TYPE, N);
      output = new Channel(Float.TYPE, N);
      this.portal = new RFtoIFPortal(rf2if);
      DELAY = 256/N;
   }

   void work() {
      boolean hopped = /* calculate if hopped */
      float newFreq =  /* detect new frequency */
      if (hopped) {
         portal.setFreq(newFreq, DELAY);
      }
   }
}

class Radio extends Stream {
   int N = 64;
   int BANDS = 10;

   void init() {
      add(new ReadFromAtoD());
      RFtoIF rf2if = add(new RFtoIF(N));
      add(new FFT(N));
      add(new CheckHop(rf2if));
      add(new Equalizer(10));
      add(new Speaker());
   }
}
\end{verbatim}
\vspace{-16pt}
\caption{StreamIt code for a software radio.
\protect\label{fig:radiocode}}
\end{figure}

There are two other stream constructors besides {\tt Pipeline}: {\tt
SplitJoin} and {\tt FeedbackLoop}.  The former is used to specify
independent parallel streams that diverge from a common {\it splitter}
and merge into a common {\it joiner} (see the Equalizer in Figure
\ref{fig:radiocode}).  There are two kinds of splitters: 1) Duplicate,
which replicates each data item and sends a copy to each parallel
stream, and 2) RoundRobin($w_1, \dots, w_n$), which sends the first
$w_1$ items to the first stream, the next $w_2$ items to the second
stream, and so on.  RoundRobin is the only joiner type.  The parallel
streams are specified by successive calls to {\tt add}; the $i$'th
call sets the $i$'th stream in the SplitJoin.

%% The splitter
%% and joiner type are specified with calls to {\tt setSplitter} and {\tt
%% setJoiner}, respectively; the parallel streams are specified by
%% successive calls to {\tt add}, with the $i$'th call setting the $i$'th
%% stream in the SplitJoin.  We do not discuss FeedbackLoops do to space
%% constraints.

%% Evident in all of these examples is another feature of the StreamIt
%% syntax: {\it inlining}.  The definition of any stream or filter can be
%% inlined at the point of its instantiation, thereby preventing the
%% definition of many small classes that are used only once, and,
%% moreover, providing a syntax that reveals the hierarchical structure
%% of the streams from the indentation level of the code.  In our Java
%% syntax, we make use of anonymous classes for inlining \cite{java}.

StreamIt differs from other languages in that it imposes a
well-defined structure on the streams: all stream graphs are built out
of a hierarchical composition of Pipelines, SplitJoins, and
FeedbackLoops.  This structure enables the stream to be mapped
efficiently onto a grid target, since all of the communication is
between neighboring filters.  Moreover, we are developing fission and
fusion algorithms that can, for example, collapse a large Pipeline
into a single filter for execution on a single processor, thereby
allowing us to adjust the granularity of the stream graph to match the
granularity of a given target.

%% The comparison of StreamIt's structure with arbitrary stream graphs
%% could be likened to the difference between structured control flow and
%% GOTO statements.  Though sometimes the structure restricts the
%% expressiveness of the programmer, the gains in robustness,
%% readability, and compiler analysis are immense.

\subsection{Messages}

StreamIt provides a dynamic messaging system for passing irregular,
low-volume control information between non-neighboring filters.
Messages are sent from within the body of a filter's {\tt work}
function, perhaps to change a parameter in another filter.  For
example, in the {\tt CheckHop} filter of our software radio example
(Figure \ref{fig:radiocode}), a message is sent upstream to change the
frequency of the receiver if the downstream component detects that the
transmitter is about to change frequencies.  The sender can continue
to execute while the message is en route, and the {\tt setFreq} method
will be invoked in the receiver with argument {\tt newFreq} when the
message arrives.  Since message delivery is asynchronous, there can be
no return value; only void methods can be message targets.

The central aspect of the messaging system is a sophisticated timing
mechanism that allows filters to specify when a message will be
received relative to the flow of information between the sender and
the receiver.  This mechanism enables a consistent definition of
message delivery timing across varying architectures that have no
global clock.  For example, the {\tt CheckHop} filter sends a
message with a latency of {\tt DELAY}.  This means that the target
will receive the message when it is processing the data item that the
the {\tt CheckHop} filter sees in {\tt DELAY} executions of its
own work function.  That is, the sender of a message specifies the
delivery timing in terms of its own local time; the translation to the
receiver's time is done relative to data items in the stream.  This
timing mechanism is important not only for portability, as it is
independent of any architecture's clock or topology, but also for
programmability, as it is often important to synchronize events with
wavefronts of data in the stream.  Please see \cite{streamittech} for
a precise definition of the message delivery semantics.

%% In StreaMIT, one can specify a range of latencies for a message to get
%% delivered.  This latency is measured in terms of an information
%% ``wavefront'' from one filter to another.  For example, in the {\tt
%% CheckHop} example of Figure \ref{fig:portal-code}, the sender
%% indicates an interval of latencies between $4N$ and $6N$.  This means
%% that the receiver will receive the message immediately following the
%% last invocation of its own {\tt work} function which produces an item
%% affecting the some output of the {\it sender's} $4N$'th to $6N$'th
%% work functions, counting the sender's current work function as number
%% 0.  Defining this notion precisely is the subject of Section
%% \ref{sec:time}, but the general idea is simple:  the receiver is
%% invoked when it sees the information wavefront that the sender sees in
%% $4N$ to $6N$ execution steps.  

%% In some cases, the ability to synchronize the arrival of a message
%% with some element of the data stream is very important.  For example,
%% {\tt CheckHop} knows that the transmitter will change the
%% frequency between $4N$ and $6N$ steps later, in terms of the frame
%% that {\tt CheckHop} is inputting.  To ensure that the radio
%% changes frequencies at the same time--so as not to lose any data at
%% the old or new frequency--{\tt CheckHop} instructs the receiver to
%% switch frequencies when the {\it receiver} sees one of the last data
%% items at the old frequency.

%% \subsubsection{Rationale}

%% Stream programs present a challenge in that filters need both regular,
%% high-volume data transfer and irregular, low-volume control
%% communication.  Moreover, there is the problem of reasoning about the
%% relative ``time'' between filters when they are running asynchronously
%% and in parallel.

%% A different approach to messaging is to embed control messages in the
%% data stream instead of providing a separate mechanism for dynamic
%% message passing.  This does have the effect of associating the message
%% time with a data item, but it is complicated, error-prone, and leads
%% to unreadable code.  Further, it could hurt performance in the steady
%% state (if each filter has to check whether or not a data item is
%% actual data or control, instead) and complicates compiler analysis,
%% too.  Finally, one can't send messages upstream without creating a
%% separate data channel for them to travel in.

%% Another solution is to treat messages as synchronous method calls.
%% However, this delays the progress of the stream when the message is en
%% route, thereby degrading the performance of the program and
%% restricting the compiler's freedom to reorder filter executions.  

%% We feel that the StreaMIT messaging model is an advance in that it
%% separates the notions of low-volume and high-volume data
%% transfer--both for the programmer and the compiler--without losing a
%% well-defind semantics where messages are {\it timed} relative to the
%% high-volume data flow.  Further, by separating message communication
%% into its own category, fewer connections are needed for steady-state
%% data transfer and the resulting stream graphs are more amenable to
%% structured stream programming.

\section{Status and Conclusions}

We have implemented a fully-functional prototype of the StreamIt
optimizing compiler.  It generates code for a uniprocessor which
performs comparably to SpectrumWare \cite{spectrumware}, a streaming
runtime library implemented in C++.  We are currently implementing a
backend for the Raw processor \cite{raw}, as well as developing a
cleaner syntax for StreamIt.

We belive that StreamIt represents a viable common machine language
for grid-based architectures.  It abstracts away the target's
granularity, memory layout, and network interconnect, while capturing
the notion of replicated processors that communicate in regular
patterns.  With this representation, we believe that the StreamIt
compiler will match the performance of C code that was hand-tailored
for a given grid-based machine.

%%%%%%%%%%%%%%%%%%%%%%%%%%%%%%%%%%%%%%%%%%%%%%%%%%%%%%%%%%%%%%%%%%%%%%%%

\begin{small}
\bibliographystyle{abbrv}
\bibliography{references}
\end{small}
\end{document}
