\begin{figure}
\centering
\vspace{6pt}
\psfig{figure=speedup-graph.eps,width=2.95in}
\vspace{-6pt}
\caption{\protect\small StreamIt throughput on a 16-tile Raw machine,
normalized to throughput of hand-written C running on a single Raw
tile.  \protect\label{fig:compare-raw}}
\vspace{-12pt}
\end{figure}

\begin{figure}
\centering
\psfig{figure=throughput-graph.eps,width=3.2in}
\vspace{-6pt}
\caption{Throughput of StreamIt code running on 16 tiles and C code
running on a single tile, normalized to throughput of C code on a
Pentium IV. \protect\label{fig:compare-pentium}}
\vspace{-12pt}
\end{figure}

\section{Results}
\label{sec:results}

Our current implementation of StreamIt supports fully automatic
compilation through the Raw backend.  We have also implemented the
optimizations that we have described: synchronization elimination,
modulo expression elimination (vertical fusion), buffer localization
(vertical fusion), and buffer sharing (horizontal fusion).

We evaluate the StreamIt compiler for the set of applications shown in
Table~\ref{tab:benchmarks}; our results appear in
Table~\ref{tab:performance}.
%  For each benchmark, we show the number of
%lines of StreamIt code, the occurrence of each stream construct, and
%the number of nodes required to execute the expanded graph on Raw.
For each application, we compare the throughput of StreamIt with a
hand-written C program, running the latter on either a single tile of
Raw or on a Pentium IV.  For Radio, GSM, and Vocoder, the C source
code was obtained from a third party; in other cases, we wrote a C
implementation following a reference algorithm.  For each benchmark,
we show MFLOPS (which is N/A for integer applications), processor
utilization (the percentage of time that an {\it occupied tile} is not
blocked on a send or receive), and throughput.  We also show the
performance of the C code, which is not available for C programs that
did not fit onto a single Raw tile (Radar, GSM, and Vocoder).
Figures~\ref{fig:compare-raw} and~\ref{fig:compare-pentium} illustrate
the speedups obtained by StreamIt compared to the C
implementations\footnote{FFT and Filterbank perform better on a Raw
tile than on the Pentium 4.  This could be because Raw's single-issue
processor has a larger data cache and a shorter processor pipeline.}.

The results are encouraging.  In many cases, the StreamIt compiler
obtains good processor utilization--over 60\% for four benchmarks and
over 40\% for two additional ones.  For GSM, parallelism is limited by
a feedbackloop that sequentializes much of the application.  Vocoder
is hindered by our work estimation phase, which has yet to accurately
model the cost of library calls such as {\tt sin} and {\tt tan}; this
impacts the partitioning algorithm and thus the load balancing.  3GPP
also has difficulties with load balancing, in part because our current
implementation fuses all the children of a stream construct at once.

StreamIt performs respectably compared to the C implementations,
although there is room for improvement.  The aim of StreamIt is to
provide a higher level of abstraction than C without sacrificing
performance.  Our current implementation has taken a large step
towards this goal.  For instance, the synchronization removal
optimization improves the throughput of 3GPP by a factor of 1.8 on 16
tiles (and by a factor of 2.5 on 64 tiles.)  Also, our partitioner can
be very effective--as illustrated in Figure~\ref{fig:beam-blood},
partitioning the \Radar~ application improves performance by a factor
of 11 even though it executes on less than one third of the tiles.

The StreamIt optimization framework is far from complete, and the
numbers presented here represent a first step rather than an upper
bound on our performance.  We are actively implementing aggressive
inter-node optimizations and more sophisticated partitioning
strategies that will bring us closer to achieving linear speedups for
programs with abundant parallelism.

%% We show the performance of the original application, which maps each
%% Filter in the original program to a single Raw tile.  In some cases,
%% we need to use an 8x8 Raw processor in order to have enough tiles for
%% the Filters; otherwise, we target a 4x4 configuration of Raw.  To
%% obtain the ``optimized'' numbers, we perform a series of fusion,
%% fission and reordering transformations to create a load-balanced set
%% of filters that can be mapped onto a 4x4 Raw processor.
%% Figures~\ref{fig:opt-diagram} and \ref{fig:utilization-diagram} depict
%% the impact of our load balancing transformations on throughput and
%% tile utilization, respectively.

%% The results show that for programs with substantial computation
%% requirements, the StreamIt compiler is able to extract good
%% performance out of the Raw processor.  For example, the \Radar
%% application (see Figure~\ref{fig:beamcode} for source code) shows a
%% sustained 1.47 GFLOPS rate, after a 145\% improvement due to our
%% optimizations.  Figures~\ref{fig:beam-orig} and~\ref{fig:beam-opt}
%% give the original and optimized configurations of \Radar's stream
%% graph, and Figure~\ref{fig:beam-blood} illustrates the impact of our
%% load balancing optimizations .

%We have evaluated our compiler with StreamIt versions applications: 
%\begin{itemize}
%
%\item A software-based FM Radio with
%equalizer, 
%
%\item A GSM Decoder, which takes GSM-encoded parameters as
%inputs, and uses these to synthesize audible speech, 
%
%\item A \Radar application
%which encapsulates the core functionality of modern radar, sonar, and
%communications signal processors -- modeled after a system from the
%Polymorphic Computing Architecture (PCA) \cite{pca}

%\item A 64-element FFT filter implemented as a pipeline of a bit reversal 
%filter followed by multiple-stage butterfly filters.

%\item A Cyclic Redundancy Check (CRC) Encoder/Decoder, which is an error detection system utilizing a predefined 32 bit generator polynomial.  It can detect errors in bit strings of up to 64KB in size with 99.99999997% accuracy.

%\item Matrix Multiplication 

%\item A MP3

%\item A performance test from the SpectrumWare system that
%implements an Orthogonal Frequency Division Multiplexor (OFDM)
%\cite{spectrumware}.  Table \ref{tab:benchmarks} gives characteristics
%of the above applications including the number of filters implemented
%and the size of the stream graph as coded.

%%\end{itemize}

