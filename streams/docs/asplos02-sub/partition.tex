\begin{figure}
  \psfig{figure=beam-blood-key.eps,width=3.2in} \\
  \subfigure[\vspace{-6pt}
    Original.\label{fig:beam-blood1}]{\psfig{figure=beam-blood-orig.eps,width=3.2in}}
  \hspace{0.3in} \subfigure[\vspace{-6pt}
    Load-balanced.\label{fig:beam-blood2}]{\psfig{figure=beam-blood-opt.eps,width=3.2in}
    \vspace{-6pt}} \vspace{-6pt} \caption{Execution traces for the (a)
    original and (b) load-balanced partitionings of the \Radar~
    application.  The $x$ axis denotes time, and the $y$ axis denotes
    the processor.  The dark bands indicate periods where processors
    are blocked waiting to receive an input or send an output.  Light
    regions indicate periods of useful work.  The thin stripes in the
    light regions represent pipeline
    stalls. \protect\label{fig:beam-blood}}
\end{figure}

\begin{figure}
\psfig{figure=beam-graph-orig.eps,width=2.5in}
\caption{\protect\small Stream graph of the original 12x4 \Radar~
application.  The 12x4 \Radar application has 12 channels and 4 beams;
it is the largest version that fits onto 64 tiles without filter
fusion.  \protect\label{fig:beam-orig}}
\vspace{36pt}
\psfig{figure=beam-graph-opt.eps,width=2.5in}
\caption{\protect\small Stream graph of the load-balanced 12x4
\Radar~ application.  Vertical fusion is applied to collapse each pipeline
into a single filter, and horizontal fusion is used to transform the
4-way splitjoin into a 2-way splitjoin.  Figure~\ref{fig:beam-blood}
shows the benefit of these
transformations. \protect\label{fig:beam-opt}}
\end{figure}

\section{Partitioning}
\label{sec:partition}

StreamIt provides the filter construct as the basic abstract unit of
autonomous stream computation.  The programmer should decide the
boundaries of each filter according to what is most natural for the
algorithm under consideration.  While one could envision each filter
running on a separate machine in a parallel system, StreamIt hides the
granularity of the target machine from the programmer.  Thus, it is
the responsibility of the compiler to adapt the granularity of the
stream graph for efficient execution on a particular architecture.

We use the word {\it partitioning} to refer to the process of dividing
a stream program into a set of balanced computation units.  Given a
number $N$, which represents the maximum number of computation units
that can be supported, the partitioning stage transforms a stream
graph into a set of no more than $N$ filters, each of which performs
approximately the same amount of work during the execution of the
program.  Following this stage, each filter can be run on a separate
processor to obtain a load-balanced executable.

%% Load balancing is particularly important in the streaming domain,
%% since the throughput of a stream graph is equal to the {\it minimum}
%% throughput of each of its stages.  This is in contrast to scientific
%% programs, which often contain a number of stages which process a given
%% data set; the running time is the {\it sum} of the running times of
%% the phases, such that a high-performance, parallel phase can partially
%% compensate for an inefficient phase.  In mathematical terms, Amdahl's
%% Law captures the maximum realizable speedup for scientific
%% applications.  However, for streaming programs, the maximum
%% improvement in throughput is given by the following expression:
%% \begin{align*}
%% \mbox{\it Maximum speedup}(w, c) = \frac{\sum_{i=1}^N{w_i \cdot c_i}}{MAX_i(w_i \cdot c_i)}
%% \end{align*}
%% where $w_1 \dots w_m$ denote the amount of work in each of the $N$
%% partitions of a program, and $c_i$ denotes the multiplicity of work
%% segment $i$ in the steady-state schedule.  Thus, if we double the load
%% of the heaviest node ({\it i.e.}, the node with the maximum $w_i \cdot
%% c_i$), then the performance could suffer by as much as a factor of
%% two.  The impact of load balancing on performance places particular
%% value on the partitioning phase of a stream compiler.

% This can be posed in
% mathematical terms by appealing to Amdahl's Law, which expresses the
% maximum speedup that can be achieved on $N$ processors when there are
% $S_1 \dots S_x$ units of sequential work and $P_1 \dots P_y$ units of
% parallel work:
% \begin{align*}
% \mbox{\it Maximum speedup}(S, P, N) = \frac{\sum_{i=1}^x{S_i}+\sum_{i=1}^y{P_i}}{\sum_{i=1}^x{S_i}+\frac{\sum_{i=1}^y{P_i}}{N}}
% \end{align*}
% However, in a stream program, the measure of performance is throughput
% rather than running time.  Using $w_1 \dots w_m$ to denote the amount
% of work in each of the $m$ partitions of a program, and denoting the
% multiplicity of work segment $i$ in the steady-state schedule by
% $c_i$, we have that:

\subsection{Overview}

Our partitioner employs a set of fusion, fission, and reordering
transformations to incrementally adjust the stream graph to the
desired granularity.  To achieve load balancing, the compiler
estimates the number of instructions that are executed by each filter
in one steady-state cycle of the entire program; then, computationally
intensive filters can be split, and less demanding filters can be
fused.  Currently, a simple greedy algorithm is used to automatically
select the targets of fusion and fission, based on the estimate of the
work in each node.

For example, in the case of the \Radar~ application, the original
stream graph (Figure~\ref{fig:beam-orig}) contains 52 nodes.  These
nodes have unbalanced amounts of computation, as evidenced by the
execution trace in Figure~\ref{fig:beam-blood1}.  The partitioner
fuses all of the pipelines in the graph, and then fuses the bottom
4-way splitjoin into a 2-way splitjoin, yielding the stream graph in
Figure~\ref{fig:beam-opt}.  As illustrated by the execution trace in
Figure~\ref{fig:beam-blood2}, the partitioned graph has much better
load balancing.  In the following sections, we describe in more detail
the transformations utilized by the partitioner.

\subsection{Fusion Transformations}

Filter fusion is a transformation whereby several adjacent filters are
combined into one.  Fusion can be applied to decrease the granularity
of a stream graph so that an application will fit on a given target,
or to improve load balancing by merging small filters so that there is
space for larger filters to be split.  Analogous to loop fusion in the
scientific domain, filter fusion can enable other optimizations by
merging the control flow graphs of adjacent nodes, thereby shortening
the live ranges of variables and allowing independent instructions to
be reordered.

\subsubsection{Naive Fusion}

In the domain of structured stream programs, there are two types of
fusion that we are interested in: {\it vertical fusion} for collapsing
pipelined filters into a single unit, and {\it horizontal fusion} for
combining the parallel components of a splitjoin into one.  Given that
each StreamIt filter has a constant I/O rate, it is possible to
implement both vertical and horizontal fusion as a plain compile-time
simulation of the execution of the stream graph.  A high-level
algorithm for doing so is as follows:
\begin{enumerate}
\item Calculate a legal initialization and steady-state schedule for
the nodes of interest.

\item For each pair of neighboring nodes, introduce a circular buffer
that is large enough to hold all the items produced during the initial
schedule and one iteration of the steady-state schedule.  For each
buffer, maintain indices to keep track of the head and tail of the
FIFO queue.

\item Simulate the execution of the graph according to the calculated
schedules, replacing all push, pop, and peek operations in the fused
region with appropriate accesses to the circular buffers.
\end{enumerate}
That is, a naive approach to filter fusion is to simply implement the
channel abstraction and to leverage StreamIt's static rates to
simulate the execution of the graph.  However, the performance of
fusion depends critically on the implementation of channels, and there
are several high-level optimizations that the compiler can employ to
improve upon the performance of a general-purpose buffer
implementation.  We describe some of these optimizations in detail in
the following sections.

\begin{figure}
\centering
\psfig{figure=fuse-pipeline-a.eps,width=1in}
\subfigure{\label{fig:fuse-pipe-a}}
\psfig{figure=fuse-pipeline-b.eps,width=2.2in}
\subfigure{\label{fig:fuse-pipe-b}}
\vspace{-6pt}
\caption{\protect\small Vertical fusion with buffer localization and
modulo-division optimizations. \label{fig:fuse-pipe}}
\vspace{12pt}
\psfig{figure=fuse-splitjoin-a.eps,width=3in}
\subfigure{\label{fig:fuse-splitjoin-a}}
\psfig{figure=fuse-splitjoin-b.eps,width=3in}
\subfigure{\label{fig:fuse-splitjoin-b}}
\caption{\protect\small Horizontal fusion of a duplicate splitjoin
construct with buffer sharing optimization. \protect\label{fig:fuse-splitjoin}}
%% \vspace{18pt}
%% \psfig{figure=fuse-splitjoin2.eps,width=4.93in}
%% \caption{\protect\small Fusion of a roundrobin splitjoin construct.
%% The fusion transformation for splitjoins containing roundrobin
%% splitters is similar to those containing duplicate splitters.  One
%% filter simulates the execution of a steady-state cycle in the
%% splitjoin by inlining the code from each filter.  This filter is
%% surrounded by Reorderroundrobin filters that recreate the reordering of
%% the roundrobin nodes.  In the above example, the difference the
%% splitter's weights, the filter's I/O rates, and the joiner's weights
%% adds complexity to the reordering.  Though this transformation is
%% fully automated in the StreamIt compiler, a general formulation is
%% beyond the scope of this paper.  \protect\label{fig:fuse-splitjoin2}}
\end{figure}

\subsubsection{Optimizing Vertical Fusion}

Figure~\ref{fig:fuse-pipe} illustrates two of our optimizations for
vertical fusion: the localization of buffers and the elimination of
modulo operations.  In this example, the UpSampler pushes $K$ items on
every step, while the MovingAverage filter peeks at $N$ items but only
pops 1.  The effect of the optimizations are two-fold.  First, buffer
localization splits the channel between the filters into a local {\tt
buffer} (holding items that are transfered within {\tt work}) and a
persistent {\tt peek\_buffer} (holding items that are stored between
iterations of {\tt work}).  Second, modulo elimination arranges copies
between these two buffers so that all index expressions are known at
compile time, preventing the need for a modulo operation to wrap
around a circular buffer.

The execution of the fused filter proceeds as follows.  In the initial
work function, which is called only on the first invocation, the {\tt
peek\_buffer} is filled with initial values from the UpSampler.  The
steady work function implements a steady-state schedule in which
$\mt{LCM}(N, K)$ items are transferred between the two original
filters--these items are communicated through a local, temporary {\tt
buffer}.  Before and after the execution of the MovingAverage code,
the contents of the {\tt peek\_buffer} are transferred in and out of
the temporary {\tt buffer}.  If the {\tt peek\_buffer} is small, this
copying can be eliminated with loop unrolling and copy propagation.
Note that the {\tt peek\_buffer} is for storing items that are
persistent from one firing to the next, while the local {\tt buffer}
is just for communicating values during a single firing.

\subsubsection{Optimizing Horizontal Fusion}

The naive fusion algorithm maintains a separate input buffer for each
parallel stream in a splitjoin.  However, in the case of a duplicate
splitter, the input buffer can be shared between the streams, as
illustrated in Figure~\ref{fig:fuse-splitjoin}.  As in pipeline
fusion, the code of the component filters is inlined into a single
filter with repetition according to the steady-state schedule.
However, there are some modifications: all {\tt pop} statements are
converted to {\tt peek} statements, and the {\tt pop}'s are performed
at the end of the fused work function.  This allows all the filters to
see the data items before they are consumed.  Finally, the roundrobin
joiner is simulated by a reorder-roundrobin filter that re-arranges
the output of the fused filter according to the weights of the joiner.
Separating this reordering phase from the fused filter also decreases
the buffer requirements within the fused node.  When appropriate,
pipeline fusion can be applied to fuse these two nodes into a single
filter that represents the entire splitjoin.

\begin{figure}
\centering
%\psfig{figure=duplicate-fission-2.eps,width=3.2in}
%\caption{\protect\small Fission of a filter that does not peek.
%For filters such as a VectorMultiply that consumes every item they look
%at, horizontal fission consists of embedding copies of the filter in a
%$K$-way roundrobin splitjoin.  The weights of the splitter and joiner
%are set to match the $pop$ and $push$ rates of the filter, respectively.  
%\protect\label{fig:fission-nopeek}}
%\vspace{18pt}
\psfig{figure=duplicate-fission-a.eps,width=1.2in}
\subfigure{\label{fig:fission-peek-a}}
\psfig{figure=duplicate-fission-b.eps,width=2in}
\subfigure{\label{fig:fission-peek-b}}
\caption{\protect\small Fission of a filter that peeks.  
%Since the MovingAverage filter reads items that it does not consume, the
%duplicated versions of the filter need to access overlapping portions
%of the input stream.  For this reason, horizontal fission creates a
%duplicate splitjoin in which each component filter has additional code
%to filter out items that are irrelevant to a given path.  This
%decimation occurs in two places: once in the initial work function, to
%disregard items considered by previous filters on the first iteration
%of the splitjoin, and once at the end of the steady work function, to
%account for items consumed by other components.
\protect\label{fig:fission-peek}}
\end{figure}

\subsection{Fission Transformations}

Filter fission is the analog of parallelization in the streaming
domain.  It can be applied to increase the granularity of a stream
graph to utilize unused processor resources, or to break up a
computationally intensive node for improved load balancing.

\subsubsection{Vertical Fission}

Some filters can be split into a pipeline, with each stage performing
some part of the {\tt work} function.  In addition to the original
input data, these pipelined stages might need to communicate
intermediate results from within {\tt work}, as well as fields within
the filter.  This scheme could apply to filters with state if all
modifications to the state appear at the top of the pipeline (they
could be sent over the data channels), or if changes are infrequent
(they could be sent via StreamIt's messaging system.)  Also, some
state can be identified as induction variables, in which case their
values can be reconstructed from the {\tt work} function instead of
stored as fields.  We have yet to automate vertical filter fission in
the StreamIt compiler.

\subsubsection{Horizontal Fission}

We refer to ``horizontal fission'' as the process of distributing a
single filter across the parallel components of a splitjoin.  We have
implemented this transformation for ``stateless'' filters--that is,
filters that contain no fields that are written on one invocation of
{\tt work} and read on later invocations.  Let us consider such a
filter $F$ with steady-state I/O rates of $peek$, $pop$, and $push$,
that is being parallelized into an $K$-way splitjoin.  There are two
cases to consider:
\begin{enumerate}
\item If {\bf $peek = pop$}, then $F$ can simply be duplicated $K$
ways in the splitjoin.  The splitter is a roundrobin that routes $pop$
elements to each copy of $F$, and the joiner is a roundrobin that
reads $push$ elements from each component.  Since $F$ does not peek at
any items which it does not consume, its code does not need to be
modified in the component streams--we are just distributing the
invocations of $F$.

\item If {\bf $peek > pop$}, then a different transformation is
applied (see Figure~\ref{fig:fission-peek}).  In this case, the
splitter is a duplicate, since the component filters need to examine
overlapping parts of the input stream.  The $i$'th component has a
steady-state work function that begins with the work function of $F$,
but appends a series of $(K-1)*pop$ pop statements in order to account
for the data that is consumed by the other components.  Also, the
$i$'th filter has an initial work function that pops $(i-1)*pop$ items
from the input stream, to account for the consumption of previous
filters on the first iteration of the splitjoin.  As before, the
joiner is a roundrobin that has a weight of $push$ for each stream.
\end{enumerate}

\begin{figure}
\centering
\psfig{figure=filter-hoisting.eps,width=3.2in}
\caption{\protect\small Filter hoisting.  This transformation allows a
stateless filter to be moved across a joiner node if its $push$ value
evenly divides the weights of the joiner.
\protect\label{fig:filter-hoisting}}
\vspace{12pt}
\psfig{figure=splitjoin-split.eps,width=3.2in}
\caption{\protect\small Breaking a splitjoin into hierarchical units.
Though our horizontal fusion algorithms work on the granularity of an
entire splitjoin, it is straightforward to transform a large splitjoin
into a number of smaller pieces, as shown here.  Following this
transformation, the fusion algorithms can be applied to obtain an
intermediate level of granularity.  This technique was employed to
help load-balance the \Radar~ application (see Section~\ref{sec:results}).
\protect\label{fig:splitjoin-split}}
\vspace{12pt}
\psfig{figure=sync-removal.eps,width=3.2in}
\caption{\protect\small Synchronization removal.  If a splitter is
bordering a joiner with identical weights, the nodes can be removed
and the component streams can be connected.  This is especially useful
in the context of libraries--many distinct components can employ
splitjoins for processing interleaved data streams, and the modules
can be composed without having to synchronize all the streams at each
boundary.  \protect\label{fig:sync-removal}}
\vspace{-6pt}
\end{figure}

\subsection{Reordering Transformations}
There are a multitude of ways to reorder the elements of a stream
graph so as to facilitate fission and fusion transformations.  For
instance, identical stateless filters can be pushed through a splitter
or joiner node if the weights are adjusted accordingly
(Figure~\ref{fig:filter-hoisting}); a splitjoin construct can be
divided into a hierarchical set of splitjoins to enable a finer
granularity of fusion (Figure~\ref{fig:splitjoin-split}); and
neighboring splitters and joiners with matching weights can be
eliminated (Figure~\ref{fig:sync-removal}).  Many of these
transformations were useful in load balancing our test applications.

\subsection{Automatic Partitioning}

In order to drive the partitioning process, we have implemented a
simple greedy algorithm that performs well on most applications.  The
algorithm analyzes the {\tt work} function of each filter and
estimates the number of cycles required to execute it.  Because of the
static I/O rates in StreamIt, most loops within {\tt work} can be
unrolled, allowing a close approximation of the actual cycle count.
Multiplying this cycle count by the multiplicity of the filter in the
steady-state schedule, the algorithm obtains an estimate of the
steady-state computational requirements for each filter.

In the case where there are fewer filters than tiles, the partitioner
considers the filters in decreasing order of their computational
requirements and attempts to split them using the filter fission
algorithm described above.  Fission proceeds until there are enough
filters to occupy the available machine resources, or until the
heaviest node in the graph is not ammenable to a fission
transformation.  Generally, it is not beneficial to split nodes other
than the heaviest one, as this would introduce more synchronization
without alleviating the bottleneck in the graph.

If the stream graph contains more nodes than the target architecture,
then the partitioner works in the opposite direction and repeatedly
fuses the least demanding stream construct until the graph will fit on
the target.  The work estimates of the filters are tabulated
hierarchically and each construct ({\it i.e.,} pipeline, splitjoin,
and feedbackloop) is ranked according to the sum of its children's
computational requirements.  At each step of the algorithm, an entire
stream construct is collapsed into a single filter.  The only
exception is the last fusion operation, which only collapses to the
extent necessary to fit on the target; for instance, a 4-element
pipeline could be fused into two 2-element pipelines if no more
collapsing was necessary.

Despite its simplicity, this greedy strategy works well in practice
because most applications have many more filters than can fit on the
target architecture; since there is a long sequence of fusion
operations, it is easy to compensate from a short-sighted greedy
decision.  However, we can construct cases in which a greedy strategy
will fail.  For instance, graphs with wildly unbalanced filters will
require fission of some components and fusion of others; also, some
graphs have complex symmetries where fusion or fission will not be
beneficial unless applied uniformly to each component of the graph.
We are working on improved partitioning algorithms that take these
measures into account.
