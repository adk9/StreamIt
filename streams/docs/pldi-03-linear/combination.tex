\section{Container Combination(propagation)}
We can use our knowledge of the linear representations of individual filters
to determine linear representations for the StreamIt's heirarchal structures.
To determine a structure's overall linear representation, we require that all 
of the structure's sub {\tt streams} have linear representations. 
In the following discussion we discuss methods for determining overall representations for 
{\tt pipeline}s and {\tt splitjoin}s. Combining {\tt feedbackloop}s will not be discussed
here as it requires the notion of ``linear state'' (see Future Work).



\subsection{Rate Matching}
The StreamIt can determine the input and output rates of each {\tt filter} and compile
time. The rates of data production and data consumption for structures is determined 
by the compiler, and the critical determination of a scheduling for the order to execute 
the {\tt filter}s is described elsewhere \cite{karczma-thesis}. However,  
in order to describe the overall action of StreamIt structures, we must take
be able to match the input and output rates effectively. 

For example, if there are two {\tt filters} $F_1$ and $F_2$ in 
a {\tt pipeline} and $F_1$ produces $2$ elements 
during its {\tt work} function ($u_{F_1}=2$) but $F_2$ needs to examine $4$ elements 
($e_{F_2}=4$) to produce any output, it is necessary to execute $F_1$
twice before executing $F_2$. 

\begin{figure}
\center
\epsfxsize=3.0in
\epsfbox{images/expanding-a-filter.eps}
\caption{Expanding {\tt stream} $S$ by a factor $f$}
\label{fig:expanding-a-filter}
\vspace{-12pt}
\end{figure}

Therefore, a fundamental operation necessary to combine any of StreamIt's structures is 
an expansion operation on our linear forms which represents the data produced
by multiple invocations of a {\tt stream}'s {\tt work} function.
We denote $F'$, the expansion of $F$ by a factor $f$ as $F' = expand(F, f)$
and this process is depicted in Figure~\ref{fig:expanding-a-filter}.
The rates of $F'$ are $e_{F'}=e_F + (f-1)o_F$, $o_{F'}=fo_F$, and $u_{F'}=fu_F$.

\begin{figure}
\center
\epsfxsize=3.0in
\epsfbox{images/expanding-a-matrix.eps}
\caption{Expanding a linear form by factor $f$.}
\label{fig:expanding-a-matrix}
\vspace{-12pt}
\end{figure}

Thus the linear form of $F'=expand(F, f)$ describes the formula for computing
exactly $fu_F$ output elements from $e_F + (f-1)o_F$ input elements.
Figure~\ref{fig:expanding-a-matrix} shows the result of expanding the linear 
form $lf = (A,b)$ by a factor $f$ such that $lf' = expand(lf, f) = (A',b')$. 
$A'$ is created by starting with a zero matrix with $e_F+(f-1)o_F$ rows 
and $fu_F$. $A$ is then copied $f$ times along the
diagonal. Starting at the top left, each copy of $A$ is offset from the previous copy 
by $u_F$ columns and $o_F$ rows. To expand the constant vector $b$ to $b'$ we 
create a row vector with $fu_F$ rows containing $f$ adjacent copies of $b$.

\subsection{Pipelines}

\begin{figure}
\center
\epsfxsize=3.0in
\epsfbox{images/pipeline-combination.eps}
\caption{A {\tt pipeline} of two linear forms $(A,b)$ and $(C,d)$ (above) and the same {\tt pipeline} with rate matched forms (below).}
\label{fig:combining-pipeline}
\vspace{-12pt}
\end{figure}

Serial combinations of {\tt streams} are formed with {\tt pipelines}.
Combining a {\tt pipeline} with an arbitrary number of linear forms can be accomplished 
by continually combining adjacent pairs until a single linear form remains. 
The method for combining a single pair of linear forms, $l_1=(A_1,b_1)$ and $l_2=(A_2,b_2)$ follows
directly from the equations relating the inputs and the outputs of the 
two linear forms. Defining our notation: $l_1$ takes $x$ as input and produces 
$y_1[n]$ as output. $l_2$ inputs $y_1$ and produces $y_2$ as output. 
Using our linear input to output equations from above, $y_1 = xA_1 + b_1$ 
and $y_2 = y_1A_2 + b_2$. Combining the two previous equations, $y_2 = xA_1A_2 + (b_1A_2 + b_2)$ 
which corresponds to a new linear form $l'=(A_1A_2, b_1A_2+b_2)$ which computes
$y_2$ directly from $x$.

Because the expressions for $l'$ are matrix equations, the dimensions of 
$A_1$ and $A_2$ might not be compatible for general programs.
The problem of incompatible matrix sizes is the same problem of scheduling execution 
such that the first {\tt stream} produces enough data for the second {\tt stream} 
to read. To make the dimensions match up, we use the $expand$ operation. We choose
expansions factors $f_1$ and $f_2$ such that $l_1' = expand(l_1,f_1) = (A_1',b_1')$ and 
$l_2' = expand(l_2,f_2) = (A_2',b_2')$. 
The equation $f_1u_{l_1}$ = $e_{l_2} + (f_2-1)o_{lf_2}$ must hold, where $u_{lf_1}$, $e_{lf_2}$, and $o_{lf_2}$
are program defined. For most pairs of linear forms encountered in real programs, 
there are multiple pairs of $f_1$ and $f_2$ that satisfy the constraint, from which
we choose the smallest.


Once we have determined $f_1$ and $f_2$ such 
that we can multiply $A'C'$, we are guaranteed the be able to multiply $b'C'$.

Figure~\ref{fig:combining-pipeline} shows graphically the process of expanding
two linear forms in a {\tt pipeline} so that they can be combined into a single linear filter.


However, in general problems, this constraint is not always satisfiable. For instance if
$u_{lf_1}=2$, $e_{lf_2}=3$ and $o_{lf_2}=2$, there are pair of integers $f_1$ and $f_2$ to 
satisfy the above equation. Due to the buffering between the filters, we can not generate an 
overall linear form that represents the pair's computation except by introducing redundant
computation. However, by recognizing that the output from the first execution of the structure
can be written as a linear form $lf'_i$, and that the output from each subsequent execution
of the structure can be written as a linear form $lf'_j$ we can store the two linear forms $lf'_i$ and
$lf'_j$ which represent the initial execution and each subsequent execution respectively.

In some cases the combined linear form
actually represents a less efficient way of computing the results of the
filters. Specifically, when the number of non zero entries of $A'C'$ is greater
than the number of non zero entries $A'$ plus the non zero entries of $C'$, the overall
representation requires more computation.


\subsection{SplitJoins}
{\tt splitjoin}s allow a StreamIt programmer to express explicitly parallel computation. 
Data elements that arrive at the {\tt splitjoin} are directed
to the parallel sub {\tt streams} in one of two ways.
The simpler of the two is to send copy of each data element to the each of the sub {\tt streams}. 
The second of the two is specified by $N$ weights, $v_k$ for $k\in[0..N-1]$, 
where $N$ is the number of sub {\tt streams} in the {\tt splitjoin}. The first $v_0$ data elements
are sent to the leftmost {\tt Stream}. The next $v_1$ elements are sent 
to the next {\tt stream} and so on. When all $\sum_{k=0}^{N-1} v_k$ elements
have been seen, the splitter starts the same pattern over again.

The data from the parallel {\tt streams} are combined back into a single stream by means of
a joiner specified by weights $w_k$ for $k\in[0..N-1]$. First, $w_0$ items from the output tape of the 
leftmost {\tt stream} are placed onto the overall output tape, then 
$w_1$ elements are taken from the second leftmost {\tt stream} and so on. 
Again, the process repeats itself after one complete set of $\sum_{k=0}^{N-1} w_k$ 
elements has been pushed.

If we know that all subcomponents of a {\tt splitjoin} 
have linear representations, then we can describe the action of the entire {\tt splitjoin} 
with a single linear representation.
{\tt splitjoin}s containing all linear components are not as infrequenct as they might seem -- 
most {\tt splitjoin} sub {\tt streams} often perform very similar computations, so when one 
is linear there is a high probability that all linear.

The joiner of a {\tt splitjoin} specifies a specific order of the output data 
that is produced by the parallel {\tt stream} blocks. Because each column 
of a linear representation's matrix represents the formula for calculating a single output value, 
the overall linear representation of a combined {\tt splitjoin} will be exactly 
the same columns of the parallel {\tt streams} in an ordering that depends on the joiner weights $w_i$.

Below we will describe the combination rules for a {\tt splitjoin}
with a duplicate splitter. Then we will describe how we can transform 
a roundrobin splitter {\tt splitjoin} into a duplicate splitter {\tt splitjoin}. 


\subsubsection{Duplicate Splitter}

\begin{figure}
\center
\epsfxsize=3.0in
\epsfbox{images/splitjoin-duplicate-ratematch.eps}
\caption{Expanding sub {\tt streams} to match their output rates in a linear {\tt SplitJoin}.}
\label{fig:splitjoin-duplicate-ratematch}
\end{figure}

In the following discussion, we focus on the matrix $A$ of a linear representation.
The operations for the constant vector $a$ occur in a very similar way.

Each row in the matrix $A$ of a linear form corresponds to the
weight associated with a particular input item. For a duplicate splitter,
the same rows in the sub matrices correspond to the same
input so long as the number of rows in the sub matries are the same.

We wish to produce an overall representation by determining the correct
ordering the columns of the sub {\tt stream}' linear representations. We
need the number of rows to be the same, the total number of columns 
from all the sub matrices to equal some multiple of $\sum_{i=0}^{N-1}w_i$
and we need all of the sub linear forms to have the same $o$. 

If the joiner is ``run'' an integer number of times it needs to consume in aggregate
the exact number of items produced by an integer number of firings of
each sub {\tt stream}. To guarantee that there are never any ``spare'' columns,
we expand each of the sub {\tt streams} $F_i$ by a factor $f_i$.

Let $x=lcm(\forall k, lcm(u_{F_k},w_k))$, which is the $lcm$ of the $lcm$s of 
the push rates and the joiner weights. $x$ is the total number of elements that 
produced by a {\tt splitjoin} whose output is the same as the original but each 
of its sub {\tt streams} can be executed an integer number of times so that 
the joiner executes an integer number of times.

Therefore the appropriate expansion factors are  $f_i=\frac{x}{u_{F_i}}$ 
$F''_i = expand(F_i,f_i)$ will produce $f_i*u_{F_i}$ items, and it will take 
exactly $f_{w_i}=\frac{x}{w_i}$ firings of the 
joiner (each of which consumes $w_i$ elements), to consume all $f_iu_i$ items produced by $F''_i$.
It is important to note that we require that the {\tt splitjoin} does not need
infinite buffer space to schedule see \cite{karczma-thesis} for an explination of schedules. 
For valid {\tt splitjoin}s $f_w=f_{w_i}$ is constant for all $i$. We also require that $o=o_{F''_i}$
constant for all $i$. Figure~\ref{fig:splitjoin-duplicate-ratematch} depicts the 
combination process graphically.

\begin{figure}
\center
\epsfxsize=3.0in
\epsfbox{images/splitjoin-duplicate-matrix.eps}
\caption{Matrix resulting from combining a {\tt splitjoin} of rate matched sub {\tt streams}.}
\label{fig:splitjoin-duplicate-matrix}
\end{figure}

After the rates are matched as described above, we are guaranteed to have the same number of rows
in each of the expanded sub representations. To construct the overall matrix, $A'''$, 
all that remains is to order the columns of the expanded representations appropriately. 
First we create $A'''$ which has $e_{F''_i}$ rows (which is the same for all $i$) and has 
$f_w(\sum_{i=0}^{N-1}w_i)$ columns. 
%We know that the first $w_0$ output elements of the overall {\tt splitjoin} will be the first $w_0$ output elements from $F''_1$. 
We place the $w_0$ rightmost columns of $A''_0$ as the $w_0$ rightmost 
columns of $A'''$. We then place the $w_1$ rightmost columns of $A''_1$ into the next $w_1$ columns 
of $A'''$, and so forth. When we have placed $\sum_{i=0}^{N-1}w_{i}$ (the number of outputs resulting from 
completly executing the joiner once), we start the process again from the beginning (taking the remaining
rightmost $w_0$ columns from $A''_0$, etc) until all of the columns from the expanded representations have
been fitted into $A'''$. Figure~\ref{fig:splitjoin-duplicate-matrix} graphically depicts the process of 
creating the overall $A'''$ from the smaller $A''_i$s.



\subsubsection{Roundrobin Splitter}

In the case of roundrobin splitters, data is directed to the various sub {\tt streams} 
according to weights $v_i$: the first $v_0$ input elements are directed to $F_0$, the next
$v_1$ elements are directed to $F_1$, etc. If the sub {\tt streams} are
all linear, we can determine an overall linear representation given the same
assumptions of finite buffer requirements, and uniform $o_i$.
Our strategy is to transform the splitter from roundrobin to duplicate by
modifying the sub {\tt stream}s' linear representations. 

Because the splitter is a roundrobin and not a duplicate, the same rows of the sub 
{\tt stream}s' representation matrices do not represent the same input elements 
even when the sub {\tt stream}s' representations are the same size. 
The same rows don't represent the same data because each {\tt stream} 
doesn't see any of the data directed to any other {\tt stream}. Therefore simply 
arranging columns as described above will not produce a correct overall linear representation. 

\begin{figure}
\center
\epsfxsize=3.0in
\epsfbox{images/splitjoin-roundrobin-matrix.eps}
\caption{Corresponding matrix for splitter conversion from roundrobin to duplicate.}
\label{fig:splitjoin-roundrobin-matrix}
\end{figure}

To perform the transformation from roundrobin to duplicate, the same rows
in each sub {\tt Stream}'s matrix must represent the same data items. Changing from splitjoin to 
duplicate implies that each sub {\tt streams} will be fed all of the input data. All data
that was nor originally bound for other {\tt stream}s needs to be ignored, which requires
changing the original {\tt stream}s $F_i$ to $F'_i$. In our linear
framework, we represent the independence of an output from an input as a zero in the appropriate
element of the linear representation matrix. Therefore, we transform $F_i$ to $F'_i$ by inserting
rows of zeros into linear representation of the sub {\tt streams} at positions such that
the data bound for another sub {\tt stream} is ignored. Each of the new matrices $F'_i$ 
has exactly $\sum_{i=0}^{N-1}v_i$ rows and the same number of columns of $F_i$, and 
$o_{F'_i}=o_{F_i}$.
Figure~\ref{fig:splitjoin-roundrobin-matrix} illustrates how the new matrix is constructed.
We have now transformed our {\tt splitjoin} with a roundrobin splitter into a {\tt splitjoin}
with a duplicate splitter which we know how to combine already.
