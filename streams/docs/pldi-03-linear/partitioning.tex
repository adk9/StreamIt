\begin{figure}[t]
  \psfig{figure=images/part-algorithm.eps,width=3.5in}
  \caption{Algorithm for optimization selection.
  \protect\label{fig:part-alg}}
\end{figure}

\begin{figure}[t]
  \psfig{figure=images/part-algorithm2.eps,width=3.5in}
  \caption{Type declarations for code in Figure~\ref{fig:part-alg}.}
\end{figure}

%% \begin{figure}[t]
%%   \psfig{figure=images/part-algorithm3.eps,width=3.5in}
%%   \caption{Cost functions for optimization selection.}
%% \end{figure}

\section{Optimization Selection}
\label{sec:partitioning}

To reap the maximum benefit from the optimizations described in the
previous two sections, it is important to apply them selectively.
There are two components of the optimization selection problem: first,
to determine the sequence of optimizations that will give the highest
performance for a given arrangement of the stream graph, and second,
to determine the arrangement of the stream graph that will give the
highest performance overall.  In this section, we explain the
relevance of each of these problems, and we present an effective
selection algorithm that relies on dynamic programming to quickly
explore a large space of possible configurations.

First, the selection of optimizations for a given stream graph can
have a large impact on performance.  As alluded to in
Section~\ref{sec:combine}, linear combination can increase the number
of arithmetic operations required, {\it e.g.} if combining a
two-element pipeline where the second filter pushes more items than it
peeks.  However, such a combination could be justified if it enables
further combination with other components and leads to a benefit
overall.  Another consideration is that a node can be converted to
frequency only if it has $\mt{pop}=1$; thus, excessive combination
could preclude frequency transformations.  This is the case for our
RateConverter benchmark, in which three filters could be combined, but
it is better to combine only two in order to enable a frequency
transformation (see Figure~\ref{fig:rategraph}).

Second, the arrangement of the stream graph can directly effect the
set of transformations that can be applied.  Since our transformations
operate on an entire {\tt pipeline} or {\tt splitjoin} construct, the
graph often needs to be refactored to put linear nodes in their own
hierarchical unit.  For example, in our TargetDetect benchmark, we can
divide the {\tt splitjoin} into two pieces and collapse the top piece
before converting to the frequency domain, thereby amortizing the cost
of the FFT on the input items (see Figure~\ref{fig:targetgraph}).

\subsection{Dynamic Programming Algorithm}

Our optimization selection algorithm automatically derives both of the
example transformations described above.  The algorithm works by
estimating the minimum cost for each structure in the stream graph.
The minimum cost represents the best of three configurations: 1)
collapsed and implemented in the time domain, 2) collapsed and
implemented in the frequency domain, and 3) uncollapsed and
implemented as a hierarchical unit.  The cost functions for the
collapsed cases are guided by profiler feedback, as described below.
For the uncollapsed case, the cost is the sum of each child's minimum
cost.  However, instead of considering the children directly, the
children are refactored into many different configurations, and the
cost is taken as the minimum over all configurations.  This allows the
algorithm to simultaneously solve for the best set of transformations
and the best arrangement of the stream graph.

The key to the algorithm's efficiency is its 

In order to consider multiple arrangements of the stream graph, the
algorithm considers many alternative arrangements of the children when
it performs this sum.  The cost for the uncollapsed case is then taken
as the minimum across all possible arrangements of the children.

ALGORITHM

- intuitive explanation

- reference to pseudocode

COST FUNCTION

we consider:
  - when/where a frequency transformation is possible
  - number of multiplies and adds (takes into account zeros, ones)
  - fact that frequency must operate on whole size of matrix
  - profiling feedback to get factor of 50

one could also consider:

  - efficacy of matrix multiply on a given node
    - architectural support for FIR operations
    - size of matrix; caching effects
    - symmetry or sparseness of matrix that could lead to more efficient implementation

  - overhead of runtime system
    - size of stream graph
    - number of splitters and joiners
    - on parallel system, could be additional load balancing issues

OPTIMIZING EXECUTION

can identify rectangles of stream graph that are equivalent, and use this to:
 - save on computing linear node representation
 - save on computing partitioning of childre
 - automatically compute closed form for some linear sections
