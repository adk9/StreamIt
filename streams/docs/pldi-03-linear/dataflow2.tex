\newcommand{\la}{$\leftarrow$}
\newcommand{\IND}{\begin{ALC@g}}
\newcommand{\UND}{\end{ALC@g}}
\newcommand{\tup}[2]{\langle{#1}, {#2}\rangle}

\section{Linear Extraction Algorithm}
\label{sec:dataflow}

Our linear extraction algorithm can identify a linear filter and
construct a linear node $\lambda$ that fully captures its behavior.
The technique, which appears as Algorithm~\ref{alg:dataflow} on the
next page, is a flow-sensitive, forward dataflow analysis similar to
constant propagation.  Unlike a standard dataflow analysis, we can
afford to symbolically execute all loop iterations, since most loops
within a filter's work function have small bounds that are known at
compile time (if a bound is statically unresolvable, the filter is
unlikely to be linear and we disregard it).

During symbolic execution, the algorithm computes the following for
each point of the program (refer to Figure~\ref{fig:types} for
notation):
\begin{itemize}

\vspace{-6pt}
\item A $\mt{map}$ between each program variable $y$ and a linear form
$\tup{\vec{v}}{c}$ where $\vec{v}[0 \cdots Peek]$ is a column vector 
and $c$ is a scalar constant. In an actual execution, the value of $y$ would
be given by $y = \vec{v} \cdot \vec{x} + c$, where $\vec{x}$ 
represents the input items.
\vspace{-6pt}

\item Matrix $A$ and vector $\vec b$, which will represent the linear node.
These values are constructed during the operation of the algorithm.
\vspace{-6pt}

\item $\mt{pushcount}$, which indicates how many items have been
pushed so far.  This is used to determine which column of $A$ and
$\vec{b}$ correspond to a given push statement.
\vspace{-6pt}

\item $\mt{popcount}$, which indicates how many items have been popped
so far.  This is used to determine the row in $\vec{v}$ that a 
given peek or pop expression refers to.
\vspace{-6pt}

\end{itemize}

We now briefly discuss the operation of {\bf Extract} at each program
node.  The algorithm is formulated in terms of a simplified set of
instructions, which appear in Figure~\ref{fig:types}.  First are the
nodes that generate fresh linear forms.  A constant assignment $y = c$
creates a form $\tup{\vec 0}{c}$ for $y$, since $y$ has constant part
$c$ and does not yet depend on the input.  A pop operation
creates a form $\tup{\mbox{\bf BuildCoeff}(\mt{popcount})}{0}$, where
{\bf BuildCoeff} introduces a coefficient of $1$ for the current index
on the input stream.  A peek$(i)$ operation is similar, but
offset by the index $i$.

\begin{algorithm}[t]
\caption{Linear extraction analysis.\protect\label{alg:dataflow}}
proc {\bf Toplevel}(filter $F$) returns linear node for $F$ \vspace{-4pt}
\begin{enumerate}
\item Set globals Peek, Pop, Push to I/O rates of filter $F$. \vspace{-4pt}
\item Let $A_{0} \leftarrow \mbox{new float[Peek, Push] with each entry =~} \bot$ \vspace{-14pt}
\item Let ${\vec b_{0}} \leftarrow \mbox{new float[Push] with each entry =~} \bot$ \vspace{-6pt}
\item $(\mt{map}, A, {\vec b}, \mt{popcount}, \mt{pushcount}) \leftarrow$ \\ 
\verb+      +{\bf Extract}$(F_{work}, (\lambda x . \bot), A_{0}, {\vec b_{0}}, 0, 0)$ \vspace{-6pt}
\item {\bf if} $A$ and ${\vec b}$ contain no $\top$ or $\bot$ entries {\bf then} \\
\verb+ + return linear node $\lambda = \{A, {\vec b}, \mbox{Peek}, \mbox{Pop}, \mbox{Push}\}$ \\
 {\bf else} \\
\verb+ + {\it fail} \\
 {\bf endif}
\end{enumerate}
proc {\bf BuildCoeff}(int $pos$) returns $\vec v$ for peek at index $pos$ \\ \vspace{-12pt}
\begin{algorithmic}
\STATE $\vec{v} = \vec{0}$
\STATE $\vec{v}[\mbox{Peek}-1-\mt{pos}] = 1$
\STATE return $\vec{v}$
\end{algorithmic}
\end{algorithm}

\begin{figure}[t]
\vspace{-12pt}
\begin{equation} \nonumber
\begin{array}{rcl}
y & \in & \mbox{program-variable} \\
c & \in & \mbox{constant}^{\top} \\
\vec v, \vec b & \in & \mbox{vector}^{\top} \\
\tup{\vec v}{c} & \in & \mbox{linear-form}^{\top} \\
map & \in & \mbox{program-variable} \rightarrow \mbox{linear-form  (a hashtable)} \\
A & \in & \mbox{matrix}^{\top} \\
code & \in & \mbox{list of instructions, each of which can be:} \\
\multicolumn{3}{l}{\parbox{3in}{
    \vspace{-6pt}
    \begin{equation} \nonumber
      \begin{array}{ll}
	y_1 := \mt{const} & \mbox{{\tt push}}(y_1) \\
	y_1 := \mbox{\tt pop}() & (\mbox{{\tt loop}} ~N~ \mt{code}) \\
        y_1 := \mbox{\tt peek}(i) & (\mbox{{\tt branch}} ~\mt{code}_1~ \mt{code}_2) \\
	y_1 := y_2~\mt{op}~y_3 & ~
      \end{array}
\end{equation}}}
\end{array} 
\end{equation}
\vspace{-18pt}
\caption{Data types for the extraction analysis.\protect\label{fig:types}}
\vspace{-14pt}
\makeline
\vspace{-14pt}
\end{figure}

\begin{algorithm}
proc {\bf Extract}($code$, $map$, $A$, $\vec b$, int $\mt{popcount}$, int $\mt{pushcount}$) \\
\verb+   + returns updated $\mt{map}$, $A$, ${\vec b}$, $\mt{popcount}$, and $\mt{pushcount}$ \\ \vspace{-12pt}
\begin{algorithmic}
\FOR {$i \leftarrow 1$ to $\mt{code}$.length}
\STATE {\bf switch} $\mt{code}$[i]
\IND
\STATE $\mbox{\bf case}~y := \mt{const}$
\IND
\STATE $\mt{map}.\mt{put}(y, (\vec 0, \mt{const}))$
\UND
\STATE \vspace{-6pt}
\STATE $\mbox{\bf case}~y := \mbox{\tt pop}()$
\IND
\STATE $\mt{map}.\mt{put}(y, \tup{\mbox{\bf BuildCoeff}(\mt{popcount})}{0})$
\STATE $\mt{popcount}$\verb|++|
\UND
\STATE \vspace{-6pt}
\STATE $\mbox{\bf case}~y := \mbox{\tt peek}(i)$
\IND
\STATE $\mt{map}.\mt{put}(y, \tup{\mbox{\bf BuildCoeff}(\mt{popcount}+i)}{0})$
\UND
\STATE \vspace{-6pt}
\STATE $\mbox{\bf case}~\mbox{\tt push}(y)$
\IND
\STATE $\tup{\vec v}{c} \leftarrow \mt{map}.\mt{get}(y)$
\STATE {\bf if} $\mt{pushcount} = \top$ {\bf then} $\mt{fail}$
\STATE $A[*, \mbox{Push} - 1 - \mt{pushcount}] \leftarrow \vec v$
\STATE $\vec{b}[\mbox{Push} - 1 - \mt{pushcount}] \leftarrow c$
\STATE $\mt{pushcount}$\verb|++|
\UND
\STATE \vspace{-6pt}
\STATE $\mbox{\bf case}~ y_1 := y_2 \mt{~op~} y_3$, for $\mt{~op~} \in \{+, -\}$
\IND
\STATE $\tup{\vec v_2}{c_2} \leftarrow \mt{map}.\mt{get}(y_2)$
\STATE $\tup{\vec v_3}{c_3} \leftarrow \mt{map}.\mt{get}(y_3)$
\STATE $\mt{map}.\mt{put}(y_1, \tup{\vec v_2 \mt{~op~} \vec v_3}{c_2 \mt{~op~} c_3})$
\UND
\STATE \vspace{-6pt}
\STATE $\mbox{\bf case}~y_1 := y_2 * y_3$
\IND
\STATE $\tup{\vec v_2}{c_2} \leftarrow \mt{map}.\mt{get}(y_2)$
\STATE $\tup{\vec v_3}{c_3} \leftarrow \mt{map}.\mt{get}(y_3)$
\STATE {\bf if} $~\vec v_2 = \vec 0$ {\bf then}
\IND
\STATE $\mt{map}.\mt{put}(y_1, (c_2*\vec v_3, c_2*c_3))$
\UND
\STATE {\bf else if} $~\vec v_3=\vec 0$ {\bf then}
\IND
\STATE $\mt{map}.\mt{put}(y_1, (c_3*\vec v_2, c_3*c_2))$
\UND
\STATE {\bf else}
\IND
\STATE $\mt{map}.\mt{put}(y_1, \top)$
\UND
\UND
\STATE \vspace{-6pt}
\STATE $\mbox{\bf case}~y_1 := y_2 / y_3$
\IND
\STATE $\tup{\vec v_2}{c_2} \leftarrow \mt{map}.\mt{get}(y_2)$
\STATE $\tup{\vec v_3}{c_3} \leftarrow \mt{map}.\mt{get}(y_3)$
\STATE {\bf if} $~\vec v_3 = \vec 0 \wedge c_3 \ne 0$ {\bf then}
\IND
\STATE $\mt{map}.\mt{put}(y_1, (\frac{1}{c_3}*\vec v_2, c_2/c_3))$
\UND
\STATE {\bf else}
\IND
\STATE $\mt{map}.\mt{put}(y_1, \top)$
\UND
\UND
\STATE \vspace{-6pt}
\STATE $\mbox{\bf case}~y_1 := y_2 ~\mt{op}~ y_3$, for $\mt{op} \in \{\&, |, \wedge, \&\&, ||, !, \mt{etc.}\}$
\IND
\STATE $\tup{\vec v_2}{c_2} \leftarrow \mt{map}.\mt{get}(y_2)$
\STATE $\tup{\vec v_3}{c_3} \leftarrow \mt{map}.\mt{get}(y_3)$
\STATE $\mt{map}.\mt{put}(y_1, (\vec 0 \sqcup \vec v_2 \sqcup \vec v_3, c_2 ~\mt{op}~ c_3))$
\UND
\STATE \vspace{-6pt}
\STATE \mbox{\bf case}~({\tt loop} N $code'$)
\IND
\STATE \bf{for} $j \leftarrow 1$ to $N$ {\bf do}
\IND
\STATE $(\mt{map}, A, {\vec b}, \mt{popcount}, \mt{pushcount})~\leftarrow~$ \\
\verb+   +\bf{Extract}$(\mt{code}, \mt{map}, A, {\vec b}, \mt{popcount}, \mt{pushcount})$
\UND
\UND
\STATE \vspace{-6pt}
\STATE \mbox{\bf case}~({\tt branch} $code_1~code_2)$
\IND
\STATE $(\mt{map}_1, A_1, {\vec b_1}, \mt{popcount}_1, \mt{pushcount}_1) \leftarrow$ \\
\verb+   +${\mbox{\bf Extract}}(\mt{code}_1, \mt{map}, A, {\vec b}, \mt{popcount}, \mt{pushcount})$
\STATE $(\mt{map}_2, A_2, {\vec b_2}, \mt{popcount}_2, \mt{pushcount}_2) \leftarrow$ \\ 
\verb+   +${\mbox{\bf Extract}}(\mt{code}_2, \mt{map}, A, {\vec b}, \mt{popcount}, \mt{pushcount})$
\STATE $\mt{map} \leftarrow \mt{map}_1 \sqcup \mt{map}_2$
\STATE $A \leftarrow A_1 \sqcup A_2$
\STATE ${\vec b} \leftarrow {\vec b_1} \sqcup {\vec b_2}$
\STATE $\mt{popcount} \leftarrow \mt{popcount}_1 \sqcup \mt{popcount}_2$
\STATE $\mt{pushcount} \leftarrow \mt{pushcount}_1 \sqcup \mt{pushcount}_2$
\UND
\UND %end case
\ENDFOR
\STATE return ($\mt{map}$, $A$, ${\vec b}$, $\mt{popcount}$, $\mt{pushcount}$)
\end{algorithmic}
\end{algorithm}

Next are the instructions which combine linear forms.  In the case of
addition or subtraction, we simply add the components of the linear
forms.  In the case of multiplication, the result is still a linear
form if either of the terms is a known constant ({\it e.g.}
$\vec{v}=\vec{0}$).  For division, the result is linear only if the
divisor is a non-zero constant\footnote{{\small Note that if the
dividend is zero and the divisor has a non-zero coefficients vector,
we cannot conclude that the result is zero, since certain runtime
inputs might cause a singularity.}} and for non-linear operations ({\it
e.g.,} bit-level and boolean), both operands must be known constants.
If any of these conditions are not met, then the LHS is assigned a
value of $\top$, which will mark the filter as non-linear if the value
is ever pushed.

The final set of instructions deal with control flow.  For loops, we
resolve the bounds at compile time and execute the body an appropriate
number of times.  For branches, we have to ensure that all the linear
state is modified consistently on both sides of the branch.  For this
we apply the confluence operator $\sqcup$, which we define for scalar
constants, vectors, matrices, linear forms, and maps.  $c_1 \sqcup
c_2$ is defined according to the lattice constant$^{\top}$.  That is,
$c_1 \sqcup c_2 = c_1$ if and only if $c_1 = c_2$; otherwise, $c_1
\sqcup c_2 = \top$.  For vectors, matrices, and linear forms, $\sqcup$
is defined element-wise; for example, $A' = A_1 \sqcup A_2$ is
equivalent to $A'[i,j] = A_1[i,j] \sqcup A_2[i,j]$.  For maps, the
join is taken on the values: $\mt{map}_1 \sqcup \mt{map}_2$ $=$
$\mt{map'}$, where $\mt{map'}.\mt{get}(x) = \mt{map}_1.\mt{get}(x)
\sqcup \mt{map}_2.\mt{get}(x)$.

Our implementation of linear extraction is also interprocedural.  It
is straightforward to transfer the linear state across a call site,
although we omit this from the pseudocode for the sake of
presentation.  Also implicit in the algorithm description is the fact
that all variables are local to the work function.  If a filter
has persistent state, all accesses to that state are marked as $\top$.
