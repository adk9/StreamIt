\section{Linear Representations}
\begin{figure}
\center
\epsfxsize=3.0in
\epsfbox{images/general-picture-matrix.eps}
\caption{Linear filter as a vector-matrix operation}
\label{fig:overview-matrix}
\vspace{-12pt}
\end{figure}

There is no general relationship between a {\tt filter}'s input data and
its output data. In actual applications, however, the output is typically 
derived from the input, but the relationship is not always clear 
since a {\tt filter} has state and can call external functions.

However, we note that a large subset of DSP operations produce outputs
that are some affine function of their input, and we call {\tt filters} that 
implement such operations ``linear.'' To use standard DSP notation to talk about
StreamIt programs, we will call the $e_F$ items that {\tt filter} $F$ uses as input
$x[n]$, where  $n\in[0,e_F-1]$. Similarly, we  
define the output of a {\tt filter} as $y[n]$ where  $n\in[0,push]$. For 
linear {\tt filter}, the output can be related to the input by 
$y[k] = (\sum_{i=0}^{e_F-1} a_{k,i}x[i])+b_k$.
where $a_{k,i}$ and $b_{k}$ are constants with complex values. 

Each $y[k]$ can be represented as the dot product of two vectors, plus a scalar constant. 
Thus, we can represent the action of the entire linear {\tt filter} 
in an affine matrix-vector equation as illustrated in 
Figure~\ref{fig:overview-matrix}.

The input data is a row vector $x$ of length $e_F$
and the output data is a row vector $y$ of length $u_F$. The {\tt work} function
of $F$ is represented as a weight matrix $A$ and a constant row vector $b$. 
$A$ has $e_F$ rows and $u_F$ columns while $b$ has $1$ row and $u_F$ columns. 
$x$ and $y$ are related by $y$ = $xA + b$. To fully characterize the computation of $F$ 
we need only to store the triple ($A$, $b$, $o_F$) which we call 
the ``linear representation'' of $F$.

It is instructive to interpret the the linear representations of a {\tt filter}
in the following way. The values in column $i$ of $A$ and column $i$ of $b$ 
represent the formula to compute the $(u_F-i)$th output. 
Row $j$ of $A$ represents the weights given to the $(e_F-j)$ input element 
when computing each output. The rightmost column of $A$ represents the first output element 
that is produced (eg the value of the first {\tt push()} statement). The bottom row 
of $A$ represents the weight use for the first input 
(eg {\tt peek(0)}), the second to bottom row the second input ({\tt peek(1)}) and so forth.




