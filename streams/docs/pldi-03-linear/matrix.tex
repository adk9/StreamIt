\section{Linear Nodes}
\begin{figure}
\center
\epsfxsize=3.0in
\epsfbox{images/general-picture-matrix.eps}
\caption{Linear filter as a vector-matrix operation}
\label{fig:overview-matrix}
\vspace{-12pt}
\end{figure}

There is no general relationship between a {\tt filter}'s input data and
its output data. In actual applications, however, the output is typically 
derived from the input, but the relationship is not always clear 
since a {\tt filter} has state and can call external functions.

However, we note that a large subset of DSP operations produce outputs
that are some affine function of their input, and we call {\tt filters} that 
implement such operations ``linear.'' To use DSP notation for {\tt streams} in
StreamIt, we will call the items that are peeked and used as input
$x[n]$, where  $n\in[0,e-1]$. Similarly, we define the output of 
a {\tt stream} as $y[n]$ where  $n\in[0,u]$. For a linear {\tt stream}, 
the output is related to the input by $y[k] = (\sum_{i=0}^{e-1} a_{k,i}x[i])+b_k$.
where $a_{k,i}$ and $b_{k}$ are comlpex valued constants. 

Each output element can be represented as the dot product of two vectors 
plus a scalar constant.  Thus, we can represent the action of the 
entire linear {\tt stream} in the affine matrix-vector equation ${\mathbf x} {\mathbf A} + {\mathbf b}$ 
as illustrated in Figure~\ref{fig:overview-matrix}.

The input is a row vector ${\mathbf x}$ of length $e$
and the output is a row vector ${\mathbf y}$ of length $u$. The {\tt work} function
is characterized by a matrix ${\mathbf A}$ and a constant vector ${\mathbf b}$. 
${\mathbf A}$ has $e$ rows and $u$ columns and $b$ has $1$ row and $u$ columns. 
${\mathbf x}$ and ${\mathbf y}$ are related by ${\mathbf y} = {\mathbf x}{\mathbf A} + {\mathbf b}$. 
We characterize the computation of a linear {\tt stream} with the tuple 
$\lambda = ({\mathbf A},{\mathbf b},e,o,u)$ which we call a linear node.

The intuition of the computation represented by a linear node is simply
that specific columns generate specific outputs and specfic rows 
correspond to using specific inputs.
The values in column $u-i-1$ of ${\mathbf A}$ and column $u-i-1$ of ${\mathbf b}$ 
(ie the $i$th column from the right) represent the formula to compute the $i$th output.
The values found in row $e-j-1$ of ${\mathbf A}$ represents the weights given to the $j$th 
input element when computing each output.




