\begin{figure}[t]
\vspace{-6pt}
\center
\epsfxsize=3.0in
\epsfbox{images/streamit-structures.eps}
\vspace{-12pt}
\caption{StreamIt structures: {\tt pipeline}, {\tt splitjoin}, and {\tt feedbackloop}.
\protect\label{fig:structures}}
\vspace{-12pt}
\end{figure}

\section{Representing Linear Nodes}
\label{sec:linearrep}

There is no general relationship that must hold between a {\tt
filter}'s input data and its output data. In actual applications, the
output is typically derived from the input, but the relationship is
not always clear since a filter has state and can call external
functions.

However, we note that a large subset of DSP operations produce outputs
that are some affine function of their input, and we call
filters that implement such operations ``linear.''  Examples of such
filters are finite impulse response (FIR) filters, compressors,
expanders and signal processing transforms such as the discrete
Fourier transform (DFT) and discrete cosine transformation (DCT).  Our
formal definition of a linear node is as follows (refer to
Figure~\ref{fig:overview-matrix} for an illustration). \vspace{-4pt}
\begin{definition}(Linear node)
A linear node $\lambda$ $=$ $\{A,$ ${\vec b},$ $e,$ $o,$ $u\}$
represents an abstract stream block which performs an affine
transformation $y = x A + {\vec b}$ from input elements $x$ to output
elements $y$.  $A$ is a $e \times u$ matrix, ${\vec b}$ is a
$u$-element row vector, and $e$, $o$, and $u$ are the peek, pop, and
push rates, respectively. \\ ~ \vspace{-8pt} \\
%
A ``firing'' of a linear node $\lambda$ corresponds to the following
series of abstract execution steps.  First, an $e$-element row vector
$x$ is constructed with $x[i] = \mbox{\tt peek}(i)$.  The node
computes $y = x A + {\vec b}$, and then pushes the $u$ elements of $y$ onto
the output tape, starting with $y[u-1]$ and proceeding through $y[0]$.
Finally, $o$ items are popped from the input tape.
\end{definition} \vspace{-2pt}

\begin{figure}[t]
\vspace{-6pt}
\center
\epsfxsize=3.0in
\epsfbox{images/general-picture-matrix.eps}
\vspace{-12pt}
\caption{Linear filter as a vector-matrix operation}
\label{fig:overview-matrix}
\vspace{-12pt}
\end{figure}

%% To use DSP notation for {\tt streams} in StreamIt, we will call the
%% items that are peeked and used as input $x[n]$, where
%% $n\in[0,e-1]$. Similarly, we define the output of a {\tt stream} as
%% $y[n]$ where $n\in[0,u]$. For a linear {\tt stream}, the output is
%% related to the input by $y[i] = (\sum_{j=0}^{e-1} a_{i,j}x[j])+b_i$.
%% where $a_{i,j}$ and $b_{i}$ are complex-valued constants.

%% Each output element can be represented as the dot product of two
%% vectors plus a scalar constant.  Thus, we can represent the action of
%% the entire linear {\tt stream} in the affine matrix-vector equation
%% ${\mathbf x} {\mathbf A} + {\mathbf b}$ as illustrated in
%% Figure~\ref{fig:overview-matrix}.

%% The input is a row vector ${\mathbf x}$ of length $e$ and the output
%% is a row vector ${\mathbf y}$ of length $u$. The {\tt work} function
%% is characterized by a matrix ${\mathbf A}$ and a constant vector
%% ${\mathbf b}$.  ${\mathbf A}$ has $e$ rows and $u$ columns and $b$ has
%% $1$ row and $u$ columns.  ${\mathbf x}$ and ${\mathbf y}$ are related
%% by ${\mathbf y} = {\mathbf x}{\mathbf A} + {\mathbf b}$.  We
%% characterize the computation of a linear {\tt stream} with the tuple
%% $\lambda = ({\mathbf A},{\mathbf b},e,o,u)$ which we call a linear
%% node.

The intuition of the computation represented by a linear node is
simply that specific columns generate specific outputs and specific
rows correspond to using specific inputs.  The values found in row
$e-i-1$ of ${\mathbf A}$ represents the weights given to the $i$th
input element when computing each output.  The values in column
$u-j-1$ of ${\mathbf A}$ and column $u-j-1$ of ${\mathbf b}$ ({\it
i.e.} the $j$th column from the right) represent the formula to
compute the $j$th output.