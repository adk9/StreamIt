As DSP programming is becoming more complex, there is an increasing
need for high-level abstractions that can be efficiently compiled.
Towards this end, we present a set of aggressive optimizations that
target linear sections of a stream program.  Our input language is
StreamIt, which represents programs as a hierarchical graph of
autonomous filters.  A filter is linear if each of its outputs can be
represented as an affine combination of its inputs.  Linear filters
are common in DSP applications; examples include FIR filters,
expanders, compressors, FFTs and DCTs.

We present a linear extraction analysis that automatically detects
linear filters based on the C-like code in their {\tt work} function.
Once linear filters are identified, we show how neighboring nodes can
be collapsed into a single linear representation, thereby eliminating
many redundant computations.  Also, we describe a method for
automatically translating linear nodes into the frequency domain,
thereby yielding algorithmic savings for convolutional filters.

We have completed a fully-automatic implementation of the above
techniques as part of the StreamIt compiler.  \todo{Performance
number.}

