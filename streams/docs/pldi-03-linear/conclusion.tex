\section{conclusion}


\subsection{Future Work}
We could extend linear analysis to incorporate the set of filters
that has a state between executions. Our formulation would then be very
much like a standard state space model. The current state would be a set of
variables. Each iteration of the filter would update the new state to be
a linear combination of the input and the current state. The output of the 
filter would also be some linear combination of the input and the current state.
This representation would be able to capture the fundamental operation of
a large set of programs -- specifically digital control systems (where state
space models are used frequently to model real systems ) and IIR filters (which
is represented as a feed back loop). An transformation that we could do with 
a feed back loop that we determine is IIR would be to calculate enough of its
frequency response so that its behavior could be captured in a FFT, multiplie,
IFFT sequence. Another possible use would be to remove a feedback loop that was
performing IIR filtering and convert it into a single linear state filter.  
