\section{conclusion}

(this is the conclusion)



\subsection{Future Work}

Linear analysis is easily extended to incorporate a notion of linear state.
A stateful linear node is characterized by

\begin{equation} \nonumber
\lambda=(({\mathbf A}_x, {\mathbf A}_s), ({\mathbf C}_x, {\mathbf C}_s), 
({\mathbf b}_x, {\mathbf b}_s))
\end{equation}

Each {\tt filter} in the streamgraph contains a state vector ${\mathbf s}$
such that ${\mathbf x}$ at time $i$ and ${\mathbf s}$ at time $i+1$ is given by

\begin{equation} \nonumber
y_i={\mathbf A}_o{\mathbf x} + {\mathbf A}_s{\mathbf s}_i + {\mathbf b}_x
\end{equation}
\begin{equation} \nonumber
{\mathbf s}_{i+1}={\mathbf C}_x{\mathbf x} + {\mathbf C}_s{\mathbf s}_i + {\mathbf b}_s
\end{equation}

The additon of stateful nodes allows us to describe a larger class of programs 
using our linear analysis framework.
Using linear state, our structure combination rules can be extended to include {\tt feedbackloops}.
Examples of programs that exhibit stateful linear nodes are control systems
and infinite impulse response (IIR) filters.

Another promising avenue of research is attempting to exploit work done
on matrix factorizations in order to automatically derive fast implementations of 
large compuatations such as DSP transforms.

To increase the class of programs that would fit into our linear framework, 
the entries of the ${\mathbf A}$ and ${\mathbf b}$ of linear nodes could contain symbolic
runtime resolvable information.


\subsection{Related Works}
Several interesting projects are underway to determine how to generate efficient programs
for DSP applications. The SPIRAL project\cite{spiral, xiong-thesis,xiong01spl,johnson01searching,egner01automatic} 
aims to determine the fastest implementation for various transforms DSP transforms
using several techniques such as group theory and stochastic search algorithms. SPIRAL
requires the user to enter a matrix corresponding to the desired transform, whereas our
system does this automatically. 
The ATLAS project \cite{whaley01automated} aims to automatically produce fast
libraries for linear algebra manipulations, focusing on adaptive library generation. 
FFTW \cite{frigo99fast, fftw} is a runtime tuned library for computing the FFT as
fast as possible.

% Something should be said about other stream projects and their attempts at optimizations...
% which I know very little about.

