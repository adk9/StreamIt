\section{Conclusion}
\label{sec:conclusion}

This paper presents a set of automated analyses for detecting,
analyzing, and optimizing linear filters in streaming applications.
Though the mathematical optimization of linear filters has been a
longtime focus of the DSP community, our techniques are novel in the
automated application of these techniques to programs that are written
in a flexible and high-level programming language.  We demonstrate
that using our linear dataflow analysis, linear combination, 
frequency translation and automated optimization selection
we can improve execution speed by a factor of $XXXXXXXXX$.

The ominous rift between the design and implementation of signal 
processing applications is growing by the day.  Algorithms are
designed at a conceptual level utilizing modular processing blocks that
that naturally express the computation.  However, in order to obtain
good performance, each hand-tuned implementation is forced to
disregard the abstraction layers and painstakingly consider
specialized whole-program optimizations. The StreamIt project aims to
reduce this process to a single stage in which the designers and
implementors share a set of high-level abstractions that can be
efficiently handled by the compiler.  

The linear analysis described in this paper represents a first step
toward this goal.  By automatically performing linear combination and
frequency translation, it allows programmers to write linear stream
operations in a natural and modular fashion without any performance
penalty.

%% \subsection{Future Work}

%% Linear analysis can easily be extended to incorporate a notion of
%% linear state.  A stateful linear node is characterized by
%% \begin{equation} \nonumber
%% \lambda=(({\mathbf A}_x, {\mathbf A}_s), ({\mathbf C}_x, {\mathbf C}_s), 
%% ({\mathbf b}_x, {\mathbf b}_s))
%% \end{equation}
%% Each {\tt filter} in the stream graph contains a state vector
%% ${\mathbf s}$ such that ${\mathbf y}$ at time $i$ and ${\mathbf s}$ at
%% time $i+1$ is given by
%% \begin{equation} \nonumber
%% y_i={\mathbf A}_o{\mathbf x} + {\mathbf A}_s{\mathbf s}_i + {\mathbf b}_x
%% \end{equation}
%% \begin{equation} \nonumber
%% {\mathbf s}_{i+1}={\mathbf C}_x{\mathbf x} + {\mathbf C}_s{\mathbf s}_i + {\mathbf b}_s
%% \end{equation}
%% The additon of stateful nodes allows us to describe a larger class of programs 
%% using our linear analysis framework.
%% Using linear state, our structure combination rules can be extended to include {\tt feedbackloops}.
%% Examples of programs that exhibit stateful linear nodes are control systems
%% and infinite impulse response (IIR) filters.

%% Another promising avenue of research is to exploit matrix
%% factorization in order to automatically derive fast implementations of
%% large compuatations such as DSP transforms.  Matrix representations
%% will also be useful for generating load-balanced parallel code in the
%% StreamIt RAW backend~\cite{streamit-asplos}.  Finally, to increase the
%% class of programs that would fit into our linear framework, the
%% entries of the ${\mathbf A}$ and ${\mathbf b}$ of linear nodes could
%% contain symbolic constants that are resolved at runtime.

