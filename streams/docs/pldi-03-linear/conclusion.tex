\section{conclusion}




\subsection{Future Work}
Linear analysis is easily extended to incorporate a notion of linear state.
Our formulation would then be very much like a standard state space model used 
frequently in the area of control system analysis. A stateful linear node would
be characterized by
\begin{equation} \nonumber
\lambda=(({\mathbf A}_x, {\mathbf A}_s), ({\mathbf C}_x, {\mathbf C}_s), 
({\mathbf b}_x, {\mathbf b}_s))
\end{equation}

Each {\tt filter} in the streamgraph contains a state vector ${\mathbf s}$
such that the output at time $i$ is given by 
$y_i={\mathbf A}_o{\mathbf x} + {\mathbf A}_s{\mathbf s} + {\mathbf b}_x$
and the state vector at time $i+1$ is given by
$s_{i+1}={\mathbf C}_x{\mathbf x} + {\mathbf C}_s{\mathbf s} + {\mathbf b}_s$

Each iteration of the filter would update the new state to be
a linear combination of the input and the current state. The output of the 
filter would also be some linear combination of the input and the current state.
This representation would be able to capture the fundamental operation of
a large set of programs -- specifically digital control systems (where state
space models are used frequently to model real systems ) and IIR filters (which
is represented as a feed back loop). An transformation that we could do with 
a feed back loop that we determine is IIR would be to calculate enough of its
frequency response so that its behavior could be captured in a FFT, multiplie,
IFFT sequence. Another possible use would be to remove a feedback loop that was
performing IIR filtering and convert it into a single linear state filter.  




\subsection{Related Works}
Several interesting projects are underway to determine how to generate efficient programs
for computing DSP transforms that are expressed as matrices. For instance, the SPIRAL
project\cite{spiral} aims to determine the fastest implementation for various transforms
using several techniques ranging from group theory to stochastic search algorithms. See
 \cite{xiong-thesis,xiong01spl,johnson01searching,egner01automatic}.
Similiarly, the ATLAS project \cite{whaley01automated} aims to automatically produce fast
performance tuned libraries for linear algebra manipulations.

