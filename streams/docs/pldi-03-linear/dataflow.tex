\section{Dataflow Analysis}
\begin{figure}[t]
\scriptsize
\begin{verbatim}
float->float filter LowPassFilter(float g, float cutoffFreq, int N) {
  float[N] h;

  /* since the impulse response is symmetric, 
   * don't worry about reversing h[n]. */
  init {
    int OFFSET = N/2;
    for (int i=0; i<N; i++) {
      int idx = i + 1;
      // generate real part
      if (idx == OFFSET) 
	/* take care of div by 0 error 
	(lim x->oo of sin(x)/x actually equals 1)*/
	h[i] = g * cutoffFreq / pi; 
      else 
	h[i] = g * sin(cutoffFreq * (idx-OFFSET)) / (pi*(idx-OFFSET));
    }
  }

  /* implement the FIR filtering operation 
   * as the convolution sum. */
  work peek N pop 1 push 1 {
    float sum = 0;
    for (int i=0; i<N; i++) { 
      sum += h[i]*peek(i);
    }
    push(sum);
    pop();
  }
}

\end{verbatim}
\vspace{-12pt}
\caption{\protect\small StreamIt code for a low pass filter.
\protect\label{fig:lowpasscode}}
\vspace{-12pt}
\end{figure}


To understand the dataflow analysis used to extract the linear representation from
a {\tt Filter}'s {\tt work} function, one needs to understand what the code of a 
{\tt work} function contains. Most of the code is arbitrary C-like code that would
be very familiar to a C programmer. There are three function calls that have special 
meaning in a {\tt work} function.

\begin{enumerate}
\item {\tt pop()}: Consumes one data item from the input tape and returns its value.
\item {\tt peek(i)}: Returns the value of the $i$th value on the input tape.
\item {\tt push(v)}: pushes the value of {\tt v} on to the output tape.
\end{enumerate}


Figure~\ref{fig:lowpasscode} contains the code to implement a rectangularly windowed 
low pass filter. Note that the {\tt init} function gets called once when the program
starts up.
 
The central idea of the data analysis pass is very much like constant propagation
except that we keep mappings from variables to linear forms instead of variables to
constants. Linear forms are pairs ($v$, $c$) of column vectors $v$ of size $peek_{F}$ 
and a constant $c$. A linear form represents a formula for computing a value from a
linear combination of the {\tt Filter}'s inputs such that the value of a 
linear form at runtime would be $xv + c$.

A linear form can be created in one of three ways. 
\begin{enumerate}
\item A constant $r$ generates a linear form with $v=0$ and $c=r$. 
\item A {\tt peek(i)} statement generates a linear form with $v$ a column vector of all zeros with 
a one in the $i$th row from the bottom and $c=0$. 
\item A {\tt pop()} expression generates a linear form with a one in the row that is associated with 
the current input element. 
\end{enumerate}

Of course, there is record keeping to calculate how many {\tt pop()}s
the {\tt work} function has performed up to a particular point so that the correct
row can be set to one when we encounter {\tt pop()}s and {\tt peeks(i)}s.

Linear forms propagate through the dataflow graph much like constants are propagated
during a standard constant propagation pass. Two linear forms $l_{1}$ and $l_{2}$ can
be added together by adding their vectors element wise and adding their constants, eg 
$l_1+l_2=(v_1+v_2, c_1+c_2)$. Linear forms can only be multiplied by constants $r$, 
the result of which is simply a scaling of all the elements of $v$ and of $c$ by a factor
$r$. Eg $l*r = (rv,rc)$. If two linear forms are multiplied together 
then the result is not a linear combination of the {\tt Filter}'s input, rather it is
some polynomial function in the input.

{\tt if} statements are handled just like for constant propagation. The confluence operator
is set intersection, which only propagates variable to linear form mappings that are
maintained along both branches of control flow. We also have to impose the special restriction that the
same number of {\tt pops()} occured along both branches or we don't know what the current offset
for generating linear forms from {\tt peek(i)} and {\tt pop()} expressions is. 
We currently only handle loops that can be unrolled in the 
compiler (eg those loops whose bounds are determinable at compile time). 
A loop can't be unrolled at compile time only if its terminiation condition is data 
dependent, and data dependent loops probably don't compute linear 
functions of the data anyways.


The final piece of generating a linear representation from a linear form is to 
When the {\tt expr} in a {\tt push(expr)} expression is a linear form $l=(v,c)$, 
we copy $v$ to the appropriate column in the {\tt Filter}'s linear representation matrix $A$ 
and we copy $c$ to the appropriate column in the linear representation's constant vector $a$.
The first {\tt push()} expression places the linear form in the last column of $A$. The second
{\tt push()} the second to last column of $A$, etc. The last column of $A$ represents 
the calculations necessary for the first output item produced by the filter. The second to last
column represents the calculatons for the second output item of the filter, etc.
The same positions are chosen for the linear forms' constants in the constant vector $a$.

If all the {\tt push} expressions in a filter have linear forms as arguments, 
then we have determined that the filter is linear and remember the mapping from 
filter to linear representation.
