\section{Linear Dataflow Analysis}
\label{sec:dataflow}

%\begin{figure}[t]
%\scriptsize
%\begin{verbatim}
%float->float filter LowPassFilter(float g, float cutoffFreq, int N) {
%  float[N] h;%
%
%  /* since the impulse response is symmetric, 
%   * don't worry about reversing h[n]. */
%  init {
%    int OFFSET = N/2;
%    for (int i=0; i<N; i++) {
%      int idx = i + 1;
%      // generate real part
%      if (idx == OFFSET) 
%	/* take care of div by 0 error 
%	(lim x->oo of sin(x)/x actually equals 1)*/
%	h[i] = g * cutoffFreq / pi; 
%      else 
%	h[i] = g * sin(cutoffFreq * (idx-OFFSET)) / (pi*(idx-OFFSET));
%    }
%  }
%
%  /* implement the FIR filtering operation 
%   * as the convolution sum. */
%  work peek N pop 1 push 1 {
%    float sum = 0;
%    for (int i=0; i<N; i++) { 
%      sum += h[i]*peek(i);
%    }
%    push(sum);
%    pop();
%  }
%}
%
%\end{verbatim}
%\vspace{-12pt}
%\caption{\protect\small StreamIt code for a low pass filter.
%\protect\label{fig:lowpasscode}}
%\vspace{-12pt}
%\end{figure}

%Figure~\ref{fig:lowpasscode} contains the code to implement a rectangularly windowed 
%low pass filter. Note that the {\tt init} function gets called once when the program
%starts up.


Since each {\tt filter}'s {\tt work} function can contain arbitrary C-like code,
the compiler must determine which {\tt filters} can be represented by linear nodes
automatically. 
 
The data analysis pass that is currently employed is very much like standard constant propagation. 
Instead of mappings from variables to compile time constants, we keep mappings to linear forms.
A linear forms $l$ is defined as a pair $({\mathbf v}, c)$ of a column vector $\mathbf{v}$ of size $e$ 
and a constant $c$. For each program variable, its linear form at a particular program 
point corresponds to the formula for computing a value from a linear 
combination of {\tt filter} inputs. The value of a linear form at runtime is 
${\mathbf xv} + c$.

The analysis must keep track of how many {\tt pop()}s the {\tt work} 
function has performed at any particular program point so that 
the position of the data returned by each {\tt pop()} or {\tt peek(i)} 
encountered is determinable. We define $popcount$ for each program point 
as the number of {\tt pops} encountered along all possible
execution paths. 
$pushcount$ is defined to be the number of push expressions already
seen at a particular program point.
If different execution paths could cause $pushcount$ or 
$popcount$ to take inconsitent values, then the appropriate
variable becomes undefined.
At each point in the program, we keep a candidate linear node 
$\lambda=({\mathbf A}, {\mathbf b}, e,o,u)$ that represents the 
computations necesary to produce the output generated up to that 
point.

We define ${\mathbf v} = zerov(k)$ a row vector of size $e$ such that 
$v[i]=0, i \neq k$, and $v[i]=1,i=k$.

A linear form is generated in one of he following three ways:
\begin{enumerate}
\item A constant $r$ generates the linear form $({\mathbf 0},r)$. 
\vspace{-6pt}

\item A {\tt peek(i)} statement generates the linear form 
$(zerov(u-i-1-popcount), 0)$.
\vspace{-6pt}

\item A {\tt pop()} expression generates the linear form 
$(zerov(u-1-popcount),0)$

\vspace{-6pt}
\end{enumerate}

Linear forms propagate through the control flow graph much as constants do
during constant propagation. The value of two expressions with linear forms $l_1$ 
and $l_2$ is a new linear form $l'=l_1+l_2=({\mathbf v}_1+{\mathbf v}_2, c_1+c_2)$.

The product of a linear form with a constant $r$, is a new linear form 
$l' (r{\mathbf v},rc)$. The product of two linear forms does not produce a 
new linear form because it corresponds to multiplying inputs together rather than
a linear combination. 

Data is propagated through forward control flow as in standard dataflow analysis. 
The confluence operator is set intersection, so mappings that are retained are the same 
along all paths of control flow. Since $popcount$ and $pushcount$ must be single
valued at all possible program points, an inconsistent $popcount$ at the
confluence point means that any future {\tt pop} and {\tt peek} operations
do not generate linear forms, and an inconsistent $pushcount$ results in
terminating the analysis (concluding the filter is non-linear).

Backwards edges in control flow are handled currently by the symbolic execution
of loops. We are currently investigating the use of parameterized generator matrices 
to describe the action of loops without requiring symbolic execution.

If a function call with side effects is encountered, the operation performed
by the filter is not linear and we discard our candidate node. Otherwise, 
we can easily propagate the dataflow state through side-effect free function calls. 

When a {\tt push(expr)} expression is encountered such that {\tt expr} is a linear form 
$({\mathbf v},c)$, we modify ${\mathbf A}$ and ${\mathbf b}$ of the candidate linear node $\lambda$.
${\mathbf v}$ is copied to the $u-1-pushcount$ column in ${\mathbf A}$,
$c$ is copied to the $u-1-pushcount$ column in $a$, and $pushcount$ is incremented.

If every {\tt push} expression in a {\tt filter} has a linear form as
an argument, then candidate linear node $\lambda$ is the linear node
for the {\tt filter}.
