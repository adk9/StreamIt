\section{Results}
\label{sec:results}

Our compiler currently has two linear analysis optimizations. The
first, {\it linear replacement}, replaces the largest linear
hierarchical streams possible with filters that directly compute the
calculation specified by the corresponding linear nodes. The second
optimization, {\it frequency replacement}, replaces all streams which
implement a sufficiently long convolution using the frequency
transformation described in Section \ref{sec:freq}. Below we describe
experiments and results that demonstrate performance improvements due
to these two optimizations.

\subsection{Measurement Methodology}
%We chose to measure the strength of our optimizations in terms of 
%floating point instruction reduction. The StreamIt compiler currently
%has two code generation backends. The uniprocessor backend generates sequential C code
%that can be compiled and linked against a supporting library. 
%There is also a backend that generates code for the RAW microprocessor
%\cite{waingold97baring, raw-micro}, which features
%a grid of processors interconnected via various communication facilities. 
%Mapping a StreamIt program on to the RAW architecture is complicated
%by issues of communication, load balancing and partitioning\cite{streamit-asplos}. 
%Therefore, the effects of our linear analysis optimizations can not
%be easily differentiated from differences in placement, routing and fusion that result
%from modifying the program's stream graph (as is the case for linear replacement). 
%Therefore we chose to use the uniprocessor backend for our measurements.

%As always, the appropriate metric to measure performance is not totally clear. 
%Running time is complicated by the multitasking environment offered by 
%modern operating systems. Also, given that the uniprocessor backend for
%the StreamIt compiler is meant for prototyping, the supporting 
%library is anything but optimized. Therefore, measuring running time is
%probably not an appropriate metric.

\begin{figure}[t]
%\vspace{-6pt}
\center
\epsfxsize=3.2in
\epsfbox{images/linear-freq-both-new.eps}
\vspace{-6pt}
\caption{Percent of multiplication operations remaining after performing linear replacement, frequency replacement, and both.}
\label{fig:linear-freq-both}
\vspace{-12pt}
\end{figure}

Both linear replacement and frequency replacement increase performance by 
decreasing the number of floating point computations (principally multiplications) required per output.
Our measurement platform is a Dual Intel 2.2 GHz P4 Xenon processor system 
with 2GB of memory running GNU/Linux. We compile our benchmarks using StreamIt's uniprocessor backend
and generate executables from the resulting C files using {\tt gcc} with {\tt -O2} optimization.

To measure the number of runtime multiplications we use a simple instruction counting 
program written using the DynamoRIO\cite{dynamo99} infrastructure. 
%Floating point multiplication instructions 
%in the IA-32 instruction set are defined to be any of ({\tt fmul fmulp fimulp fdiv fdivp fidivp fdivr fdivrp fidivr}).
%We measure execution time normalized to the number of program outputs generated. 
There are no standard benchmarks written yet for StreamIt, so we use
a set of representative programs\footnote{Stream graphs appear in Appendix A.}
which perform computations that 
are found in actual streaming applications: 1) {\bf FIR}, a single 256
point rectangularly windowed low pass FIR filter
($\omega_c=\frac{\pi}{3}$); 2) {\bf TargetDetect}, four matched
filters performing threshold target detection in parallel; 3) {\bf
FilterBank}, a multi-rate signal decomposition processing block common
in communications and image processing; 4) {\bf RateConvert}, an audio
signal down sampler that converts the sampling rate by a non-integral
factor; 5) {\bf FMRadio}, an FM software radio with equalizer.

\subsection{Performance}

%\begin{figure}
%\center
%\epsfxsize=3.2in
%\epsfbox{images/overlap-add-savings.eps}
%\vspace{-12pt}
%\caption{Multiplication savings from using the overlap and add method compared to overlap and discard method.}
%\label{fig:overlap-add-savings}
%\vspace{-12pt}
%\end{figure}

%Figure~\ref{fig:overlap-add-savings} shows the required computation differences for
%the overlap-add method and the overlap-discard method. 
%In all cases, the overlap-add method performs better than the overlap-discard method.
%The performance difference declines as the number of outputs($N$) increases because
%the ratio of the amount of data discarded by the overlap-discard method to the 
%amount of data produced grows smaller. 

To determine the effects of our linear replacement and frequency
replacement optimizations, we compiled each benchmark program with
linear replacement, with frequency replacement and with both linear
replacement and frequency replacement.
Figure~\ref{fig:linear-freq-both} shows the reduction of
multiplications due to our optimizations.

For the FIR filter, all of the multiplication reduction comes from the
frequency replacement optimization because the entire application is
comprised of a single filter calculating a convolution sum so there is
nothing to combine.  The multiplication reduction in TargetDetect is
also solely due to the frequency transformation because threshold
detection is non-linear which makes the parallel computation blocks
uncollapsible. Finally, RateConvert contains a large low pass filter
which also benefits from frequency replacement.

%A complete
%implementation of the linear combination rules will allow linear replacement 
%to improve performance as well.

\begin{figure}[t]
%\vspace{-6pt}
\center
\epsfxsize=3.2in
\epsfbox{images/execution-speedup.eps}
\vspace{-6pt}
\caption{Execution speedup for each of the benchmarks with both linear replacement and frequency replacement optimizations.}
\label{fig:execution-speedup}
\vspace{-6pt}
\end{figure}

All of the multiplication reduction in the FMRadio benchmark is due to
the automatic combination of parallel equalizer computations.
Table~\ref{fig:benchmark-statistics} shows the number of nodes in each
benchmark both before and after linear replacement.

FilterBank is the only benchmark where multiplications are reduced  
more by both optimizations than either alone. Linear replacement reduces
the required multiplications because it can combine the action of parallel analysis 
and synthesis channels into an overall FIR filter. Frequency replacement alone helps 
multiplication reduction only somewhat because FilterBank contains fairly small FIR filters.
However, frequency replacement speeds up the calculation of the overall filter
generated by linear replacement, and thus decreases the number of multiplies even further.

Reducing computation does not necessarily translate into execution time improvement, but as
Figure~\ref{fig:execution-speedup} demonstrates, our benchmarks speedup on average by a factor of five 
and by a factor of $6.5$ in the best case. Our current implementation takes advantage of the machine tuned 
FFT package FFTW~\cite{frigo99fast}), to perform the necessary time-frequency conversions. 
The benchmarks where multiplication reduction is due only to linear replacement (FMRadio) 
also show a large speedup.

\begin{table*}[t]
%\vspace{-6pt}
\centering
\small
\begin{tabular}{|l|c|c|c||c||c|c|c|} 
\hline
          & \multicolumn{3}{|c||}{Originally}  &             & \multicolumn{3}{|c|}{After Linear Replacement} \\
\hline
Benchmark & Filters & Pipelines & SplitJoins & Average     & Filters      & Pipelines         & SplitJoins \\
          & (linear)& (linear)  & (linear)   & vector size &              &                   &            \\
\hline
FIR       & 3 (1)  & 1(0)      & 0 (0)      & 256         & 3            & 1                 & 0 \\
\hline
TargetDetect & 10 (4)& 1 (0)    & 1 (0)      & 300         & 10           & 1                 & 1 \\
\hline
FilterBank & 27 (25) & 17 (9)   & 4 (3)      & 51          & 15           & 8                 & 1 \\
\hline
RateConvert& 5 (3)   & 2 (0)    & 0 (0)      & 335         & 5            & 2                 & 0 \\
\hline
FMRadio    & 25 (22) & 3 (1)    & 2 (2)      & 33          & 5            & 1                 & 0 \\
%\hline
%Radar      & 10 (7) & 5 (2)     & 2 (0)      & 46          & 6            & 3                 & 2 \\
\hline
\end{tabular}
\vspace{-6pt}
\caption{Statistics for benchmarks before and after transformations.
\protect\label{fig:benchmark-statistics}}
\vspace{-10pt}
\end{table*}
