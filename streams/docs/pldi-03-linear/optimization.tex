% this thing used to be about optimizations, now it is all about ``Translation to Frequency Domain''

\section{Translation to Frequency Domain}
\label{sec:freq}

Our linear analysis framework provides a compile time formulation of
the computation that a linear {\tt stream} is performing and we can
use this information to exploit well known domain specific optimizing
transformations.  Using our linear information, our compiler
identifies convolution regions that require substantially fewer
computations if they are translated into the frequency domain.

Calculating a convolution sum is a common and fundamental operation in
discrete time signal processing.  If the convolution region is
sufficiently large, transforming the data to the frequency domain,
performing a simple vector multiply and converting back to the time
domain requires fewer operations than the original convolution
computation.

% modified the tense so that it's clear we aren't talking about OUR 
% system.  --bft
The transformation from convolution sum into frequency multiplication
has always been done explicitly by the programmer because no compiler
analysis has had the information to determine when a convolution sum
is being computed.  As the complexity of DSP programs grow,
determining the disparate regions across which these optimizations can
be applied is an ever more daunting task. For example, individual
filters may not have sufficiently large convolutions to merit this
transformation, but after a liner combination of multiple filters the
transformation will be beneficial.  Furthermore, differing
architectural features makes the task of portable program
transformations even harder.

\subsection{Transformation Overview}
The convolution sum $y[n]=x[n]*h[n]$ is defined as
$y[n]=\sum_{k=-\infty}^{\infty}x[k]h[n-k]$.  In StreamIt, if a {\tt
stream} is calculating a convolution sum we know that the input
($x[n]$) and output ($y[n]$) correspond exactly to the input and
output tapes.  Furthermore, a {\tt stream} will be computing a
convolution sum when $o=1$ in which case we can identify the values
$h[n]$ as exactly the columns of $A$.

Calculating the convolution in the frequency domain is more efficient
because of the existence of the Fast Fourier Transform (FFT) algorithm
to calculate the Discrete Fourier Transform (DFT). Calculating a
convolution takes $O(N^2)$ time, and performing the equivalent
frequency computation domain using an FFT takes only two $O(N lg(N))$
time-frequency conversions coupled with an $O(N)$ frequency domain
vector multiply.

%This seeming roundabout calculation
%is feasible because a class of fast algorithms known as the FFT are known that convert
%to frequency and back again. For a thorough treatment of the theory of discrete time 
%signal processing, including using the DFT to implement convolution, see \cite{oppenheim-discrete}.

To compute the convolution of two discrete time signals, $x[n]*h[n]$,
one can calculate the DFT of both sequences to produce $X[k]$ and
$H[k]$. Multiplying $X[k]$ and $H[k]$ element-wise produces a new
sequence $Y[k]$, and taking the inverse DFT (IDFT) of $Y[k]$ produces
$y[n]$ which is exactly the same as $x[n]*h[n]$.

When we identify a {\tt stream} that computes a convolution sum, the
compiler computes $H[k]$ at compile time and stores it during runtime. 
The {\tt stream}'s {\tt work} function is changed so that first $X[k]$ is 
$x[n]$ is calculated using an FFT algorithm. Then the {\tt work} function 
multiplies $X[k]$ with $H[k]$, to produce $Y[k]$. Finally $y[n]$ is obtained by
transforming $Y[k]$ back to the time domain using an inverse FFT.

\subsection{Automatic Transformation}
%this is where the fun starts.  

To implement this transformation, the compiler needs to compute $H[k]$ at
compile time. The compiler is transforming an FIR filter which has $h[n]$ of length $e$
and push rate $u=1$.
In order to overcome the constant factors hidden in the asymptotic savings
the transformed $F$ needs to produce more than the original $1$ output on each
execution of {\tt work}. The number of outputs, $N$, to produce on each 
execution of {\tt work} is set to be approximately $2e$, a number determined
by empirical observations. $N$ is then rounded up so that $N+2(e-1)$ is a power of two 
because our FFT algorithm only works on powers of two sized inputs. 

The frequency transformation generates a new {\tt stream} that
uses $N+e-1$ items each execution where the original {\tt stream} used only $e$.
Using $N+e-1$ input items results in a complete convolution results in 
$N+2e-2$ values, of which both the first and last $e_F-1$ values 
are incorrect. Without state, the transformed filter can at most produce 
the next $N$ items because any additonal output requires more than $N+e-1$ inputs.

The compiler automaticalls computes the complex values of
$H[k]=FFT(N+2e-2,h[n])$, the $N+2e-2$ point DFT of $h[n]$ at compile
time and saves them as filter state.  A new compiler-generated {\tt
work} function calculates the complex valued $X[k]=DFT(N+2e-2,x[n])$,
the $N+2e-2$ DFT of the input and then calculates $Y[k]$ as the
element-wise vector product of $X[k]$ and $H[k]$. Finally, it performs
the inverse FFT $y[n]=IFFT(N+2e-2,y[n])$.

In the overlap-discard implementation, the {\tt work} function simply
uses the middle $N$ values of $y[n]$ and discards the overlapping
$2e-2$ on either end. Then the $N$ values are used to produce the next
$N$ outputs.

The overlap-add method saves the overlaping values of $y[n]$ because they contain partial
outputs due to both the previous and the next $N+e-1$ inputs.
The first $e-1$ are part of the computation from the previous invocation of the 
{\tt work} function and the last $e-1$ are part of the computation that will be done
by the next invocation. 
The reuse of partial results from previous frequency calculations is known 
as the ``overlap and add'' method ~\cite{oppenheim-discrete} and used widely.
On each {\tt work} function exection, the filter pushes the first $e-1$ items 
of $y[n]$ plus the partial results from the previous invocation. Then the middle 
$N$ elements of $y[n]$ are produced. Finally, last $e-1$ elements 
of $y[n]$ are stored in as the partial results to be used on the next 
invocation of {\tt work}.

