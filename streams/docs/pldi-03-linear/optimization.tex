\section{Optimizations}
The goal of linear analysis and combination is to allow us to do optimizations
that are hard to formulate in imperiative languages. For example, combining the action
of {\tt Streams} in a {\tt Pipeline} is equivalent to doing algebriaic simplification across
sequential loops. Combining the action of {\tt Streams} in a {\tt SplitJoin} is equivalent
to combining the action of multiple inner loops of a larger enclosing loop.

Also, as our linear analysis framework gives us a complete compile time formulation of the 
computation that a {\tt Stream} is doing, we can exploit well known domain specific optimizations
such as implementing discrete time convolution using the discrete Fourier transform (DFT). 
 


\subsection{Algebraic Simplification from Combination}
One result of combining linear representations in {\tt Pipelines} in the stream graph is that  
the compiler can perform algebriac simplification {\textbf across} the execution of {\tt Streams}. 
If the compiler simply replaces the {\tt Pipeline} structure with a {\tt Filter} that 
directly computes the overall vector matrix product, performance can be significantly improved.
For instance, instead of implementing a band pass filter as a back to back combination
of a low pass filter and a high pass filter, a clever programmer could determine the coefficients 
of a single FIR filter back to back in a pipeline. However, given the ubiquity of high and 
low pass filters, it is typically more convenient to understand and implement the action
of a bandpass filter as a the two separate pieces.

Another example of a processing block which is implemented in a different way than
usually represented is a FIR filter followed by a compressor block 
(see \cite{oppenheim-discrete}) decimates the input by a factor $M$ (eg only every $M$th 
input sample is passed on to the output filter). 
Obviously there is no need to calcluate all of the outputs. 
Instead you should only calculate every $M$th sample.
Therefore, we end up with an FIR {\tt Filter} that pops $M$ instead of poping one. While
this is a transformation that is certainly worth doing (it reduces the computational
requirements by a factor of $\frac{M-1}{M}$ the system is almost always represented
as a filter followed by the compressor in system design diagrams. The inefficient
implementation is represented because it is easier to understand the operation of the
two fundamental processing blocks in cascade than it is to understand the combined system.
Given the linear representations for both blcoks, when such a pipeline is combined into an
overall representation the efficient implementation is automatically derived. Therefore,
our compiler with the lienar analysis can perform the transformation automatically, 
letting the programmer express the system in the most natural way (seperate filter and compressor).

Another example is a multi band equalizer (such as is found in the FMRadio benchmark). 
The overall purpose of such an equalizer is to filter $N$ different frequency bands 
separately (like the graphic equalizer found on stereos). 
The system diagram that best describes the operation of the multiband equalizer
passes the input signal though the $N$ different filters, and then combines the $N$
different outputs into the final output signal.
This obvious design calls for $N$ filters, but if the filters do not vary with time, 
a single filter which has the overall response that matches the action of all the 
different individual filters. However, designing this single combined filter is 
difficult, and subsequent changes to any of the sub filters will necessitate a
redesign of the overall filter. Ideally, the designer would specify the system as
the combination of multiple filters and then let the compiler combine them. 
This automatic combination can is exactly what happens as a result of combining
linear representations as described above. Because the linear analysis can
figure out what is happening across the execution of multiple sub combonents, 
it can do implementation optimizations that normally must be done by hand.


\subsection{Frequency Implementation}
There is a wealth of knowledge about DSP implementation optimizations. One in
particularly wide use and of wide applicability is calculating a convolution sum 
using the discrete Fourier transform. Calculating a convolution sum is a fundamental
operation in discrete time signal processing as it allows one to calculate the output
of an LTI system given its response to single initial impulse.

For our compiler, if a {\tt Stream} is implementing an FIR digital filter as a direct 
convolution sum (eg it is calculating exactly $\sum_{i=0}^{N-1}a_{i}x[i]$ for each output)
we can automatically improve its performance by moving the calculation to the frequency
domain. 

The mathematics of transforming convolution in time to a multiplication in frequency 
is well understood. The main idea is that it requires fewer calculations to transform
the input signal to a different basis, do a multiplication, and then convert back 
to time than it does to do the convolution directly . This sequence is faster because 
fast algorithms collectively known as FFT are known that efficiently convert transform
signals to different bases. For a thorough treatment of the theory, see
\cite{oppenheim-discrete}.

\begin{figure}
\center
\epsfxsize=2.0in
\epsfbox{images/frequency-conversion-convolution.eps}
\caption{A filter directly implementing the convolution sum.}
\label{fig:frequency-conversion-convolution}
\end{figure}

\begin{figure}
\center
\epsfxsize=2.0in
\epsfbox{images/frequency-conversion-frequency.eps}
\caption{A filter computing the convolution in the frequency domain.}
\label{fig:frequency-conversion-frequency}
\end{figure}

Given our linear analysis information, we can recognize filters that are
computing convolution sums and we can take advantage of the frequency transformation 
outlined above. We define an FIR filter as a {\tt Stream} that uses $M$ input elements,
consumes $1$ input element and produces $1$ output item per {\tt work} function execution.
We can transform such an FIR {\tt Stream} such that it uses and consumes $N+M-1$ input
elements to produce $N+M-1$ outputs per {\tt work} function execution. We use the 
following notation in the rest of this section to remain consistent with DSP terminology.
$h[n]$ is the filters ``impulse response'' and consists of the values in the linear 
representation matrix. $x[n]$ is the next $N+M-1$ block of the input that the filter
will process. $y[n]$ is is the $N+M-1$ values the filter produces.

Our compiler computes the complex values of $H[k]=H(e^{j\omega})|_{\omega=\frac{2{\pi}k}{L}}$, 
the DFT of $h[n]$, at compile time. The filter is transformed 
so that it  calcluates $X[k]=X(e^{j\omega})|_{\omega=\frac{2{\pi}k}{L}}$, the DFT 
of $x[n]$. Then it calculates $Y[k]=X[k]H[k]$, the DFT of $h[n]*x[n]$.
Then the filter converts $Y[k]$ into $y[n]$ using the inverse DFT. Finally, 
the appropriate values from $y[n]$ are pushed on to the output tape. The
calculations of the DFT are implemented using the the well-known 
``fast Fourier transform'' (FFT) algorithm.

\begin{figure}
\center
\epsfxsize=3.0in
\epsfbox{images/frequency-example.eps}
\caption{Example of edge case in frequency results.}
\label{fig:frequency-example}
\end{figure}

As always, the devil is in the details. We have not mentioned above either 
the size of the DFT($L$), how to deal with initial conditions, or which 
values of $y[n]$ to push on to the output tape. 
Without going into detail, we use a simple overlap and add\cite{oppenheim-discrete} 
method to calculate the values of $y[n]$ for each iteration. To do so 
we need to introduce state into our filters. If the filters did not have state
we would have to throw out partial values each execution of the {\tt work} function
which makes our technique much less effective. Figure~\ref{frequency-example} shows
the basic idea of what is going on.


\begin{figure}
\center
\epsfxsize=3.0in
\epsfbox{images/frequency-win.eps}
\caption{Plot showing where frequency transformation will results in an overall savings versus M (filter size) and N(input block size)}
\label{fig:frequency-win}
\end{figure}


Below we provide a simple justification and analysis about why we expect a performance
improvement using this frequency transformation. Direct convolution requires $O(MN)$ multiplies
to generate $N$ output values. Convolution using the frequency transformation requires
$O((M+N)lg(M+N))$ multiplications for both the FFT and IFFT and $O(M+N)$ time for the 
multiplication in frequency. Figure~\ref{fig:frequency-win} shows the differences between
$O(MN)$ and $O((M+N)lg(M+N)+M+N)$. Values above $1$ indicate that the frequency transformation
will result in less computation and values below $1$ indicate that the freqency transformation
will result in more computation.



