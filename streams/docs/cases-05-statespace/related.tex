\mysection{Related Work}
\label{sec:related}

This paper builds directly on the work done to analyze and optimize
linear components in StreamIt graphs \cite{Lamb}. We extend the
theoretical framework for linear analysis to state space analysis in
order to apply our optimizations to a wider class of applications.
Specifically, state space analysis applies to filters with persistent
state, and feedback loops can be combined into a single state space
representation; neither of these cases are handled by linear analysis.
The extension from linear analysis to state space analysis required a
fundamental change to the underlying representation, as well as a
complete reformulation of the rules for combination and expansion.
Moreover, this paper introduces novel optimizations of state removal
and parameter reduction, both of which operate on the state space
representation.

Several other groups are researching methods for automated DSP
optimizations. SPIRAL \cite{Spiral} is a system developed to generate
libraries of DSP transforms. These libraries are designed for specific
architectures, and can be re-optimized when hardware is upgraded or
replaced. Other such libraries that have been designed include a
package for linear algebra manipulations by the ATLAS project
\cite{Atlas} and portable high-performance FFTs (Fast Fourier
Transforms) \cite{fftw}.

Aside from StreamIt, other programming languages have been designed
for streaming data. Synchronous languages which target embedded
applications include LUSTRE \cite{Lustre}, Esterel \cite{Esterel}, and
Signal \cite{Signal}. Other stream-based languages are Occam
\cite{Occam}, SISAL \cite{sisal}, and StreamC \cite{streamc}.  Some of
these languages are designed to exploit vector and parallel
processing. However, none of these languages have compilers that run
state space or linear analysis.
