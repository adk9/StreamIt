\mysection{Implementation}
\label{sec:results}

We have implemented the extraction, combination, and optimization
(except multiple execution stages) procedures within the StreamIt
compiler, which uses the Kopi Java Compiler
infrastructure~\cite{kopi}. We measure performance by counting the
number of floating point operations (additions and multiplications)
executed in a given benchmark. The DynamoRIO~\cite{dynamo} system is
used to count operations.

We compare linear state space optimizations with linear
optimizations~\cite{Lamb}.  For the state space numbers, all linear
state space blocks are combined and then optimizations are applied.
For applications of this size, it is primarily the combination that
yields a performance boost; we have not characterized the individual
contributions of state removal and parameter reduction.  For the
linear numbers, all linear blocks are combined.  We do not enable the
``frequency translation'' optimization~\cite{Lamb} because it has no
counterpart in the state space domain.  A hybrid optimizer could apply
frequency translation where appropriate, and linear state space
optimizations elsewhere.

\begin{table}[tbp]
\small
\begin{center}
\begin{tabular} {c|c|c}
{\bf Application} & \hspace{-8pt}~~~~~~\parbox{0.7in}{{\bf ~~~Linear} \\ {\bf State Space} \vspace{2pt}}\hspace{-8pt} & {\bf Linear} \\ \hline
FM Radio & 0.1740 & 0.1692 \\
FIR Program &  0.9961 & 0.9980 \\
Channel Vocoder & 0.2601 & 0.2620 \\
FilterBank2 & 1.000 & 1.001 \\
FFT (16 pt) & 2.938 & 3.000 \\
Linear Difference Equation & 1.005 & 1.000 \\
IIR & 1.005 & 1.000 \\
IIR + 1/2 Decimator & 0.6441 & 1.000 \\
IIR + 1/16 Decimator & 0.3393 & 1.000 \\
IIR + FIR & 0.9413 & 1.000 \\
FIR + IIR + IIR & 0.9214 & 1.000
\end{tabular}
\caption{Floating point operations with state space and linear
optimizations, normalized to no optimizations.\protect\label{tab:results}}
\end{center}
\vspace{-18pt}
\end{table}

Results appear in Table~\ref{tab:results}.  For the first 5
applications, which mainly have linear components without state,
linear optimizations and state space optimizations are equally
effective. There is a large performance degradation for an FFT (Fast
Fourier Transform) for both types of transformations. This is not
surprising, since an FFT performs its computations sparsely across
multiple filters. Combining these filters creates one filter densely
packed with computations. This is exactly a conversion from an FFT to
a DFT (Discrete Fourier Transform).  We would need staged execution
with minimal parameterization to convert the DFT back to an FFT.  Note
that the compiler could detect that the operations count has increased
and refrain from combining the filters when performance
degrades~\cite{Lamb}; in this paper we are evaluating only the
performance impact of blindly combining all filters.

The remaining applications have filters with state.  Thus, they are
not handled by linear analysis, and there is a normalized performance
of 1.0 in the linear column.  For a simple standalone Linear
Difference Equation or IIR (Infinite Impulse Response) filter,
state space transformations show a very slight degradation in
performance. The reason for this is that there are a few extra
operations performed by the prework function. In terms of steady-state
behavior, state space transformation is equivalent to no
transformation. There is no gain in performance, because the filters
are written optimally.

When we combine stateful filters with other filters, we notice a
performance improvement using state space optimizations. For example,
combining an IIR filter with a decimator that leaves 1 out of every 16
values has a 66\% improvement. Combining an IIR filter with an FIR
filter has a 6\% improvement. In the case of an IIR filter with a
decimator, there are extraneous computations performed by the IIR
filter that are thrown away by the decimator.  Combining their
respective matrices removes these computations. In the case of an IIR
filter with an FIR filter, the computations in both filters can be
merged to a single set of computations. This indicates that
state space optimizations are more useful when applied to combined
filters than when applied to individual filters.

In summary, the results show that linear state space analysis is more
general than linear analysis.  While the evaluation spans only a
modest set of micro-benchmarks, it also demonstrates that a relatively
elaborate mathematical framework is managable within the compiler.
