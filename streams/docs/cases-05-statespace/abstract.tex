Digital Signal Processing (DSP) is becoming increasingly widespread in
portable devices. Due to harsh constraints on power, latency, and
throughput in embedded environments, developers often appeal to signal
processing experts to hand-optimize algorithmic aspects of the
application.  However, such DSP optimizations are tedious,
error-prone, and very expensive, as they require sophisticated
domain-specific knowledge.

We present a general model for automatically representing and
optimizing a large class of signal processing applications. The model
is based on linear state space systems. A program is viewed as a set
of filters, each of which has an input stream, an output stream, and a
set of internal states. At each time step, the filter produces some
outputs that are a linear combination of the inputs and the state
values; the state values are also updated in a linear
fashion. Examples of linear state space filters include IIR filters
and linear difference equations.

Using the state space representation, we describe a novel set of
program transformations, including combination of adjacent filters,
elimination of redundant states and minimal parameterization of the
system. We have implemented the optimizations in the StreamIt compiler
and demonstrate performance gains over previous techniques.
