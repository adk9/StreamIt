\mysection{Conclusion}

    We present a methodology for the detection, analysis, combination, and
optimization of linear state-space filters in DSP applications.
This work is automatized in the compiler of a high-level
programming language, StreamIt, designed for streaming
applications. This frees the programmer from the burden of writing
low-level optimized code that requires expert DSP analysis.
Instead, the programmer can focus on the top-level design of a DSP
application and write modular code in a structured setting.

%%     Due to the infinite number of possible state-space transforms,
%% the optimizations discussed are not necessarily ideal.
%% Additionally, parts of DSP applications are non-linear and cannot
%% be analyzed in the state-space domain. Therefore, the work
%% presented in this thesis does not fully optimize some portions of
%% DSP applications, and does not apply towards other portions.
%% However, it does represent an additional step towards the goal of
%% a full analysis and optimization of any application. We outline
%% some of the future steps that can be taken, both to improve on the
%% work in this thesis and to expand it to other types of
%% domain-specific analyses and optimizations.
%% \begin{itemize}

%% \vspace{\itemshrink} \item As mentioned in Section 4, minimally parameterize a system,
%% then uses multiple execution stages.

%% \vspace{\itemshrink} \item Use a balanced representation \cite{Moore} to quantify the
%% relative impact of each state of a filter on its execution. Then
%% the states that have impact values below a certain threshold can
%% be removed, resulting in only a small change in the filter's
%% execution.

%% \vspace{\itemshrink} \item Formulate loops in traditional programming languages as
%% state-space filters, and use state-space work developed in this
%% thesis to detect their induction variables and optimize their
%% execution.

%% \vspace{\itemshrink} \item Create a cost function metric,
%% $f(\mathbf{A},\mathbf{B},\mathbf{C},\mathbf{D})$, that balances
%% the traditional program analysis metrics (throughput, power
%% consumption, memory allocation, etc.) in any manner desired. Then
%% find a general way to minimize this cost function over all
%% possible state-space transformations $\mathbf{T}$.

%% \vspace{\itemshrink} \item Formulate a methodology to deal with filters whose outputs
%% are a linear combination of their inputs but a non-linear
%% combination of their state variables.

%% \vspace{\itemshrink} \item Use a `black box' method to find the appropriate
%% representation of a filter. In this approach the filter is given
%% inputs and the output data is collected. The input/output
%% relations are used to formulate an appropriate state-space model
%% that may not exactly represent the filter, but does so within a
%% tolerable error margin (see \cite{Schutter}).
%% \vspace{\itemshrink} \end{itemize}
