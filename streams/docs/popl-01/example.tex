\section{Detailed Example}
\label{sec:example}
This section describes the Trunked Radio example implementation in
Figure~\ref{fig:radiocode}. The trunked radio is a frequency hopping
system. The transition times between the preset frequencies is
indicated to the receiver by transmitting a preset tone.  Our radio
also has an FIR filter to increase the gain of a weak signal. However,
in order to save power, this filter is activated only when the signal
to noise ratio is low.  This implementation relies on transforming the
signal into the frequency domain for processing using a FFT. For
brevity, we have not shown the RF processing in the front-end and the
audio processing in the back-end.  The code for the trunked radio
demonstrates many features of StreaMIT.

The high-level structure of the radio, graphically shown in
Figure~\ref{fig:radiodiagram}, is implemented in the class {\tt
TrunkedRadio}. The radio has seven stages, where the first three
stages operate in the time domain, the last three stages operate in
the frequency domain, and there is a conversion phase in between.  At
this high level, the structure of the system is a pipeline of either
six or seven stream stages. The difference is due to the {\tt Booster}
stage, which can be active or inactive. The switching on and off of
the {\tt Booster} stage, which happens infrequently, is accomplished
using a reinitialization message from the {\tt CheckQuality} stage. We
also use a message from the {\tt CheckFreqHop} stage to the {\tt
RFtoIF} stage to change the baseband when a frequency hop tone is
present.

The {\tt RFtoIF} stage modulates the input signal from RF to a
frequency band around the current IF frequency. This stage is
implemented as a filter that multiplies the current signal with a sine
wave at the IF frequency.  To support a change in the IF frequency
when frequency hopping occurs, the filter contains a {\tt set\_freq}
method that can be invoked using a message.

The optional {\tt Booster} stage is an FIR filter that is activated
when the signal is hard to detect. During normal operation, however,
it is deactivated to conserve power. The turning on and off of the
filter is controlled by a message.  The filter itself, shown in
Figure~\ref{fig:firfilter}, is implemented as a {\tt Filter} that
peeks at N elements in the input stream.

The {\tt FFT} stage converts the program from the time domain to the
frequency domain using a multi-stage FFT. It is graphically presented
in~\ref{fig:fftfilter}. The FFT is composed of a reordering filter and
a multi-stage butterfly filter. The StreaMIT representation of the
reordering filter (a bit reverse order filter) is given in
Figure~\ref{fig:bitreverseorder}. Note that the complex data
re-shuffling is accomplished using a few SplitJoin constructs.  A
parameterized Butterfly implementation is used to abstract the
multi-stage butterfly in the FFT. As shown in
Figure~\ref{fig:butterfly}, the Butterfly filter is also implemented
using a combination of SplitJoin constructs.

The StreaMIT implementation of the FFT filter is clean and
intuitive. It already has a large amount of pipelined parallelism. Due
to the simple and straightforward mapping from the algorithm to the
implementation, compiler analyses should be able to extract the
parallel structure of the FFT when hardware resources are available.

The next stage, {\tt CheckFreqHop}, checks four different frequencies
for the change frequency tone. When the stage detects this tone, it
has to change the frequency within a time limit. This task is
accomplished by sending a message to the {\tt RFtoIF} stage.  The
message requires the {\tt RFtoIF} stage to deliver between $4N$ and
$6N$ items using the old modulation before changing to the new
frequency.

The {\tt CheckQuality} stage checks if the signal has a distinct
frequency spectrum. If all the frequencies have similar amplitudes,
the stage assumes that the signal-to-noise ratio is low and sends a
message to activate the {\tt Booster}. This message is sent using
best-effort delivery.

\begin{figure}[t]
\centering
\psfig{figure=fir-block.eps,width=3.2in}
\caption{A block diagram of a five tap FIR filter.}
\label{fig:firfilter}
\end{figure}

\begin{figure}[t]
\centering
\psfig{figure=fft-block.eps,width=3.2in}
\caption{The multi-stage FFT algorithm}
\label{fig:fftfilter}
\end{figure}