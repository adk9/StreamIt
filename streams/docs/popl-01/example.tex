\section{Detailed Example}
\label{sec:example}
This section describes the Trunked Radio example implementation in
Figure~\ref{fig:radiocode}. The code for the trunked radio
demonstrates many features of StreaMIT.

The high-level structure of the radio, graphically shown in
Figure~\ref{fig:radiodiagram}, is implemented in the class {\tt
TrunkedRadio}. The radio has seven stages, where the first three
stages operate in the time domain, the last three stages in the
frequency domain, with a conversion phase in between.  At this high
level, the structure of the system is a pipeline of six or seven
stream stages. The difference is due to the {\tt Booster} stage, which
can be active or inactive. The switching on and off of the {\tt
Booster} stage, which happens infrequently, is accomplished using a
message from the {\tt CheckQuality} stage. We also use another message
from the {\tt CheckFreqHop} stage to {\tt RFtoIF} stage to change the
baseband when a frequency hop tone is present.

In this example we will omit the implementation of first and the last
stages, {\tt ReadFromAtoD} and {\tt AudioBackEnd}.

The {\tt RFtoIF} stage modulates the input signal from RF to a
frequency band around the current IF frequency. This stage is
implented as a filter that multiplies the current signal with a sine
wave at the IF frequency.  To support a change in the IF frequency
when frequency hopping occurs, the filter contains a {\tt set\_freq}
method that can be invoked using a message.

The optional {\tt Booster} stage is a FIR filter that is activated
when the signal is hard to detect. During normal operation, however,
it is deactivated to conserve power. The turning on and off of the
filter is controlled by a message.  The filter itself, shown in
Figure~\ref{fig:firfilter}, is implemented as a Filter that uses peek
to look at N elements in the input stream.

The {\tt FFT} stage converts the program from the time domain to the
frequency domain using a multi-stage FFT. It is graphically presented
in~\ref{fig:fftfilter}. The FFT is composed of a reordering filter and
a multi-stage butterfly filter. The StreaMIT representation of the
reordering filter (a bit reverse order filter) is given in
Figure~\ref{fig:bitreverseorder}. Note that the complex data
reshuffling is accomplished using a few splitjoin constructs.  A
parameterized Butterfly implementation is used to abstract the
multi-stage butterfly in the FFT. As shown in
Figure~\ref{fig:butterfly}, the Butterfly filter is also implemented
using a combination of splitjoin constructs.

StreaMIT implementation of the FFT filter is clean and intutive. It
already has a large amount of pipelined parallelism. Due to the simple
and straighforward mapping from the algorithm to implementation,
compiler analyses should be able to extract the parallel structure of
the FFT when hardware resources are available.

The next stage, {\tt CheckFreqHop}, checks four different frequencies
for the change frequency tone. When the stage detects this tone, it
has to change the frequency within a time limit. This task is
accomplished by sending a message to the {\tt RFtoIF} stage.  The
message requires the {\tt RFtoIF} stage to deliver between 4N and 6N
items using the old modulation before changing to the new frequency.

The {\tt CheckQuality} stage checks if the signal has a distinct
frequency spectrum. If all the frequencies have similar amplitudes,
the stage assumes that the signal to noise ratio is low and sends a
message to activate the {\tt Booster}. Note that this message is send
using best effort delivery.



\begin{figure}
\centering
\psfig{figure=fft-block.eps,width=3.2in}
\caption{The multi-stage FFT algorithm}
\label{fig:fftfilter}
\end{figure}

\begin{figure*}
\centering
\psfig{figure=fft-pre-tape.eps,width=4.8in}
\caption{The bit reverse order filter in FFT. The tapes at each
channel illustrates the data reshuffling when N=8. }
\label{fig:bitreverseorder}
\end{figure*}

\begin{figure*}
\centering
\psfig{figure=fft-butterfly-tape.eps,width=5.8in}
\caption{The 4x4 butterfly stage in the FFT. The tapes at each channel illusrates the data transformation and computation. }
\label{fig:butterfly}
\end{figure*}

\begin{figure}
\centering
\psfig{figure=fir-block.eps,width=3.2in}
\caption{A block diagram of a five tap FIR filter.}
\label{fig:firfilter}
\end{figure}



\newpage

\begin{figure}[t]
\scriptsize
\begin{verbatim}
class Butterfly extends Stream {
   void init(int N, int W) {
      add(new SplitJoin() {
         void init() {
            setSplitter(WEIGHTED_ROUND_ROBIN(N, N));
            add (new Identity());
            add (new Filter() {
               float weights[W];
               int curr;
               Channel input = new FloatChannel();
               Channel output = new FloatChannel();
               void init() {
                  for(int i=0; i<W; i++)
                     weights[i] = calcWeight(i, N, W);
                  curr = 0;
               }
               void work() {
                  output.push(input.pop()*weights[curr++]);
                  if(curr>= W) curr = 0;
               }    
            });
            setJoiner(ROUND_ROBIN);
      }});
      add(new SplitJoin() {
         void init() {
            setSplitter(DUPLICATE);
            add (new Filter() {   
               Channel input = new FloatChannel();
               Channel output = new FloatChannel();
               void work() {
                  output.push(input.pop() + input.pop());
               }
            });
            add (new Filter() {   
               Channel input = new FloatChannel();
               Channel output = new FloatChannel();
               void work() {
                  output.push(input.pop() - input.pop());
               }
            });
            setJoiner(WEIGHTED_ROUND_ROBIN(N, N));
      }});
}}

class FFT extends Stream {
   void init(int N) {
      add(new SplitJoin() {
         void init() {
            setSplitter(WEIGHTED_ROUND_ROBIN(N/2, N/2));
            for(int i=0; i<2; i++) 
               add(new SplitJoin() {
                  void init() {
                     setSplitter(ROUND_ROBIN);
                     add (new Identity());
                     add (new Identity());
                     setJoiner(WEIGHTED_ROUND_ROBIN(N/4, N/4));
               }});
            setJoiner(ROUND_ROBIN);
      }});
      for(int i=2; i<=N/2; i *= 2)
        add(new Butterfly(i, N));
}}

class FIR extends Filter {
   Channel input = new FloatChannel();
   Channel output = new FloatChannel();           
   int N;
   void init(int N) {
      this.N = N;
   }
   void work() {
      float sum = 0;
      for (int i=0; i<N; i++) {
         sum += input.peek(i)*firCoeff[i][N];
      }
      input.pop();
      output.push(sum);
   }
}

class Booster extends Stream {
   void init(boolean adds) {
      if(adds) add(new FIR(N));
   }
}
\end{verbatim}
\vspace{-12pt}
\caption{\protect\small The StreaMIT program for the Trunked Radio Receiver.
\protect\label{fig:radiocode}}
\vspace{-12pt}
\end{figure}

\begin{figure}[t]
\scriptsize
\begin{verbatim}
class RFtoIF extends Filter {
   Channel input = new FloatChannel();
   Channel output = new FloatChannel();
   int sz, i;
   float weight[];
   void init(float f) {
      setf(f);
   }
   void work() {
      output.push(input.pop()*weight[i++]);
      if(i==sz) i = 0;
   }
   void setf(float f) {
      i = 0;
      sz = CarrierFreq/f*N;
      weight = new float[sz];
      for(int i=0; i<sz; i++)
         weight[i] = sine(i*PI/sz);
   }
}

class CheckFreqHop extends SplitJoin {
   RFtoIFPortal fh;
   void init(RFtoIFPortal fh) {
      this.fh = fh;
      setSplitter(WEIGHTED_ROUND_ROBIN(N/4-2,1,1,N/2,1,1,N/4-2));
      int k = 0;
      for(int i=0; i<4; i++) {
         if((i==0)||(i==2)) {
            for(int j=0; j<2; j++) {
               add(new Filter() {
                  Channel input = new FloatChannel();
                  Channel output = new FloatChannel();
                  void work() {
                     float val = input.pop();
                     if(val >= MIN_THRESHOLD) 
                        fh.setf(Freq[k], new TimeInterval(4*N, 6*N)); 
                     output.push(val);
                  }
               });
               k++;
            }
         } else add(new Identity());
      }
      setJoiner(WEIGHTED_ROUND_ROBIN(N/4-2,1,1,N/2,1,1,N/4-2));
   }
}

class CheckQuality extends Filter {
   Channel input = new FloatChannel();
   Channel output = new FloatChannel();
   float aveHi, aveLo;   
   BoosterPortal onOffSwitch;
   void init(BoosterPortal onOffSwitch) {
      aveHi = 0; aveLo = 0;
      this.onOffSwitch = onOffSwitch;
   }
   void work() {
      float val = input.pop();
      aveHi = max(0.9*aveHi, val);
      aveLo = min(1.1*aveLo, val);
      if(aveHi - aveLo < QUAL_BAD_THRESHOLD)
         onOffSwitch.init(true);
      if(aveHi - aveLo > QUAL_GOOD_THRESHOLD)
         onOffSwitch.init(false);
      output.push(val);
   }
}

class TrunkedRadio extends Stream {
   RFtoIFPortal freqHop = new RFtoIFPortal();
   BoosterPortal onOff = new BoosterPortal().
   void init() {
      add(new ReadFromAtoD());
      RFtoIF rf2if = new RFtoIF(STARTFREQ);
      add(rf2if);
      freqHop.register(rf2if);
      Booster iss = new Booster(false);
      add(iss);
      onOff.register(iss);
      add(new FFT());
      add(new CheckFreqHop(freqHop));
      add(new CheckQuality(onOff));
      add(new AudioBackEnd());
   }
}
\end{verbatim}
\vspace{-12pt}
\end{figure}
