\section{Introduction}

Applications that are structured around some notion of a ``stream''
are becoming increasingly important and widespread.  There is evidence
that streaming media applications are already consuming most of the
cycles on consumer machines \cite{Rix98}, and their use is continuing
to grow.  In the embedded domain, applications for hand-held
computers, cell phones, and DSP's are centered around a stream of
voice or video data.  The stream abstraction is also fundamental to
high-performance applications such as intelligent software routers,
cell phone base stations, and HDTV editing consoles.

Despite the prevalence of these applications, there is surprisingly
little language and compiler support for practical, large-scale stream
programming.  Of course, the notion of a stream as a programming
abstraction has been around for decades \cite{SICP}, and a number of
special-purpose stream languages have been designed (see
\cite{survey97} for a review).  Many of these languages and
representations are elegant and theoretically sound, but they often
lack features and are too inflexible to support straightforward
development of modern stream applications, or their implementations
are too inefficient to use in practice.

Consequently, most programmers turn to general-purpose languages such
as C or C++ to implement stream programs, resorting to low-level
assembly codes for performance-critical loops.  This practice is
labor-intensive, error-prone, and very costly, since the
performance-critical sections must be re-implemented for each target
architecture.  Moreover, general purpose languages do not provide a
natural and intuitive representation of streams, thereby having a
negative effect on readability, robustness, and programmer
productivity.

StreaMIT is a language and compiler specifically designed for modern
stream programming.  Its goals are two-fold: first, to raise the
abstraction level in stream programming, thereby improving programmer
productivity and program robustness, and second, to provide a compiler
that performs stream-specific optimizations to achieve the performance
of an expert assembly programmer.  

To address the first of these goals, this paper describes the first
version of the StreaMIT language.  The first version is special in
that it concentrates on streams with static input and output rates;
applications such as compression that have dynamic flow rates will be
the focus of the next version of StreaMIT.  This paper also presents a
notion of ``information flow'' that we believe is fundamental to the
streaming domain.  Using this notion, we give a clear semantics to
StreaMIT's messaging system and describe simple algorithms for static
deadlock and overflow detection.

The paper is organized as follows. In Section {\ref{sec:domain}}, we
characterize the domain of streaming programs that motivates the
design of StreaMIT.  In Section~\ref{sec:overview}, we provide a
detailed description of StreaMIT, and in Section~\ref{sec:time} we
formally define the semantics of StreaMIT using a notion of
``information flow''. We describe related work in
Section~\ref{sec:related} and present a detailed example of a software
radio in Section~\ref{sec:example}.  Section~\ref{sec:conc} contains
our conclusions, and the Appendix gives an overview of StreaMIT
syntax.


