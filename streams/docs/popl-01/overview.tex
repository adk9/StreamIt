\section{Language Overview}

\begin{figure*}
\centering
\psfig{figure=Radio.eps,width=5.2in}
\caption{ fig1}
\label{fig:data_structs}
\end{figure*}

In this section, we give an overview of the StreaMIT language.

\begin{verbatim}

- motivated by properties above
- this is interface language; developing higher-level streamit
- legal java syntax for now, to ease development cycle
- assume static push, pop, peek

- point to example, talk through major features

---------------------------------------------------------------------------
- representing filters 
- others:
	- procedural language
		- complicated loops that don't express well the feedback,
			parallelism in streaming systems
		-> loop overhead dominates, data leaves the cache

	- object oriented language
		- complicated runtime-model, pulling data and trying to 
			get block size right for machine configuration
			(scheduling is complicated)

- us: only define base-level blocks, leave the rest to language/compiler
	- init, work, message handlers

	- define in init function anything that has to be set up when
          the block is first defined.

	- define in work function the finest-grained step of
          execution, expressing the simple transfer function of one
          stream into another

	- no global data or global time

- syntax presented:  filter/channel declarations, init & work functions
---------------------------------------------------------------------------
- representing streams (how to ``connect the blocks together'')
- others:
	- straight-line code declaring & connecting blocks
		problem: can't see the structure.
			 easy to make mistakes
	- ad-hoc graph languages for declaring & connecting blocks
		problem: awkward to use
			 no good textual representation

	  generally, arbitrary connections are too hard to think
		about. and too hard for compiler to analyze.
		make connection with unstructured control flow

- us: structured streams.
	- compose out of simple structures:  
		pipe, feedbackloop, splitjoin

	   advantages: can see structure in code, with indentation
		level corresponding to hierarchical stream
		depth (pointer to natural syntax argument?)

		- can have clean abstractions about *hierarchical*
		structure.  Provides better abstraction and
		modularity than *flat* layout of blocks.

- syntax presented:  stream, feedbackloop, splitjoin.  inlining.
---------------------------------------------------------------------------
- integrated support for infrequent, low-bandwidh messages
- others:
	- embed control messages in data stream
		- complicated, error-prone, unreadable code
		- hurts performance of steady state (if you have to 
			check if everything is message or data)
		- complicates compiler analysis
		- can't send messages upstream without explicit
			feedback loop
	- treat messages as synchronous method calls
		- but delays stream progress when method en route, 
			disrupting scheduling freedom of compiler, as
			well as efficiency when at time of call
		- usually there is no return value (it's a message!)
		- no good notion of ``when'' message arrives relative
			to sender
- us: 	- talk about PORTALS
	- support for asynchronous messages with timing relative to
		data items
	- (define the ``wavefront'' of a data item, and say that
		messages sent with range of times)
	- avoids problems of two above approaches (it's readable, out
		of the way of steady-state data, and out of way of compiler)
	- this RELATIVE time gives more freedom to compiler for
		scheduling blocks
	- absence of global time good for parallel programming

- syntax presented:  sendMessage, messageHandlers, messageStubs.
---------------------------------------------------------------------------
- integrated support for re-initialization sections of a stream
- others:	
	- ad-hoc halting of whole stream's data flow so that some
		restructuring can take place
		
		- complicated programming to interface between the time of
			reinitialization and starting/stopping the stream

- us:
	- re-invoke the original initialization routine with different 
		parameters.

		- talk about draining the pipeline.  the compiler
                  worries about doing the reinitialization at the
                  right time so that the re-initialized segment drains
                  completely.  but the re-init message is delivered
                  just like a message, and a warning is given if the
                  message delivery schedule does not encompass a clean
                  draining point.

		- (alternatively, could destroy the pipeline.)

		- so improvement #1 is that the interface to the user is 
			cleaner.  Also it's easier for the compiler to 
			analyze.

		- improvement #2: very clear targets of
			re-initiailization.  Can reinitialize any
			block in the hierarchy in a very well-defined
			way.

		- improvement #3 is that the draining analysis is done
                  automatically, which otherwise is very complicated
                  for the programmer to deal with.

- syntax presented:  special reset method

\end{verbatim}






