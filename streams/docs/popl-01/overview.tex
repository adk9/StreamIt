\section{Language Overview}

\begin{figure*}
\centering
\psfig{figure=Radio.eps,width=5.2in}
\caption{A block diagram of a software radio.  A detailed
implementation in StreaMIT appears in the Appendix.}
\label{fig:radiodiagram}
\end{figure*}

StreaMIT includes stream-specific abstractions and representations
that are designed to improve programmer productivity for the domain of
programs described above.  In this paper, we present StreaMIT in legal
Java syntax for ease of presentation.  Though this syntax can express
the fundamental ideas of StreaMIT, in the longer term we plan to
develop a cleaner and more abstract syntax that is designed
specifically for stream programs.

In the Appendix we have included a detailed example of a software
radio implemented in StreaMIT; a block diagram of the system appears
in Figure \ref{fig:radiodiagram}.  In the following sections, we draw
on different components of this example to describe and justify the
major features of StreaMIT.

\subsection{Filters}

\begin{figure}
\scriptsize
\begin{verbatim}
class FIR extends Filter {
   Channel input = new ComplexChannel();
   Channel output = new ComplexChannel();           
   int N;

   void init(int N) {
      this.N = N;
   }

   void work() {
      Complex sum = 0;
      for (int i=0; i<N; i++) {
         sum += input.peek(i)*fir_coeff[i][N];
      }
      input.pop();
      output.push(sum);
   }
}
\end{verbatim}
\vspace{-12pt}
\caption{\protect\small A Finite Input Response (FIR) filter in StreaMIT.
\protect\label{fig:firfilter}}
\vspace{-12pt}
\end{figure}

\subsubsection{StreaMIT Approach}

The basic unit of computation in StreaMIT is the {\tt Filter}.  An
example of a Filter is the {\tt FIRFilter}, a component of our
software radio (see Figure \ref{fig:firfilter}).  Each {\tt Filter}
contains an {\tt init} function that is called at initialization time;
in this case, the {\tt FIRFilter} records {\tt N}, the number of items
it should filter at once.

The {\tt work} function describes the most fine grained execution step
fo the filter in the steady state.  Within the {\tt work} function,
the filter can communicate with neighboring blocks using the {\tt
input} and {\tt output} channels, which are typed FIFO queue's
declared as fields at the top of the class.  These high-volume
channels support the three intuitive operations: 1) {\tt pop} removes
an item from the end of the channel and returns its value, 2) {\tt
peek(i)} returns the value of the item $i$ spaces from the end of the
channel without removing it, and 3) {\tt push(v)} writes the value $v$
to the front of the channel.  For now, we require that the number of
items popped, peeked, and pushed by each filter is constant from one
invocation of the {\tt work} function to the next.

\subsubsection{Rationale}

This representation of a filter is an improvement over general-purpose
languages.  In a procedural language, the analog of a filter is a
block of statements in a complicated loop nest.  This representation
is unnatural for expressing the feedback and parallelism that is
inherent in streaming systems.  Also, there is no clear abstraction
barrier between one filter and another, and high-volume stream
processing is muddled with global variables and control flow.  The
loop nest must be re-arranged if the input or output ratios of a
filter changes, and scheduling optimizations further inhibit the
readability of the code.  In contrast, StreaMIT places the filter in
its own independent unit, making explicit the parallelism and
inter-filter communication while hiding the grungy details of
scheduling and optimization from the programmer.

One could also use an object-oriented language to implement a stream
abstraction.  This avoids some of the problems associated with a
procedural loop nest, but the programming model is again complicated
by efficiency concerns.  That is, a runtime library might exectute
filters according to a pull model, where a filter operates on a block
of data that it retrieves from the input channel.  The block size is
often optimized for the cache size of a given architecture, which
hampers portability.  Moreover, operating on large-grained blocks
obscures the fundamental fine-grained algorithm that is visible in a
StreaMIT filter.  Thus, the absence of a runtime model in favor of
automated scheduling and optimization again distinguishes StreaMIT.

\subsection{Connecting Filters}

\subsubsection{StreaMIT Approach}

\begin{figure}
\scriptsize
\begin{verbatim}
class FFT extends Stream {
   void init(int N) {
      add(new SplitJoin() {
         void init() {
            setSplitter(WEIGHTED_ROUND_ROBIN(N/2, N/2));
            for(int i=0; i<2; i++) 
               add(new SplitJoin() {
                  void init() {
                     round_robin();
                     add (new Identity());
                     add (new Identity());
                     weighted_round_robin(N/4, N/4);
               }});
            setJoiner(ROUND_ROBIN);
      }});
      for(int i=2; i<=N/2; i *= 2)
        add(new Butterfly(i, N));
}}
\end{verbatim}
\vspace{-12pt}
\caption{\protect\small A Fast Fourier Transform (FFT) in StreaMIT.
\protect\label{fig:fft}}
\vspace{-12pt}
\end{figure}

The basic construct for composing filters into a communicating network
is a {\tt Stream}.  The FFT in Figure \ref{fig:firfilter} is an
example of a {\tt Stream} that appears in our software radio.  Like a
{\tt Filter}, a {\tt Stream} has an {\tt init} function that is called
upon its instantiation.  However, there is no {\tt work} function, and
all input and output channels are implicit; instead, the stream
behaves as the sequential composition of filters that are specified
with successive calls to {\tt add} from within {\tt init}.  That is,
{\tt Stream} creates a single pipeline.

There are two other stream constructors besides {\tt Stream}: {\tt
SplitJoin} and {\tt FeedbackLoop}.  The former is used to specify
independent parallel streams that diverge from a common {\it splitter}
and merge into a common {\it joiner}.  There are three kinds of
splitters:
\begin{enumerate}
\item WEIGHTED\_ROUND\_ROBIN($i_1$, $i_2$, $\dots$, $i_k)$),
which sends the first $i_1$ data items to the first stream, the next
$i_2$ data items to the second stream, and so on.
\item ROUND\_ROBIN, which is just a weighted round robin where all
weights are 1.
\item DUPLICATE, which replicates each data item and sends a copy to each
parallel stream.
\item NULL, which means that all of the parallel components are
sources and there is no input to split.
\end{enumerate}
Similarly, there are three kinds of joiners: 
\begin{enumerate}
\item WEIGHTED\_ROUND\_ROBIN($i_1$, $i_2$, $\dots$, $i_k)$), which reads the
first $i_1$ data items from the first stream, the next $i_2$ data
items to the second stream, and so on
\item ROUND\_ROBIN, which is just a weighted round robin where all
weights are 1
\item COMBINE, which reads all from all the streams in parallel and
combines the results in a structure which is passed on.
\item NULL, which means that all of the parallel components are sinks
and there is no output to join together.
\end{enumerate}
The splitter and joiner type are specified with calls to {\tt
setSplitter} and {\tt setJoiner}, respectively (see Figure \ref{fig:fft}).

\begin{figure}
\scriptsize
\begin{verbatim}
class Fibonnacci extends FeedbackLoop {
   void init() {
      setDelay(2);
      setBody(new Filter() {
          Channel input = new IntChannel();
          Channel output = new IntChannel();
          void work() {
              output.push(input.peek(0)+input.peek(1));
              input.pop();
          }
      });
   }

   int initPath(int index) {
      return index;
   }
}
\end{verbatim}
\vspace{-12pt}
\caption{\protect\small A FeedbackLoop version of Fibonnacci.
\protect\label{fig:feed}}
\vspace{-12pt}
\end{figure}

The last control construct provides a way to create cycles in the
stream graph: the {\tt FeedbackLoop}.  It contains a body stream (set
with {\tt setBody}, a feedback stream (set with {\tt setLoop}), and
round robin splitters and joiners (see Figure \ref{fig:feed}).  The
feedback loop has a special semantics when the stream is first
starting to run.  Since there are no items on the feedback path at
first, the stream instead inputs items from an {\tt initPath} function
defined by the {\tt FeedbackLoop}; {\tt initPath} is called with the
number of the data item that is being fabricated (starting from 0).
With a call to {\tt setDelay} from within the {\tt init} function, the
user can specify how many items should be calculated with {\tt
initPath} before the joiner looks for data items from the feedback
channel.

Evident in all of these examples is another feature of the StreaMIT
syntax: {\it inlining}.  The definition of any stream or filter can be
inlined at the point of its instantiation, thereby preventing the
definition of many small classes that are used only once, and,
moreover, providing a syntax that reveals the hierarchical structure
of the streams from the indentation level of the code.  In our Java
syntax, we make use of anonymous classes for inlining \cite{java}.

\subsubsection{Rationale}

StreaMIT differs from other languages in that it imposes a
well-defined structure on the streams; all stream graphs are built out
of a hierarchical composition of Streams, SplitJoins, and
FeedbackLoops.  This is in contrast to other environments, which
generally regard a stream as a flat and arbitrary network of filters
that are connected by channels.  However, arbitrary graphs are very
hard for the compiler to analyze, and equally difficult for a
programmer to describe.  Most programmers either resort to
straight-line code that links one filter to another (thereby making it
very hard to visualize the stream graph), or using an ad-hoc graphical
programming environment that is awkward to use and admits no good
textual representation.

In contrast, StreaMIT is a clean textual representation
that--especially with inlined streams--makes it very easy to see the
shape of the computation from the indentation level of the code.  The
comparison of StreaMIT's structure with arbitrary stream graphs could
be likened to the difference between structured control flow and GOTO
statements.  Though sometimes the structure restricts the
expressiveness of the programmer, the gains in robustness,
readability, and compiler analysis are immense.

A final benefit of stream graph construction in StreaMIT is the
ability to do {\it scripting} to parameterize graphs.  For instance,
both the FFT stream in Figure \ref{fig:fft} inputs a parameter {\tt N}
and adjusts the number of butterfly stages appropriately.  This
further improves readability and decreases code size.

\subsection{Messages}

\subsubsection{StreaMIT Approach}

\begin{figure}
\scriptsize
\begin{verbatim}
class CheckFreqHop extends SplitJoin {
   RFtoIFPortal freq_hop;
   void init(Portal fh) {
      freq_hop = fh;
      weighted_round_robin(N/4-2,1,1,N/2,1,1,N/4-2);
      int k = 0;
      for(int i=0; i<4; i++) {
         if((i==0)||(i==2)) {
            for(int j=0; j<2; j++) {
               add(new Filter() {
                  Channel input = new FloatChannel();
                  Channel output = new FloatChannel();
                  void work() {
                     Float val = input.pop();
                     if(val >= MIN_TRASHOLD) 
                        freq_hop.set_freq(Freq[k], new TimeInterval(4*N, 6*N)); 
                     output.push(val);
                  }
               });
               k++;
            }
         } else
            add(new Identity());
      }
      weighted_round_robin(N/4-2,1,1,N/2,1,1,N/4-2);
   }
}
\end{verbatim}
\vspace{-12pt}
\caption{\protect\small The CheckFreqHop stream from our software
   radio provides an example of the messaging system in StreaMIT.
\protect\label{fig:mess}}
\vspace{-12pt}
\end{figure}

StreaMIT provides a dynamic messaging system for passing irregular,
low-volume control information between filters and streams.  Messages
are sent from within the body of a filter's {\tt work} function,
perhaps to change a parameter in another filter.  For example, in the
CheckFreqHop stream of our software radio example (Figure
\ref{fig:mess}), a message is sent upstream to change the frequency of
the receiver if the downstream component detects that the transmitter
is about to change frequencies.  The sender can continue to execute
while the message is en route, and the {\tt set\_freq} method will be
invoked in the receiver with argument {\tt Freq[k]} when the message
arrives.  Since message delivery is asynchronous, there can be no
return value; only void methods can be message targets.

{\bf Message timing.}The central aspect of the messaging system is a
sophisticated timing mechanism that allows filters to specify when a
message will be received relative to the flow of information between
the sender and the receiver.  Recall that each filter executes
independently, without any notion of global time.  Thus, the only way
for two filters to talk about a time that is meaningful for both of
them is in terms of the data items that are passed through the streams
from one to the other.

In StreaMIT, one can specify a range of latencies for a message to get
delivered.  This latency is measured in terms of an information
``wavefront'' from one filter to another.  For example, in the {\tt
CheckFreqHop} example of Figure \ref{fig:mess}, the sender indicates
an interval of latencies between $4N$ and $6N$.  This means that the
receiver will receive the message immediately following the last
invocation of its own {\tt work} function which produces an item
affecting the output of the {\it sender's} 4'th, 5'th, or 6'th work
functions, counting the sender's current work function as number 0.
Defining this notion precisely is the subject of Section
\ref{sec:time}, but the general idea is simple:  the receiver is
invoked when it sees the information wavefront that the sender sees in
4-6 execution steps.  

In some cases, the ability to synchronize the arrival of a message
with some element of the data stream is very important.  For example,
{\tt CheckFreqHop} knows that the transmitter will change the
frequency between 4 and 6 steps later, in terms of the frame that {\tt
CheckFreqHop} is inputting.  To ensure that the radio changes
frequencies at the same time--so as not to lose any data at the old or
new frequency--{\tt CheckFreqHop} instructs the receiver to switch
frequencies when the {\it receiver} sees one of the last data items at
the old frequency.

{\bf Portals for broadcast messaging.}  StreaMIT also has support for
modular broadcast messaging.  When a sender wants to send a message
that will invoke method $M$ of the receiver $R$ upon arrival, it does
not call $M$ on the object $R$.  Rather, it calls $M$ on a {\it
Portal} of which $R$ is a member.  Portals are typed containers that
forward all messages they receive to the elements of the container.
Portals could be useful in cases when a component of a filter library
needs to announce a message (e.g., that it is shutting down) but does
not know the list of recipients; the user of the library can pass the
filter a Portal containing all interested receivers.

\begin{figure}
\scriptsize
\begin{verbatim}
class TrunkedRadio extends Stream {
   RFtoIFPortal freq_hop = new RFtoIFPortal();
   BoosterPortal onoff = new BoosterPortal().
   void init() {
      add(new ReadFromAtoD());

      RFtoIF RF2IF = new RFtoIF();
      add(RF2IF);
      freq_hop.register(RF2IF);

      Booster ISS = new Booster();
      add(ISS);
      onoff.register(ISS);
 
      add(new FFT());
      add(new CheckFreqHop(freq_hop));
      add(new CheckQuality(onoff));
      add(new AudioBackEnd());
   }
}
\end{verbatim}
\vspace{-12pt}
\caption{\protect\small TrunkedRadio is the top-level class for our
      software radio, demonstrating the use of Portals for messaging.
\protect\label{fig:radiocode}}
\vspace{-12pt}
\end{figure}

In a language with generic data types, a Portal could be implemented
as a templated list.  However, since Java does not yet support
templates, we automatically generate a {\tt \{Class\}Portal} class for
every class \{Class\}.  Our syntax for using Portals is evident in the
high-level radio code in Figure \ref{fig:radiocode}.

\subsubsection{Rationale}

Stream programs present a challenge in that filters need both regular,
high-volume data transfer and irregular, low-volume control
communication.  Moreover, there is the problem of reasoning about the
relative ``time'' between filters when they are running asynchronously
and in parallel.

A different approach to messaging is to embed control messages in the
data stream instead of providing a separate mechanism for dynamic
message passing.  This does have the effect of associating the message
time with a data item, but it is complicated, error-prone, and leads
to unreadable code.  Further, it could hurt performance in the steady
state (if each filter has to check whether or not a data item is
actual data or control, instead) and complicates compiler analysis,
too.  Finally, one can't send messages upstream without creating a
separate data channel for them to travel in.

Another solution is to treat messages as synchronous method calls.
However, this delays the progress of the stream when the message is en
route, thereby degrading the performance of the program and
restricting the compiler's freedom to reorder filter executions.  

We feel that the StreaMIT messaging model is an advance in that it
separates the notions of low-volume and high-volume data
transfer--both for the programmer and the compiler-- without losing a
well-defind semantics where messages are {\it timed} relative to the
high-volume data flow.

\subsection{Re-Initialization}

\subsubsection{StreaMIT Approach}

One of the characteristics of a streaming application is the need for
occaisional re-initialization of sub-regions of the stream graph.
StreaMIT integrates re-initialization with its messaging model.  If a
sender targets a message at the {\tt init} function of a stream or
filter $S$, then when the message arrives, it re-executes the
initialazation code and replaces $S$ with a new version of itself.
However, the new version might have a different structure than the
original if the arguments to the {\tt init} call on re-initialization
were different than during the original initialization.

When an init message arrives, it does not kill all of the data that is
in the stream being re-initialized.  Rather, it {\it drains} the
stream until the wavefront of information (as defined for the
messaging model) from the top of the stream has reached the bottom.
The draining occurs without consuming any data from the input channels
to the re-initialized region.  Instead, a {\tt drain} function of each
filter is invoked to provide input when its other input source is
frozen.  (Each filter can override the {\tt drain} function as part of
its definition.)  If the programmer prefers to kill the data in a
stream segment instead of draining it, this can be indicated by
sending an extra argument to the message portal with the
re-initialization message.

\subsubsection{Rationale}

Re-initialization is a headache for stream programmers because--if
done manually--the entire runtime system could be put on hold to
re-initialize a portion of the stream.  The interface to starting and
stopping streams could be complicated when there is not an explicit
notion of initialization time vs. steady-state execution time.

StreaMIT improves on this situation by abstracting the
re-initialization process from the user.  Additionally, any
hierarchical stream construct automatically becomes a possible
candidate for re-initialization, due to the well-defined stream
structure and the simple interface with the {\tt init} function.
Finally, it is easy for the compiler to recognize stream
re-initialization possibilities and to account for all possible
configurations of the stream flow graph during analysis and optimization.

\subsection{Realtime Constraints}



