\clearpage

\begin{figure}[h]
\scriptsize
\begin{verbatim}
class Butterfly extends Stream {
   void init(int N, int W) {
      add(new SplitJoin() {
         void init() {
            setSplitter(WEIGHTED_ROUND_ROBIN(N, N));
            add (new Identity());
            add (new Filter() {
               float weights[W];
               int curr;
               Channel input = new FloatChannel();
               Channel output = new FloatChannel();
               void init() {
                  for(int i=0; i<W; i++)
                     weights[i] = calcWeight(i, N, W);
                  curr = 0;
               }
               void work() {
                  output.push(input.pop()*weights[curr++]);
                  if(curr>= W) curr = 0;
               }    
            });
            setJoiner(ROUND_ROBIN);
      }});
      add(new SplitJoin() {
         void init() {
            setSplitter(DUPLICATE);
            add (new Filter() {   
               Channel input = new FloatChannel();
               Channel output = new FloatChannel();
               void work() {
                  output.push(input.pop() + input.pop());
               }
            });
            add (new Filter() {   
               Channel input = new FloatChannel();
               Channel output = new FloatChannel();
               void work() {
                  output.push(input.pop() - input.pop());
               }
            });
            setJoiner(WEIGHTED_ROUND_ROBIN(N, N));
      }});
}}

class FFT extends Stream {
   void init(int N) {
      add(new SplitJoin() {
         void init() {
            setSplitter(WEIGHTED_ROUND_ROBIN(N/2, N/2));
            for(int i=0; i<2; i++) 
               add(new SplitJoin() {
                  void init() {
                     setSplitter(ROUND_ROBIN);
                     add (new Identity());
                     add (new Identity());
                     setJoiner(WEIGHTED_ROUND_ROBIN(N/4, N/4));
               }});
            setJoiner(ROUND_ROBIN);
      }});
      for(int i=2; i<=N/2; i *= 2)
        add(new Butterfly(i, N));
}}

class FIR extends Filter {
   Channel input = new FloatChannel();
   Channel output = new FloatChannel();           
   int N;
   void init(int N) {
      this.N = N;
   }
   void work() {
      float sum = 0;
      for (int i=0; i<N; i++) {
         sum += input.peek(i)*FIR_COEFF[i][N];
      }
      input.pop();
      output.push(sum);
   }
}

class Booster extends Stream {
   void init(boolean adds) {
      if(adds) add(new FIR(N));
   }
}
\end{verbatim}
\vspace{-12pt}
\caption{\protect\small The StreaMIT program for the Trunked Radio Receiver.
\protect\label{fig:radiocode}}
\vspace{-12pt}
\end{figure}

\begin{figure}[h]
\scriptsize
\begin{verbatim}
class RFtoIF extends Filter {
   Channel input = new FloatChannel();
   Channel output = new FloatChannel();
   int sz, i;
   float weight[];
   void init(float f) {
      setf(f);
   }
   void work() {
      output.push(input.pop()*weight[i++]);
      if(i==sz) i = 0;
   }
   void setf(float f) {
      i = 0;
      sz = CarrierFreq/f*N;
      weight = new float[sz];
      for(int i=0; i<sz; i++)
         weight[i] = sine(i*PI/sz);
   }
}

class CheckFreqHop extends SplitJoin {
   RFtoIFPortal fh;
   void init(RFtoIFPortal fh) {
      this.fh = fh;
      setSplitter(WEIGHTED_ROUND_ROBIN(N/4-2,1,1,N/2,1,1,N/4-2));
      int k = 0;
      for(int i=0; i<4; i++) {
         if((i==0)||(i==2)) {
            for(int j=0; j<2; j++) {
               add(new Filter() {
                  Channel input = new FloatChannel();
                  Channel output = new FloatChannel();
                  void work() {
                     float val = input.pop();
                     if(val >= MIN_THRESHOLD) 
                        fh.setf(Freq[k], new TimeInterval(4*N, 6*N)); 
                     output.push(val);
                  }
               });
               k++;
            }
         } else add(new Identity());
      }
      setJoiner(WEIGHTED_ROUND_ROBIN(N/4-2,1,1,N/2,1,1,N/4-2));
   }
}

class CheckQuality extends Filter {
   Channel input = new FloatChannel();
   Channel output = new FloatChannel();
   float aveHi, aveLo;   
   BoosterPortal onOffSwitch;
   void init(BoosterPortal onOffSwitch) {
      aveHi = 0; aveLo = 1;
      this.onOffSwitch = onOffSwitch;
   }
   void work() {
      float val = input.pop();
      aveHi = max(0.9*aveHi, val);
      aveLo = min(1.1*aveLo, val);
      if(aveHi - aveLo < QUAL_BAD_THRESHOLD)
         onOffSwitch.init(true, BEST_EFFORT);
      if(aveHi - aveLo > QUAL_GOOD_THRESHOLD)
         onOffSwitch.init(false, BEST_EFFORT);
      output.push(val);
   }
}

class TrunkedRadio extends Stream {
   RFtoIFPortal freqHop = new RFtoIFPortal();
   BoosterPortal onOff = new BoosterPortal().
   void init() {
      add(new ReadFromAtoD());
      RFtoIF rf2if = new RFtoIF(STARTFREQ);
      add(rf2if);
      freqHop.register(rf2if);
      Booster iss = new Booster(false);
      add(iss);
      onOff.register(iss);
      add(new FFT());
      add(new CheckFreqHop(freqHop));
      add(new CheckQuality(onOff));
      add(new AudioBackEnd());
   }
}
\end{verbatim}
\vspace{-12pt}
\end{figure}

\newpage

\begin{figure*}[h]
\centering
\psfig{figure=fft-pre-tape.eps,width=4.8in}
\caption{The bit reverse order filter in FFT. The tapes at each
channel illustrates the data re-shuffling when N=8. }
\label{fig:bitreverseorder}
\end{figure*}

\begin{figure*}[h]
\centering
\psfig{figure=fft-butterfly-tape.eps,width=5.8in}
\caption{The 4x4 butterfly stage in the FFT. The tapes at each channel illustrates the data transformation and computation. }
\label{fig:butterfly}
\end{figure*}


