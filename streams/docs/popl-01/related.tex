\section{Related Work}
\label{sec:related}
A large number of programming languages have included a concept of a
stream; see \cite{survey97} for a survey.  Those that are perhaps most
related to the static-rate version of StreaMIT are synchronous
dataflow languages such as LUSTRE~\cite{lustre} and
ESTEREL~\cite{esterel92} which require a fixed number of inputs to
arrive simultaneously before firing a stream node.  However, most
special-purpose stream languages are functional instead of imperative,
and do not contain features such as messaging and support for modular
program development that are essential for modern stream applications.
Also, most of these languages are so abstract and unstructured that
the compiler cannot perform enough analysis and optimization to result
in an efficient implementation.

At an abstract level, the stream graphs of StreaMIT share a number of
properties with synchronous dataflow (SDF) domain as considered by the
Ptolemy project~\cite{ptolemyoverview}.  Each node in an SDF graph
produces and consumes a given number of items, and there can be delays
along the arcs between nodes (corresponding loosely to items that are
peeked in StreaMIT).  As in StreaMIT, SDF graphs are guaranteed to
have a static schedule, testing for deadlock is decidable, and there
are a number of nice scheduling results incorporating code size and
execution time~\ref{leesdf}.  However, previous results on SDF
scheduling do not consider constraints imposed by point-to-point
messages, and do not include a notion of StreaMIT's information
wavefronts, re-initialization, and programming language support.

A specification package used in industry bearing some likeness to
StreaMIT is SDL: Specificacation and Description
Language~\cite{sdlrec}.  SDL is a formal, object-oriented language for
describing the structure and behavior of large, real-time systems,
especially for telecommunications applications.  It includes a notion
of asynchronous messaging based on queues at the receiver, but does
not incorporate wavefront semantics as does StreaMIT.  Moreover, its
focus is on specification and verification whereas StreaMIT aims to
produce an efficient implementation.

