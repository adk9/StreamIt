\section{Timing Semantics}

In this section we develop a more formal semantics for the message
delivery guarantees described above.  The timing model in StreaMIT is
unique in that all time is relative to {\em information
wavefronts}--that is, two filters can describe a time in terms of

- Definition of wavefront stuff

- Scheduling Constraints
	- data flow
	- messaging

- Semantics of the functions
	- in terms of messages and latency

\begin{verbatim}

let's try some new definitions:

given that there are exactly x items on your input tape, the maximum
number of items you can have on your output tape is h(x)

the inverse is now well-defined: given that there are exactly x items
on your output tape, the minimum number of items you need on your
input tape is p(x)

filter:		h(x) = push*max(0, floor((x-(peek-pop))/pop))
split:		h(x) = (ceil(x/2), floor(x/2))
join:		h(x1, x2) = min(2*x1-1, 2*x2)

filter:		p(x) = ceil(x/push)*pop+(peek-pop)
split:		p(x1, x2) = min(2*x1-1, 2*x2)
join:		p(x) = (ceil(x/2), floor(x/2))

--

scheduling constraints:

- again let n_a denote the number of items on the OUTPUT tape of filter a

if have filters a and b in pipeline (a -> b), then these are equivalent:
	- n_b <= h_b(n_a)
	- n_a >= p_b(n_b)

if have splitjoin with input a, then (b, c) in parallel, and output d:
	- there are n_a on input
	- can have up to h(n_a) = (ceil(n_a/2), floor(n_a/2)) into (b, c)
	- can have up to (h_b(ceil(n_a/2)), h_c(floor(n_a/2))) out of (b, c)
	- can have up to min(2*h_b(ceil(n_a/2))-1,2*h_c(floor(n_a/2))) into d

	so constraint is that:
	- n_d <= h_d( min(2*h_b(ceil(n_a/2))-1,2*h_c(floor(n_a/2))) )

	or, in other words, h(x) for a split/join is:
		h(x) = min(2*h_b(ceil(n_a/2))-1,2*h_c(floor(n_a/2)))

if sending message from a to b with max wavefront latency n, then:
	
	1. if b is downstream, with filters x1, x2, x3 in between a and b:
			a -> x1 -> x2 -> x3 -> b

		n_b <= h_b(h_x3(h_x2(h_x1(n_a + n * a.push))))

	2. if b is upstream, with filters x1, x2, x3 between b and a:
			b -> x1 -> x2 -> x3 -> a

		n_b <= p_x3(p_x2(p_x1(p_a(n_a + n * a.push))))

	3. in the general case, a is sending message to b.  trace a
		path, finding y1, y2, y3 that are blocks between OUTPUT of
		a and OUTPUT of b.  a -> y1 -> y2 -> y3 -> b.  (note that
		if b is upstream, then y1 = a.)  Then we have:

		n_b <= z_y3(z_y2(z_y1(n_ a + n * a.push)))

		where z is replaced by h if we are traveling downstream, 
		but replaced by p if we are traveling upstream.

aha, if coming from common sender or going to common receiver, the wavefront
	must have EQUAL index at the sender or receiver.  so you REFLECT
	around split/join points instead of translating through them.
	because if you go up one part of join and down the other, the
	wavefront shifts

\end{verbatim}

