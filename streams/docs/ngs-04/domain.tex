\section{Streaming Application Domain}
\label{sec:domain}

The applications  that make use  of a stream abstraction  are diverse,
with targets  ranging from embedded devices, to  consumer desktops, to
high-performance servers.  Examples include  systems such as the Click
modular router \cite{click} and the Spectrumware software radio
\cite{spectrumware,softwareradio};   specifications    such   as   the
Bluetooth communications protocol \cite{bluetooth}, the GSM Vocoder
\cite{gsm}, and the AMPS cellular base station\cite{amps}; and almost
any application developed with Microsoft's DirectShow library
\cite{directshow}, Real Network's RealSDK \cite{realsdk} or Lincoln
Lab's Polymorphous Computing Architecture \cite{pca}.


We  have identified a  number of  properties that  are common  to such
applications---enough  so as  to characterize  them as  belonging  to a
distinct class of programs, which  we will refer to as \emph{streaming
applications}.   We  believe that  the  salient  characteristics of  a
streaming application are as follows:

\begin{enumerate}
\item \emph{Large streams of data.}  Perhaps the most fundamental
  aspect of a streaming application is that it operates on a large (or
  virtually infinite) sequence of data items, hereafter referred to as
  a \emph{data stream}.  Data streams generally enter the program from
  some external source, and each  data item is processed for a limited
  time  before being  discarded.  This  is in  contrast  to scientific
  codes, which  manipulate a  fixed input set  with a large  degree of
  data reuse.

\item \emph{Independent stream filters.}  Conceptually, a streaming
  computation  represents a  sequence of  transformations on  the data
  streams in  the program.  We  will refer to  the basic unit  of this
  transformation  as  a  \emph{filter}:  an operation  that---on  each
  execution  step---reads one  or  more items  from  an input  stream,
  performs some computation, and writes one or more items to an output
  stream.   Filters  are  generally  independent  and  self-contained,
  without references  to global variables or other  filters.  A stream
  program is the composition of filters into a \emph{stream graph}, in
  which the  outputs of  some filters are  connected to the  inputs of
  others.
  
\item \emph{A stable computation pattern.}  The structure of the
  stream graph is generally constant during the steady-state operation
  of  a  stream  program.  That  is,  a  certain  set of  filters  are
  repeatedly  applied in a  regular, predictable  order to  produce an
  output stream that is a given function of the input stream.
  
\item \emph{Occasional modification of stream structure.}  Even though
  each arrangement of  filters is executed for a  long time, there are
  occasional dynamic modifications to the stream graph.  For instance,
  a software radio re-initializes a portion of the stream graph when a
  user switches  from AM  to FM.  Sometimes,  these re-initializations
  are synchronized  with some  data in the  stream, as when  a network
  protocol changes  from Bluetooth to 802.11  at a certain  point of a
  transmission.    There  is   typically  an   enumerable   number  of
  configurations that the  stream graph can adopt in  any one program,
  such that all  of the possible arrangements of  filters are known at
  compile time.
  
\item \emph{Occasional out-of-stream communication.}  In addition to
  the  high-volume data streams  passing from  one filter  to another,
  filters also communicate small  amounts of control information on an
  infrequent  and  irregular  basis.   Examples include  changing  the
  volume on  a cell phone, printing  an error message to  a screen, or
  changing a coefficient in an upstream FIR (Finite Impulse Response) filter.
  
\item \emph{High performance expectations.}  Often there are real-time
  constraints that must be  satisfied by streaming applications. Thus,
  efficiency (in  terms of both  latency and throughput) is  a primary
  concern.  Additionally, many  embedded applications are intended for
  mobile  environments where  power consumption,  memory requirements,
  and code size are also important.
\end{enumerate}
